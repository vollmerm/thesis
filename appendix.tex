\paragraph{Notation for references to well-formedness judgements}
Because there are many requirements specified inside the various
well-formedness judgements, I will introduce notation for referring
to requirements individually.
%
For example, the notation
\refwellformed{sec:well-formedness-allocation}{wf:impl-linear-alloc2}
refers to the judgement
\begin{align*}
\storewfca{\AENV}{\NENV}{\MENV}{\STOR},
\end{align*}
specified in Section~\ref{sec:well-formedness-allocation},
and in that judgement, rule number~\ref{wf:impl-linear-alloc2}.


\paragraph{Variables and Substitution}
%
I use the convention that all variables for binding values,
locations, and regions are distinct, and maintain this invariant
implicitly.
%
The bindings sites of variables are summarized by the following:
%
\begin{itemize}
\item Variables for binding values $\var$ are bound by function definitions
$\FD$ and pattern matches $\pat$.

\item Location variables $\locreg{\loc}{\reg}$ are bound by type schemes $\TS$, pattern matches
$\pat$, and $\keywd{letloc}$ binders.

\item Region variables $\reg$ are bound by type schemes $\TS$, pattern
matches $\pat$, and $\keywd{letregion}$ binders.
\end{itemize}
%
The use sites of variables are summarized by the following:
\begin{itemize}
\item Variables for binding varlues $\var$ are used by values $\VAL$.

\item Location variables $\locreg{\loc}{\reg}$ are used by concrete locations $\concreteloc{\reg}{\ind}{\locreg{\loc}{\reg}}$,
the argument list of function applications
$\fapp{\overharpoon{\locreg{l}{r}}}{\overharpoon{\VAL}}$, the location
argument of constructor applications
$\datacon{\DC}{\keywd{\locreg{\loc}{\reg}}}{\overharpoon{\VAL}}$,
located types $\hTYP$, and located expressions $\LE$.

\item Region varaibles $\reg$ are used in the same places as location variables.
\end{itemize}
%
We use the following conventions for variable substitution:
%
\begin{itemize}
\item $\subst{\EXPR}{x}{v}$: Substitute $v$ for $x$ in $e$. We let the notation extend to vectors such that
$\subst{\EXPR}{\overharpoon{x}}{\overharpoon{v}}$ denotes the iterated substitution $\subst{\EXPR}{\overharpoon{x_1}}{\overharpoon{v_1}} \ldots \subst{}{\overharpoon{x_n}}{\overharpoon{v_n}}$, where $n = |\overharpoon{x}| = |\overharpoon{v}|$.

\item $\subst{\EXPR}{\locreg{\loc_1}{\reg_1}}{\locreg{\loc_2}{\reg_2}}$: Substitute location variable $\locreg{\loc_2}{\reg_2}$ for $\locreg{\loc_1}{\reg_1}$ in $\EXPR$. We extend this notation to vectors of locations in the same fashion, as described above.

\item $\subst{\EXPR}{\reg_1}{\reg_2}$ : Substitute region variable $\reg_2$ for $\reg_1$ in $\EXPR$. We extend this notation to vectors of locations in the same fashion, as described above.

\item Finally, we extend the aforementioned notation so that
  substitution can act on environments $\CENV$, $\AENV$, and $\NENV$,
  e.g.,
  $\subst{\CENV}{\locreg{\loc_1}{\reg_1}}{\locreg{\loc_2}{\reg_2}}$.
\end{itemize}

\paragraph{Metafunctions}

\begin{itemize}
\item $Function(f)$: An environment that maps a function $f$ to its definition $\FD$.

\item $Freshen(\FD)$: A metafunction that freshens all bound variables in function definition
$\FD$ and returns the resulting function definition.

\item $TypeOfCon(\DC):$ An environment that maps a data constructor to its type.

\item $TypeOfField(\DC,i)$: A metafunction that returns the type of the \il{i}'th field
of data constructor $\DC$.

\item $ArgTysOfConstructor(\DC)$: An environment that maps a data constructor to its field types.

\item $\allocptr{\reg}{\STOR}$: $\max \set{-1} \cup \set{ \indj \; | \; {(\reg \mapsto (\indj \mapsto \DC)) \in \STOR}}$.
\end{itemize}

% \section{Well-formedness of the Store}
% \label{sec:well-formedness}

% The well formedness of the store is defined by the top-level judgement
% \begin{displaymath}
% \storewf{\SENV}{\CENV}{\AENV}{\NENV}{\MENV}{\STOR}
% \end{displaymath}
% whose definition itself uses three other judgements.
% %
% All of these judgements are summarized in Table~\ref{tbl:swf-judgements}.

% %\floatstyle{boxed}\restylefloat{table}
% \begin{table}
% \bgroup
% \def\arraystretch{1.2}
% \setlength\tabcolsep{0.5cm}
% \begin{tabular}{lclp{6cm}}
%  & \textbf{Judgement form} & \textbf{Section} &\textbf{Summary}
%  \\\\
% \parbox[t]{3.5cm}{Store \\ well formedness} & $\storewf{\SENV}{\CENV}{\AENV}{\NENV}{\MENV}{\STOR}$ &
% \ref{sec:well-formedness} &
% The store $\STOR$ along with location map $\MENV$ are well formed with respect to
% typing environments $\SENV$, $\CENV$, and $\AENV$.
% \\\\
% End witness & $\ewitness{\TYP}{\concreteloc{\reg}{\ind_{s}}{}}{\STOR}{\concreteloc{\reg}{\ind_{e}}{}}$ &
% \ref{sec:end-witness} &
% The store address $\concreteloc{\reg}{\ind_{e}}{}$ is the position one
% after the last cell of the tree of type $\TYP$ starting at
% $\concreteloc{\reg}{\ind_{s}}{}$ in store $\STOR$.
% \\\\
% \parbox[t]{3.5cm}{Constructor-application \\ well formedness}
%  & $\storewfcfa{\CENV}{\MENV}{\STOR}$ &
% \ref{sec:well-formedness-constructors} &
% All in-flight data-constructor applications in store $\STOR$ along with location map $\MENV$
% are well formed with respect to constructor-progress typing environment $\CENV$.
% \\\\
% \parbox[t]{3.5cm}{Allocation \\ well formedness} & $\storewfca{\AENV}{\NENV}{\MENV}{\STOR}$ &
% \ref{sec:well-formedness-allocation} &
% Allocation in store $\STOR$ along with location map $\MENV$ is well formed
% with respect to allocation-typing environments $\AENV$ and $\NENV$.
% \end{tabular}
% \egroup
% \caption{Summary of judgements used to establish well formedness of the store.}
% \label{tbl:swf-judgements}
% \end{table}

% \paragraph{Notation for references to well-formedness judgements}
% Because there are many requirements specified inside the various
% well-formedness judgements, we introduce notation for referring
% to requirements individually.
% %
% For example, the notation
% \refwellformed{sec:well-formedness-allocation}{wf:impl-linear-alloc2}
% refers to the judgement
% \begin{align*}
% \storewfca{\AENV}{\NENV}{\MENV}{\STOR},
% \end{align*}
% specified in Section~\ref{sec:well-formedness-allocation},
% and in that judgement, rule number~\ref{wf:impl-linear-alloc2}.

% The definition of store well formedness follows.

% \paragraph{Judgement form}

% $\storewf{\SENV}{\CENV}{\AENV}{\NENV}{\MENV}{\STOR}$

% The well-formedness judgement specifies the valid layouts of the store by using the location
% map and the various environments from the typing judgement.
% %
% Rule~\ref{wf:map-store-consistency} specifies that, for each location in the store-typing environment,
% there is a corresponding concrete location in the location map and that concrete location satisfies
% a corresponding end-witness judgement.
% %
% Rules~\ref{wf:cfc} and~\ref{wf:ca} reference the judgements for well formedness concerning
% in-flight constructor applications (\secref{sec:end-witness}) and correct allocation in
% regions (\secref{sec:well-formedness-allocation}), respectively.
% %
% Finally, Rule~\ref{wf:impl1} specifies that the nursery and store-typing environments reference
% no common locations, which is a way of reflecting that each location is either in the process
% of being constructed and in the nursery, or allocated and in the store-typing environment, but
% never both.

% \paragraph{Definition}

% \begin{enumerate}

%     \item \label{wf:map-store-consistency} $ (\locreg{\loc}{\reg} \mapsto \TYP) \in \SENV \Rightarrow \\
%             ((\locreg{\loc}{\reg} \mapsto \concreteloc{\reg}{\ind_1}{}) \in \MENV \wedge \\
%             \ewitness{\TYP}{\concreteloc{\reg}{\ind_1}{}}{\STOR}{\concreteloc{\reg}{\ind_2}{}})
%           $

%     \item \label{wf:cfc} $\storewfcfa{\CENV}{\MENV}{\STOR}$

%     \item \label{wf:ca} $\storewfca{\AENV}{\NENV}{\MENV}{\STOR}$

%     \item \label{wf:impl1} $dom(\SENV) \cap \NENV = \emptyset $
% \end{enumerate}

% \subsection{End-Witness judgement}
% \label{sec:end-witness}

% \paragraph{Judgement form}

% $\ewitness{\TYP}{\concreteloc{\reg}{\ind_{s}}{}}{\STOR}{\concreteloc{\reg}{\ind_{e}}{}}$

% The end-witness judgement specifies the expected layout in the store of a fully
% allocated data constructor.
% %
% Rule~\ref{ewitness:impl1} requires that the first cell store a constructor
% tag of the appropriate type.
% %
% Rule~\ref{ewitness:impl2} specifies the address of the cell one past the tag.
% %
% Rule~\ref{ewitness:impl3} recursively specifies the positions of the constructor
% fields.
% %
% Finally, Rule~\ref{ewitness:impl4} specifies that the end witness of
% the overall constructor is the address one past the end of either the
% tag, if the constructor has zero fields, or the final field,
% otherwise.

% \paragraph{Definition}

% \begin{enumerate}
% \item \label{ewitness:impl1} $\STOR(\reg)(\ind_s) = \DC'$ \; \text{such that} \\
%       $\; \DATA\;\TYP = \overharpoon{\DC_1 \; \overharpoon{\sTYP}_1\;} \; | \; \ldots \; | \; \DC' \; \overharpoon{\sTYP}' \; | \; \ldots \; | \; \overharpoon{\DC_m \; \overharpoon{\sTYP}_m\;}$

% \item \label{ewitness:impl4} $\overharpoon{w_0} = \ind_s + 1$

% \item \label{ewitness:impl2}
%   $\ewitness{\overharpoon{\TYP'_1}}{\concreteloc{\reg}{\overharpoon{w_0}}{}}{\STOR}{\concreteloc{\reg}{\overharpoon{w_1}}{}} \wedge$ \\
%   $\ewitness{\overharpoon{\TYP'_{j+1}}}{\concreteloc{\reg}{\overharpoon{w_j}}{}}{\STOR}{\concreteloc{\reg}{\overharpoon{w_{j+1}}}{}}$
%   \\ where $\indj \in \set{1,\ldots,n-1} ; n = | \overharpoon{\TYP'} |$

% \item \label{ewitness:impl3}
%   $\ind_e = \overharpoon{w_n}$
% \end{enumerate}

% \subsection{Well-formedness of constructor application}
% \label{sec:well-formedness-constructors}

% \paragraph{Judgement form}

% $\storewfcfa{\CENV}{\MENV}{\STOR}$

% The well-formedness judgement for constructor application specifies the various constraints
% that are necessary for ensuring correct formation of constructors, dealing with constructor
% application being an incremental process that spans multiple \ourcalc{} instructions.
% %
% Rule~\ref{wfconstr:constraint-start} specifies that, if a location corresponding to the
% first address in a region is in the constraint environment, then there is a corresponding
% entry for that location in the location map.
% %
% Rule~\ref{wfconstr:constraint-tag} specifies that, if a location corresponding to the address one past a constructor
% tag is in the constraint environment, then there are corresponding locations for the address
% of the tag and the address after in the location map.
% %
% Rule~\ref{wfconstr:constraint-after} specifies that, if a location corresponding to the address
% one past after a fully allocated constructor application is in the constraint environment,
% then there are corresponding locations for the address one past the constructor application
% and for the address of the start of that constructor application in the location map, as well as the existence
% of an end witness for that fully allocated location.

% \paragraph{Definition}

% \begin{enumerate}
%     \item \label{wfconstr:constraint-start} $ (\locreg{\loc}{\reg} \mapsto \startr{\reg}) \in \CENV \Rightarrow \\
%             (\locreg{\loc}{\reg} \mapsto \concreteloc{\reg}{0}{}) \in \MENV $

%     \item \label{wfconstr:constraint-tag} $ (\locreg{\loc}{\reg} \mapsto (\locreg{\loc'}{\reg} + 1)) \in \CENV \Rightarrow
%             \\
%             (\locreg{\loc'}{\reg} \mapsto \concreteloc{\reg}{\ind_l}{})  \in \MENV \wedge \\
%             (\locreg{\loc}{\reg} \mapsto \concreteloc{\reg}{\ind_l + 1}{})  \in \MENV
%             $

%     \item \label{wfconstr:constraint-after} $ (\locreg{\loc}{\reg} \mapsto \afterl{\tyatlocreg{\TYP}{\locreg{\loc'}{\reg}}{\reg}}) \in \CENV \Rightarrow \\
%             ((\locreg{\loc'}{\reg} \mapsto \concreteloc{\reg}{\ind_1}{}) \in \MENV \wedge \\
%             \ewitness{\TYP}{\concreteloc{\reg}{\ind_1}{}}{\STOR}{\concreteloc{\reg}{\ind_2}{}} \wedge \\
%             (\locreg{\loc}{\reg} \mapsto \concreteloc{\reg}{\ind_2}{}) \in \MENV)
%             $

% \end{enumerate}

% \subsection{Well-formedness concerning allocation}
% \label{sec:well-formedness-allocation}

% \paragraph{Judgement form}

% $\storewfca{\AENV}{\NENV}{\MENV}{\STOR}$

% The well-formedness judgement for safe allocation specifies the various properties
% of the location map and store that enable continued safe allocation, avoiding in particular
% overwriting cells, which could, if possible, invalidate overall type safety.
% %
% Rule~\ref{wf:impl-linear-alloc} requires that, if a location is in both the allocation
% and nursery environments, i.e., that address represents an in-flight
% constructor application, then there is a corresponding location in the location
% map and the address of that location is the highest address in the store.
% %
% Rule~\ref{wf:impl-linear-alloc2} requires that, if there is an address in the allocation
% environment and that address is fully allocated, then the address of that location is the
% highest address in the store.
% %
% Rule~\ref{wf:impl-write-once} requires that, if there is an address in the nursery, then
% there is a corresponding location in the location map, but nothing at the corresponding
% address in the store.
% %
% Finally, Rule~\ref{wf:impl-empty-region} requires that, if there is a region that has been
% created but for which nothing has yet been allocated, then there can be no addresses
% for that region in the store.

% \paragraph{Definition}

% \begin{enumerate}
%     \item \label{wf:impl-linear-alloc} $ ((\reg \mapsto \locreg{\loc}{\reg}) \in \AENV \wedge \locreg{\loc}{\reg} \in \NENV) \Rightarrow \\
%           ((\locreg{\loc}{\reg} \mapsto \concreteloc{\reg}{\ind}{}) \in \MENV \wedge
% %          \ind = \max \set{0} \cup \set{\indj \; | \; \concreteloc{\reg}{\indj}{} \in \MENV} \wedge \\
%           \ind > \allocptr{\reg}{\STOR})
%           $

%     \item \label{wf:impl-linear-alloc2} $ ((\reg \mapsto \locreg{\loc}{\reg}) \in \AENV \wedge
%     \, (\locreg{\loc}{\reg} \mapsto \concreteloc{\reg}{\ind_s}{}) \in \MENV \wedge \locreg{\loc}{\reg} \not \in \NENV \wedge
%     \, \ewitness{\TYP}{\concreteloc{\reg}{\ind_s}{}}{\STOR}{\concreteloc{\reg}{\ind_e}{}}) \Rightarrow \\
%           \ind_e > \allocptr{\reg}{\STOR}
%           $

%     \item \label{wf:impl-write-once} $ \locreg{\loc}{\reg} \in \NENV \Rightarrow \\
%           ((\locreg{\loc}{\reg} \mapsto \concreteloc{\reg}{\ind}{}) \in \MENV \wedge \\
%           (\reg \mapsto (\ind \mapsto \DC)) \not \in \STOR)
%           $

%     \item \label{wf:impl-empty-region} $(\reg \mapsto \emptyset) \in \AENV \Rightarrow \\
%     \reg \not \in dom(\STOR)$
% \end{enumerate}

% \newcommand{\substlemmasubsts}[1]{\subst{#1}{\overharpoon{\var}}{\overharpoon{\VAL}} \subst{}{\overharpoon{\locreg{\loc}{\reg}}}{\overharpoon{\locreg{\loc'}{\reg'}}} \subst{}{\locreg{\loc}{\reg}}{\locreg{\loc'}{\reg'}}}

\paragraph{Progress and Preservation}
\label{proof:typesafety}

\begin{lemma}[Substitution lemma]
  \label{lemma:substitution}
  \begin{align*}
  \text{If} \quad & \TENV \cup \set{\overharpoon{\var_1} \mapsto \overharpoon{\TYP_1} \ensuremath{@} \overharpoon{\locreg{\loc_1}{\reg_1}}, \ldots, \overharpoon{\var_n} \mapsto  \overharpoon{\TYP_n} \ensuremath{@} \overharpoon{\locreg{\loc_n}{\reg_n}}}; \SENV; \CENV; \AENV; \NENV \vdash \AENV'; \NENV'; \EXPR : \tyatlocreg{\TYP}{\loc}{\reg}\\
  \text{and} \quad & \TENV; \SENV'; \CENV'; \AENV'; \NENV' \vdash \AENV'; \NENV'; \overharpoon{\VAL_i} : \overharpoon{\TYP_i} \ensuremath{@} \overharpoon{\locreg{\loc'_i}{\reg'_i}} \qquad \; i \in \set{1, \ldots, n}\\
  \text{then} \quad & \TENV; \SENV'; \CENV'; \AENV'; \NENV' \vdash \AENV'''; \NENV'''; \substlemmasubsts{\EXPR} : \tyatlocreg{\TYP}{\loc'}{\reg'}\\
   \text{where} \quad & \SENV = \SENV_0 \cup \set{\overharpoon{\locreg{\loc_1}{\reg_1}} \mapsto \overharpoon{\TYP_1}, \ldots, \overharpoon{\locreg{\loc_n}{\reg_n}} \mapsto \overharpoon{\TYP_n}}\\
   \text{and} \quad & \forall_{(\var \mapsto \TYP'' \ensuremath{@} \locreg{\loc''}{\reg''}) \in \TENV} . (\locreg{\loc''}{\reg''} \mapsto \TYP'') \in \SENV_0 \\
  \text{and} \quad & dom(\SENV) \cap \NENV = \emptyset\\
  \text{and} \quad & \NENV = \NENV_0 \cup \locreg{\loc}{\reg}\\
  \text{and} \quad & \SENV' = \SENV \cup \set{\overharpoon{\locreg{\loc'_1}{\reg'_1}} \mapsto \overharpoon{\TYP_1}, \ldots, \overharpoon{\locreg{\loc'_n}{\reg'_n}} \mapsto \overharpoon{\TYP_n}}\\
  \text{and} \quad & \CENV' = \subst{\CENV}{\overharpoon{\locreg{\loc}{\reg}}}{\overharpoon{\locreg{\loc'}{\reg'}}} \subst{}{\locreg{\loc}{\reg}}{\locreg{\loc'}{\reg'}} \\
  \text{and} \quad & \AENV' = \subst{\AENV}{\overharpoon{\locreg{\loc}{\reg}}}{\overharpoon{\locreg{\loc'}{\reg'}}} \subst{}{\locreg{\loc}{\reg}}{\locreg{\loc'}{\reg'}} \subst{}{\reg}{\reg'}\\
  \text{and} \quad & \NENV' = \subst{\NENV}{\locreg{\loc}{\reg}}{\locreg{\loc'}{\reg'}}
  \end{align*}
\end{lemma}
\begin{nproof}
  The proof is by rule induction on the given typing derivation.
  %
    \begin{bcase} \tvar{}, \tconcreteloc{}\\
    These cases discharge vacuously because
    the corresponding typing judgements cannot establish that the expression $\EXPR$ has type
    $\tyatlocreg{\TYP}{\loc}{\reg}$, as required by the premise of the lemma.
    %
    The reason is that the premise of the lemma also assumes that
    $\locreg{\loc}{\reg} \in \NENV$ and $dom(\SENV) \cap \NENV = \emptyset$, but by inversion
    on the respective typing judgements, it must be that $(\locreg{\loc}{\reg} \mapsto \TYP) \in \SENV$,
    thereby resulting in a contradiction.
  \end{bcase}

  \begin{bcase}
    \begin{mathpar}
    \rtdatacon{}
    \end{mathpar} \\
    By inversion on the typing judgement, there are three proof obligations for this case.
    %
    The first one concerns the subtitution of
    location $\locreg{\loc}{\reg}$, which changes the type of the
    term $\EXPR$ from $\tyatlocreg{\TYP}{\loc}{\reg}$ to
    $\tyatlocreg{\TYP}{\loc'}{\reg'}$.
    %
    The specific obligation is to establish that
    all uses of $\locreg{\loc}{\reg}$ in the
    typing judgement are properly substituted by $\locreg{\loc'}{\reg'}$, thereby
    satisfying the corresponding parts of the typing judgement that need to
    reflect the change in the result location.
    %
    The uses of $\locreg{\loc}{\reg}$ in the typing judgement
    are the first argument of the constructor application,
    the result type, the constraint environment $\CENV$, and environments $\AENV$, $\NENV$,
    $\AENV'$, and $\NENV'$.
    %
    The corresponding updates are established by inspection of the various substitutions
    in the consequent of the lemma, which affect $\EXPR$ and the typing environments.
    %
    The second obligation concerns the locations used by the typing judgement in $\CENV$, each
    of which is substituted as needed in the environment $\CENV'$.

    The third and final obligation is to establish typing judgements required by the premise of \tdatacon{}
    that concern the arguments of the constructor application.
    %
    To distinguish the constructor arguments from the values $\overharpoon{\VAL}$ that are being substituted,
    let the constructor arguments be $\overharpoon{\VAL'}$, and $m = |\overharpoon{\VAL'}|$.
    %
    Then the specific obligation is to establish the typing judgements
    %
    \begin{displaymath}
       \TENV;\SENV';\CENV';\AENV';\NENV' \vdash \AENV';\NENV';\substlemmasubsts{\overharpoon{\VAL'_{k}}} : \overharpoon{\tyatlocreg{\TYP_{k}'}{\loc''_{k}}{\reg}},
     \end{displaymath}
     %
     for all $k \in \set{1, \ldots, m}$, and for some suitable corresponding locations
     $\locreg{\loc''_{k}}{\reg}$.
     %
     Each value $\overharpoon{\VAL'_k}$ is either a variable or a concrete location.
    %
    \begin{itemize}
    \item Case $\overharpoon{\VAL'_k} = \yvar$, for some variable $\yvar$:\\
      \begin{itemize}
      \item Case $\yvar = \overharpoon{\var_{\indj}}$, for some $\indj$:\\
        Now, the obligation is to establish that the value resulting from the substitution of $\yvar$, namely
        $\overharpoon{\VAL_{\indj}}$, has type $\overharpoon{\TYP'_{k}} \ensuremath{@} \overharpoon{\locreg{\loc'_{\indj}}{\reg}}$.
        %
        From the premise of the lemma, we have that
        \begin{displaymath}
        \TENV; \SENV'; \CENV'; \AENV'; \NENV' \vdash \AENV'; \NENV'; \overharpoon{\VAL_j} : \overharpoon{\TYP_j} \ensuremath{@} \overharpoon{\locreg{\loc'_j}{\reg}},
        \end{displaymath}
        and, moreover, by inversion on \tdatacon{}, we can conclude that $\overharpoon{\TYP_j} = \overharpoon{\TYP'_k}$,
        thereby establishing that
       \begin{displaymath}
        \TENV; \SENV'; \CENV'; \AENV'; \NENV' \vdash \AENV'; \NENV'; \overharpoon{\VAL_j} : \overharpoon{\TYP'_k} \ensuremath{@} \overharpoon{\locreg{\loc'_j}{\reg}},
        \end{displaymath}
        and thus discharging this case.
      \item Case $\yvar \neq \overharpoon{\var_{\indj}}$, for all $\indj$:\\
        This case discharges immediately by implication of the typing judgement of the
        source term given in the premise of this lemma, and by inversion on \tvar{}.
      \end{itemize}
    \item Case $\overharpoon{\VAL'_k} = \concreteloc{\reg}{\ind'''}{\locreg{\loc''}{\reg}}$,
      for some location $\locreg{\loc''}{\reg}$, $\ind'''$\\
      \begin{itemize}
      \item Case $\locreg{\loc''}{\reg} = \overharpoon{\locreg{\loc_{\indj}}{\reg}}$, for some
        $\indj$:\\
        The specific obligation is to establish the type of
        the concrete location affected by the substitution of the location $\locreg{\loc''}{\reg}$
        for $\overharpoon{\locreg{\loc'_{\indj}}{\reg}}$, that is,
        \begin{displaymath}
        \TENV; \SENV'; \CENV'; \AENV'; \NENV' \vdash \AENV'; \NENV'; \concreteloc{\reg}{\ind'''}{\overharpoon{\locreg{\loc'_{\indj}}{\reg}}} : \overharpoon{\TYP'_k} \ensuremath{@} \overharpoon{\locreg{\loc'_j}{\reg}}.
        \end{displaymath}
        The above follows from the facts $\SENV'(\overharpoon{\locreg{\loc'_{\indj}}{\reg}}) = \overharpoon{\TYP_{\indj}}$
        and $\overharpoon{\TYP_j} = \overharpoon{\TYP'_k}$, using
        similar reasoning to the previous case, thus discharging this case.
      \item Case $\locreg{\loc''}{\reg} = \locreg{\loc}{\reg}$:\\
      Impossible, because $\locreg{\loc''}{\reg} \in dom(\SENV)$, but from the premise of this lemma,
      $\locreg{\loc}{\reg} \in \NENV$ and $dom(\SENV) \cap \NENV = \emptyset$.
      \item Case $\locreg{\loc''}{\reg} \neq \overharpoon{\locreg{\loc_{\indj}}{\reg}}$, for all
        $\indj$, and $\locreg{\loc''}{\reg} \neq \locreg{\loc}{\reg}$:\\
        This case discharges straightforwardly because, by inversion on \tconcreteloc{},
        $(\locreg{\loc''}{\reg} \mapsto \TYP'') \in \SENV$, thus implying that
        $(\locreg{\loc''}{\reg} \mapsto \TYP'') \in \SENV'$,
        as needed.
      \end{itemize}
    \end{itemize}
  \end{bcase}

  \begin{bcase}
    \begin{mathpar}
    \rtlet{}
    \end{mathpar} \\
    This case discharges via straightforward uses of the induction hypothesis
    for the let-bound expression and the body.
  \end{bcase}


  \begin{bcase} \tlregion{}, \tllstart{}, \tlltag{}, \tllafter{}, \tapp{}, \tcase{} \\
  These remaining cases discharge by similar uses of the induction hypothesis.
  \end{bcase}

$\blacksquare$
\end{nproof}

\begin{lemma}[Progress]
  \label{lemma:progress}
  \begin{displaymath}
  \begin{aligned}
  \text{if} \;\; & \emptyset;\SENV;\CENV;\AENV;\NENV \vdash \AENV';\NENV';\EXPR : \hTYP \\
  \text{and} \;\; & \storewf{\SENV}{\CENV}{\AENV}{\NENV}{\MENV}{\STOR} \\
  \text{then} \;\; & \EXPR \; \mathit{value} \\
  \text{else} \;\; & \STOR;\MENV;\EXPR \stepsto \STOR';\MENV';\EXPR'
  \end{aligned}
  \end{displaymath}
\end{lemma}

\begin{nproof}
  The proof is by rule induction on the given typing derivation.
  %
  \begin{bcase}
    \begin{mathpar}
    \rtdatacon{}
    \end{mathpar} \\

    Because $\EXPR = \datacon{\DC}{\keywd{\locreg{\loc}{\reg}}}{\overharpoon{\VAL}}$ is not
    a value, the proof obligation is to show that there is a rule in the dynamic semantics whose
    left-hand side matches the machine configuration $\STOR;\MENV;\EXPR$.
    %
    The only rule that can match is \ddatacon{}, but to establish the
    match, there remains one obligation, which is obtained
    by inversion on \ddatacon{}.
    %
    The particular obligation is to establish that
    $\concreteloc{\reg}{\ind}{} = \MENV(\locreg{\loc}{\reg})$,
    for some $\ind$.
    %
    To obtain this result, we need to use the well formedness
    of the store, given by the premise of this lemma, and in particular rule
    \refwellformed{sec:well-formedness-allocation}{wf:impl-write-once}.
    %
    But a precondition for using
    \refwellformed{sec:well-formedness-allocation}{wf:impl-write-once} that
    the location is in the nursery, i.e., $\locreg{\loc}{\reg} \in \NENV$.
    %
    This precondition is satisfied by inversion on \tdatacon{}.
    %
    Our application of rule \refwellformed{sec:well-formedness-allocation}{wf:impl-write-once}
    therefore yields the desired result, thereby discharging this case.
  \end{bcase}

  \begin{bcase}
    \begin{mathpar}
    \rtllafter{}
    \end{mathpar}

    Because $\EXPR
    = \letloc{\locreg{\loc}{\reg}}{\afterl{\tyatlocreg{\TYP'}{\loc_{1}}{\reg}}}{\EXPR'}$
    is not a value, the proof obligation is to show that
    there is a rule in the dynamic semantics whose left-hand side matches the machine
    configuration $\STOR;\MENV;\EXPR$.
    %
    The only rule that can match is \dletlocafter{}, but the match is
    dependent on two further obligations, which are
    implied by inversion on \dletlocafter{}.
    %
    The first one is to establish that
    $\concreteloc{\reg}{\ind}{} = \MENV(\locreg{\loc_1}{\reg})$.
    %
    To do so, we need to use
    rule \refwellformed{sec:well-formedness}{wf:map-store-consistency}
    of the well-formedness of the store.
    %
    This rule requires that $\SENV(\locreg{\loc_1}{\reg}) = \TYP'$,
    which is established by inversion on \tllafter{}.
    %
    As such,
    we have $(\loc_1 \mapsto \concreteloc{\reg}{\ind}{}) \in \MENV$, as needed.
    %
    The second and final obligation is to establish that, for some $\indj$,
    $\ewitness{\TYP'}{\concreteloc{\reg}{\ind}{}}{\STOR}{\concreteloc{\reg}{\indj}{}}$.
    %
    Again, we use well-formedness
    rule \refwellformed{sec:well-formedness}{wf:map-store-consistency} to
    discharge the obligation, and thus this case.
    %
  \end{bcase}

  \begin{bcase}
    \tlltag\\ Similar to the previous case.
  \end{bcase}

  \begin{bcase}
    \label{lemma:progress:trivials}
    \tllstart, \tlregion, \tapp \\ These cases discharge immediately because \dletlocstart{},
    \dletregion{}, and \dapp{} match their corresponding machine configurations unconditionally.
  \end{bcase}

  \begin{bcase}
    \tvar, \tconcreteloc \\ These cases discharge immediately because $\EXPR$ is a value.
  \end{bcase}

  \begin{bcase}
    \begin{mathpar}
    \rtlet{}
  \end{mathpar}

     Because $\EXPR = \letpack{\var : \tyatlocreg{\TYP_1}{\loc_1}{\reg_1}}{\EXPR_1}{\EXPR_2}$ is not a value, the proof obligation is to
    show that there is a rule in the dynamics whose left-hand side matches the machine
    configuration $\STOR;\MENV;\EXPR$.
    %
    If $\EXPR_1$ is a value, then the rule discharges immediately because \dletval{} matches
    $\EXPR$ unconditionally.
    %
    Otherwise, if $\EXPR_1$ is not a value, then the only other rule that can match
    is \dletexp.
    %
    To match \dletexp{}, the only requirement is to match the left-hand side of the rule
    $\STOR;\MENV;\EXPR_1 \stepsto \STOR';\MENV';\EXPR'_1$ in the premise, for some $\STOR'$, $\MENV'$, and $\EXPR'_1$.
    %
    To obtain this result, we need to use the induction hypothesis, which is in this instance
    %
    \begin{displaymath}
    \begin{aligned}
      \text{if} \;\; & \emptyset;\SENV;\CENV;\AENV;\NENV \vdash \AENV';\NENV';\EXPR_1 : \tyatlocreg{\TYP}{\loc_1}{\reg_1} \\
      \text{and} \;\; & \storewf{\SENV}{\CENV}{\AENV}{\NENV}{\MENV}{\STOR} \\
      \text{then} \;\; & \EXPR_1 \; \mathit{value} \\
      \text{else} \;\; & \STOR;\MENV;\EXPR_1 \stepsto \STOR';\MENV';\EXPR'_1.
    \end{aligned}
    \end{displaymath}
    %
    By inversion on \tlet{}, we have $\emptyset;\SENV;\CENV;\AENV;\NENV \vdash \AENV';\NENV';\EXPR_1: \tyatlocreg{\TYP_1}{\loc_1}{\reg_1}$, and, from the premise of this lemma, we have
    $\storewf{\SENV}{\CENV}{\AENV}{\NENV}{\MENV}{\STOR}$.
    %
    Thus, by the consequent of the  induction hypothesis, we have that either $\EXPR_1$ is a value (which we have
    already ruled out) or that
    $\STOR;\MENV;\EXPR_1 \stepsto \STOR';\MENV';\EXPR'_1$, thereby discharging this case.

  \end{bcase}

  \begin{bcase}
    \begin{mathpar}
    \rtcase{} \\
    \text{and} \\
    \rtpat{}
  \end{mathpar}

    Because the given expression $\EXPR = \case{\VAL}{\overharpoon{\pat}}$ is not a value, the proof obligation
    is to show that there is a rule in the dynamic semantics whose left-hand side matches the machine configuration $\STOR;\MENV;\EXPR$.
    %
    The only rule that can match is \dcase{}, and
    there are three requirements to match \dcase{}.
    %
    The first of which is that the value
    $\VAL$ is a concrete location of the form $\concreteloc{\reg'}{\ind}{\locreg{\loc'}{\reg'}}$.
    %
    Any value $\VAL$ is, by inspection of the grammar of \ourcalc{}, either a variable or a concrete location.
    %
    But because $\VAL$ is well typed with respect to the
    empty typing environment $\TENV = \emptyset$, the value $\VAL$
    cannot be a variable in this instance, owing to inversion on \tvar{} and \tconcreteloc{},
    thereby ensuring $\VAL$ is a concrete
    location, and thus discharging this requirement.
    %
    The second requirement for \dcase{} is that the tag is in the
    expected location in the store, i.e., $\STOR(\reg')(\ind) = \DC$.
    %
    To satisfy this requirement, we start by using the jugement $\storewf{\SENV}{\CENV}{\AENV}{\NENV}{\MENV}{\STOR}$,
    from the premise of this lemma,
    and in particular, unpacking from this judgement the property \refendwitness{sec:end-witness}{ewitness:impl1}.
    %
    To use this property, we need that $(\locreg{\loc'}{\reg'} \mapsto \TYP') \in \SENV$, which
    is given by inversion on the given typing rule \tcase{}.
    %
    From the unpacking, we obtain that
    \begin{align}
    (\locreg{\loc'}{\reg'} \mapsto \concreteloc{\reg'}{\ind}{} \in \MENV) \wedge \\
      \ewitness{\TYP'}{\concreteloc{\reg'}{\ind}{}}{\STOR}{\concreteloc{\reg'}{\ind'}{}} \label{prf-loc1}.
      \end{align}
    %
    From the end-witness judgement, in particular,
    \refendwitness{sec:end-witness}{ewitness:impl1},
    we establish that
    $\STOR(\reg')(\ind) = \DC$, thereby discharging the second requirement.
    %
    The third and final requirement for \dcase{} is that the arguments succeeding
    the tag are in the expected locations, i.e.,
        \begin{align*}
    & \ewitness{\overharpoon{\TYP'_{1}}}{\concreteloc{\reg'}{\ind + 1}{}}{\STOR}{\concreteloc{\reg'}{\overharpoon{w_1}}{}} \wedge \\
    & \ewitness{\overharpoon{\TYP'_{j+1}}}{\concreteloc{\reg'}{\overharpoon{w_j}}{}}{\STOR}{\concreteloc{\reg'}{\overharpoon{w_{j+1}}}{}}
    \end{align*}
    %
    The above is established by expanding the judgement obtained in~\ref{prf-loc1}, namely
    $\ewitness{\TYP'}{\concreteloc{\reg'}{\ind}{}}{\STOR}{\concreteloc{\reg'}{\ind'}{}}$,
    using in particular, the end-witness rule
    \refendwitness{sec:end-witness}{ewitness:impl2} to obtain the
    needed judgements.
    %
    This final requirement discharges the case.
  \end{bcase}

$\blacksquare$

\end{nproof}

\begin{lemma}[Preservation]
  \label{lemma:preservation}
  \begin{displaymath}
    \begin{aligned}
      \text{If} \;\; & \emptytenv;\SENV;\CENV;\AENV;\NENV \vdash \AENV';\NENV';\EXPR : \hTYP \\
      \text{and} \;\; & \storewf{\SENV}{\CENV}{\AENV}{\NENV}{\MENV}{\STOR}\\
      \text{and} \;\; & \STOR;\MENV;\EXPR \stepsto \STOR';\MENV';\EXPR' \\
      \text{then for some} \;\; & \SENV' \supseteq \SENV, \CENV' \supseteq \CENV ,\\
      & \emptytenv;\SENV';\CENV';\AENV';\NENV' \vdash \AENV'';\NENV'';\EXPR' : \hTYP \\
      \text{and} \;\; & \storewf{\SENV'}{\CENV'}{\AENV'}{\NENV'}{\MENV'}{\STOR'}\\
    \end{aligned}
  \end{displaymath}
\end{lemma}

\begin{nproof}
  The proof is by rule induction on the given derivation of the dynamic semantics.

  \begin{bcase}
    \begin{mathpar}
    \rddatacon{}
    \end{mathpar}
    \begin{itemize}
    \item
    The first of two proof obligations is to show that
    the result $\EXPR' = \concreteloc{\reg}{\ind}{\locreg{\loc}{\reg}}$ of
    the given step of evaluation is well typed, that is,
    \begin{displaymath}
    \emptytenv;\SENV';\CENV';\AENV';\NENV' \vdash \AENV''; \NENV''; \concreteloc{\reg}{\ind}{\locreg{\loc}{\reg}} : \hTYP,
    \end{displaymath}
    where $\hTYP = \tyatlocreg{\TYP}{\loc}{\reg}$.
    %
    As implied by inversion on \tconcreteloc{}, the only obligation is to establish
    that $\SENV'(\locreg{\loc}{\reg})=\TYP$.
    %
    This obligation discharges by appropriately instantiating typing
    environments: $\SENV' = \SENV \cup \set{\locreg{\loc}{\reg} \mapsto \TYP}$, so that $\SENV' \supseteq \SENV$ and
    $\SENV'(\locreg{\loc}{\reg})=\TYP$, and $\CENV' = \CENV$, so that $\CENV' \supseteq \CENV$.
    %
    \item Given the instantiations of $\SENV'$ and $\CENV'$ used by the previous step, the second obligation
    for this proof case is to show that
    \begin{displaymath}
    \storewf{\SENV'}{\CENV}{\AENV'}{\NENV'}{\MENV}{\STOR'}.
    \end{displaymath}
    The individual requirements, labeled
    \refwellformed{sec:well-formedness}{wf:map-store-consistency} -
        \refwellformed{sec:well-formedness}{wf:ca},
        are handled by the following case analysis.
    \begin{itemize}
      \item
      Case (\refwellformed{sec:well-formedness}{wf:map-store-consistency}):
      for each $(\locreg{\loc'}{\reg} \mapsto \TYP) \in \SENV'$, there exists some $\ind_1, \ind_2$ such that
      \begin{align}
      (\locreg{\loc'}{\reg'} \mapsto \concreteloc{\reg'}{\ind_1}{}) \in \MENV \wedge \\
        \ewitness{\TYP}{\concreteloc{\reg'}{\ind_1}{}}{\STOR'}{\concreteloc{\reg'}{\ind_2}{}} \label{prf:dc-cj2}
      \end{align}
      %
      The first conjunct above discharges
      by inversion on \ddatacon{}, but
      to establish the second one, we need to distinguish
      between the case in which the given location $\locreg{\loc'}{\reg'}$ is the one
      affected by the constructor application, or not.
      \begin{itemize}
      \item Case $\locreg{\loc'}{\reg'} = \locreg{\loc}{\reg}$: \\
        For this case, the obligation is to show that the constructor being
        allocated by the constructor application, namely $\locreg{\loc}{\reg}$,
        has the end witness given above.
        %
        As such, for this case, it is the case that $\reg' = \reg$ and $\ind_1 = \ind$, which is
        a consequence of inversion on \ddatacon{}.
        %
        To establish the end witness, the first obligation therein, namely \refendwitness{sec:end-witness}{ewitness:impl1},
        is to establish $\STOR'(r)(i) = \DC$.
        %
        This obligation discharges by inspection of $\STOR'$, which is obtained by inversion on \ddatacon{}.
        %
        The second part is to establish the requirement
        \refendwitness{sec:end-witness}{ewitness:impl2} of the end-witness
        judgement, which pertains to the arguments passed to the constructor.
        The specific obligation is, if $n = | \overharpoon{\TYP'} | \geq 1$, then
           \begin{align}
           \ewitness{\overharpoon{\TYP'_1}}{\concreteloc{\reg}{\ind + 1}{}}{\STOR'}{\concreteloc{\reg}{\overharpoon{w_1}}{}} \wedge \label{prf:ew-req1} \\
           \ewitness{\overharpoon{\TYP'_{j+1}}}{\concreteloc{\reg}{\overharpoon{w_j}}{}}{\STOR'}{\concreteloc{\reg}{\overharpoon{w_{j+1}}}{}} \label{prf:ew-req2}
           \end{align}
           for some $\overharpoon{w}$, where
           $\indj \in J = J' - \set{n}$, $\indj' \in J' = \set{1,\ldots,n}$, and
           $\overharpoon{\TYP'} = \kargtys(\DC)$.
           %
           To establish the above, we need
           to reason backward from what
           the corresponding typing rules establish regarding the
           arguments passed to the constructor application.
           %
           First, we establish that, for each location corresponding to a constructor argument
           $\overharpoon{\locreg{\loc}{\reg}_{\indj'}}$,
           there is a corresponding mapping in the store-typing environment, i.e.,
           $(\overharpoon{\locreg{\loc}{\reg}_{\indj'}} \mapsto \overharpoon{\TYP_{\indj'}'}) \in \SENV$.
           %
           To establish these mappings, we first
           obtain by inversion on \tdatacon{} that the
           constructor arguments are well typed:
           \begin{align*}
           \emptytenv;\SENV;\CENV;\AENV;\NENV \vdash \AENV;\NENV;\overharpoon{\VAL_{\indj'}} : \overharpoon{\tyatlocreg{\TYP_{\indj'}'}{\loc_{\indj'}}{\reg}}
           \end{align*}
           %
           Each value $\overharpoon{\VAL_{\indj'}}$ is either a variable or a concrete location, and
           as such, by inversion on the typing rules \tvar{} and \tconcreteloc{}, respectively,
           we establish the required mappings in $\SENV$.
           %
           Thus, we can now combine the well-formedness
           of the store in the premise of this lemma,
           in particular requirement \refwellformed{sec:well-formedness}{wf:map-store-consistency},
           with the mappings of constructor arguments in $\SENV$ to establish
           the end witnesses in $\overharpoon{\ind}$ corresponding to the constructor arguments:
           %
           \begin{align}
           (\overharpoon{\locreg{\loc}{\reg}_{\indj'}} \mapsto \concreteloc{\reg}{\overharpoon{\ind_{\indj'}}}{}) \in \MENV \wedge \label{prf:dc-ew1} \\
           \ewitness{\overharpoon{\TYP_{\indj'}'}}{\concreteloc{\reg}{\overharpoon{\ind_{\indj'}}}{}}{\STOR}{\concreteloc{\reg}{\overharpoon{\ind_{\indj' + 1}}}{}} \label{prf:dc-ew2}
           \end{align}
           %
           We first address the obligation pertaining to the first constructor argument,
           and then the remaining ones.
           %
           From inversion on \tdatacon{}, we establish a mapping for the location of the
           first constructor argument.
           %
           \begin{align*}
           \CENV(\overharpoon{\locreg{\loc}{\reg}_1}) = \locreg{\loc}{\reg} + 1
           \end{align*}
           %
           Now, using this result, we can establish from well
           formedness rule \refwellformed{sec:well-formedness-constructors}{wfconstr:constraint-tag}
           that the following mappings exist in the location environment.
           %
           \begin{align*}
           (\locreg{\loc}{\reg} \mapsto \concreteloc{\reg}{\ind}{})  \in \MENV \wedge \\
                      (\overharpoon{\locreg{\loc}{\reg}_1} \mapsto \concreteloc{\reg}{\ind + 1}{})  \in \MENV
           \end{align*}
           %
           Next, combining the fact from line \ref{prf:dc-ew1} above regarding $\overharpoon{\locreg{\loc}{\reg}_1}$, the
           end witness corresponding to $\overharpoon{\ind_1}$ from the end witnesses of
           constructor arguments line~\ref{prf:dc-ew2} from above, we
           establish the requirement on line \ref{prf:ew-req1} above, such that $\overharpoon{w_1} = \overharpoon{i_1}$, i.e.,
           %
           \begin{align}
           \ewitness{\overharpoon{\TYP'_1}}{\concreteloc{\reg}{\ind + 1}{}}{\STOR}{\concreteloc{\reg}{\overharpoon{w_1}}{}} \label{prf:dc-ew-f1}.
           \end{align}
%           Notice that the end witness of the first field is by definition given above $\overharpoon{w_1}$.

           For the remaining constructor arguments, the structure of the proof is similar.
           %
           We establish mappings in $\CENV$ for the locations of these constructor arguments
           by inversion on \tdatacon{}.
           %
           \begin{align*}
           \CENV(\overharpoon{\locreg{\loc}{\reg}_{\indj+1}}) = \afterl{\overharpoon{\TYP'_{\indj}} \ensuremath{@} \overharpoon{\locreg{\loc}{\reg}_{\indj}}}
           \end{align*}
           %
           The following end witnesses $\overharpoon{\ind}$ are established by combining the property
           on the constraint environment
           with the property
           \refwellformed{sec:well-formedness-constructors}{wfconstr:constraint-after},
           which is obtained from the well formedness of the store in the premise of this lemma.
           \begin{align*}
           ((\overharpoon{\locreg{\loc}{\reg}_{\indj}} \mapsto \concreteloc{\reg}{\overharpoon{\ind_{\indj}}}{}) \in \MENV \wedge \\
                       \ewitness{\overharpoon{\TYP'_{\indj}}}{\concreteloc{\reg}{\overharpoon{\ind_{\indj}}}{}}{\STOR}{\concreteloc{\reg}{\overharpoon{\ind_{\indj+1}}}{}} \wedge \\
                       (\overharpoon{\locreg{\loc}{\reg}_{\indj+1}} \mapsto \concreteloc{\reg}{\overharpoon{\ind_{\indj+1}}}{}) \in \MENV)
           \end{align*}
           %
           To isolate the indices of any constructor arguments succeeding the
           first argument, we let $\indj'' \in J - \set{1}$, and thus deduce
           from the above that the end witnesses
           \begin{align*}
           \ewitness{\overharpoon{\TYP_{\indj''+1}'}}{\concreteloc{\reg}{\overharpoon{\ind_{\indj''+1}}}{}}{\STOR}{\concreteloc{\reg}{\overharpoon{\ind_{\indj'' + 2}}}{}}.
           \end{align*}
           %
           exist.
           %
           We obtain the needed result for the remaining end witnesses
           by instantiating for $\overharpoon{w}$, yielding
           \begin{align}
           \ewitness{\overharpoon{\TYP_{\indj''+1}'}}{\concreteloc{\reg}{\overharpoon{w_{\indj''}}}{}}{\STOR}{\concreteloc{\reg}{\overharpoon{w_{\indj'' + 1}}}{}} \label{prf:dc-ew-f2}.
           \end{align}
           %
           The original end witness required by~\ref{prf:dc-cj2} is now established by letting
           $\ind_1 = \ind$ and $\ind_2 = \overharpoon{w_{n+1}}$.

           Finally, to discharge this case, the end witnesses of the constructor
           arguments established in lines~\ref{prf:dc-ew-f1} and~\ref{prf:dc-ew-f2}
           need to hold for the new store
           $\STOR' = \STOR \cup \set{\reg \mapsto (\ind \mapsto \DC)}$.
           %
           To this end, in $\STOR'$, the newly
           written tag at address $\ind$ cannot overlap with the cells
           occupied by any of the constructor arguments.
           %
           Therefore, the desired end witnesses exist in $\STOR'$, thereby
           discharging this case.
      \item Case $\locreg{\loc'}{\reg'} \neq \loc$: \\
      This case requires we establish that, for such a given location $\locreg{\loc'}{\reg'}$, its
      corresponding end witness in the original store $\STOR$ also exists
      in the new store, $\STOR'$, that is, supposing $(\locreg{\loc'}{\reg'} \mapsto \concreteloc{\reg'}{\ind_1}{}) \in \MENV$, then $\ewitness{\TYP}{\concreteloc{\reg'}{\ind_1}{}}{\STOR}{\concreteloc{\reg'}{\ind_2}{}}$ implies $\ewitness{\TYP}{\concreteloc{\reg'}{\ind_1}{}}{\STOR'}{\concreteloc{\reg'}{\ind_2}{}}$.
      %
      But the only way that any such end witness can be invalidated
      is if the write of the constructor tag at index $\ind$ in
      $\STOR' = \STOR \cup \set{\reg \mapsto (\ind \mapsto \DC)}$ affects
      any address in the end witness corresponding
      to location $\locreg{\loc'}{\reg'}$, that is, any address
      in the right-open range $[\ind_1, \ind_2)$.
      %
      The proof obligation therefore amounts to ruling out
      aliasing, that is, $\ind$ falling in the range
      $[\ind_1, \ind_2)$.
      %
      To this end, we start by working backwards from the typing of
      the location $\locreg{\loc}{\reg}$, which corresponds to address $\ind$, the (only) address
      written by the constructor application.
      %
      By inversion on \tdatacon{}, we establish that $\locreg{\loc}{\reg} \in \NENV$.
      %
      As such, given the well formedness of the store $\STOR$ in the premise
      of this lemma, we obtain
      from \refwellformed{sec:well-formedness-allocation}{wf:impl-write-once}
      that
      $(\reg \mapsto (\ind \mapsto \DC)) \not \in \STOR$.
      %
      However, by the end-witness rule, for each $\indj \in [\ind_1, \ind_2)$,
      there exists a mapping from the address in the original store to its constructor tag
      $\DC_j$, which is
      $(\reg \mapsto (\indj \mapsto \DC_j)) \in \STOR$.
      %
      Therefore, the end witness judgement remains valid in store $\STOR'$,
      thus discharging this case.
      \end{itemize}
      \item Case (\refwellformed{sec:well-formedness}{wf:cfc}):
      \begin{align*}
      \storewfcfa{\CENV}{\MENV}{\STOR'}
      \end{align*}
      The first two proof obligations of this judgement, namely \refwellformed{sec:well-formedness-constructors}{wfconstr:constraint-start} and \refwellformed{sec:well-formedness-constructors}{wfconstr:constraint-tag}, discharge immediately, because the environments used by these
      rules are unaffected in a data-constructor application.
      %
      The only remaining obligation is \refwellformed{sec:well-formedness-constructors}{wfconstr:constraint-after}, because that requirement is affected by the write of the constructor tag, which is reflected in the new store $\STOR'$.
      %
      The obligation is to establish the preservation of the
      end witnesses of the locations in the domain of $\CENV$.
      %
      A similar proof obligation was already addressed by the proof of Property~\ref{prf:dc-cj2},
      in particular the subcase for $\locreg{\loc'}{\reg'} \neq \locreg{\loc}{\reg}$.
      %
      The only difference in that case is the locations range over the
      domain of the store-typing environment $\SENV$, whereas in this case
      the obligation concerns locations
      in the domain of the constraint environment $\CENV$.
      %
      However, the same proof steps apply in both cases, thus discharging
      this case.
      \item Case (\refwellformed{sec:well-formedness}{wf:ca}):
      \begin{align*}
      \storewfca{\AENV'}{\NENV'}{\MENV}{\STOR'}
      \end{align*}
      Obligations \refwellformed{sec:well-formedness-allocation}{wf:impl-linear-alloc} and
      \refwellformed{sec:well-formedness-allocation}{wf:impl-write-once} discharge immediately
      because $\locreg{\loc}{\reg} \not \in \NENV'$.
      %
      It remains to discharge the obligation corresponding to
      \refwellformed{sec:well-formedness-allocation}{wf:impl-linear-alloc2}.
      %
      Because it is the case that
      \begin{align*}
      (\reg \mapsto \locreg{\loc}{\reg}) \in \AENV' \wedge
      (\locreg{\loc}{\reg} \mapsto \concreteloc{\reg}{\ind_1}{}) \in \MENV \wedge \locreg{\loc}{\reg} \not \in \NENV' \wedge
    \, \ewitness{\TYP}{\concreteloc{\reg}{\ind_1}{}}{\STOR'}{\concreteloc{\reg}{\ind_2}{}},
    \end{align*}
    the obligation amounts to showing that the end witness of the
    constructor application is the new highest address in the store
    $\STOR'$, i.e., $\ind_2 > \allocptr{\reg}{\STOR'}$.
    %
    There are two cases, based on the number of constructor arguments $n$:
      \begin{itemize}
      \item Case $n = 0$:\\
      We need to appeal to the well formedness of the store, as given by the premise of this lemma,
      and in particular rule \refwellformed{sec:well-formedness-allocation}{wf:impl-linear-alloc}.
      %
      To use  this rule, we need to first establish
      $(\reg \mapsto \locreg{\loc}{\reg}) \in A$ and $\locreg{\loc}{\reg} \in \NENV$,
      which follows immediately by inversion on \tdatacon{}.
      %
      It therefore follows that
      \begin{align*}
%      \ind_1 = \max \set{0} \cup \set{\indj \; | \; \concreteloc{\reg}{\indj}{} \in \MENV} \wedge \\
      \ind_1 > \allocptr{\reg}{\STOR}.
      \end{align*}
      From this property, and by inspection on $\STOR'$, we discharge
      this case by establishing that the end witness of the constructor
      application is the highest address allocated in the new store $\STOR'$, i.e.,
      \begin{align*}
      \ind_1+1 = \ind_2 > \allocptr{\reg}{\STOR'}.
      \end{align*}
      \item Case $n \geq 1$:\\
      To discharge this case, we need to show that the end witness of the
      last constructor argument, i.e., the one at position $n$,
      is the highest address in the new store $\STOR'$.
      %
      This obligation follows from the well formedness of the
      store $\STOR$ given by the premise of this lemma, and
      in particular the application of rule
      \refwellformed{sec:well-formedness-allocation}{wf:impl-linear-alloc2}
      to the end witness of the last constructor argument, i.e.,
      \begin{align*}
      (\reg \mapsto \overharpoon{\locreg{\loc}{\reg}_n}) \in A \wedge
    \, (\overharpoon{\locreg{\loc}{\reg}_n} \mapsto \concreteloc{\reg}{\overharpoon{w_n}}{}) \in \MENV \wedge
    \, \ewitness{\TYP}{\concreteloc{\reg}{\overharpoon{w_n}}{}}{\STOR}{\concreteloc{\reg}{\overharpoon{w_{n+1}}}{}}
      \end{align*}
      The first two conjuncts follow from inversion on \tdatacon{}
      and \tconcreteloc{}, respectively, and the final one from Property~\ref{prf:dc-ew-f2}.
      %
      Thus, we have that $\overharpoon{w_{n+1}} > \allocptr{\reg}{\STOR}$.
      %
      It follows that $\overharpoon{w_{n+1}} > \allocptr{\reg}{\STOR'}$, because the
      newly written address in $\STOR'$, namely $i_1$, is such that $i_1 < \overharpoon{w_{n+1}}$.
      %
      By defintion of the end witness, we discharge this case by establishing that
      $\overharpoon{w_{n+1}} = i_2 > \allocptr{\reg}{\STOR'}$.
      \end{itemize}
      The final obligation of this case concerns the requirement
      \refwellformed{sec:well-formedness-allocation}{wf:impl-empty-region}.
      %
      Part of this obligation is given by the premise of this lemma, for the original
      store $\STOR$, and yields in particular that,
      for each $(\reg' \mapsto \emptyset) \in \AENV$, it
      is the case that $\reg' \not \in dom(\STOR)$.
      %
      The remaining obligation is to show the property holds for the new store
      $\STOR'$, which discharges immediately because, although $\reg \in \STOR'$,
      by inversion on \tdatacon{}, it must be that
      $(\reg \mapsto \emptyset) \not \in \AENV$.
      \item
      Case (\refwellformed{sec:well-formedness}{wf:impl1}):
      \begin{displaymath}
      dom(\SENV') \cap \NENV' = \emptyset
      \end{displaymath}
      From the premise of the lemma, we have that the store is well formed with respect to typing
      environments $\SENV$ and $\NENV$, and as such, we have that
      $dom(\SENV) \cap \NENV = \emptyset$.
      %
      Therefore, we discharge this case by inspection of typing rule \tdatacon{}, which
      shows that $N' = N - \set{\loc}$.
    \end{itemize}
    \end{itemize}
  \end{bcase}

  \begin{bcase}
    \begin{mathpar}
    \rdcase{}
    \end{mathpar}
    \begin{itemize}
    \item
    The first of two proof obligations is to show that
    the result $\EXPR' = \subst{\EXPR}{\overharpoon{\var}}{\concreteloc{\reg}{\overharpoon{w}}{\overharpoon{\locreg{\loc}{\reg}}}}$ of
    the given step of evaluation is well typed, that is,
    \begin{displaymath}
    \emptytenv;\SENV';\CENV;\AENV;\NENV \vdash \AENV; \NENV; \EXPR' : \hTYP,
    \end{displaymath}
    %
    where $\hTYP = \tyatlocreg{\TYP}{\loc}{\reg}$.
    %
    To establish the above, we start by obtaining the type
    for the body of the pattern, then the types of the
    concrete locations being substituted into the body,
    and finally use these two results
    with the substitution lemma to discharge the case.
    %
    First, by inversion on the typing rules \tcase{} and \tpat{}, we
    establish that the body of the pattern, namely $\EXPR$, is well typed, i.e.,
    \begin{align*}
    \TENV';\SENV';\CENV;\AENV;\NENV \vdash \AENV;\NENV;\EXPR : \tyatlocreg{\TYP}{\loc}{\reg},
    \end{align*}
    where
    \begin{align*}
    \TENV' &= \set{\overharpoon{\var_1} \mapsto \overharpoon{\TYP_1} \ensuremath{@} \overharpoon{\locreg{\loc_1}{\reg}}, \ldots, \overharpoon{\var_1} \mapsto \overharpoon{\TYP_n} \ensuremath{@} \overharpoon{\locreg{\loc_n}{\reg}}} \\
    \SENV' &= \SENV \cup \set{\overharpoon{\locreg{\loc_1}{\reg}}\mapsto\overharpoon{\TYP}_1,\ldots,\overharpoon{\locreg{\loc_n}{\reg}}\mapsto\overharpoon{\TYP}_n}.
    \end{align*}
    %
    Second, we establish that the concrete locations being substituted for the
    pattern variables $\overharpoon{x}$ are well typed.
    %
    The specific obligation is, for each $i \in \set{1, \ldots, n}$, to establish that
    \begin{align*}
    \emptyset;\SENV';\CENV;\AENV;\NENV \vdash \AENV;\NENV; \concreteloc{\reg}{\overharpoon{w_i}}{\overharpoon{\locreg{\loc_i}{\reg}}} : \overharpoon{\TYP}_i \ensuremath{@} \overharpoon{\locreg{\loc_i}{\reg}}.
    \end{align*}
    %
    The above holds because, by inversion on \tconcreteloc{}, the obligation is
    to show that, for each such $i$, $(\locreg{\overharpoon{\loc_i}}{\reg} \mapsto \overharpoon{\TYP_i}) \in \SENV'$,
    which is immediate by inspection on $\SENV'$ above.
    %
    Third, and finally, to establish the typing judgement for $\EXPR'$, we use the Substitution
    Lemma \ref{lemma:substitution}, which yields
    %
    \begin{align*}
    \emptyset;\SENV';\CENV;\AENV;\NENV \vdash \AENV;\NENV; \subst{\EXPR}{\overharpoon{\var_1}}{\concreteloc{\reg}{\overharpoon{w_1}}{\overharpoon{\locreg{\loc_1}{\reg}}}}
    \ldots \subst{}{\overharpoon{\var_n}}{\concreteloc{\reg}{\overharpoon{w_1}}{\overharpoon{\locreg{\loc_n}{\reg}}}}: \hTYP,
    \end{align*}
    as needed, thereby discharging this obligation.
    \item The second obligation
    for this proof case is, given the affected environments, namely
    $\SENV'$ and $\MENV'$, to establish the well formedness
    of the resulting store, i.e.,
    %
    \begin{displaymath}
    \storewf{\SENV'}{\CENV}{\AENV}{\NENV}{\MENV'}{\STOR}.
    \end{displaymath}
    We omit most of the details of this proof obligation because they
    discharge straightforwardly.
    %
    The only part that requires attention is rule
    \refwellformed{sec:well-formedness}{wf:map-store-consistency},
    which is affected by the fresh locations in the location
    environment $\MENV'$.
    %
    This requirement discharges by inspection of \dcase{}, thereby
    discharging this obligation.
    \end{itemize}
  \end{bcase}

  \begin{bcase}
    \begin{mathpar}
    \rdletloctag{}
    \end{mathpar}
    \begin{itemize}
    \item
    The first of two proof obligations is to show that
    the result $\EXPR$ of
    the given step of evaluation is well typed, that is,
    \begin{displaymath}
    \emptytenv;\SENV;\CENV';\AENV';\NENV' \vdash \AENV''; \NENV''; \EXPR : \hTYP,
    \end{displaymath}
    where $\hTYP = \tyatlocreg{\TYP}{\loc}{\reg}$, $\AENV' = \AENV \cup \set{\reg \mapsto \locreg{\loc}{\reg}}$,
    and $\NENV' = \NENV \cup \set{\locreg{\loc}{\reg}}$.
    %
    This proof obligation follows straightforwardly by inversion
    on \tlltag{}.
    \item The second obligation for this proof case is to show that
    \begin{displaymath}
    \storewf{\SENV}{\CENV'}{\AENV'}{\NENV'}{\MENV'}{\STOR}.
    \end{displaymath}
    The individual requirements, labeled
    \refwellformed{sec:well-formedness}{wf:map-store-consistency} -
        \refwellformed{sec:well-formedness}{wf:ca},
        are handled by the following case analysis.
    \begin{itemize}
      \item
      Case (\refwellformed{sec:well-formedness}{wf:map-store-consistency}):
      for each $(\locreg{\loc'}{\reg} \mapsto \TYP) \in \SENV$, there exists some $\ind_1, \ind_2$ such that
      \begin{align*}
      (\locreg{\loc'}{\reg} \mapsto \concreteloc{\reg}{\ind_1}{}) \in \MENV' \wedge \\
        \ewitness{\TYP}{\concreteloc{\reg}{\ind_1}{}}{\STOR}{\concreteloc{\reg}{\ind_2}{}}
      \end{align*}
      %
      By the well formedness of the store given in the premise of this lemma,
      the above already holds for the location environment $\MENV$.
      %
      The obligation discharges by inspecting the only new location
      in $\MENV'$, namely $\locreg{\loc}{\reg}$, which
      is fresh and therefore cannot be in the domain of $\SENV$.
      \item Case (\refwellformed{sec:well-formedness}{wf:cfc}):
      \begin{align*}
      \storewfcfa{\CENV'}{\MENV'}{\STOR}
      \end{align*}
      Of the requirements for this judgement, the only one that is
      not satisfied immediately by the well formedness of the store
      given in the premise of the lemma is requirement
      \refwellformed{sec:well-formedness-constructors}{wfconstr:constraint-tag}
      %
      The specific requirement is to establish that
      \begin{align*}
      (\locreg{\loc'}{\reg} \mapsto \concreteloc{\reg}{\ind}{})  \in \MENV' \wedge \\
      (\locreg{\loc}{\reg} \mapsto \concreteloc{\reg}{\ind + 1}{})  \in \MENV',
      \end{align*}
      which follows immediately by inversion on \dletloctag{}.
      \item Case (\refwellformed{sec:well-formedness}{wf:ca}):
      \begin{align*}
      \storewfca{\AENV'}{\NENV'}{\MENV'}{\STOR}
      \end{align*}
        \begin{itemize}
        \item Case (\refwellformed{sec:well-formedness-allocation}{wf:impl-linear-alloc}):
        \begin{align*}
                  (\locreg{\loc}{\reg} \mapsto \concreteloc{\reg}{\ind+1}{}) \in \MENV' \wedge
%          \ind+1 = \max \set{0} \cup \set{\indj \; | \; \concreteloc{\reg}{\indj}{} \in \MENV'} \wedge \\
          \ind+1 > \allocptr{\reg}{\STOR}
        \end{align*}
        The first conjunct follows immediately from inversion on \dletloctag{}.
        %
        To establish the second, however, we first need to establish
        that the address corresponding to location $\locreg{\loc'}{\reg}$ is the highest index in
        the store $\STOR$.
        %
        To do so, we need to appeal to the well formedness of the store given by the
        premise of this lemma.
        %
        In particular, we need to use the same requirement we are trying to prove,
        namely \refwellformed{sec:well-formedness-allocation}{wf:impl-linear-alloc}, but in this case,
        instantiating for $\locreg{\loc'}{\reg}$ in the original location environment $\MENV$.
        %
        By inversion on \tlltag{}, we have that $\AENV(\reg) = \locreg{\loc'}{\reg}$ and $\locreg{\loc'}{\reg} \in \NENV$,
        and as a consequence of \refwellformed{sec:well-formedness-allocation}{wf:impl-linear-alloc},
        \begin{align*}
                  (\locreg{\loc'}{\reg} \mapsto \concreteloc{\reg}{\ind}{}) \in \MENV \wedge
%          \ind = \max \set{0} \cup \set{\indj \; | \; \concreteloc{\reg}{\indj}{} \in \MENV} \wedge \\
          \ind > \allocptr{\reg}{\STOR}.
        \end{align*}
        Using the second conjunct above, this case discharges immediately.
        \item Case (\refwellformed{sec:well-formedness-allocation}{wf:impl-linear-alloc2}):
        This obligation discharges immediately because, by inversion on \tlltag{}, $\locreg{\loc}{\reg} \in \NENV'$.
        \item Case (\refwellformed{sec:well-formedness-allocation}{wf:impl-write-once}):
        The proof obligation is to establish that, for any constructor tag $\DC$,
        \begin{align*}
         ((\locreg{\loc}{\reg} \mapsto \concreteloc{\reg}{\ind+1}{}) \in \MENV' \wedge \\
          (\reg \mapsto (\ind+1 \mapsto \DC)) \not \in \STOR)
        \end{align*}
        The first conjunct discharges by inversion on \dletloctag{},
        and the second as a consequence of having already
        established just above that $\ind+1 > \allocptr{\reg}{\STOR}$.
        \item Case (\refwellformed{sec:well-formedness-allocation}{wf:impl-empty-region}):
        The proof obligation is to establish that, for each
        $(\reg \mapsto \emptyset) \in \AENV'$, it is the case that
        $\reg \not \in dom(\STOR)$.
        %
        This case discharges because, from the premise of the
        lemma, this property holds for the original environment
        $\AENV$ and store $\STOR$, and, by inversion on \tlltag{},
        continues to hold for $\AENV'$ and $\STOR'$.
      \end{itemize}
      \item
      Case (\refwellformed{sec:well-formedness}{wf:impl1}):
      \begin{displaymath}
      dom(\SENV) \cap \NENV' = \emptyset
      \end{displaymath}
      Because it is a bound location, $\loc \not \in dom(\SENV)$, and by inversion on \tlltag{},
      $\loc \in \NENV'$, which discharges the obligation.
      \end{itemize}
    \end{itemize}
  \end{bcase}

  \begin{bcase}
    \begin{mathpar}
    \rdletlocafter{}
    \end{mathpar}
    \begin{itemize}
    \item
    The first of two proof obligations is to show that
    the result $\EXPR'$ of
    the given step of evaluation is well typed, that is,
    \begin{displaymath}
    \emptytenv;\SENV;\CENV';\AENV';\NENV' \vdash \AENV''; \NENV''; \EXPR' : \hTYP,
    \end{displaymath}
    where $\hTYP = \tyatlocreg{\TYP}{\loc'}{\reg'}$.
    %
    This proof obligation follows straightforwardly by inversion
    on \tllafter{}.
    \item The second obligation for this proof case is to show that
    \begin{displaymath}
    \storewf{\SENV}{\CENV'}{\AENV'}{\NENV'}{\MENV'}{\STOR}.
    \end{displaymath}
    The individual requirements, labeled
    \refwellformed{sec:well-formedness}{wf:map-store-consistency} -
        \refwellformed{sec:well-formedness}{wf:ca},
        are handled by the following case analysis.
    \begin{itemize}
      \item
      Case (\refwellformed{sec:well-formedness}{wf:map-store-consistency}):
      for each $(\locreg{\loc'}{\reg} \mapsto \TYP) \in \SENV$, there exists some $\ind_1, \ind_2$ such that
      \begin{align*}
      (\locreg{\loc'}{\reg} \mapsto \concreteloc{\reg}{\ind_1}{}) \in \MENV' \wedge \\
        \ewitness{\TYP}{\concreteloc{\reg}{\ind_1}{}}{\STOR}{\concreteloc{\reg}{\ind_2}{}}
      \end{align*}
      %
      By the well formedness of the store given in the premise of this lemma,
      the above already holds for the location environment $\MENV$.
      %
      The obligation discharges by inspecting the only new location
      in $\MENV'$, namely $\locreg{\loc}{\reg}$, which
      is fresh and therefore cannot be in the domain of $\SENV$.
      \item Case (\refwellformed{sec:well-formedness}{wf:cfc}):
      \begin{align*}
      \storewfcfa{\CENV'}{\MENV'}{\STOR}
      \end{align*}
      Of the requirements for this judgement, the only one that is
      not satisfied immediately by the well formedness of the store
      given in the premise of the lemma is requirement
      \refwellformed{sec:well-formedness-constructors}{wfconstr:constraint-after}
      %
      The specific requirement is to establish that
      \begin{align*}
       (\locreg{\loc_1}{\reg} \mapsto \concreteloc{\reg}{\ind}{}) \in \MENV' \wedge \\
       \ewitness{\TYP}{\concreteloc{\reg}{\ind}{}}{\STOR}{\concreteloc{\reg}{\indj}{}} \wedge \\
       (\loc \mapsto \concreteloc{\reg}{\indj}{}) \in \MENV'
      \end{align*}
      which follows immediately by inversion on \dletlocafter{}.
      \item Case (\refwellformed{sec:well-formedness}{wf:ca}):
      \begin{align*}
      \storewfca{\AENV'}{\NENV'}{\MENV'}{\STOR}
      \end{align*}
        \begin{itemize}
        \item Case (\refwellformed{sec:well-formedness-allocation}{wf:impl-linear-alloc}):
        \begin{align*}
                  (\loc \mapsto \concreteloc{\reg}{\indj}{}) \in \MENV' \wedge
%          \indj = \max \set{0} \cup \set{\indj' \; | \; \concreteloc{\reg}{\indj'}{} \in \MENV'} \wedge \\
          \indj > \allocptr{\reg}{\STOR}
        \end{align*}
        The first conjunct follows immediately from inversion on \dletlocafter{}.
        %
        To establish the second, however, we first need to establish
        that the end witness $\indj$ of location $\locreg{\loc_1}{\reg}$ is the maximum index in the
        store $\STOR$.
        %
        To do so, we need to appeal to the well formedness of the store given by the
        premise of this lemma.
        %
        In particular, we need to use the requirement \refwellformed{sec:well-formedness-allocation}{wf:impl-linear-alloc2},
        instantiating for $\locreg{\loc_1}{\reg}$ in the original location environment $\MENV$.
        %
        By inversion on \tllafter{}, we have that $\AENV(\reg) = \locreg{\loc_1}{\reg}$, $\locreg{\loc_1}{\reg} \not \in \NENV$,
        and $\ewitness{\TYP}{\concreteloc{\reg}{\ind}{}}{\STOR}{\concreteloc{\reg}{\indj}{}}$.
        %
        Thus, as a consequence of \refwellformed{sec:well-formedness-allocation}{wf:impl-linear-alloc2},
        \begin{align*}
%                  (\loc' \mapsto \concreteloc{\reg}{\ind}{}) \in \MENV \wedge
%          \ind = \max \set{0} \cup \set{\indj \; | \; \concreteloc{\reg}{\indj}{} \in \MENV} \wedge \\
          \indj > \allocptr{\reg}{\STOR}.
        \end{align*}
        Using the second and third conjuncts above, this case discharges immediately.
        \item Case (\refwellformed{sec:well-formedness-allocation}{wf:impl-linear-alloc2}):
        This obligation discharges immediately because, by inversion on \tllafter{}, $\loc \in \NENV'$.
        \item Case (\refwellformed{sec:well-formedness-allocation}{wf:impl-write-once}):
        The proof obligation is to establish that, for any constructor tag $\DC$,
        \begin{align*}
         ((\loc \mapsto \concreteloc{\reg}{\indj}{}) \in \MENV' \wedge \\
          (\reg \mapsto (\indj \mapsto \DC)) \not \in \STOR)
        \end{align*}
        The first conjunct discharges by inversion on \dletlocafter{},
        and the second as a consequence of having already
        established just above that $\indj > \allocptr{\reg}{\STOR}$.
        \item Case (\refwellformed{sec:well-formedness-allocation}{wf:impl-empty-region}):
        This case discharges straightforwardly, in a similar fashion
        to the previous case, for \dletloctag{}.
        \end{itemize}
      \item
      Case (\refwellformed{sec:well-formedness}{wf:impl1}):
      \begin{displaymath}
      dom(\SENV) \cap \NENV' = \emptyset
      \end{displaymath}
      Because it is a bound location, $\loc \not \in dom(\SENV)$, and by inversion on \tllafter{}
      $\loc \in \NENV'$, which discharges this obligation.
      \end{itemize}
    \end{itemize}
  \end{bcase}

  \begin{bcase}
    \begin{mathpar}
    \rdletlocstart{}
    \end{mathpar}
    \begin{itemize}
    \item
    The first of two proof obligations is to show that
    the result $\EXPR'$ of
    the given step of evaluation is well typed, that is,
    \begin{displaymath}
    \emptytenv;\SENV;\CENV';\AENV';\NENV' \vdash \AENV''; \NENV''; \EXPR' : \hTYP,
    \end{displaymath}
    where $\hTYP = \tyatlocreg{\TYP}{\loc'}{\reg'}$.
    %
    This obligation follows straightforwardly by inversion
    on \tllstart{}.
    \item The second obligation for this proof case is to show that
    \begin{displaymath}
    \storewf{\SENV}{\CENV'}{\AENV'}{\NENV'}{\MENV'}{\STOR}.
    \end{displaymath}
    The individual requirements, labeled
    \refwellformed{sec:well-formedness}{wf:map-store-consistency} -
        \refwellformed{sec:well-formedness}{wf:ca},
        are handled by the following case analysis.
    \begin{itemize}
      \item
      Case (\refwellformed{sec:well-formedness}{wf:map-store-consistency}):
      for each $(\loc' \mapsto \TYP) \in \SENV$, there exists some $\ind_1, \ind_2$ such that
      \begin{align*}
      (\loc' \mapsto \concreteloc{\reg}{\ind_1}{}) \in \MENV' \wedge \\
        \ewitness{\TYP}{\concreteloc{\reg}{\ind_1}{}}{\STOR}{\concreteloc{\reg}{\ind_2}{}}
      \end{align*}
      %
      By the well formedness of the store given in the premise of this lemma,
      the above already holds for the location environment $\MENV$.
      %
      The obligation discharges by inspecting the only new location
      in $\MENV'$, namely $\locreg{\loc}{\reg}$, which
      is fresh and therefore cannot be in the domain of $\SENV$.
      \item Case (\refwellformed{sec:well-formedness}{wf:cfc}):
      \begin{align*}
      \storewfcfa{\CENV'}{\MENV'}{\STOR}
      \end{align*}
      Of the requirements for this judgement, the only one that is
      not satisfied immediately by the well formedness of the store
      given in the premise of the lemma is requirement
      \refwellformed{sec:well-formedness-constructors}{wfconstr:constraint-start}.
      %
      The specific requirement is to establish that
      \begin{align*}
      (\locreg{\loc}{\reg} \mapsto \concreteloc{\reg}{0}{}) \in \MENV',
      \end{align*}
      which follows immediately by inversion on \dletlocstart{}.
      \item Case (\refwellformed{sec:well-formedness}{wf:ca}):
      \begin{align*}
      \storewfca{\AENV'}{\NENV'}{\MENV'}{\STOR}
      \end{align*}
        \begin{itemize}
        \item Case (\refwellformed{sec:well-formedness-allocation}{wf:impl-linear-alloc}):
        \begin{align*}
                  (\loc \mapsto \concreteloc{\reg}{0}{}) \in \MENV' \wedge
          0 > \allocptr{\reg}{\STOR}
        \end{align*}
        The first conjunct follows immediately from inversion on \dletlocstart{}.
        %
        To establish the second conjunct above, it suffices establish
        that $\reg \not \in dom(\STOR)$,
        because, as such, $\allocptr{\reg}{\STOR} = -1$, by the definition of $MaxIdx$.
        %
        This property follows from the well formedness of the
        store, in particular, from rule
        \refwellformed{sec:well-formedness-allocation}{wf:impl-empty-region}.
        %
        The rule guarantees that, if $(\reg \mapsto \emptyset) \in \AENV$, then
        $\reg \not \in dom(\STOR)$, as needed.
        %
        By inversion on \tllstart{}, we establish this precondition, thereby
        discharging the case.
        \item Case (\refwellformed{sec:well-formedness-allocation}{wf:impl-linear-alloc2}):
        This obligation discharges immediately because, by inversion on \tllstart{}, $\loc \in \NENV'$.
        \item Case (\refwellformed{sec:well-formedness-allocation}{wf:impl-write-once}):
        The proof obligation is to establish that, for any constructor tag $\DC$,
        \begin{align*}
         ((\locreg{\loc}{\reg} \mapsto \concreteloc{\reg}{0}{}) \in \MENV' \wedge \\
          (\reg \mapsto (0 \mapsto \DC)) \not \in \STOR)
        \end{align*}
        The first conjunct discharges by inversion on \dletlocstart{},
        and the second as a consequence of having already
        established just above that $0 > \allocptr{\reg}{\STOR}$.
        \item Case (\refwellformed{sec:well-formedness-allocation}{wf:impl-empty-region}):
        The obligation for this case is to establish that
        for each $(\reg \mapsto \emptyset) \in \AENV' = \AENV \cup \set{\reg \mapsto \locreg{\loc}{\reg}}$, it
        is the case that $\reg \not \in dom(\STOR)$.
        %
        The part of this obligation pertaining to environment $\AENV$
        is given by the premise of this lemma, and thus it only remains
        to establish that the property holds for the rest, namely
        $\set{\reg \mapsto \locreg{\loc}{\reg}}$.
        %
        This part discharges trivially, because $(\reg \mapsto \emptyset) \not \in \AENV'$,
        thereby discharging this case.
        \end{itemize}
      \item
      Case (\refwellformed{sec:well-formedness}{wf:impl1}):
      \begin{displaymath}
      dom(\SENV) \cap \NENV' = \emptyset
      \end{displaymath}
      This case discharges straightforwardly.
      \end{itemize}
    \end{itemize}
  \end{bcase}

  \begin{bcase}
    \begin{mathpar}
    \rdletregion{}
    \end{mathpar}
    \begin{itemize}
    \item
    The first of two proof obligations is to show that
    the result $\EXPR'$ of
    the given step of evaluation is well typed, that is,
    \begin{displaymath}
    \emptytenv;\SENV;\CENV';\AENV';\NENV' \vdash \AENV''; \NENV''; \EXPR' : \hTYP,
    \end{displaymath}
    where $\hTYP = \tyatlocreg{\TYP}{\loc'}{\reg'}$.
    %
    This proof obligation follows straightforwardly by inversion
    on \tlregion{}.
    \item The second obligation for this proof case is to show that
    \begin{displaymath}
    \storewf{\SENV}{\CENV}{\AENV'}{\NENV}{\MENV}{\STOR}.
    \end{displaymath}
    The individual requirements, labeled
    \refwellformed{sec:well-formedness}{wf:map-store-consistency} -
        \refwellformed{sec:well-formedness}{wf:ca},
        are handled by the following case analysis.
    \begin{itemize}
      \item
      Case (\refwellformed{sec:well-formedness}{wf:map-store-consistency}):
      for each $(\locreg{\loc'}{\reg} \mapsto \TYP) \in \SENV$, there exists some $\ind_1, \ind_2$ such that
      \begin{align*}
      (\locreg{\loc'}{\reg} \mapsto \concreteloc{\reg}{\ind_1}{}) \in \MENV \wedge \\
        \ewitness{\TYP}{\concreteloc{\reg}{\ind_1}{}}{\STOR}{\concreteloc{\reg}{\ind_2}{}}
      \end{align*}
      %
      This case discharges immediately by inversion of \tlregion{} and \dletregion{},
      because none of the relevant environments are affected by the transition.
      \item Case (\refwellformed{sec:well-formedness}{wf:cfc}):
      \begin{align*}
      \storewfcfa{\CENV}{\MENV}{\STOR}
      \end{align*}
      The case discharges in a fashion similar to the previous one.
      \item Case (\refwellformed{sec:well-formedness}{wf:ca}):
      \begin{align*}
      \storewfca{\AENV'}{\NENV}{\MENV}{\STOR}
      \end{align*}
      Of the requirements in this judgement, the only one that is affected by the
      new environment $\AENV'$ is requirement
      \refwellformed{sec:well-formedness-allocation}{wf:impl-empty-region}.
      %
      The specific obligation is to establish that,
      for each $(\reg \mapsto \emptyset) \in \AENV'$, it
      is the case that $\reg \not \in dom(\STOR)$.
      %
      By inversion on \tlregion{},
      $\AENV' = \AENV \cup \set{\reg \mapsto \emptyset}$, and therefore,
      the first part of the obligation, that is, for $\AENV$, is already
      given by the premise of this lemma.
      %
      As such, it only remains to establish that $\reg \not \in dom(\STOR)$,
      which follows from $\reg$ being a fresh region, thereby ruling out
      it being in the store, and thus discharging this case.
      \item
      Case (\refwellformed{sec:well-formedness}{wf:impl1}):
      \begin{displaymath}
      dom(\SENV) \cap \NENV' = \emptyset
      \end{displaymath}
      This case discharges straightforwardly.
      \end{itemize}
    \end{itemize}
  \end{bcase}

  \begin{bcase}
    \begin{mathpar}
    \rdletval{}
    \end{mathpar}
    \begin{itemize}
    \item
    The first of two proof obligations is to show that
    the result $\subst{\EXPR_2}{\var}{\VAL_1}$ of
    the given step of evaluation is well typed, that is,
    \begin{displaymath}
    \emptytenv;\SENV';\CENV;\AENV;\NENV \vdash \AENV; \NENV; \subst{\EXPR_2}{\var}{\VAL_1} : \tyatlocreg{\TYP_2}{\loc_2}{\reg_2}.
    \end{displaymath}
    By inversion on \tlet{}, we obtain the type of the value being bound
    \begin{align*}
    \emptyset;\SENV;\CENV;\AENV;\NENV \vdash \AENV;\NENV;\VAL_1 : \tyatlocreg{\TYP_1}{\loc_1}{\reg_1},
    \end{align*}
    and we obtain the type of the body
    \begin{align*}
    \TENV';\SENV';\CENV;\AENV;\NENV \vdash \AENV;\NENV;\EXPR_2 : \tyatlocreg{\TYP_2}{\loc_2}{\reg_2}
    \end{align*}
    where
    \begin{align*}
    \TENV' &= \set{\var \mapsto \tyatlocreg{\TYP_1}{\loc_1}{\reg_1}} \\
    \SENV' &= \SENV \cup \set{\locreg{\loc_1}{\reg_1} \mapsto \TYP_1}.
    \end{align*}
    As such we can apply the Substitution
    Lemma \ref{lemma:substitution}, as follows
    \begin{displaymath}
    \emptytenv;\SENV';\CENV;\AENV;\NENV \vdash \AENV; \NENV; \subst{\EXPR_2}{\var}{\VAL_1} \subst{}{\locreg{\loc_1}{\reg_1}}{\locreg{\loc_1}{\reg_1}} : \tyatlocreg{\TYP_2}{\loc_2}{\reg_2},
    \end{displaymath}
    which discharges our obligation, given that the substitution of the
    bound location $\locreg{\loc_1}{\reg_1}$ is the identity substitution.
    \item Given the instantiations of $\SENV'$ and $\MENV'$
    used by the previous step, the second obligation
    for this proof case is to show that
    \begin{displaymath}
    \storewf{\SENV'}{\CENV}{\AENV}{\NENV}{\MENV'}{\STOR}.
    \end{displaymath}
    The individual requirements, labeled
    \refwellformed{sec:well-formedness}{wf:map-store-consistency} -
        \refwellformed{sec:well-formedness}{wf:ca},
        are handled by the following case analysis.
    \begin{itemize}
      \item
      Case (\refwellformed{sec:well-formedness}{wf:map-store-consistency}):
      for each $(\locreg{\loc'}{\reg} \mapsto \TYP) \in \SENV' = \SENV \cup \set{\locreg{\loc_1}{\reg_1} \mapsto \TYP_1}$, there exists some $\ind_1, \ind_2$ such that
      \begin{align*}
      (\locreg{\loc'}{\reg} \mapsto \concreteloc{\reg}{\ind_1}{}) \in \MENV \wedge \\
        \ewitness{\TYP}{\concreteloc{\reg}{\ind_1}{}}{\STOR}{\concreteloc{\reg}{\ind_2}{}}
      \end{align*}
      %
      This obligation amounts to showing the above holds for the bound location
      $\locreg{\loc_1}{\reg_1}$, because the well formedness of the
      store given by the premise of this lemma guarantees the property
      holds for locations bound in $\SENV$.
      %
      The value $\VAL_1$ bound at location $\locreg{\loc_1}{\reg_1}$ is
      a value and is well typed, and as such, there are only two typing
      rules that could apply, namely \tvar{} and \tconcreteloc{}.
      %
      By inversion on these rules, we establish that
      \begin{align*}
      (\locreg{\loc_1}{\reg_1} \mapsto \TYP_1) \in \SENV.
      \end{align*}
      %
      Therefore, we can discharge this obligation by application of well formedness of the
      store, in particular, the rule
      \refwellformed{sec:well-formedness}{wf:map-store-consistency} we
      are currently considering.
      %
      Concretely, we discharge this obligation by instantiating that rule to
      %
      \begin{align*}
      (\locreg{\loc_1}{\reg_1} \mapsto \concreteloc{\reg_1}{\ind_1}{}) \in \MENV \wedge \\
        \ewitness{\TYP_1}{\concreteloc{\reg_1}{\ind_1}{}}{\STOR}{\concreteloc{\reg_1}{\ind_2}{}}.
      \end{align*}
      \item Case (\refwellformed{sec:well-formedness}{wf:cfc}):
      \begin{align*}
      \storewfcfa{\CENV}{\MENV}{\STOR}
      \end{align*}
      This case discharges immediately because the relevant environments
      are affected by neither the of the relevant typing nor the dynamic-semantic judgement.
      \item Case (\refwellformed{sec:well-formedness}{wf:ca}):
      \begin{align*}
      \storewfca{\AENV}{\NENV}{\MENV}{\STOR}
      \end{align*}
      This case discharges immediately because the relevant environments
      are affected by neither the of the relevant typing nor the dynamic-semantic judgement.
      \item
      Case (\refwellformed{sec:well-formedness}{wf:impl1}):
      \begin{displaymath}
      dom(\SENV') \cap \NENV = \emptyset
      \end{displaymath}
      This case discharges straightforwardly.
      \end{itemize}
    \end{itemize}
  \end{bcase}

  \begin{bcase}
    \begin{mathpar}
    \rdletexp{}
    \end{mathpar}
    \begin{itemize}
    \item
    The first of two proof obligations is to show that
    the result $\letpack{\var : \hTYP}{\EXPR'_1}{\EXPR_2}$ of
    the given step of evaluation is well typed, that is,
    \begin{displaymath}
    \emptytenv;\SENV;\CENV;\AENV';\NENV' \vdash \AENV''; \NENV''; \letpack{\var : \hTYP}{\EXPR'_1}{\EXPR_2} : \tyatlocreg{\TYP_2}{\loc_2}{\reg_2},
    \end{displaymath}
    %
    The induction hypothesis is
    \begin{displaymath}
      \begin{aligned}
        \text{If} \;\; & \emptytenv;\SENV;\CENV;\AENV;\NENV \vdash \AENV';\NENV';\EXPR_1 : \tyatlocreg{\TYP_1}{\loc_1}{\reg_1} \\
        \text{and} \;\; & \storewf{\SENV}{\CENV}{\AENV}{\NENV}{\MENV}{\STOR}\\
        \text{and} \;\; & \STOR;\MENV;\EXPR_1 \stepsto \STOR';\MENV';\EXPR_1' \\
        \text{then for some} \;\; & \SENV' \supseteq \SENV, \CENV' \supseteq \CENV ,\\
        & \emptytenv;\SENV';\CENV';\AENV';\NENV' \vdash \AENV'';\NENV'';\EXPR_1' : \tyatlocreg{\TYP_1}{\loc_1}{\reg_1} \\
        \text{and} \;\; & \storewf{\SENV'}{\CENV'}{\AENV'}{\NENV'}{\MENV'}{\STOR'}.
      \end{aligned}
    \end{displaymath}
    %
    By inversion on \tlet{}, we establish that
    \begin{align*}
    \emptytenv;\SENV;\CENV;\AENV;\NENV \vdash \AENV';\NENV';\EXPR_1 : \tyatlocreg{\TYP_1}{\loc_1}{\reg_1},
    \end{align*}
    and, by the premise of this lemma, we establish that
    \begin{align*}
    \storewf{\SENV}{\CENV}{\AENV}{\NENV}{\MENV}{\STOR}
    \end{align*}
    and by inversion on \dletexp{} we establish that
    \begin{align*}
    \STOR;\MENV;\EXPR_1 \stepsto \STOR';\MENV';\EXPR_1'.
    \end{align*}
    %
    Now, we can apply the above to the induction hypothesis to establish
        \begin{displaymath}
      \begin{aligned}
        \text{For some} \;\; & \SENV' \supseteq \SENV, \CENV' \supseteq \CENV ,\\
        & \emptytenv;\SENV';\CENV';\AENV';\NENV' \vdash \AENV'';\NENV'';\EXPR_1' : \tyatlocreg{\TYP_1}{\loc_1}{\reg_1} \\
        \text{and} \;\; & \storewf{\SENV'}{\CENV'}{\AENV'}{\NENV'}{\MENV'}{\STOR'}.
      \end{aligned}
    \end{displaymath}
    %
    By inversion on \tlet{}, we also have that
    \begin{align*}
    \TENV';\SENV';\CENV;\AENV';\NENV' \vdash \AENV'';\NENV'';\EXPR_2 : \tyatlocreg{\TYP_2}{\loc_2}{\reg_2},
    \end{align*}
    where
    \begin{align*}
    \TENV' &= \set{\var \mapsto \tyatlocreg{\TYP_1}{\loc_1}{\reg_1}} \\
    \SENV' &= \set{\locreg{\loc_1}{\reg_1} \mapsto \TYP_1}.
    \end{align*}
    %
    By inspection on \tlet{} and the previous two typing judgements, that is, for $\EXPR_1'$
    and $\EXPR_2$, we discharge this case.
    \item The second obligation
    \begin{displaymath}
    \storewf{\SENV'}{\CENV'}{\AENV'}{\NENV'}{\MENV'}{\STOR'}
    \end{displaymath}
    discharges immediately from the result of the induction hypothesis, which is
    established by the above.
    \end{itemize}
  \end{bcase}

  \begin{bcase}
    \begin{mathpar}
    \rdapp{}
    \end{mathpar}
    \begin{itemize}
    \item
    The first of two proof obligations is to show that
    the result $\EXPR' = \subst{\EXPR}{\overharpoon{\var}}{\overharpoon{\VAL}} \subst{}{\overharpoon{\locreg{\loc'}{\reg'}}}{\overharpoon{\locreg{\loc}{\reg}}}$ of
    the given step of evaluation is well typed, that is,
    \begin{displaymath}
    \emptytenv;\SENV';\CENV;\AENV;\NENV' \vdash \AENV'; \NENV''; \subst{\EXPR}{\overharpoon{\var}}{\overharpoon{\VAL}} \subst{}{\overharpoon{\locreg{\loc'}{\reg'}}}{\overharpoon{\locreg{\loc}{\reg}}} : \hTYP,
    \end{displaymath}
    where $\hTYP = \tyatlocreg{\TYP}{\loc}{\reg}$.
    %
    To this end, we first establish typing judgements for the body of the
    callee and then the arguments of the function, and finally
    discharge the first obligation by combining the two results using
    the substitution lemma.
    By inversion on \tfunctiondef{}, the type judgement
    \begin{align*}
    \TENV;\SENV'';\CENV;\AENV;\NENV \vdash \AENV;\NENV';
    \EXPR : \tyatlocreg{\TYP}{\loc}{\reg},
    \end{align*}
    holds for body of the callee $\EXPR$,
    with constrants
    for any caller, such that $\locreg{\loc}{\reg} \in \NENV$, $\locreg{\loc}{\reg} \not \in \NENV'$ and $\AENV(\reg) = \locreg{\loc}{\reg}$, where
    \begin{align*}
    \TENV & = \set{\overharpoon{\var_1} \mapsto \overharpoon{{\tyatlocreg{\TYP_1}{\loc'_1}{\reg'}}}, \; \ldots \; , \overharpoon{\var_n} \mapsto \overharpoon{{\tyatlocreg{\TYP_n}{\loc'_n}{\reg'}}}}\\
    \SENV'' & = \set{\overharpoon{\locreg{\loc'_1}{\reg'}} \mapsto \overharpoon{\TYP_1}, \; \ldots \; , \overharpoon{\locreg{\loc'_n}{\reg'}} \mapsto \overharpoon{\TYP_n}}.
    \end{align*}
    Regarding the arguments to the call, we obtain by inversion on \tapp{} that
    \begin{align*}
    \emptyset;\SENV;\CENV;\AENV;\NENV \vdash \AENV;\NENV;\overharpoon{\VAL_i} : \overharpoon{\tyatlocreg{\TYP_i}{\loc_i}{\reg}}
    \end{align*}
    for $\ind \in \set{1 \ldots n}$.
    Furthermore, by inversion on \tapp{}, we obtain that $\locreg{\loc}{\reg} \in \NENV$,
    $\locreg{\loc}{\reg} \not \in \NENV'$, and $\AENV(\reg) = \locreg{\loc}{\reg}$,
    which altogether satisfy the requirements of \tfunctiondef{}.
    Now, by application of the Substitution Lemma,
    we have that
    \begin{align*}
    \emptytenv;\SENV;\CENV;\AENV;\NENV' \vdash \AENV; \NENV'; \subst{\EXPR}{\overharpoon{\var_1}}{\overharpoon{\VAL_1}} \subst{}{\overharpoon{\locreg{\loc'_1}{\reg'}}}{\overharpoon{\locreg{\loc_1}{\reg}}} \ldots \subst{}{\overharpoon{\var_n}}{\overharpoon{\VAL_n}} \subst{}{\overharpoon{\locreg{\loc'_n}{\reg'}}}{\overharpoon{\locreg{\loc_n}{\reg}}} : \tyatlocreg{\TYP}{\loc}{\reg}.
    \end{align*}
    \item Given the new environment $\NENV'$
    used by the previous step, the second obligation
    for this proof case is to show that
    \begin{displaymath}
    \storewf{\SENV}{\CENV}{\AENV}{\NENV'}{\MENV}{\STOR}.
    \end{displaymath}
    The individual requirements, labeled
    \refwellformed{sec:well-formedness}{wf:map-store-consistency} -
        \refwellformed{sec:well-formedness}{wf:ca},
        are handled by the following case analysis.
    \begin{itemize}
      \item
      Case (\refwellformed{sec:well-formedness}{wf:map-store-consistency}):
      for each $(\locreg{\loc'}{\reg} \mapsto \TYP) \in \SENV$, there exists some $\ind_1, \ind_2$ such that
      \begin{align}
      (\locreg{\loc'}{\reg} \mapsto \concreteloc{\reg}{\ind_1}{}) \in \MENV \wedge \\
        \ewitness{\TYP}{\concreteloc{\reg}{\ind_1}{}}{\STOR'}{\concreteloc{\reg}{\ind_2}{}}
      \end{align}
      %
      This case discharges immediately from the well formedness of the store
      given by the premise of this lemma.
      \item Case (\refwellformed{sec:well-formedness}{wf:cfc}):
      \begin{align*}
      \storewfcfa{\CENV}{\MENV}{\STOR}
      \end{align*}
      This case discharges immediately from the well formedness of the store
      given by the premise of this lemma.
      \item Case (\refwellformed{sec:well-formedness}{wf:ca}):
      \begin{align*}
      \storewfca{\AENV}{\NENV'}{\MENV}{\STOR}
      \end{align*}
      Of the requirements pertaining to this judgement, the only one potentially
      affected by the new environment $\NENV'$ is requirement
      \refwellformed{sec:well-formedness-allocation}{wf:impl-linear-alloc2}.
      %
      The specific obligation therein is to establish that
      \begin{align*}
      ((\reg \mapsto \locreg{\loc}{\reg}) \in \AENV \wedge
    \, (\locreg{\loc}{\reg} \mapsto \concreteloc{\reg}{\ind_s}{}) \in \MENV \wedge \locreg{\loc}{\reg} \not \in \NENV' \wedge
    \, \ewitness{\TYP}{\concreteloc{\reg}{\ind_s}{}}{\STOR}{\concreteloc{\reg}{\ind_e}{}}) \Rightarrow \\
          \ind_e > \allocptr{\reg}{\STOR}.
          \end{align*}
      %
      The reason the change to environment $\NENV'$ might affect
      the above is because, if all the conjuncts above hold, then
      it remains to establish that $\ind_e > \allocptr{\reg}{\STOR}$
      holds.
      %
      However, it turns out that the fourth conjunct above does not
      hold, i.e., there is no such end witness in the store $\STOR$,
      thus relieving the obligation to establish $\ind_e > \allocptr{\reg}{\STOR}$.
      %
      The reason the end witness does not exist is yielded by
      the well formedness of the store given by the premise of this
      lemma, in particular requirement
      \refwellformed{sec:well-formedness-allocation}{wf:impl-linear-alloc}.
      %
      That is, by inversion on \tapp{}, it is the case that
      \begin{align*}
      (\reg \mapsto \locreg{\loc}{\reg}) \in \AENV \wedge \locreg{\loc}{\reg} \in \NENV.
      \end{align*}
      %
      Therefore, requirement \refwellformed{sec:well-formedness-allocation}{wf:impl-linear-alloc}
      implies that
      \begin{align*}
      \ind_s > \allocptr{\reg}{\STOR}.
      \end{align*}
      %
      As such, given that the store $\STOR$ remains unchanged
      and the above, it is straightforward to show that the end witness
      starting at $\ind_s$ cannot exist, thereby discharging this case.
      \end{itemize}
      \item
      Case (\refwellformed{sec:well-formedness}{wf:impl1}):
      \begin{displaymath}
      dom(\SENV) \cap \NENV' = \emptyset
      \end{displaymath}
      This case discharges because,
      from the well formedness of the store given by the
      premise of this lemma, $dom(\SENV) \cap \NENV = \emptyset$,
      and because $\NENV' = \NENV - \set{\locreg{\loc}{\reg}}$.
    \end{itemize}
  \end{bcase}

$\blacksquare$

\end{nproof}

%% \subsection{Type Safety}

The type safety theorem for \ourcalc{} was stated in~\ref{theorem:type-safety} and is restated here.

\begin{theorem}[Type safety]
  %% \label{theorem:type-safety}
\begin{displaymath}
  \begin{aligned}
  \text{If} \;\; & (\emptyset;\SENV;\CENV;\AENV;\NENV \vdash \AENV';\NENV';\EXPR : \hTYP) \wedge
                   (\storewf{\SENV}{\CENV}{\AENV}{\NENV}{\MENV}{\STOR}) \\
  \text{and} \;\; & \STOR;\MENV;\EXPR \stepsto^n \STOR';\MENV';\EXPR' \\
  \text{then} \;\; & (\EXPR' \; \mathit{value}) \vee
                     (\exists \STOR'', \MENV'', \EXPR'' . \; \STOR';\MENV';\EXPR' \stepsto \STOR'';\MENV'';\EXPR'')
  \end{aligned}
  \end{displaymath}
\end{theorem}

\begin{nproof}
  The type safety follows from an induction with
  \ref{lemma:progress} (progress lemma) and \ref{lemma:preservation} (preservation lemma).
\end{nproof}

\mv{Overview of type safety of LoCal and why we should care.}
% Full type safety proof is give in the appendix of \cite{local-extended}.

\subsubsection{Variables and substitution}
I use the convention that all variables for binding values,
locations, and regions are distinct, and maintain this invariant
implicitly.
%
The bindings sites of variables are summarized by the following:
%
\begin{itemize}
\item Variables for binding values $\var$ are bound by function definitions
$\FD$ and pattern matches $\pat$.

\item Location variables $\locreg{\loc}{\reg}$ are bound by type schemes $\TS$, pattern matches
$\pat$, and $\keywd{letloc}$ binders.

\item Region variables $\reg$ are bound by type schemes $\TS$, pattern
matches $\pat$, and $\keywd{letregion}$ binders.
\end{itemize}
%
The use sites of variables are summarized by the following:
\begin{itemize}
\item Variables for binding varlues $\var$ are used by values $\VAL$.

\item Location variables $\locreg{\loc}{\reg}$ are used by concrete locations $\concreteloc{\reg}{\ind}{\locreg{\loc}{\reg}}$,
the argument list of function applications
$\fapp{\overharpoon{\locreg{l}{r}}}{\overharpoon{\VAL}}$, the location
argument of constructor applications
$\datacon{\DC}{\keywd{\locreg{\loc}{\reg}}}{\overharpoon{\VAL}}$,
located types $\hTYP$, and located expressions $\LE$.

\item Region varaibles $\reg$ are used in the same places as location variables.
\end{itemize}
%
We use the following conventions for variable substitution:
%
\begin{itemize}
\item $\subst{\EXPR}{x}{v}$: Substitute $v$ for $x$ in $e$. We let the notation extend to vectors such that
$\subst{\EXPR}{\overharpoon{x}}{\overharpoon{v}}$ denotes the iterated substitution $\subst{\EXPR}{\overharpoon{x_1}}{\overharpoon{v_1}} \ldots \subst{}{\overharpoon{x_n}}{\overharpoon{v_n}}$, where $n = |\overharpoon{x}| = |\overharpoon{v}|$.

\item $\subst{\EXPR}{\locreg{\loc_1}{\reg_1}}{\locreg{\loc_2}{\reg_2}}$: Substitute location variable $\locreg{\loc_2}{\reg_2}$ for $\locreg{\loc_1}{\reg_1}$ in $\EXPR$. We extend this notation to vectors of locations in the same fashion, as described above.

\item $\subst{\EXPR}{\reg_1}{\reg_2}$ : Substitute region variable $\reg_2$ for $\reg_1$ in $\EXPR$. We extend this notation to vectors of locations in the same fashion, as described above.

\item Finally, we extend the aforementioned notation so that
  substitution can act on environments $\CENV$, $\AENV$, and $\NENV$,
  e.g.,
  $\subst{\CENV}{\locreg{\loc_1}{\reg_1}}{\locreg{\loc_2}{\reg_2}}$.
\end{itemize}

% \subsubsection{Global environment and metafunctions}
% \begin{itemize}
% \item $Function(f)$: An environment that maps a function $f$ to its definition $\FD$.

% \item $Freshen(\FD)$: A metafunction that freshens all bound variables in function definition
% $\FD$ and returns the resulting function definition.

% \item $TypeOfCon(\DC):$ An environment that maps a data constructor to its type.

% \item $TypeOfField(\DC,i)$: A metafunction that returns the type of the \il{i}'th field
% of data constructor $\DC$.

% \item $ArgTysOfConstructor(\DC)$: An environment that maps a data constructor to its field types.

% \item $\allocptr{\reg}{\STOR}$: $\max \set{-1} \cup \set{ \indj \; | \; {(\reg \mapsto (\indj \mapsto \DC)) \in \STOR}}$.
% \end{itemize}


\subsubsection{Store typing}
\label{sec:well-formedness}
A key part of the safety of \ourcalc{} programs is the following property: if a
term $e$ is type $\tyatlocreg{\TYP}{\loc}{\reg}$, then if we look in the store
under region $\reg$ at the location represented by $\loc$, we will find the
start of a serialized data structure which is a valid serialization of a value
of type $\TYP$. In other words, the types tell us the truth about the values in
the store.
%
To achieve this, we rely on the store being \emph{well-formed}, whose definition
itself uses three other judgements (shown in Table~\ref{tbl:swf-judgements}).
%
% \begin{displaymath}
%   \storewf{\SENV}{\CENV}{\AENV}{\NENV}{\MENV}{\STOR}
% \end{displaymath}

\begin{table}
\bgroup
% \def\arraystretch{1.2}
% \setlength\tabcolsep{0.5cm}
\begin{tabular}{llp{8cm}}
 & \textbf{Judgement form} & \textbf{Summary}
 \\\\
\parbox[t]{3.5cm}{Store \\ well formedness} & $\storewf{\SENV}{\CENV}{\AENV}{\NENV}{\MENV}{\STOR}$ &

The store $\STOR$ along with location map $\MENV$ are well formed with respect to
typing environments $\SENV$, $\CENV$, and $\AENV$.
\\\\
End witness & $\ewitness{\TYP}{\concreteloc{\reg}{\ind_{s}}{}}{\STOR}{\concreteloc{\reg}{\ind_{e}}{}}$ &

The store address $\concreteloc{\reg}{\ind_{e}}{}$ is the position one
after the last cell of the tree of type $\TYP$ starting at
$\concreteloc{\reg}{\ind_{s}}{}$ in store $\STOR$.
\\\\
\parbox[t]{3.5cm}{Constructor-application \\ well formedness}
 & $\storewfcfa{\CENV}{\MENV}{\STOR}$ &

All in-flight data-constructor applications in store $\STOR$ along with location map $\MENV$
are well formed with respect to constructor-progress typing environment $\CENV$.
\\\\
\parbox[t]{3.5cm}{Allocation \\ well formedness} & $\storewfca{\AENV}{\NENV}{\MENV}{\STOR}$ &

Allocation in store $\STOR$ along with location map $\MENV$ is well formed
with respect to allocation-typing environments $\AENV$ and $\NENV$.
\end{tabular}
\egroup
\caption{Summary of judgements used to establish well formedness of the store.}
\label{tbl:swf-judgements}
\end{table}


% The store typing specifies three categories of invariants.
% %
% \begin{enumerate}
% \item The first enforces that allocations occur in the sequence
% specified by the constraint environment $\CENV$.
% %
% In particular, if there is some location $\loc$ in the domain of
% $\CENV$, then the location map and store have the expected
% allocations at the expected types.
% %
% For instance, if $(\loc \mapsto
% \afterl{\tyatlocreg{\TYP}{\loc'}{\reg}}) \in \CENV$, then $\loc'$ maps
% to $\concreteloc{\reg}{\ind_1}{}$ and $\loc$ to
% $\concreteloc{\reg}{\ind_2}{}$ in the location map, and $\ind_2$ is
% the end witness of $\ind_1$ at type $\TYP$ in the store, at region
% $\reg$.
% %
% \item The second category enforces that, for each symbolic location such
% that $(\loc \mapsto \TYP) \in \SENV$, there is some
% $\concreteloc{\reg}{\ind_1}{}$ for $\loc$ in the location map and
% $\ind_1$ has some end witness $\ind_2$ at type $\TYP$.
% %
% \item The final category enforces that each address in the store is written
% once.
% %
% This property is asserted by insisting that, if $\loc \in \NENV$, then
% there is some $\concreteloc{\reg}{\ind}{}$ for $\loc$ in the location
% map, but there is no write to for $\ind$ at $\reg$ in the store.
% %
% To support this property, there are two additional conditions which
% require that the most recently allocated location (tracked by
% to $\AENV, \NENV$) is at the end of its respective region.
% \end{enumerate}

% To support this behavior, there are two additional conditions in this
% category that enforce linearity of allocation.
% %
% The first one captures the situation when there is an allocation in
% flight, such as is the case in the postcondition of allocating
% a tag (and the corresponding binding the cell to $\loc$).
% %
% In this case, the typing rule specifies that $(\reg \mapsto \loc) \in
% A$ and $\loc \in \NENV$.
% %
% The requirement is that the address bound to $\loc$ is the maximum
% position in the location map for the region $\reg$ and that this
% position is past the end of any address already written in the store
% at $\reg$.
% The second condition captures the other situation, just after
% a field has been allocated and finalized, which is occurs
% when $(\reg \mapsto \loc) \in A$ and $(\loc \mapsto \TYP) \in \SENV$
% for the location $\loc$ representing the field.
% %
% In this case, the condition requires that the address of the end
% witness of the field is the maximum of all other addresses in the
% store at region $\reg$.



\paragraph{Notation for references to well-formedness judgements}
Because there are many requirements specified inside the various
well-formedness judgements, we introduce notation for referring
to requirements individually.
%
For example, the notation
\refwellformed{sec:well-formedness-allocation}{wf:impl-linear-alloc2}
refers to the judgement
\begin{align*}
\storewfca{\AENV}{\NENV}{\MENV}{\STOR},
\end{align*}
specified in Section~\ref{sec:well-formedness-allocation},
and in that judgement, rule number~\ref{wf:impl-linear-alloc2}.

The definition of store well formedness follows.

\paragraph{Judgement form}

$\storewf{\SENV}{\CENV}{\AENV}{\NENV}{\MENV}{\STOR}$

The well-formedness judgement specifies the valid layouts of the store by using the location
map and the various environments from the typing judgement.
%
Rule~\ref{wf:map-store-consistency} specifies that, for each location in the store-typing environment,
there is a corresponding concrete location in the location map and that concrete location satisfies
a corresponding end-witness judgement.
%
Rules~\ref{wf:cfc} and~\ref{wf:ca} reference the judgements for well formedness concerning
in-flight constructor applications (\secref{sec:end-witness}) and correct allocation in
regions (\secref{sec:well-formedness-allocation}), respectively.
%
Finally, Rule~\ref{wf:impl1} specifies that the nursery and store-typing environments reference
no common locations, which is a way of reflecting that each location is either in the process
of being constructed and in the nursery, or allocated and in the store-typing environment, but
never both.

\paragraph{Definition}

\begin{enumerate}

    \item \label{wf:map-store-consistency} $ (\locreg{\loc}{\reg} \mapsto \TYP) \in \SENV \Rightarrow \\
            ((\locreg{\loc}{\reg} \mapsto \concreteloc{\reg}{\ind_1}{}) \in \MENV \wedge \\
            \ewitness{\TYP}{\concreteloc{\reg}{\ind_1}{}}{\STOR}{\concreteloc{\reg}{\ind_2}{}})
          $

    \item \label{wf:cfc} $\storewfcfa{\CENV}{\MENV}{\STOR}$

    \item \label{wf:ca} $\storewfca{\AENV}{\NENV}{\MENV}{\STOR}$

    \item \label{wf:impl1} $dom(\SENV) \cap \NENV = \emptyset $
\end{enumerate}


\subsubsection{End-Witness judgement}
\label{sec:end-witness}

\paragraph{Judgement form}

$\ewitness{\TYP}{\concreteloc{\reg}{\ind_{s}}{}}{\STOR}{\concreteloc{\reg}{\ind_{e}}{}}$

The end-witness judgement specifies the expected layout in the store of a fully
allocated data constructor.
%
Rule~\ref{ewitness:impl1} requires that the first cell store a constructor
tag of the appropriate type.
%
Rule~\ref{ewitness:impl2} specifies the address of the cell one past the tag.
%
Rule~\ref{ewitness:impl3} recursively specifies the positions of the constructor
fields.
%
Finally, Rule~\ref{ewitness:impl4} specifies that the end witness of
the overall constructor is the address one past the end of either the
tag, if the constructor has zero fields, or the final field,
otherwise.

\paragraph{Definition}

\begin{enumerate}
\item \label{ewitness:impl1} $\STOR(\reg)(\ind_s) = \DC'$ \; \text{such that} \\
      $\; \DATA\;\TYP = \overharpoon{\DC_1 \; \overharpoon{\sTYP}_1\;} \; | \; \ldots \; | \; \DC' \; \overharpoon{\sTYP}' \; | \; \ldots \; | \; \overharpoon{\DC_m \; \overharpoon{\sTYP}_m\;}$

\item \label{ewitness:impl4} $\overharpoon{w_0} = \ind_s + 1$

\item \label{ewitness:impl2}
  $\ewitness{\overharpoon{\TYP'_1}}{\concreteloc{\reg}{\overharpoon{w_0}}{}}{\STOR}{\concreteloc{\reg}{\overharpoon{w_1}}{}} \wedge$ \\
  $\ewitness{\overharpoon{\TYP'_{j+1}}}{\concreteloc{\reg}{\overharpoon{w_j}}{}}{\STOR}{\concreteloc{\reg}{\overharpoon{w_{j+1}}}{}}$
  \\ where $\indj \in \set{1,\ldots,n-1} ; n = | \overharpoon{\TYP'} |$

\item \label{ewitness:impl3}
  $\ind_e = \overharpoon{w_n}$
\end{enumerate}

\subsubsection{Well-formedness of constructor application}
\label{sec:well-formedness-constructors}

\paragraph{Judgement form}

$\storewfcfa{\CENV}{\MENV}{\STOR}$

The well-formedness judgement for constructor application specifies the various constraints
that are necessary for ensuring correct formation of constructors, dealing with constructor
application being an incremental process that spans multiple \ourcalc{} instructions.
%
Rule~\ref{wfconstr:constraint-start} specifies that, if a location corresponding to the
first address in a region is in the constraint environment, then there is a corresponding
entry for that location in the location map.
%
Rule~\ref{wfconstr:constraint-tag} specifies that, if a location corresponding to the address one past a constructor
tag is in the constraint environment, then there are corresponding locations for the address
of the tag and the address after in the location map.
%
Rule~\ref{wfconstr:constraint-after} specifies that, if a location corresponding to the address
one past after a fully allocated constructor application is in the constraint environment,
then there are corresponding locations for the address one past the constructor application
and for the address of the start of that constructor application in the location map, as well as the existence
of an end witness for that fully allocated location.

\paragraph{Definition}

\begin{enumerate}
    \item \label{wfconstr:constraint-start} $ (\locreg{\loc}{\reg} \mapsto \startr{\reg}) \in \CENV \Rightarrow \\
            (\locreg{\loc}{\reg} \mapsto \concreteloc{\reg}{0}{}) \in \MENV $

    \item \label{wfconstr:constraint-tag} $ (\locreg{\loc}{\reg} \mapsto (\locreg{\loc'}{\reg} + 1)) \in \CENV \Rightarrow
            \\
            (\locreg{\loc'}{\reg} \mapsto \concreteloc{\reg}{\ind_l}{})  \in \MENV \wedge \\
            (\locreg{\loc}{\reg} \mapsto \concreteloc{\reg}{\ind_l + 1}{})  \in \MENV
            $

    \item \label{wfconstr:constraint-after} $ (\locreg{\loc}{\reg} \mapsto \afterl{\tyatlocreg{\TYP}{\locreg{\loc'}{\reg}}{\reg}}) \in \CENV \Rightarrow \\
            ((\locreg{\loc'}{\reg} \mapsto \concreteloc{\reg}{\ind_1}{}) \in \MENV \wedge \\
            \ewitness{\TYP}{\concreteloc{\reg}{\ind_1}{}}{\STOR}{\concreteloc{\reg}{\ind_2}{}} \wedge \\
            (\locreg{\loc}{\reg} \mapsto \concreteloc{\reg}{\ind_2}{}) \in \MENV)
            $

\end{enumerate}


\subsubsection{Well-formedness concerning allocation}
\label{sec:well-formedness-allocation}

\paragraph{Judgement form}

$\storewfca{\AENV}{\NENV}{\MENV}{\STOR}$

The well-formedness judgement for safe allocation specifies the various properties
of the location map and store that enable continued safe allocation, avoiding in particular
overwriting cells, which could, if possible, invalidate overall type safety.
%
Rule~\ref{wf:impl-linear-alloc} requires that, if a location is in both the allocation
and nursery environments, i.e., that address represents an in-flight
constructor application, then there is a corresponding location in the location
map and the address of that location is the highest address in the store.
%
Rule~\ref{wf:impl-linear-alloc2} requires that, if there is an address in the allocation
environment and that address is fully allocated, then the address of that location is the
highest address in the store.
%
Rule~\ref{wf:impl-write-once} requires that, if there is an address in the nursery, then
there is a corresponding location in the location map, but nothing at the corresponding
address in the store.
%
Finally, Rule~\ref{wf:impl-empty-region} requires that, if there is a region that has been
created but for which nothing has yet been allocated, then there can be no addresses
for that region in the store.

\paragraph{Definition}

\begin{enumerate}
    \item \label{wf:impl-linear-alloc} $ ((\reg \mapsto \locreg{\loc}{\reg}) \in \AENV \wedge \locreg{\loc}{\reg} \in \NENV) \Rightarrow \\
          ((\locreg{\loc}{\reg} \mapsto \concreteloc{\reg}{\ind}{}) \in \MENV \wedge
%          \ind = \max \set{0} \cup \set{\indj \; | \; \concreteloc{\reg}{\indj}{} \in \MENV} \wedge \\
          \ind > \allocptr{\reg}{\STOR})
          $

    \item \label{wf:impl-linear-alloc2} $ ((\reg \mapsto \locreg{\loc}{\reg}) \in \AENV \wedge
    \, (\locreg{\loc}{\reg} \mapsto \concreteloc{\reg}{\ind_s}{}) \in \MENV \wedge \locreg{\loc}{\reg} \not \in \NENV \wedge
    \, \ewitness{\TYP}{\concreteloc{\reg}{\ind_s}{}}{\STOR}{\concreteloc{\reg}{\ind_e}{}}) \Rightarrow \\
          \ind_e > \allocptr{\reg}{\STOR}
          $

    \item \label{wf:impl-write-once} $ \locreg{\loc}{\reg} \in \NENV \Rightarrow \\
          ((\locreg{\loc}{\reg} \mapsto \concreteloc{\reg}{\ind}{}) \in \MENV \wedge \\
          (\reg \mapsto (\ind \mapsto \DC)) \not \in \STOR)
          $

    \item \label{wf:impl-empty-region} $(\reg \mapsto \emptyset) \in \AENV \Rightarrow \\
    \reg \not \in dom(\STOR)$
\end{enumerate}



\subsubsection{Type safety theorem}

\newcommand{\substlemmasubsts}[1]{\subst{#1}{\overharpoon{\var}}{\overharpoon{\VAL}} \subst{}{\overharpoon{\locreg{\loc}{\reg}}}{\overharpoon{\locreg{\loc'}{\reg'}}} \subst{}{\locreg{\loc}{\reg}}{\locreg{\loc'}{\reg'}}}


% \begin{lemma}[Substitution lemma]
%   \label{lemma:substitution}
%   \begin{align*}
%   \text{If} \quad & \TENV \cup \set{\overharpoon{\var_1} \mapsto \overharpoon{\TYP_1} \ensuremath{@} \overharpoon{\locreg{\loc_1}{\reg_1}}, \ldots, \overharpoon{\var_n} \mapsto  \overharpoon{\TYP_n} \ensuremath{@} \overharpoon{\locreg{\loc_n}{\reg_n}}}; \SENV; \CENV; \AENV; \NENV \vdash \AENV'; \NENV'; \EXPR : \tyatlocreg{\TYP}{\loc}{\reg}\\
%   \text{and} \quad & \TENV; \SENV'; \CENV'; \AENV'; \NENV' \vdash \AENV'; \NENV'; \overharpoon{\VAL_i} : \overharpoon{\TYP_i} \ensuremath{@} \overharpoon{\locreg{\loc'_i}{\reg'_i}} \qquad \; i \in \set{1, \ldots, n}\\
%   \text{then} \quad & \TENV; \SENV'; \CENV'; \AENV'; \NENV' \vdash \AENV'''; \NENV'''; \substlemmasubsts{\EXPR} : \tyatlocreg{\TYP}{\loc'}{\reg'}\\
%    \text{where} \quad & \SENV = \SENV_0 \cup \set{\overharpoon{\locreg{\loc_1}{\reg_1}} \mapsto \overharpoon{\TYP_1}, \ldots, \overharpoon{\locreg{\loc_n}{\reg_n}} \mapsto \overharpoon{\TYP_n}}\\
%    \text{and} \quad & \forall_{(\var \mapsto \TYP'' \ensuremath{@} \locreg{\loc''}{\reg''}) \in \TENV} . (\locreg{\loc''}{\reg''} \mapsto \TYP'') \in \SENV_0 \\
%   \text{and} \quad & dom(\SENV) \cap \NENV = \emptyset\\
%   \text{and} \quad & \NENV = \NENV_0 \cup \locreg{\loc}{\reg}\\
%   \text{and} \quad & \SENV' = \SENV \cup \set{\overharpoon{\locreg{\loc'_1}{\reg'_1}} \mapsto \overharpoon{\TYP_1}, \ldots, \overharpoon{\locreg{\loc'_n}{\reg'_n}} \mapsto \overharpoon{\TYP_n}}\\
%   \text{and} \quad & \CENV' = \subst{\CENV}{\overharpoon{\locreg{\loc}{\reg}}}{\overharpoon{\locreg{\loc'}{\reg'}}} \subst{}{\locreg{\loc}{\reg}}{\locreg{\loc'}{\reg'}} \\
%   \text{and} \quad & \AENV' = \subst{\AENV}{\overharpoon{\locreg{\loc}{\reg}}}{\overharpoon{\locreg{\loc'}{\reg'}}} \subst{}{\locreg{\loc}{\reg}}{\locreg{\loc'}{\reg'}} \subst{}{\reg}{\reg'}\\
%   \text{and} \quad & \NENV' = \subst{\NENV}{\locreg{\loc}{\reg}}{\locreg{\loc'}{\reg'}}
%   \end{align*}
% \end{lemma}
% \begin{nproof}
% The proof is by rule induction on the given typing derivation.
% \end{nproof}

\begin{lemma}[Progress]
  \label{lemma:progress}
  \begin{displaymath}
  \begin{aligned}
  \text{if} \;\; & \emptyset;\SENV;\CENV;\AENV;\NENV \vdash \AENV';\NENV';\EXPR : \hTYP \\
  \text{and} \;\; & \storewf{\SENV}{\CENV}{\AENV}{\NENV}{\MENV}{\STOR} \\
  \text{then} \;\; & \EXPR \; \mathit{value} \\
  \text{else} \;\; & \STOR;\MENV;\EXPR \stepsto \STOR';\MENV';\EXPR'
  \end{aligned}
  \end{displaymath}
\end{lemma}
\begin{nproof}
The proof is by rule induction on the given typing derivation.
\end{nproof}

\begin{lemma}[Preservation]
  \label{lemma:preservation}
  \begin{displaymath}
    \begin{aligned}
      \text{If} \;\; & \emptytenv;\SENV;\CENV;\AENV;\NENV \vdash \AENV';\NENV';\EXPR : \hTYP \\
      \text{and} \;\; & \storewf{\SENV}{\CENV}{\AENV}{\NENV}{\MENV}{\STOR}\\
      \text{and} \;\; & \STOR;\MENV;\EXPR \stepsto \STOR';\MENV';\EXPR' \\
      \text{then for some} \;\; & \SENV' \supseteq \SENV, \CENV' \supseteq \CENV ,\\
      & \emptytenv;\SENV';\CENV';\AENV';\NENV' \vdash \AENV'';\NENV'';\EXPR' : \hTYP \\
      \text{and} \;\; & \storewf{\SENV'}{\CENV'}{\AENV'}{\NENV'}{\MENV'}{\STOR'}\\
    \end{aligned}
  \end{displaymath}
\end{lemma}
\begin{nproof}
The proof is by rule induction on the given derivation of the dynamic semantics.
\end{nproof}


The type safety of \ourcalc{} follows from the following result.

\begin{theorem}[Type safety]
  \label{theorem:type-safety}
\begin{displaymath}
  \begin{aligned}
  \text{If} \;\; & (\emptyset;\SENV;\CENV;\AENV;\NENV \vdash \AENV';\NENV';\EXPR : \hTYP) \wedge
                   (\storewf{\SENV}{\CENV}{\AENV}{\NENV}{\MENV}{\STOR}) \\
  \text{and} \;\; & \STOR;\MENV;\EXPR \stepsto^n \STOR';\MENV';\EXPR' \\
  \text{then} \;\; & (\EXPR' \; \mathit{value}) \vee
                     (\exists \STOR'', \MENV'', \EXPR'' . \; \STOR';\MENV';\EXPR' \stepsto \STOR'';\MENV'';\EXPR'')
  \end{aligned}
  \end{displaymath}
\end{theorem}

\begin{nproof}
  The type safety follows from an induction with
  the progress and preservation lemmas.
\end{nproof}

\mv{Show some cases of the proof as examples!}

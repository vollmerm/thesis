\documentclass[showabstract,showacknowledgments,showpreface,showdedication]{iuphd} 

\PassOptionsToPackage{dvipsnames}{xcolor}
\usepackage{marvosym,listings,etoolbox,amsmath}
\usepackage{mathtools}
\usepackage{MnSymbol}
\usepackage{xspace}
\usepackage{mathpartir}
\usepackage{stmaryrd}
\usepackage{hyperref}
\usepackage[noabbrev]{cleveref}
\usepackage{graphicx}
\usepackage{subcaption}
\usepackage{xspace}
\usepackage{float}
\usepackage{makecell}
\usepackage{adjustbox}
\usepackage{balance}
\usepackage{MnSymbol}%
\makeatletter
\let\th@plain\relax
\makeatother
\usepackage[thmmarks,thref,amsmath]{ntheorem}

\newtheorem{theorem}{Theorem}[section]
\newtheorem{lemma}[theorem]{Lemma}

% Cases in a proof.
\theorempreskip{0pt}
\theorempostskip{10pt}
\theoremheaderfont{\scshape}
\theorembodyfont{\upshape}
\theoremindent10pt
\newtheorem*{goal}{Obl.}

%\theorempreskip{5pt}
%\theorempostwork{\setcounter{goal}{0}}
\theoremindent10pt
\newtheorem{subcase}{SubCase}
\theorempostwork{\setcounter{subcase}{0}}
%\theoremindent0pt
\newtheorem{ncase}{Case}
\newtheorem*{bcase}{Case}


% Proof sketches.
\theoremstyle{nonumberplain}
\theorempreskip{0pt}
\theorempostskip{10pt}
\theoremheaderfont{\scshape}
\theorembodyfont{\upshape}

\theorempostwork{\setcounter{ncase}{0}}
\newtheorem{proofsketch}{Proof Sketch}

\theorempostwork{\setcounter{ncase}{0}}
\newtheorem{nproof}{Proof}



% For title and abstract page

\title{A Language-based Approach to Programming with Serialized Data}
\author{Michael Vollmer}
\date{Month 2020} % Completion date of Dissertation
\department{Computer Science} % Change this to your department if not Mathematics

% For acceptance and abstract page

\committeechair{Ryan Newton, PhD}
\readertwo{Jeremy Siek, PhD}
\readerthree{Sam Tobin-Hochstadt, PhD}
\readerfour{Larry Moss, PhD}
\defensedate{Month Day, 2020} % Date of PhD defense

% For Copyright Page
\cryear{2020} % Copyright year

% ================================================================================
% extra \newcommand's specific to this paper.
% ================================================================================

%% Oh no, we don't actually HAVE lambda!
%% Later we'll have to make this a higher-order calculus!
%% \newcommand{\ourcalc}[0]{\ensuremath{\lambda^{loc}}}
%% \newcommand{\lamadt}{\ensuremath{\lambda^{adt}}}
%% \newcommand{\lamcur}{\ensuremath{\lambda^{cur}}}

%% Proposal 1 - hey, we have capital lambda, and it's even the thing that binds
%% our lovely locations.
%% \newcommand{\ourcalc}[0]{\ensuremath{\Lambda^{loc}}}
%% \newcommand{\lamadt}{\ensuremath{\Lambda^{adt}}}
%% \newcommand{\lamcur}{\ensuremath{\Lambda^{cur}}}

% TODO: coming up with a new name.  RECONCILE usages with \ourcalc below.
% Ones using this macro have already been audited/reconciled.
\newcommand{\sysname}[0]{LoCal\xspace}

%% %% Proposal 2 - Core, like GHC Core.
% \newcommand{\ourcalc}[0]{\ensuremath{Core^{loc}}}

%% Helpers
\newcommand{\ourcalc}[0]{\sysname}
% \newcommand{\lamadt} [0]{\ensuremath{Core^{adt}}}
\newcommand{\lamadt} [0]{HiCal\xspace}
% \newcommand{\lamcur} [0]{\ensuremath{Core^{cur}}}
% \newcommand{\lamcur} [0]{\textsf{Cursored}}
% \newcommand{\lamcur} [0]{\textsf{CursedCal}}

\newcommand{\lamcur}[0]{NoCal\xspace}

\newcommand{\gramdef}{\; ::= \;}
\newcommand{\gramor}{\; | \;}
\newcommand{\keywd}[1]{\mathit{#1}}
\newcommand{\gramwd}[1]{\texttt{#1}}
\newcommand{\sgramwd}[1]{\texttt{#1}}
\newcommand{\skeywd}[1]{\mathit{#1}}

%% Grammar
\newcommand{\PROG}{\keywd{top}}
\newcommand{\DD}{\keywd{dd}}
\newcommand{\VD}{\keywd{vd}}
\newcommand{\FD}{\keywd{fd}}
\newcommand{\DC}{\keywd{K}}
\newcommand{\sDC}{\skeywd{K}}
\newcommand{\TC}{\keywd{T}}
\newcommand{\EXPR}{\keywd{e}}
\newcommand{\sEXPR}{\skeywd{e}}
\newcommand{\DATA}{\gramwd{data}}
\newcommand{\TYP}{\keywd{\tau}}
\newcommand{\hTYP}{\keywd{\hat{\tau}}}
\newcommand{\sTYP}{\skeywd{\tau}}
\newcommand{\var}{\svar}
\newcommand{\svar}{x}
\newcommand{\fvar}{\sfvar}
\newcommand{\sfvar}{f}
\newcommand{\yvar}{y}
\newcommand{\num}{n}
\newcommand{\ARROW}{\rightarrow}
\newcommand{\RP}{\keywd{dl}}
\newcommand{\sRP}{\skeywd{dl}}
\newcommand{\loc}{\skeywd{l}}
\newcommand{\sloc}{\skeywd{l}}
\newcommand{\concreteloc}[3]{\ensuremath{\langle #1, #2 \rangle ^{#3}}}
\newcommand{\reg}{\skeywd{r}}
\newcommand{\sreg}{\skeywd{r}}
\newcommand{\LS}{\keywd{ls}}
\newcommand{\LC}{\keywd{lc}}
\newcommand{\LE}{\keywd{le}}
\newcommand{\letpack}[3]{\gramwd{let}\;#1=#2\;\gramwd{in}\;#3}
\newcommand{\letloc}[3]{\gramwd{letloc}\;#1 = #2\;\gramwd{in}\;#3}
\newcommand{\letreg}[2]{\gramwd{letregion}\;#1\;\gramwd{in}\;#2}
\newcommand{\fapp}[2]{\fvar \;[#1]\; #2}
\newcommand{\TS}{\keywd{ts}}
\newcommand{\ssetbool}[4]{#4 = #3[#2\mapsto \keywd{#1}=\keywd{True}]}
\newcommand{\sinlinesetbool}[3]{#3[#2\mapsto \keywd{#1}=\keywd{True}]}
\newcommand{\CS}{\keywd{cs}}
\newcommand{\pat}{\keywd{pat}}
\newcommand{\ptr}[2]{(\gramwd{ptr}\;\keywd{#1}\;\keywd{#2})}
\newcommand{\sind}{i}
\newcommand{\ind}{\keywd{i}}
\newcommand{\indj}{\keywd{j}}
\newcommand{\VAL}{\keywd{v}}
\newcommand{\ec}{\keywd{\mathcal{E}}}
\newcommand{\evalc}[1]{\ec \llbracket #1 \rrbracket }
\newcommand{\STOR}{\keywd{S}}
\newcommand{\stepsto}{\Rightarrow}
\newcommand{\Name}{\Red{Name }}
\newcommand{\q}[1]{\texttt{#1}}
\newcommand{\frl}[1]{{\ensuremath \mathit{frl}(#1)}}
\newcommand{\caseclause}[2]{#1 \;\rightarrow \;#2}
\newcommand{\case}[2]{\gramwd{case}\; #1 \;\gramwd{of}\;#2}
\newcommand{\spat}{\keywd{spat}}
\newcommand{\switch}[2]{\gramwd{switch}\; #1 \;\gramwd{of}\;#2}
\newcommand{\readInt}[1]{\gramwd{\text{readInt}} \; #1}
\newcommand{\writeInt}[2]{\gramwd{\text{writeInt}} \; #1 \; #2}
\newcommand{\readTag}[1]{\gramwd{\text{readTag}} \; #1}
\newcommand{\writeTag}[2]{\gramwd{\text{writeTag}} \; #1 \; #2}
\newcommand{\readCursor}[1]{\gramwd{\text{readCursor}} \; #1}
\newcommand{\writeCursor}[2]{\gramwd{\text{writeCursor}} \; #1 \; #2}
\newcommand{\datacon}[3]{\ensuremath{#1 \;#2 \;#3}}
\newcommand{\litcon}[2]{\datacon{\keywd{L}}{#1}{#2}}
\newcommand{\litnum}{\keywd{n}}
\newcommand{\CP}{\keywd{cp}}
\newcommand{\tptr}[3]{\langle \mathit{ptr}\;#2\;#3 \rangle_{#1}}
\newcommand{\LM}{\keywd{M}}
\newcommand{\locis}[2]{\ensuremath{#1:#2}}
\newcommand{\lvar}{\keywd{lx}}
\newcommand{\has}[3]{\ensuremath{#1(#2) = #3}}
\newcommand{\startr}[1]{(\gramwd{start}\; #1)}
\newcommand{\afterl}[1]{(\gramwd{after}\; #1)}
\newcommand{\tightoverset}[2]{%
  \mathop{#2}\limits^{\vbox to -.5ex{\kern-0.75ex\hbox{$#1$}\vss}}}
\newcommand\set[1]{\{ \, \ensuremath{#1} \,\}}
\newcommand{\ecdot}{\bullet}
\newcommand{\heap}{\keywd{h}}
\newcommand{\alphaequiv}{=_{\alpha}}

%% Environments
\newcommand{\LENV}{\keywd{L}}
\newcommand{\RENV}{\keywd{R}}
\newcommand{\CENV}{\keywd{C}}
\newcommand{\TENV}{\keywd{\Gamma}}
\newcommand{\EENV}{\keywd{E}}
\newcommand{\SENV}{\keywd{\Sigma}}
\newcommand{\MENV}{\keywd{M}}
\newcommand{\FENV}{\keywd{F}}
\newcommand{\AENV}{\keywd{A}}
\newcommand{\NENV}{\keywd{N}}

%% Loc helpers
\newcommand{\inloc}[0]{\ensuremath{\downarrow\sloc}}
\newcommand{\outloc}[0]{\ensuremath{\uparrow\sloc}}
\newcommand{\locreg}[2]{\ensuremath{{#1}^{#2}}}
\newcommand{\tyatlocreg}[3]{#1 \ensuremath{@} \locreg{#2}{#3}}
\newcommand{\Cursorize}[0]{Cursorize}
\newcommand{\fresh}[0]{\keywd{fresh}}

%% Meta functions

% Takes FROM,TO, i.e. \subst{e}{x}{v}.
\newcommand{\subst}[3]{\ensuremath{#1[#3/#2]}}

\newcommand{\typeofcon}{\keywd{TypeOfCon}}
\newcommand{\typeoffield}{\keywd{TypeOfField}}
\newcommand{\kargtys}{\keywd{ArgTysOfConstructor}}
\newcommand{\allocptr}[2]{\keywd{MaxIdx}(#1,#2)}

% Update(M,key,val), e.g. M[key>val]
%\newcommand{\update}[3]{\ensuremath{((#2 \mapsto #3)\; #1)}}
\newcommand{\update}[3]{\ensuremath{#1+\{#2 \mapsto #3\}}}

\newcommand{\ewitness}[4]{\ensuremath{#1;#2;#3 \vdash_{ew} #4}}
\newcommand{\storewf}[6]{\ensuremath{#1;#2;#3;#4 \vdash_{wf} #5;#6}}
\newcommand{\storewfcfa}[3]{\ensuremath{#1 \vdash_{wf_{cfc}} #2;#3}}
\newcommand{\storewfca}[4]{\ensuremath{#1;#2 \vdash_{wf_{ca}} #3;#4}}

\newcommand{\tcfun}{\ensuremath{\vdash_{fun}}}
\newcommand{\tcpat}{\ensuremath{\vdash_{pat}}}
\newcommand{\tcts}{\ensuremath{\vdash_{ts}}}

%% Special references for proofs.
\newcommand{\refwellformed}[2]{WF~\ref{#1};\ref{#2}}
\newcommand{\refendwitness}[2]{EW~\ref{#1};\ref{#2}}
\newcommand{\refts}[1]{T-#1}
\newcommand{\refcase}[1]{Case-\ref{#1}}
\newcommand{\elimexists}[0]{$\exists$ elim}
\newcommand{\refinst}[0]{Inst.}

\newcommand{\tfunctiondef}{T-Function-Definition}
\newcommand{\tdatacon}{T-DataConstructor}
\newcommand{\ddatacon}{D-DataConstructor}
\newcommand{\dletloctag}{D-LetLoc-Tag}
\newcommand{\dletlocafter}{D-LetLoc-After}
\newcommand{\dletlocstart}{D-LetLoc-Start}
\newcommand{\dapp}{D-App}
\newcommand{\dletregion}{D-LetRegion}
\newcommand{\tcase}{T-Case}
\newcommand{\dcase}{D-Case}
\newcommand{\tpat}{T-Pattern}
\newcommand{\tprogram}{T-Program}
\newcommand{\tllafter}{T-LetLoc-After}
\newcommand{\tlltag}{T-LetLoc-Tag}
\newcommand{\tllstart}{T-LetLoc-Start}
\newcommand{\tlregion}{T-LetRegion}
\newcommand{\tvar}{T-Var}
\newcommand{\tlet}{T-Let}
\newcommand{\dletexp}{D-Let-Expr}
\newcommand{\dletval}{D-Let-Val}
\newcommand{\tapp}{T-App}
\newcommand{\tconcreteloc}{T-Concrete-Loc}

\newcommand{\MPL}{\text{MaPLe}}

\newcommand{\vsmaple}[1]{$\inferrule{\MPL{}}{\text{Ours}}{#1}$}
\newcommand{\vsocaml}[1]{$\inferrule{\text{OCaml}}{\text{Ours}}{#1}$}
\newcommand{\vsghc}[1]{$\inferrule{\text{GHC}}{\text{Ours}}{#1}$}


\newcommand{\emptytenv}{\emptyset}

%% Formatting

%% fdtools, harpoon are packages that provide this feature, but
%% have their own set of problems. We do this by hand instead:
%% https://tex.stackexchange.com/questions/304622
%% \makeatletter
%% \newcommand*\MY@rightharpoonupfill@{%
%%   \arrowfill@\relbar\relbar\rightharpoonup
%% }
%% \newcommand*\overrightharpoon{%
%%   \mathpalette{\overarrow@\MY@rightharpoonupfill@}%
%% }
%% \makeatother
\newcommand{\overharpoon}[1]{\overrightharpoon{#1}}


\newcommand{\rtvar}{
  \inferrule*[lab={\;\text{[\tvar]}}]{\TENV(\var) = \tyatlocreg{\TYP}{\loc}{\reg} \\ \SENV(\locreg{\loc}{\reg})=\TYP}{\TENV;\SENV;\CENV;\AENV;\NENV \vdash \AENV; \NENV; \var : \tyatlocreg{\TYP}{\loc}{\reg}}
}

\newcommand{\rtconcreteloc}{
  \inferrule*[lab={\;\text{[\tconcreteloc]}}]{\SENV(\locreg{\loc}{\reg})=\TYP}{\TENV;\SENV;\CENV;\AENV;\NENV \vdash \AENV; \NENV; \concreteloc{r}{i}{l} : \tyatlocreg{\TYP}{\loc}{\reg}}
}

\newcommand{\rtlet}{
  \inferrule*[lab={\;\text{[\tlet]}}]{\TENV;\SENV;\CENV;\AENV;\NENV \vdash \AENV';\NENV';\EXPR_1 : \tyatlocreg{\TYP_1}{\loc_1}{\reg_1} \\\\
    \TENV';\SENV';\CENV;\AENV';\NENV' \vdash \AENV'';\NENV'';\EXPR_2 : \tyatlocreg{\TYP_2}{\loc_2}{\reg_2}}
                   {\TENV;\SENV;\CENV;\AENV;\NENV \vdash \AENV''; \NENV''; \letpack{\var : \tyatlocreg{\TYP_1}{\loc_1}{\reg_1}}{\EXPR_1}{\EXPR_2} : \tyatlocreg{\TYP_2}{\loc_2}{\reg_2} 
    \\\\{ \begin{aligned}
          \text{where} \;
          & \; \TENV' = \TENV \cup \set{\var \mapsto \tyatlocreg{\TYP_1}{\loc_1}{\reg_1}} ;
          \;   \SENV' = \SENV \cup \set{\locreg{\loc_1}{\reg_1} \mapsto \TYP_1}
          \end{aligned}
        }
   }
}

\newcommand{\rtlregion}{
  \inferrule*[lab={\;\text{[\tlregion]}}]{\TENV;\SENV;\CENV;\AENV';\NENV \vdash \AENV''; \NENV'; \EXPR : \hTYP}{\TENV;\SENV;\CENV;\AENV;\NENV \vdash \AENV''; \NENV'; \letreg{\sreg}{\sEXPR} : \hTYP
    \\\\{ \begin{aligned}
          \text{where} \;
          \AENV'= \AENV \cup \set{\reg \mapsto \emptyset}
          \end{aligned}
        }
    }
}

\newcommand{\rtllstart}{
  \inferrule*[lab={\;\text{[\tllstart]}}]{\AENV(\reg)=\emptyset \\ \locreg{\loc}{\reg} \not \in \NENV'' \\ \locreg{\loc'}{\reg'} \neq \locreg{\loc}{\reg} \\ \TENV;\SENV;\CENV';\AENV';\NENV' \vdash \AENV'';\NENV'';\EXPR : \tyatlocreg{\TYP'}{\loc'}{\reg'}}{\TENV;\SENV;\CENV;\AENV;\NENV \vdash \AENV''; \NENV''; \letloc{\locreg{\loc}{\reg}}{\startr{\reg}}{\EXPR} : \tyatlocreg{\TYP'}{\loc'}{\reg'}
    \\\\ {\begin{aligned}
            \text{where} \; 
            \; \CENV' = \CENV \cup \set{\locreg{\loc}{\reg} \mapsto \startr{r}} ;
            \; \AENV' = \AENV \cup \set{\reg \mapsto \locreg{\loc}{\reg}} ; \\
            \; \NENV' = \NENV \cup \set{\locreg{\loc}{\reg}}
          \end{aligned}}
    }
}

\newcommand{\rtlltag}{
  \inferrule*[lab={\;\text{[\tlltag]}}]{\AENV(\reg)=\locreg{\loc'}{\reg} \\ \locreg{\loc'}{\reg} \in \NENV \\ \locreg{\loc}{\reg} \not \in \NENV'' \\ \locreg{\loc}{\reg} \neq \locreg{\loc''}{\reg''} \\\\ \TENV;\SENV;\CENV';\AENV';\NENV' \vdash \AENV'';\NENV'';\EXPR : \tyatlocreg{\TYP''}{\loc''}{\reg''}}{\TENV;\SENV;\CENV;\AENV;\NENV \vdash \AENV''; \NENV''; \letloc{\locreg{\loc}{\reg}}{(\locreg{\loc'}{r} + 1)}{\EXPR} : \tyatlocreg{\TYP''}{\loc''}{\reg''}
  \\\\{\begin{aligned}
         \text{where} \;
         \; \CENV' = \CENV \cup \set{\locreg{\loc}{\reg} \mapsto (\locreg{\loc'}{\reg} + 1)} ;
         \; \AENV' = \AENV \cup \set{\reg \mapsto \locreg{\loc}{\reg}} ; \\
         \; \NENV' = \NENV \cup \set{\locreg{\loc}{\reg}}
       \end{aligned}
      }
  }
}


\newcommand{\rtllafter}{
  \inferrule*[lab={\;\text{[\tllafter]}}]{\AENV(\reg)=\locreg{\loc_1}{\reg} \\ \SENV(\locreg{\loc_1}{\reg}) = \TYP' \\ \locreg{\loc_1}{\reg} \not \in \NENV \\ \locreg{\loc}{\reg} \not \in \NENV'' \\ \locreg{\loc}{\reg} \neq \locreg{\loc'}{\reg'} \\\\ \TENV;\SENV;\CENV';\AENV';\NENV' \vdash \AENV'';\NENV'';\EXPR : \tyatlocreg{\TYP'}{\loc'}{\reg'}}{\TENV;\SENV;\CENV;\AENV;\NENV \vdash \AENV''; \NENV''; \letloc{\locreg{\loc}{\reg}}{\afterl{\tyatlocreg{\TYP'}{\loc_{1}}{\reg}}}{\EXPR} : \tyatlocreg{\TYP'}{\loc'}{\reg'}
  \\\\{\begin{aligned}
         \text{where} \;
         \; \CENV' = \CENV \cup \set{\locreg{\loc}{\reg} \mapsto \afterl{\tyatlocreg{\TYP'}{\loc_{1}}{\reg}}} ;
         \; \AENV' = \AENV \cup \set{\reg \mapsto \locreg{\loc}{\reg}} ; \\
         \; \NENV' = \NENV \cup \set{\locreg{\loc}{\reg}}
       \end{aligned}
      }
  }
}

\newcommand{\rtdatacon}{
   \inferrule*[lab={\;\text{[\tdatacon]}}]{\typeofcon(\DC)=\TYP \\ \typeoffield(\DC,\ind)=\overharpoon{\hTYP_{\ind}} \\\\ \locreg{\loc}{\reg} \in \NENV \\
 \AENV(\reg) = \overharpoon{\locreg{\loc_n}{\reg}} \quad \text{if}\;n \neq 0 \quad \text{else} \; \locreg{\loc}{\reg} \\\\
 \CENV(\overharpoon{\locreg{\loc_1}{\reg}}) = \locreg{\loc}{\reg} + 1 \\ \CENV(\overharpoon{\locreg{\loc_{\indj + 1}}{\reg}}) = \afterl{(\overharpoon{\TYP'_{\indj}} \ensuremath{@} \overharpoon{\locreg{\loc'_{\indj}}{\reg}})} \\\\
%   \tyatlocreg{\overharpoon{\TYP'_{\indj}}}{\overharpoon{\loc_{\indj}}}{\reg})} \\\\
 \TENV;\SENV;\CENV;\AENV;\NENV \vdash \AENV;\NENV;\overharpoon{\VAL_{\ind}} : \overharpoon{\tyatlocreg{\TYP_{\ind}'}{\loc_{\ind}}{\reg}}
 }{\TENV;\SENV;\CENV;\AENV;\NENV \vdash \AENV';\NENV'; \datacon{\DC}{\locreg{\loc}{\reg}}{\overharpoon{\VAL}} : \tyatlocreg{\TYP}{\loc}{\reg}
 \\\\{\begin{aligned}
        \text{where} \;
        & \; \AENV' = \AENV \cup \set{\reg \mapsto \locreg{\loc}{\reg}} ;
          \; \NENV' = \NENV - \set{\locreg{\loc}{\reg}} \\[-2mm]
        & \; \litnum = |\overharpoon{\VAL}| ;
          \; \ind \in \keywd{I} = \set{1, \; \ldots \; , \litnum} ;
          \; \indj \in \keywd{I} - \set{\litnum}
       \end{aligned}
     }
   }
}

\newcommand{\rtfunctiondef}{
  \inferrule*
   [lab={\;\text{[\tfunctiondef]}}]
   {\TENV;\SENV;\CENV;\AENV;\NENV \vdash \AENV;\NENV';
    \EXPR : \tyatlocreg{\TYP}{\loc}{\reg} \qquad \locreg{\loc}{\reg} \not \in \NENV'\\\\
    \forall_{i \in \set{1, \ldots, n}} . \exists_j . \overharpoon{\locreg{\loc_i}{\reg_i}} = \overharpoon{\locreg{\loc'_j}{\reg'_j}} \\ \exists_j . \locreg{\loc}{\reg} = \overharpoon{\locreg{\loc'_j}{\reg'_j}}
   }
   {\tcfun 
    \fvar : \forall _{\overharpoon{\locreg{l'}{r'}}}.
            \overharpoon{\tyatlocreg{\TYP}{\loc}{\reg}}
            \ARROW \tyatlocreg{\TYP}{\loc}{\reg};
    \fvar \overharpoon{\var} = \EXPR
    \\\\{\begin{aligned}
            \text{where} \;
            & \; \TENV = \set{\overharpoon{\var_1} \mapsto \overharpoon{{\tyatlocreg{\TYP_1}{\loc_1}{\reg_1}}}, \; \ldots \; , \overharpoon{\var_n} \mapsto \overharpoon{{\tyatlocreg{\TYP_n}{\loc_n}{\reg_n}}}}\\[-2mm]
            & \; \SENV = \set{\overharpoon{\locreg{\loc_1}{\reg_1}} \mapsto \overharpoon{\TYP_1}, \; \ldots \; , \overharpoon{\locreg{\loc_n}{\reg_n}} \mapsto \overharpoon{\TYP_n}} \\[-2mm]
            & \; \CENV = \emptyset; \; \AENV = \set{\reg \mapsto \locreg{\loc}{\reg}}; \; \NENV =  \set{\locreg{\loc}{\reg}}; \\[-2mm]
            & \; n = |\overharpoon{\var}| = |\overharpoon{\tyatlocreg{\TYP}{\loc}{\reg}}|\\[-2mm]
         \end{aligned}
        }
   }
}

\newcommand{\rtprogram}{
  \inferrule*
   [lab={\;\text{[\tprogram]}}]
   { \tcfun \overharpoon{\FD} \\
    \TENV;\SENV;\CENV;\AENV;\NENV \vdash \AENV';\NENV';\EXPR:\tyatlocreg{\TYP}{\loc}{\reg}}
   {\vdash_{prog} \AENV';\NENV';\overharpoon{\DD} \;; \overharpoon{\FD} \;; \EXPR : \tyatlocreg{\TYP}{\loc}{\reg} \\\\{\begin{aligned}
            \text{where} \;
            & \; \TENV = \emptyset ;
              \; \SENV = \emptyset \\[-2mm]
            & \; \CENV = \set{\locreg{\loc}{\reg} \mapsto \startr{\reg}} ;
              \; \AENV = \set{\reg \mapsto \locreg{\loc}{\reg}} ;
              \; \NENV = \set{\locreg{\loc}{\reg}}
          \end{aligned}
         }
   }
}

\newcommand{\rtapp}{
  \inferrule*[lab={\;\text{[\tapp]}}]{
    |\overharpoon{\locreg{\loc'}{\reg'}}| = |\overharpoon{\locreg{\loc''}{\reg''}}| \quad |\overharpoon{\VAL}| = |\overharpoon{\var}|\\\\
    \TENV;\SENV;\CENV;\AENV;\NENV \vdash \AENV;\NENV;\overharpoon{\VAL_i} : \overharpoon{\tyatlocreg{\TYP_i}{\loc_i}{\reg_i}} \\
    \locreg{\loc}{\reg} \in \NENV \\ \AENV(\reg) = \locreg{\loc}{\reg}\\\\
    \forall_i . \exists_j . \overharpoon{\locreg{\loc'''_i}{\reg'''_i}} = \overharpoon{\locreg{\loc''_j}{\reg''_j}} \wedge \overharpoon{\locreg{\loc_i}{\reg_i}} = \overharpoon{\locreg{\loc'_j}{\reg'_j}}  \quad \exists_j . \locreg{\loc'''}{\reg'''} = \overharpoon{\locreg{\loc''_j}{\reg''_j}} \wedge \locreg{\loc}{\reg} = \overharpoon{\locreg{\loc'_j}{\reg'_j}}
    }{\TENV;\SENV;\CENV;\AENV;\NENV \vdash \AENV;\NENV';
      \fapp{\overharpoon {\locreg{\loc'}{\reg'}}}{\overharpoon{\VAL} : \tyatlocreg{\TYP}{\loc}{\reg}}
     \\\\{\begin{aligned}
            \text{where} \;
            & \; \fvar : \forall _{\overharpoon{\locreg{\loc''}{r''}}}.
            \overharpoon{\tyatlocreg{\TYP_i}{\loc'''_i}{\reg'''_i}} \ARROW \tyatlocreg{\TYP}{\loc'''}{\reg'''} ; (\fvar \overharpoon{\var} = \EXPR) = Function(f)\\[-2mm]
%            & \; \overharpoon{\hTYP_{\fvar_{i}}} = \subst{\overharpoon{\TYP_{\ind}}}{\overharpoon{\locreg{\loc_{\fvar}}{\reg}}}{\overharpoon{\loc_{\reg}}} \\[-2mm]
%              \; \tyatlocreg{\TYP}{\loc}{\reg} = \subst{\hTYP_{\fvar}}{\overharpoon{\locreg{\loc_{\fvar}}{\reg}}}{\overharpoon{\loc_{\reg}}} \\[-2mm]
%            & \; \AENV' = \AENV \cup \set{\reg \mapsto \locreg{\loc}{\reg}} ;
             & \; \NENV' = \NENV - \set{\locreg{\loc}{\reg}}; 
              \; n = |{\overharpoon{\VAL}}| ;
              \; \ind \in \set{1, \; \ldots \;, n} 
          \end{aligned}
         }
     }
}

\newcommand{\rtcase}{
  \inferrule*[lab={\;\text{[\tcase]}}]{
     \TENV;\SENV;\CENV;\AENV;\NENV \vdash \AENV;\NENV; \VAL : \tyatlocreg{\TYP'}{\loc'}{\reg'} \\\\
     \TYP';\TENV;\SENV;\CENV;\AENV;\NENV \vdash_{\pat} \AENV';\NENV'; \overharpoon{pat_{\ind}} : \hTYP}
     {\TENV;\SENV;\CENV;\AENV;\NENV \vdash \AENV';\NENV';
      \case{\VAL}{\overharpoon{\pat} : \hTYP}
      \\\\{\begin{aligned}
             \text{where} \;
               \; \litnum = |\overharpoon{\pat}| ;
               \; \ind \in \set{1, \; \ldots \; , \litnum}
             \end{aligned}
           }
      }
}

\newcommand{\rtpat}{
  \inferrule*[lab={\;\text{[\tpat]}}]{
     \typeofcon(\DC) = \TYP'' \\ \kargtys(\DC) = \overharpoon{\TYP'} \\ \SENV(\locreg{\loc}{\reg}) = \TYP \\\\
     \locreg{\loc}{\reg} \neq \overharpoon{\locreg{\loc'_i}{\reg'}} \\ \TENV';\SENV';\CENV;\AENV;\NENV \vdash \AENV';\NENV';\EXPR : \tyatlocreg{\TYP}{\loc}{\reg}}
     {\TYP'';\TENV;\SENV;\CENV;\AENV;\NENV \tcpat \AENV';\NENV';
     \caseclause{\datacon{\DC}{}{(\overharpoon{\var : \tyatlocreg{\TYP'}{\loc'}{\reg'}})}}{\EXPR} : \tyatlocreg{\TYP}{\loc}{\reg}
     \\\\{\begin{aligned}
            \text{where} \;
            & \; \TENV' = \TENV \cup \set{\overharpoon{\var_1} \mapsto \overharpoon{\TYP'_1} \ensuremath{@} \overharpoon{{\loc_1'}^{\reg'}}, \; \ldots \; , \overharpoon{\var_n} \mapsto \overharpoon{\TYP'_n} \ensuremath{@} \overharpoon{{\loc_n'}^{\reg'}}}\\[-2mm]
            & \; \SENV' = \SENV \cup \set{\overharpoon{\locreg{\loc_1'}{\reg'}} \mapsto \overharpoon{\TYP'_1}, \; \ldots \; , \overharpoon{\locreg{\loc_n'}{\reg'}} \mapsto \overharpoon{\TYP'_n}} \\[-2mm]
            & \; \ind \in \set{1, \ldots , n} ; 
              \; \litnum = |\overharpoon{\TYP'}| = |\overharpoon{\var : \tyatlocreg{\TYP'}{\loc'}{\reg}}|
         \end{aligned}
        }
     }
}

\newcommand{\rddatacon}{
  \inferrule*[lab={\;\text{[\ddatacon]}}]{}{
    \STOR;\MENV;
    \datacon{\DC}{\locreg{\loc}{\reg}}{\overharpoon{\VAL}}
    \stepsto
    \STOR';\MENV;\concreteloc{\reg}{\ind}{\locreg{\loc}{\reg}}
    \\\\ {
          \begin{aligned}
            \text{where} \;
            \; \STOR' = \STOR \cup \set{\reg \mapsto (\ind \mapsto \DC)} ;
            \; \concreteloc{\reg}{\ind}{} = \MENV(\locreg{\loc}{\reg})
          \end{aligned}
    }
    }}

\newcommand{\rdletlocstart}{
  \inferrule*[lab={\;\text{[\dletlocstart]}}]
  {}{\STOR;\MENV;\letloc{\locreg{\loc}{\reg}}{\startr{\reg}}{\EXPR} \stepsto \STOR;\MENV';\EXPR
    \\\\{
         \begin{aligned}
           \text{where} \;
%           & \; \locreg{\loc_f}{\reg} \; \fresh \\[-2mm]
%             \; \EXPR' = \subst{\EXPR}{\locreg{\loc}{\reg}}{\locreg{\loc_f}{\reg}} \\[-2mm]
           & \; \MENV' = \MENV \cup \set{\locreg{\loc}{\reg} \mapsto \concreteloc{\reg}{0}{}}
         \end{aligned}
        }
  }
}

\newcommand{\rdletloctag}{
  \inferrule*[lab={\;\text{[\dletloctag]}}]
  {}{\STOR;\MENV;\letloc{\locreg{\loc}{\reg}}{\locreg{\loc'}{\reg} + 1}{\EXPR} \stepsto \STOR;\MENV';\EXPR
    \\\\ {
          \begin{aligned}
            \text{where} \;
%            & \; \locreg{\loc_f}{\reg} \; \fresh ;
%              \; \EXPR' = \subst{\EXPR}{\locreg{\loc}{\reg}}{\locreg{\loc_f}{\reg}} \\[-2mm]
            & \; \MENV' = \MENV \cup \set{\locreg{\loc}{\reg} \mapsto \concreteloc{\reg}{i+1}{}} ;
              \; \concreteloc{\reg}{\ind}{} = \MENV(\locreg{\loc'}{\reg})
           \end{aligned}
         }
}}

\newcommand{\rdletlocafter}{
  \inferrule*[lab={\;\text{[\dletlocafter]}}]
  {}{\STOR;\MENV;\letloc{\locreg{\loc}{\reg}}{\afterl{\tyatlocreg{\TYP}{\loc_1}{\reg}}}{\EXPR} \stepsto \STOR;\MENV';\EXPR
    \\\\ {
          \begin{aligned}
            \text{where} \;
%            & \; \locreg{\loc_f}{\reg} \; \fresh ;
%              \; \EXPR' = \subst{\EXPR}{\locreg{\loc}{\reg}}{\locreg{\loc_f}{\reg}} \\[-2mm]
            & \; \MENV' = \MENV \cup \set{\locreg{\loc}{\reg} \mapsto \concreteloc{\reg}{j}{}} ;
              \; \concreteloc{\reg}{\ind}{} = \MENV(\locreg{\loc_1}{\reg}) \\[-2mm]
            & \; \ewitness{\TYP}{\concreteloc{\reg}{\ind}{}}{\STOR}{\concreteloc{\reg}{\indj}{}}
           \end{aligned}
         }
  }
}

\newcommand{\rdcase}{
  \inferrule*[lab={\;\text{[\dcase]}}]
  {}{
    \STOR;\MENV;\case{\concreteloc{\reg}{\ind}{\locreg{\loc}{\reg}}}{[\ldots, \caseclause{\datacon{\DC}{}{(\overharpoon{\var : \tyatlocreg{\TYP}{\loc}{\reg}})}}{\EXPR}, \ldots]}
    \stepsto
    \\\\
    \STOR;\MENV';\subst{\EXPR}{\overharpoon{\var}}{\concreteloc{\reg}{\overharpoon{w}}{\overharpoon{\locreg{\loc}{\reg}}}}
    \\\\ {
         \begin{aligned}
           \text{where} \;
%           & \; \overharpoon{\locreg{\loc_f}{\reg'}} \; \fresh ;
%             \; \EXPR'' = \subst{\EXPR'}{\overharpoon{\locreg{\loc'}{\reg'}}}{\overharpoon{\locreg{\loc_f}{\reg'}}} \\[-2mm]
           & \; \MENV' = \MENV \cup \set{\overharpoon{\locreg{\loc}{\reg}_1} \mapsto \concreteloc{\reg}{\ind + 1}{}, \ldots , \overharpoon{\locreg{\loc}{\reg}_{j+1}} \mapsto \concreteloc{\reg}{\overharpoon{w_{j+1}}}{}} \\[-2mm]
%           & \; \EXPR' = \subst{\EXPR}{\overharpoon{\var}}{\concreteloc{\reg}{\overharpoon{w}}{\overharpoon{\locreg{\loc}{\reg}}}}\\[-2mm]
           & \; \ewitness{\overharpoon{\TYP_1}}{\concreteloc{\reg}{\ind + 1}{}}{\STOR}{\concreteloc{\reg}{\overharpoon{w_1}}{}} \\[-1mm]
           & \; \ewitness{\overharpoon{\TYP_{j+1}}}{\concreteloc{\reg}{\overharpoon{w_j}}{}}{\STOR}{\concreteloc{\reg}{\overharpoon{w_{j+1}}}{}} \\[-1mm]
           & \; \DC = \STOR(\reg)(\ind); \; \indj \in \set{1 , \ldots , \litnum - 1} ;
             \; \litnum = | \overharpoon{\var : \hTYP} | \\
%           & \; \DC = \STOR(\reg')(\ind) \\[-2mm]
         \end{aligned}
       }
  }
}

\newcommand{\rdletexp}{
  \inferrule*[lab={\;\text{[\dletexp]}}]
              {\STOR;\MENV;\EXPR_1 \stepsto \STOR';\MENV';\EXPR'_1 \\ \EXPR_1 \neq \VAL}
              {\STOR;\MENV;\letpack{\var : \hTYP}{\EXPR_1}{\EXPR_2} \stepsto
               \STOR';\MENV';\letpack{\var : \hTYP}{\EXPR'_1}{\EXPR_2}}
}

\newcommand{\rdletval}{
  \inferrule*[lab={\;\text{[\dletval]}}]
    {}{\STOR;\MENV;\letpack{\var : \hTYP}{\VAL_1}{\EXPR_2} \stepsto \STOR;\MENV;\subst{\EXPR_2}{\var}{\VAL_1}} 
}

\newcommand{\rdapp}{
  \inferrule*[lab={\;\text{[\dapp]}}]
    {}{\STOR;
      \MENV;
      \fapp{\overharpoon{\locreg{\loc}{\reg}}}{\overharpoon{\VAL}}
      \stepsto \STOR;
      \MENV;
      \subst{\EXPR}{\overharpoon{\var}}{\overharpoon{\VAL}} \subst{}{\overharpoon{\locreg{\loc'}{\reg'}}}{\overharpoon{\locreg{\loc}{\reg}}}
      \\\\{
        \begin{aligned}
          \text{where} \;
%          & \; \EXPR' =  \\[-2mm]
          & \; \FD = Function(f) \\[-2mm]
          & \; \fvar : \forall _{\overharpoon{\locreg{\loc'}{\reg'}}}.
          \overharpoon{\hTYP_{\fvar}} \ARROW \hTYP_{\fvar} ; (\fvar \overharpoon{\var} = \EXPR) = Freshen(\FD)
        \end{aligned}
      }
    }
}

\newcommand{\rdletregion}{
  \inferrule*[lab={\;\text{[\dletregion]}}]
    {}{\STOR;\MENV;\letreg{\reg}{\EXPR} \stepsto \STOR;\MENV;\EXPR}
}


%\usepackage[usenames,dvipsnames]{xcolor}
%\usepackage[dvipsnames]{xcolor}
%\usepackage[]{xcolor}

% [2016.10.23] Switching to margin notes for peanut gallery:
\usepackage{todonotes}
\presetkeys{todonotes}{inline}{}  % Activate to turn-off margin notes.

% [2016.05.26] This can just be a growing list from paper to paper:
\newcommand{\rn}[1]{\pgwrapper{RRN}{#1}}  % Ryan Newton

\newcommand{\tlm}[1]{\pgwrapper{TLM}{#1}} % Trevor
\newcommand{\bjs}[1]{\pgwrapper{BJS}{#1}} % Bo Joel Svenson
\newcommand{\chc}[1]{\pgwrapper{CHC}{#1}} % Chao-Hong Chen
\newcommand{\vc}[1]{\pgwrapper{VC}{#1}} % Vikraman Choudhury
\newcommand{\bc}[1]{\pgwrapper{BC}{#1}} % Buddhika Chamith
\newcommand{\mv}[1]{\pgwrapper{MV}{#1}} % Michael Vollmer
\newcommand{\mav}[1]{\pgwrapper{MV}{#1}} % Michael Vollmer
\newcommand{\osa}[1]{\pgwrapper{OSA}{#1}} % Omer Agacan
\newcommand{\mm}[1]{\pgwrapper{MM}{#1}} % Madan Musuvathi
\newcommand{\rgs}[1]{\pgwrapper{RGS}{#1}} % Ryan Scott
\newcommand{\csk}[1]{\pgwrapper{CSK}{#1}} % Chaitanya Koparkar

\newcommand{\mk}[1]{\pgwrapper{MK}{#1}} % Milind Kulkarni
\newcommand{\ls}[1]{\pgwrapper{LS}{#1}} % Laith Sakka

\newcommand{\spall}[1]{\pgwrapper{SS}{#1}} % Sarah Spall

\newcommand{\jd}[1]{\pgwrapper{JD}{#1}} % Joe Devietti
\newcommand{\onl}[1]{\pgwrapper{ONL}{#1}} % Omar Navarro-Leija

\newcommand{\mr}[1]{\pgwrapper{MR}{#1}} % Mike Rainey

% Add co-authors here or in a downstream file.

\InputIfFileExists{activateeditingmarks}{
}{
    \def\noeditingmarks{}
}

\definecolor{comment-red}{rgb}{0.8,0,0}
\definecolor{dark-green}{rgb}{0.0,0.4,0}
\definecolor{dark-blue}{rgb}{0.0,0.0,0.55}
\definecolor{very-dark-green}{rgb}{0.0,0.3,0}

\ifx\noeditingmarks\undefined
   %% This means someone else should take a look:
   \newcommand{\auditme}[1]{{\color{dark-green}{#1}}}

   %% This meants there's a serious problem that must be fixed before
   %% the paper goes out:
   \newcommand{\fixme}[1]{{\color{comment-red}{#1}}}

   %% This is for outlining, i.e., psuedotext that must be converted
   %% to real text:
   \newcommand{\note}[1]{{\begin{itemize}\item \textcolor{blue}{NOTE: #1} \end{itemize}}}

   %% A very poor-man's version of ``track changes'':
   \newcommand{\new}[1]{{\color{blue}{#1}}}

   \newcommand{\const}[1]{\textred{#1}}
   %\definecolor{mygrey}{rgb}{0.6,0.6,0.6}
   \definecolor{mygrey}{rgb}{0.7,0.7,0.7}

   \newcommand{\textred}[1]{{\color{comment-red}{#1}}}

   % Old version was inline:   
   % \newcommand{\pgwrapper}[2]{\textred{(#1: #2)}}
   % New version is in the margin:
   \setlength{\marginparwidth}{1.6cm}
   \newcommand{\pgwrapper}[2]{\todo[size=\scriptsize]{#1: #2}}

   %% RRN: Dim the background for Ryan's eyes:
   %% Hack: do this ONLY on Ryan's machine.
   %%  I don't know a way based on user or host name, so...
   %%  instead I just signal on a file, activategreybg.tex which I
   %%  don't then checkin.
   \InputIfFileExists{activategreybg}{
       \definecolor{comment-red}{rgb}{0.5,0,0}
       \pagecolor{mygrey}
   }{}

  % When they are their own para, todonotes annoyingly take up vertical space:
  \newcommand{\rnfloat}[1]{\pgwrapper{RRN}{#1}\vspace{-4mm}}  % Ryan Newton

\else
%   \newcommand{\textred}[1]{#1}
% Keep it on so we catch stuff to fix:
   \newcommand{\textred}[1]{{\color{comment-red}{#1}}}
   \newcommand{\auditme}[1]{#1}
   % This is more severe and should not go away:
   \newcommand{\fixme}[1]{\textred{#1}}
   \newcommand{\pgwrapper}[2]{}
%   \newcommand{\note}[1]{}
   \newcommand{\note}[1]{{\begin{itemize}\item {\textcolor{blue}{#1}} \end{itemize}}}
   % \newcommand{\new}[1]{#1}
   \newcommand{\new}[1]{{\color{blue}{#1}}}
   \newcommand{\const}[1]{#1}

   \newcommand{\rnfloat}[1]{}  % Ryan Newton
\fi

% Just an alias because sometimes we use this:
\newcommand{\Red}[1]{\textred{#1}}

% These probable should be elsewhere:
\newcommand{\ie}{{i.e.,}}
\newcommand{\eg}{{e.g.,}}
\newcommand{\etal}{\textit{et al.}}

\def\scrunchit{}
\ifx\scrunchit\undefined
  \newcommand{\captionscrunch}{}
\else
%  \newcommand{\captionscrunch}{\vspace{-4mm}}
  \newcommand{\captionscrunch}{\vspace{-3mm}}
\fi

%% \newcommand{\secref}[1]{section~\ref{#1}}
%% \newcommand{\Secref}[1]{Section~\ref{#1}}
\newcommand{\secref}[1]{\S\ref{#1}}
\newcommand{\Secref}[1]{\secref{#1}}

%% \newcommand{\Figref}[1]{Figure~\ref{#1}}
%% \newcommand{\figref}[1]{figure~\ref{#1}}
\newcommand{\Figref}[1]{Fig.~\ref{#1}}
\newcommand{\figref}[1]{Fig.~\ref{#1}}
\newcommand{\tabref}[1]{Table~\ref{#1}}


\newcommand{\appendixref}[1]{\ref{#1}}
\newcommand{\Appendixref}[1]{\ref{#1}}



\usepackage{upquote}
\usepackage{listings}
% This is used in the acmart.cls, leading to problems with option clash:
% \usepackage[usenames,dvipsnames]{xcolor}
% \usepackage[dvipsnames]{xcolor}
\usepackage[]{xcolor}
% [2017.01.16] Work around option clash:
\definecolor{MidnightBlue}{rgb}{0.1, 0.1, 0.44}

\definecolor{darkgreen}{rgb}{0,0.5,0}
\definecolor{darkred}{rgb}{0.5,0,0}

% Define the language styles we will use
%
\lstset{%
    frame=none,
    rulecolor={\color[gray]{0.7}},
    numbers=none,
    basicstyle=\footnotesize\ttfamily,
%    basicstyle=\scriptsize\ttfamily,
    numberstyle=\color{Gray}\tiny\it,
    commentstyle=\color{MidnightBlue}\it,
    stringstyle=\color{Maroon},
    keywordstyle=[1],
    keywordstyle=[2]\color{ForestGreen},
    keywordstyle=[3]\color{Bittersweet},
    keywordstyle=[4]\color{RoyalPurple},
    captionpos=b,
    aboveskip=1\medskipamount,
    xleftmargin=0.5\parindent,
    xrightmargin=0.5\parindent,
    flexiblecolumns=false,
%   basewidth={0.5em,0.45em},           % default {0.6,0.45}
%    escapechar={\%},
    escapechar={\@},
    mathescape=true,
    texcl=true,                          % tex comment lines
    showstringspaces=false
}

\lstloadlanguages{Haskell}
\lstdefinestyle{haskell}{%
    language=Haskell,
    upquote=true,
    deletekeywords={case,class,data,default,deriving,do,in,instance,let,of,type,where,IO,ST,STM},
    morekeywords={[2]class,data,default,deriving,family,instance,type,where,if,then,else},
    morekeywords={[3]in,let,case,of,do,letpacked,letloc,letregion},
    morekeywords={[4]IORef,IO,ST,STM,STRef,TVar},
    literate=
        {\\}{{$\lambda$}}1
        {Lambda}{{$\Lambda$}}1
        {\\\\}{{\char`\\\char`\\}}1
        {>->}{>->}3
        {>>=}{>>=}3
        {->}{{$\rightarrow$}}2
        {->.}{{$\multimap$}}2
        {>=}{{$\geq$}}2
        {<-}{{$\leftarrow$}}2
        {<=}{{$\leq$}}2
        {=>}{{$\Rightarrow$}}2
        {|}{{$\mid$}}1
        {forall}{{$\forall$}}1
        {exists}{{$\exists$}}1
        {...}{{$\cdots$}}3
%       {`}{{\`{}}}1
%       {\ .}{{$\circ$}}2
%       {\ .\ }{{$\circ$}}2
%
%    deletekeywords={insert},
%    deletekeywords={map,sort,zipWith,replicate,Num,Char,Bool,Array,Int,Double
%                   ,sqrt,not,filter,IO,Maybe,Either,quot,scanl,scanr,reverse,fst,id},
%    literate=
%        {+}{{$+$}}1
%        {/}{{$/$}}1
%        {*}{{$*$}}1
%        % {=}{{$=$}}1
%        {>}{{$>$}}1 {<}{{$<$}}1
%        {\\}{{$\lambda$}}1
%        {\\\\}{{\char`\\\char`\\}}1
%        {->}{{$\rightarrow\;$}}2
%        {>=}{{$\geq$}}2
%        {<-}{{$\leftarrow\;$}}2
%        {<=}{{$\leq$}}2
%        {=>}{{$\Rightarrow\;$}}2
%        {\ .}{{$\circ$}}2
%        {\ .\ }{{$\circ$}}2
%        {>>}{{>>}}2
%        {>>=}{{>>=}}2
%        {=<<}{{=<<}}2
%        {|}{{$\mid$}}1
%        {dotdotdot}{{$\ldots$}}3
}

\lstdefinestyle{inline}{%
    style=haskell,
    basicstyle=\footnotesize\ttfamily,
    keywordstyle=[1],
    keywordstyle=[2],
    keywordstyle=[3],
    keywordstyle=[4],
    escapechar={\@},
    mathescape=true,
    literate=
        {\\}{{$\lambda$}}1
        {Lambda}{{$\Lambda$}}1        
        {\\\\}{{\char`\\\char`\\}}1
        {>->}{>->}3
        {>>=}{>>=}3
        {->}{{$\rightarrow$\space}}3    % include forced space
        {->.}{{$\multimap$\space}}3
        {>=}{{$\geq$}}2
        {<-}{{$\leftarrow$}}2
        {<=}{{$\leq$}}2
        {=>}{{$\Rightarrow$}}2
        {|}{{$\mid$}}1
%        {~}{{$\sim$}}1
        {forall}{{$\forall$}}1
        {exists}{{$\exists$}}1
        {...}{{$\cdots$}}3
}

\lstnewenvironment{code}
    {\lstset{style=haskell}%
      \csname lst@SetFirstLabel\endcsname}
    {\csname lst@SaveFirstLabel\endcsname}
    {}

% Default all listings to Haskell style
\lstset{style=haskell}

% \newcommand{\inl}[1]{\lstinline[style=inline];#1;}
\newcommand{\il}[1]{\lstinline[style=inline,mathescape=true];#1;}

\newcommand{\makeatcode}{\lstMakeShortInline[style=inline]@}
\newcommand{\makeatchar}{\lstDeleteShortInline@}


\begin{document}
\maketitle
\acceptancepage

% This page is optional
%\copyrightpage


% This page is optional but generally included

\begin{acknowledgments}
This is the (optional) acknowledgments page, which is designed to recognize people or agencies to whom you feel grateful for any academic, technical, financial, or personal aid in the preparation of your thesis. As a matter of courtesy, you should ordinarily mention the members of your committee here, as well as institutions that provided funding.
\end{acknowledgments}

% This page is optional

\begin{dedication}
This is the (optional) dedication page. Per Graduate School standards, this page should appear with no title and should be centered horizontally and vertically.
\end{dedication}

% This page is optional

\begin{preface}
This is the (optional) preface page which can be used if you wish. This page should appear after the dedication (or acknowledgements page if there is no dedication page) and before the abstract page.
\end{preface}

% % This page is required

\begin{abstract}
This abstract page is required. Make sure to adhere to the word count and other limits set by ProQuest. It should appear in your dissertation that is submitted to the Graduate School \emph{without} signatures.
\end{abstract}

\newpage

% This page is required

\tableofcontents

\chapter{Introduction}

Words. Words.

\chapter{The Location Calculus}

\section{Overview}

Traditional compilers are built on a number of well-defined
intermediate abstractions and translations that close the semantic gap
between source and target.  What we {need are analogous way-points to
  structure compilers that target serialized-data traversals}
(stream-processors, essentially).  Indeed, there is quite a semantic
gap between the low-level, buffer-mutating, pointer-bumping programs,
and a source language of high-level, pure, recursive functions on
algebraic datatypes.  To structure the space between, we propose a
location calculus, \ourcalc, augmented to track {\em locations} within
regions (\eg{} byte offsets).

\ourcalc follows in the tradition of typed assembly language~\cite{TAL},
region calculi~\cite{mlkit-retrospective}, and Cyclone~\cite{cyclone-pldi}
in that it uses types to both
expose and make safe low-level implementation mechanisms.
%
%% Ultimately, our compiler output will increment pointers into memory buffers, as
%% seen in code of the previous section.
%
The basic idea of \ourcalc{} is to first establish what data \emph{share} which
logical memory regions (essentially, buffers), and in what \emph{order} those data reside,
abstracting the details of computing exact addresses.
%
For example, data constructor applications, such as \lstinline{Leaf 3}, take an extra
location argument in \ourcalc{}, specifying where the data constructor
should place the resulting value in memory:
% starts in memory, that is, where the data
% constructor itself is written:
\lstinline[mathescape]{Leaf $\;\locreg{l}{}\;$ 3}.
%
This location becomes part of type of the value: \lstinline[mathescape]{$\,\tyatlocreg{Tree}{\,l}{}$}.
Every location resides in a region, and when we want to 
name that region, we write $\locreg{l}{r}$.


Locations represent information about where values are in a store, but
are less flexible than pointers. They are introduced {relative} to other
locations. A location variable is either \emph{after} another
variable, or it is at the beginning of a region,
thus specifying a serial order.
%so they essentially describe what \emph{order} values are in within a region.
If location $l_2$ is declared as
\lstinline[mathescape]{$l_2$ = after($\tyatlocreg{Tree}{l_1}{\reg}$)},
%% where $x$ is a tree at $l_1$,
then $l_2$ is \emph{after} every element of the tree
rooted at $l_1$.

{\emph{Regions} in \ourcalc{} represent the memory buffers containing serialized data
  structures.  Unlike some other region calculi, in \ourcalc{}, values in a
  region \emph{may} escape the static scope which binds and allocates that region. In
  fact, an extension introduced later in~\secref{sec:indirections}
  specifically relies on inter-region pointers and coarse-grained garbage
  collection of regions.}

%% In this paper, we present a new method for building compilers to lift functions
%% over serialized data.
%% We also present a new calculus, \ourcalc{}, which corresponds to a simplified
%% presentation of the core intermediate language of our compiler.
%% %
%% It was designed both to form a foundation for the optimizations we describe in
%% this paper, generalizing previous work on region calculi to represent recursive
%% programs operating on serialized data. Crucially, in addition to associating
%% values with \emph{regions}, in \ourcalc{} values are associated with a

% As shown in the grammar in \figref{fig:grammar},
\ourcalc{} is a first-order,
call-by-value functional language with algebraic datatypes and pattern
matching. Programs consist of a series of datatype definitions, function
definitions, and a main expression.
%
%% The algebraic data types work essentially like one would expect, but with
%% additional machinery to handle \emph{where} a particular value is in the store
%% and the \emph{relative position} of its fields in the store.  Every invocation
%% of a data constructor is annotated with a location variable ---
%
\ourcalc{} programs can be written directly by hand, and \ourcalc{} also serves
as a practical {\em intermediate language} for other tools or front-ends that
want to convert computations to run on serialized data (essentially fusing a
consuming recursion with the deserialization loop).
%
We return to this use-case in \secref{sec:impl-hical}.

\paragraph{Allocating to output regions}
%
% \note{buildtree-shared?}
Now that we have seen how data constructor applications are parameterized by
locations, let us look at a more complex example than those of the prior section.
Consider \lstinline[mathescape]{buildtree}, which constructs the same trees consumed by \lstinline[mathescape]{sum}
and \lstinline[mathescape]{rightmost} above.  First, in the source language without locations:
\rn{Some redundancy here w sec 3.1.. if needed could throw outsome of it and move this
  example down there, or remove it entirely.}

%% \floatstyle{plain}\restylefloat{figure}
%% \begin{figure}
%% \centering
\begin{code}
buildtree : Int -> Tree
buildtree n = if n == 0
              then Leaf 1
              else Node (buildtree (n - 1))
                        (buildtree (n - 1))
\end{code}
%% \vspace{-1mm}
%% \caption{Example: Building an output data structure}
%% \label{fig:buildtree}
%% \vspace{-5mm}  
% \end{figure}
% \note{TODO: vertically align the stmts}
%
Then in {\ourcalc}, where the type scheme binds an output rather than input location:
% \floatstyle{plain}\restylefloat{figure}
%\vspace{-1mm}
%\floatstyle{boxed}\restylefloat{figure}
%\begin{figure}[h]
%\centering
\begin{code}
buildtree : forall @\locreg{l}{r}@ . Int -> @\tyatlocreg{Tree}{l}{r}@
buildtree [@\locreg{l}{r}@] n =
  if n == 0 then (Leaf @\locreg{l}{r}@ 1) -- write tag + int to output 
  else -- skip past tag:
       letloc $\locreg{l_a}{r}$ = $\locreg{l}{r}$ + 1 in
       -- build left in place:
       let left : @\tyatlocreg{Tree}{l_a}{\reg}@ =
           buildtree [@\locreg{l_a}{r}@] (n - 1) in
       -- find start of right:
       letloc $\locreg{l_b}{r}$ = after(@\tyatlocreg{Tree}{l_a}{\reg}@) in
       -- build right in place:
       let right : @\tyatlocreg{Tree}{l_b}{\reg}@ =
           buildtree [@\locreg{l_b}{r}@] (n - 1) in
       -- write datacon tag, connecting things together:
       (Node @\locreg{l}{r}@ left right)
  \end{code}
%\end{figure}
%\vspace{-3mm}

\noindent
Here, we see that \ourcalc{} must represent locations that have {\em not yet
  been written}, \ie{} they are output destinations.  Nevertheless, in the
recursive calls of \lstinline[mathescape]{buildtree} this location is passed as an argument: a form
of destination-passing style~\cite{destination-passing}.
The type system guarantees that memory will be initialized and written exactly once.
%
The output location is threaded through the recursion to build the left subtree,
and then offset to compute the starting location of the right subtree.
%
%% Computing \lstinline[mathescape]{after($\tyatlocreg{Tree}{l_a}{\reg}$)} is a potentially expensive operation, and
%% implementing it---or rearranging the program to avoid needing it---is the
%% responsibility of the \ourcalc implementation (\secref{sec:impl-local}).
{It might appear that computing
\lstinline[mathescape]{after($\tyatlocreg{Tree}{l_a}{\reg}$)} could be quite expensive,
if there is a large tree at that location. This does not need to be the case.
In~\secref{sec:impl-local} we will present different techniques for efficiently
compiling \ourcalc programs without requiring linear walks through serialized data.}

%% Later
%% (in~\secref{sec:route-ends}), we describe a compilation technique that
%% makes obtaining the \lstinline[mathescape]{after} of a location in cases like this
%% straightforward by having computations return the ending locations of
%% whatever trees they traverse.
%
%% Furthermore, while the above \lstinline[mathescape]{buildTree} code writes the left and right subtrees {\em
%%   before} writing the data constructor tag into the output, we will want the
%% compiler to eventually reorder these writes to achieve a linear memory access
%% pattern on the output region.

%% \rn{May want to modify the discussion of ``size information'' below.}
{One of the goals of \ourcalc{} is to support several compilation
  strategies. One extreme is compiling programs to work with a representation of
  data structures that do not include \emph{any} pointers or indirections at
  run-time---within such a representation, the size of a value can be observed
  by threading through ``end witnesses'' while consuming packed values: for
  example, \lstinline[mathescape]{buildtree} above would \emph{return} $\locreg{l_b}{r}$, rather than computing
  it with an \lstinline[mathescape]{after} operation.
  %
  (The end-witness strategy was first used in
  {\em Gibbon}~\cite{ecoop17-gibbon} prior to the design of \ourcalc{},
  which previously compiled functions on fully serialized data,
  while not preserving asymptotic complexity.)
  %% %% Our language allows for this by ensuring that
  %% %% when location variables are introduced after values, a variable binding the
  %% %% preceeding value must be in scope.
  %% %
  %% In \secref{sec:compiler}, we will discuss in detail how the invariants
  %% enforced by \ourcalc{} make subsequent transformations and optimizations
  %% more tractable.
}
%
%% In practice, as a whole-program compiler, our compiler will decide
%% whether or not particular data constructor needs {\em random-access}
%% based on how it is used in the program. The analyses required for this
%% are described in more detail in \secref{sec:compiler}.
%
%% \note{Forward reference whatever section where we add back limited random-access
%%   capabilities.}
%
%% Furthermore, while our compiler does translate every program into
%% \ourcalc{}, later stages in the compiler will relax or violate the
%% invariants enforced by \ourcalc{}, such as the absolute ordering of
%% values in a region.
%% %
%% In fact, by code generation time the compiler will
%% have likely transformed the program to include explicit size
%% information and inter-region pointers.
%
Next, we will present a formalized core subset of \ourcalc, 
%% in more detail
%% (\secref{subsec:grammar}),
its type system (\secref{subsec:static}),
and operational semantics (\secref{subsec:dynamic}),
% which uses a store to  represent regions and locations.
before moving on to implementation (\secref{sec:impl-local}, \secref{sec:impl-hical})
and evaluation (\secref{sec:eval}).

\section{Formal Language and Grammar}
\label{subsec:grammar}

\begin{figure}
  \begin{displaymath}
  \begin{aligned}
  &\DC \in \; \textup{Data Constructors},\:\: \TYP \in \; \textup{Type Constructors},\\
  &\var, \yvar,\fvar \in \; \textup{Variables},\:\:
  \loc, \locreg{l}{r} \in \; \textup{Symbolic Locations},\\
  &\reg \in \; \textup{Regions},\:\: \ind, \indj \in \; \textup{Region Indices},\\
  &\concreteloc{r}{i}{l} \in \; \textup{Concrete Locations}
  \end{aligned}
\end{displaymath}
\begin{displaymath}
  \begin{aligned}
    \textup{Top-Level Programs} && \PROG && \gramdef & \overharpoon{\DD} \;; \overharpoon{\FD} \;; \EXPR \\
    \textup{Datatype Declarations} && \DD && \gramdef & \DATA\;\TYP = \overharpoon{\DC \; \overharpoon{\sTYP}\;} \\
    \textup{Function Declarations} && \FD && \gramdef & \fvar : \TS ; \fvar \overharpoon{\var} = \EXPR \\
    %\textup{Data Location} && \RP && \gramdef & \loc^{\reg} \\
%    \textup{Types} && \TYP && \gramdef & \ldots \gramor \TC \gramor \ldots \\
    %\gramor \ldots \\ % maybe don't even talk about int, is it confusing to have non-packed data?
    \textup{Located Types} && \hTYP && \gramdef & \tyatlocreg{\TYP}{\loc}{\reg} \\
    \textup{Type Scheme} && \TS && \gramdef &
      \forall _{\overharpoon{\locreg{l}{r}}}.
      \overharpoon{\hTYP} \ARROW \hTYP \\
    \textup{Values} && \VAL && \gramdef & \var \gramor \concreteloc{\reg}{\ind}{\locreg{\loc}{\reg}}\\
    \textup{Expressions} && \EXPR && \gramdef & \VAL\\[-5pt]
    && && \gramor & \fapp{\overharpoon{\locreg{l}{r}}}{\overharpoon{\VAL}} \\
    && && \gramor & \datacon{\DC}{\keywd{\locreg{\loc}{\reg}}}{\overharpoon{\VAL}}\\
    %% && && \gramor & \datacon{\DC}{\keywd{\locreg{\loc}{\reg}}}{\overline{\var}} \gramor \litcon{\locreg{\loc}{\reg}}{\EXPR} \\
    && && \gramor & \letpack{\var:\hTYP}{\EXPR}{\EXPR} \\
    && && \gramor & \letloc{\locreg{\loc}{\reg}}{\LE}{\EXPR} \\
    && && \gramor & \letreg{\reg}{\EXPR} \\
    && && \gramor & \case{\VAL}{\overharpoon{\pat}} \\
    \textup{Pattern} && \pat && \gramdef &
    \caseclause{\datacon{\DC}{}{(\overharpoon{\var : \hTYP})}}{\EXPR} \\
    \textup{Location Expressions} && \LE && \gramdef
    & \startr{\reg} \\
    && && \gramor & (\locreg{\loc}{r} + 1) \\
    && && \gramor & \afterl{\hTYP}
  \end{aligned}
\end{displaymath}

  \caption{Grammar of \ourcalc{}}
  \label{fig:grammar}  
\end{figure}

\Figref{fig:grammar} gives the grammar for a formalized core of \ourcalc{}.
%
We use the notation $\overharpoon{x}$ to denote a vector $[
x_1, \ldots, x_n]$, and $\overharpoon{x_{\ind}}$ the item at position
$\ind$.
%
To simplify presentation, the language supports 
algebraic datatypes without any base primitive types, but could be extended in a straightforward
manner to represent primitives such as an $\sgramwd{Int}$ type or tuples.
%
The expression language is based on the first-order lambda calculus,
using A-normal form.
%
The use of A-normal form simplifies our formalism and proofs
without loss of generality.
%in ways that do not limit generality.
%It would be straightforward to generalize to direct style.

Like previous work on region-based memory~\cite{regioncalcs},
\ourcalc{} has a special binding
form for introducing region variables, written as
$\sgramwd{letregion}$.
%
Location variables are similarly introduced by $\sgramwd{letloc}$.
%
The pattern-matching form $\sgramwd{case}$ binds variables to
serialized values, as well as binding the location for each variable.
%
% To simplify the formalism,
We require that each bound location in
a source program is unique.
%
% This convention rules out programs with space leaks caused by
%shadowing of \gramwd{letloc}-bound names.
% \rn{I can't easily construct hypothetical example...}

The $\sgramwd{letloc}$ expression binds locations in only three ways:
a location is either the \emph{start} of a region (meaning, the
location corresponds to the very beginning of that region), is
immediately after another location, or it occurs \emph{after} the last
position occupied by some previously allocated data constructor.
%
For the last case, the location is written to exist at
$\afterl{\tyatlocreg{\TYP}{\loc}{\reg}}$, where $\loc$ is already
bound in a region, and has a value written to it.

%% \begin{comment}
%% Second, the convention enables the dynamic semantics to treat symbolic
%% locations at runtime as a source of dynamically fresh names (i.e., a
%% gensym-like mechanism), which enables a simpler environment structure
%% for tracking locations.
%% %
%% It is straightforward to relax these requirements.
%% \end{comment}

Values in \ourcalc{} are either (non-location) variables or
\emph{concrete locations}.
%
In contrast to bound location variables, concrete locations
do not occur in source programs; rather, they appear at runtime, created by the
application of a data constructor, which has the effect of
extending the store.
%
Every application of a data constructor writes a \emph{tag} to the store, and
concrete locations allow the program to navigate through it.
%
To distinguish between concrete locations and location variables in
the formalism, we refer to the latter as \emph{symbolic locations}.
%
A concrete location is a tuple
$\concreteloc{\reg}{\ind}{\loc}$ consisting of a region, an index, and
symbolic location corresponding to its binding site.
%
The first two components are sufficient to fully describe
an \emph{address} in the store.

%% \begin{comment}
%% Rather than annotating expressions with their region (such as
%% $\sEXPR\; @ \;\sreg$) like standard region calculi, \emph{location
%% annotations} appear at various points in the program source, such as
%% in types, in function declarations, and at the site of constructor
%% application.
%% %
%% %% As a first-order language, functions are only defined at the top
%% %% level. Each function has a name, a \emph{type scheme}, an argument,
%% %% and an expression (the body of the function).  The type scheme
%% %% consists of the types of the input and output of the function,
%% %% parameterized by \emph{data locations} (location and region
%% %% pair).
%% %
%% %% {Function application similarly requires a list of data
%% %% locations.}
%% %
%% Function types are only introduced at the top level, and these
%% functions may only be polymorphic with respect to data locations.
%% \end{comment}
%
%% \mv{Explain that locations of inputs to functions must have been written,
%% and locations of outputs to functions cannot have been written.}
%
%% Data locations are annotated with an arrow indicating whether they are
%% input ($\inloc$) or output ($\outloc$) locations, which essentially is intended to represent
%% whether the location will be associated with a value that is coming
%% into the function as an input or a value that will be computed and
%% returned from the function as an output. These annotations occur both
%% in the declarations of functions and in the application of functions;
%% in the former case, as parameters for the function, and in the latter
%% case, as arguments to the function. In the case where data locations
%% occur as arguments, the locations must be in scope, and their
%% annotation of input or output must match their state at that point
%% (so, input locations have been already written to, output locations have not).

%% We elide the superscript $r$ on location variables ($\sRP$ in the
%% grammar) when that information is not relevant.
%

%% \begin{comment}
%% Naturally, a location's region matches the one it's derived from: in
%% \il{letloc $\;l^r\;$ = start $\;r$}, the locations $r$
%% match.
%% %
%% These regions model logically continuguous memory buffers that are
%% growable at one end.  This is similar to traditional region
%% systems, except that allocating to, \eg{} an MLKit region happens in
%% units of complete objects (with pointers), whereas we permit leaving
%% objects ``partially written'' for arbitrarily long periods of time ---
%% separating writing the tag of a data constructor from serializing out
%% its fields.
%% %
%% Indeed, the entire purpose of using a region system in \ourcalc{} is
%% to group objects together so that their representations can be merged
%% and compressed by ``inlining'' pointers.
%% \end{comment}

%% Pattern matching, in the form of $\sgramwd{case}$ statements, are
%% familiar from other functional languages. Each alternative in the
%% pattern match statement matches on a constructor, its location, and a
%% series of variable and annotated type pairs corresponding to each
%% field of that constructor.

%% For this presentation of \ourcalc{}, we will not be too concerned with
%% including primitive operations on, e.g., numbers, strings, etc though
%% our full implementation obviously includes these things.  At certain
%% points in this paper we will give examples of \ourcalc{} programs that
%% make use of these extensions. Extending the presentation of \ourcalc{}
%% to include such types and operations is straightforward.
%% %
%% \rn{Doesn't need that much discussion, but could combine this point
%% with a bigger point about what syntactic sugar we will allow ourselves
%% in examples.}

\subsection{Static Semantics}
\label{subsec:static}


\begin{figure}
  \begin{displaymath}
  \begin{aligned}
    \textup{Typing Env.} && \TENV && \gramdef & \; \set{\var_1 \mapsto \hTYP_1, \; \ldots \; , \var_n \mapsto \hTYP_n} \\
    \textup{Store Typing} && \SENV && \gramdef & \; \set{\locreg{\loc}{r_1}_1 \mapsto \TYP_1, \; \ldots \; , \locreg{\loc}{r_n}_n \mapsto \TYP_n} \\      
    \textup{Constraint Env.} && \CENV && \gramdef & \;
      \set{\locreg{\loc}{r_1}_1 \mapsto \LE_1, \; \ldots \; , \locreg{\loc}{r_n}_n \mapsto \LE_n} \\
    \textup{Allocation Pointers} && \AENV && \gramdef & \; \set{r_1 \mapsto ap_1, \; \ldots \; , r_n \mapsto ap_n} \\
    && && & {\text{where} \; ap = \locreg{\loc}{\reg} \gramor \emptyset} \\
    \textup{Nursery} && \NENV && \gramdef & \; \set{\locreg{\loc}{r_1}_1, \; \ldots \; , \locreg{\loc}{r_n}_n}
  \end{aligned}
\end{displaymath}

  \caption{Extended grammar of \ourcalc{} for static semantics}
  \label{fig:typegrammar}
\end{figure}
\begin{figure}
  \footnotesize
  \begin{mathpar}
    \rtvar{}\hspace{1em}
    \rtconcreteloc{}\\
    \rtlet{}\hspace{1em}
    \rtlregion{}\\
    \rtlltag{}\hspace{1em}
    \rtllstart{}\\
    \rtllafter{}\hspace{1em}
    \rtdatacon{}
  \end{mathpar}
  \normalsize
  \caption{Typing judgments for \ourcalc{} (1)}
  \label{fig:types1}
\end{figure}

%\floatstyle{boxed}\restylefloat{figure}
\begin{figure}
  \footnotesize
  \begin{mathpar}
    \rtapp{}\hspace{1em}
    \rtfunctiondef{}\\
    \rtpat{}\\
    \rtcase{}\hspace{1em}
    \rtprogram{}
  \end{mathpar}
  \normalsize
   \caption{Typing judgments for \ourcalc{} (2)}
   \label{fig:types2}

\end{figure}



In \figref{fig:typegrammar}, we extend the grammar with some extra
details necessary for describing the type system.
The typing rules for expressions in \ourcalc{} are given in
\figref{fig:types1} and \figref{fig:types2}, where the rule form is as follows:
%% $\RENV;\EENV;\CENV;\SENV;\TENV \vdash \EXPR : \TYP \tyatlocreg{\loc}{\reg} ; \RENV';\EENV'$.

\[ \TENV;\SENV;\CENV;\AENV;\NENV \vdash \AENV'; \NENV'; \EXPR : \hTYP \]

%

The five letters to the left of the turnstile are different environments.
$\TENV$ is a standard typing environment.
$\SENV$ is a store-typing environment, which maps all
\emph{materialized} symbolic
locations to their types. That is, every location in $\SENV$ {\em has been written}
and contains a value of type $\SENV(l^r)$.
$\CENV$ is a constraint environment, which keeps
track of how symbolic locations relate to each other.
$\AENV$ maps each region in scope to a location, and is used to symbolically
track the allocation and incremental construction of data structures;
$\AENV$ can be thought of as representing the
\emph{focus} within a region of the computation.
% (the exact usage of $\AENV$ will be expanded on below).
$\NENV$ is a nursery of all symbolic locations that have been allocated,
but not yet written to.
Locations are removed from $\NENV$
upon being written to, as the purpose is to prevent multiple writes to
a location.
Both $\AENV$ and $\NENV$ are threaded through the typing
rules, also occuring in the output (to the right of the
turnstile).

%% To give the typing rules, we first extend the grammar with some extra
%% details that are necessary for keeping track of location and region
%% variables. A \emph{location state} is a three-tuple of boolean values,
%% representing whether a location has been written to, whether the
%% location was introduced after some other location, and whether a
%% different location has been introduced after the location. An
%% environment maps locations to their location states.
%% %
%% The location state environment is \emph{threaded through} the
%% typing rules: it occurs as a premise (before the turnstile)
%% and alongside the type after the turnstile. This allows the typing
%% rules to encode \emph{changes} in the state of locations (for
%% example, recording that they have been written to).

The \textsc{\tvar} rule ensures that the variable is in scope, and
the symbolic location of the variable has been written to.
%
\textsc{\tconcreteloc} is very similar, and also just ensures that
the symbolic location has been written to.
%
\textsc{\tlet} is straightforward, but note that along with $\TENV$,
it also extends $\SENV$ to signify that the location $\loc$ has
materialized.

In \textsc{\tlregion}, extending $\AENV$ with an empty allocation pointer
brings the region $\reg$ in scope, and also indicates that
a symbolic location has not yet been allocated in this region.

There are three rules for introducing locations
(\textsc{\tllstart}, \textsc{\tlltag} and \textsc{\tllafter}, all shown in \Figref{fig:types1}),
corresponding to three ways of allocating a new location in a
region. A new location is either: at the start of a region, one cell
after an existing location, or after the data structure rooted at an
existing location. Introducing locations in this fashion sets up an
ordering on locations, and the typing rules must ensure that the
locations are used in a way that is consistent with this intended
ordering.
%
To this end, each such rule extends the constraint environment $\CENV$
with a constraint that is based on how the location was introduced,
and $\NENV$ is extended to indicate that the new location is in scope
and unwritten.

Additionally, the location-introduction rules use $\AENV$ to ensure that a program
must introduce locations in a certain pattern (corresponding to the
left-to-right allocation and computation of fields, as explained
in~\secref{subsec:dynamic}).
%% Recall from earlier, constructing a data structure proceeds through a
%% series of steps: allocating a location for a tag, allocating a
%% location after the tag for the first field (if there are fields), then
%% for each field alternating materializing fields and allocating after
%% them, until finally writing the initial tag.
%
In $\AENV$, each region is mapped to either the right-most allocated
symbolic location in that region (if it is unwritten), or to the
symbolic location of the most recently materialized data structure.
This mapping in $\AENV$ is used by the typing rules to ensure that:
%
(1) \textsc{\tllstart} may only introduce a location at the start
of a region once;
%
(2) \textsc{\tlltag} may only introduce a location if an unwritten location has
just been allocated in that region (to correspond to the tag of some
soon-to-be-built data structure); and
%
(3) \textsc{\tllafter} may only introduce a location if a data structure has
just been materialized at the end of the region, and the programmer wants to
allocate \emph{after} it. To attempt, for example, to allocate the location of
the right sub-tree of a binary tree \emph{before} materializing the left sub-tree would
be a type error.
%
{Each location-introduction rule also ensures that the
introduced location must be written to at some point, by checking that
it's absent from the nursery after evaluating the expression.}

%% When a fresh
%% region $\reg$ is bound, $\AENV$ is extended to map that region
%% to $\emptyset$, indicating no locations have been allocated in it,
%% and so \textsc{\tllstart} may introduce a location at the start of $\reg$
%% and update $\AENV$ to note that a location has been allocated there.
%% For allocating a location intended as the first field after a tag,
%% \textsc{\tlltag} expects $\AENV$ to map to some location $\loc'$
%% which is also in $N$, because the tag of a dat 

%% and each enforces that the location will be allocated in a region in a
%% specific way, extending the location constraint environment with
%% details of where the location was allocated relative to the rest of
%% the region. $\AENV$ and $\NENV$ are also extended to reflect this allocation.

In order to type an application of a data constructor, \textsc{\tdatacon} starts by
ensuring that the tag being written and all the fields have the correct type.
Along with that, the locations of all the fields of the constructor must
also match the expected constraints. That is, the location of the first field
should be immediately after the constructor tag, and there should be
appropriate $\mathit{after}$ constraints for other fields in the location constraint environment.
After the tag has been written, the location $\loc$ is removed from the
nursery to prevent multiple writes to a location.
%
{
%
As mentioned earlier, LoCal uses destination-passing style. To guarantee destination-passing
style, it suffices to ensure that a function returns its value in a location passed from its caller.
The LoCal type system enforces this property by using constraints of the form $l' \neq l$ in the premises
of the typing rules of the operations that introduce new locations
}

{As demonstrated by \textsc{\tdatacon}, the type system enforces a
  particular ordering of writes to ensure the resulting tree is
  serialized in a certain order. Some interesting patterns are
  expressible with this restriction (for example, writing or reading
  multiple serialized trees in one function), and, as we will address
  shortly in \secref{sec:indirections}, \ourcalc is flexible enough
  to admit extensions that soften this restriction and allow for
  programmers to make use of more complicated memory layouts.}

{A simple demonstration of the type system is shown in
Table~\ref{table:types-example}, which tracks how $\AENV$, $\CENV$,
and $\NENV$
% (allocation point, constraints, nursery)
change after each line in a simple expression that builds
a binary tree with leaf children. Introducing $\loc$ at the top
establishes that it is at the beginning of $\reg$,
$\AENV$ maps $\reg$ to $\loc$, and $\NENV$ contains $\loc$.
The location for the left sub-tree,
$\loc_a$, is defined to be $+ 1$ after it, which updates $\reg$ to
point to $\loc_a$ in $\AENV$ and adds a constraint to $\CENV$ for
$\loc_a$. Actually constructing the \texttt{Leaf} in the next line
removes $\loc_a$ to $\NENV$, because it has been written to. Once $\loc_a$
has been written, the next line can introduce a new location $\loc_b$
\emph{after} it, which updates the mapping in $\AENV$ and adds a
new constraint to $\CENV$. Once $\loc_b$ has been written and removed
from $\NENV$ in the next line, the final \texttt{Node} can be constructed,
which expects the constraints to establish that $\loc$ is before $\loc_a$,
which is before $\loc_b$.}

%% Additional rules (such as for function application, pattern matching)
%% are conventional%
%% \iftoggle{EXTND}{
%% and are in the Appendix,~\appendixref{subsec:types2}.
%% }{
%% and are available in the Appendix of the extended version~\cite{LoCal-tr}
%% }

%% Two rules (\textsc{\tcase}, \textsc{\tpat}) cover pattern matching. \ldots

%% \note{We could also potentially explain the $\loc \neq \loc_i$ bit in \tpat.}

%All the other typing judgements are straightforward.

To finish out the typing rules, \Figref{fig:types2} contains rules for
function application and definition, as well as pattern matching.
%
Function application in \textsc{\tapp} ensures the location of the result of
the application is initially unwritten, and is considered written afterward.
%
Types and locations for the function are pulled from the function signature.
%
Pattern matching is handled by \textsc{\tcase} and \textsc{\tpat}, which
are straightforward.
%
The final rule type checks a whole program, consisting of datatype
and function definitions.

To simplify the formalism and proofs, we restricted typing rules
somewhat so that, in effect, the rules restrict well-typed expressions
so that they can return only the the result of a freshly allocated
constructor application.
%
Consequently, it is not possible, for instance, to type the following
expression, because the right-hand side is a value and, as such, does
not allocate.
%
\begin{code}
let x : @\tyatlocreg{T}{l}{r}@ = y in ...
\end{code}
%
This restriction is enforced by there being an assertion of the form
$\locreg{\loc}{\reg} \in \NENV$ in the premise of the typing rules of
the non-value expressions, such that $\tyatlocreg{\TYP}{\loc}{\reg}$
is the result type of the given expression.
%
Lifting this restriction is conceptually straightforward, but would
require either added complexity to the substitution lemma or the use
of a different factoring of the grammar and typing rules.
%
Similarly, our formalism and proofs could be extended to treat
primitive types, such as ints, bools, tuples, etc., as well as with
offsets and indirections in data constructors, with some conceptually
straightforward extensions to the formalism.


%% \setlength{\tabcolsep}{2pt}
%% \floatstyle{plaintop}\restylefloat{table}

\begin{table}[]
  \centering
%% \begin{adjustbox}{width=0.75\linewidth,center}
  
\begin{tabular}{llll}
  Code & $\AENV$ & $\CENV$ & $\NENV$ \\ \hline
  \begin{code}
letloc $\locreg{l}{r}$ =
    start($r$)
  \end{code} & $\{r \mapsto \locreg{l}{r} \}$ & $\emptyset$ & $\{\locreg{l}{r}\}$ \\
  \begin{code}
letloc $\locreg{l_a}{r}$ = $\locreg{l}{r}$ + 1
  \end{code} &
  $\{r \mapsto \locreg{l_a}{r} \}$ &
  $\{ \locreg{l_a}{r} \mapsto \locreg{l}{r} + 1 \}$  &
  $\{\locreg{l}{r},\locreg{l_a}{r}\}$ \\
  \begin{code}
let x : @\tyatlocreg{T}{l_a}{r}@ =
    Leaf $\locreg{l_a}{r}$ 1
  \end{code} &
  $\{ r \mapsto \locreg{l_a}{r} \}$ &
  $\{ l_a \mapsto \locreg{l}{r} + 1 \}$ &
  $\{\locreg{l}{r}\}$ \\
  \begin{code}
letloc $\locreg{l_b}{r}$ =
    after(@\tyatlocreg{T}{l_a}{r}@)
  \end{code} &
  $\{ r \mapsto \locreg{l_b}{r} \}$ &
  \makecell[cl]{$\{ \locreg{l_a}{r} \mapsto \locreg{l}{r} + 1,$\\\;$\locreg{l_b}{r} \mapsto \mathit{after}(\tyatlocreg{T}{l_a}{r})\}$} &
  $\{ \locreg{l}{r}, \locreg{l_b}{r} \}$ \\
  \begin{code}
let y : @\tyatlocreg{T}{l_b}{r}@ =
    Leaf $\locreg{l_b}{r}$ 2 
  \end{code} &
  $\{ r \mapsto \locreg{l_b}{r} \}$ &
  \makecell[cl]{$\{ \locreg{l_a}{r} \mapsto l + 1,$\\\;$\locreg{l_b}{r} \mapsto \mathit{after}(\tyatlocreg{T}{l_a}{r})\}$} &
  $\{ \locreg{l}{r} \}$ \\
  \begin{code}
Node $\locreg{l}{r}$ x y
  \end{code} &
  $\{ r \mapsto \locreg{l}{r} \}$ &
  \makecell[cl]{$\{ \locreg{l_a}{r} \mapsto \locreg{l}{r} + 1,$\\\;$\locreg{l_b}{r} \mapsto \mathit{after}(\tyatlocreg{T}{l_a}{r})\}$} &
  $\emptyset$ \\
&&& \\
\end{tabular}
%% \end{adjustbox}
\caption{Step-by-step example of type checking a simple expression.}
\label{table:types-example}
\end{table}


\subsection{Dynamic Semantics}
\label{subsec:dynamic}
%\floatstyle{boxed}\restylefloat{figure}
\begin{figure}
  \begin{displaymath}
  \begin{aligned}
    \textup{Store} && \STOR && \gramdef & \set{\reg_1 \mapsto \heap_1 , \; \ldots \; , \reg_{n} \mapsto \heap_{n}} \\
    \textup{Heap} && \heap && \gramdef & \set{\ind_1 \mapsto \DC_1 , \; \ldots \; , \ind_{n} \mapsto \DC_{n}}\\
    \textup{Location Map} && \MENV && \gramdef & \set{\locreg{\loc}{\reg_1}_1 \mapsto \concreteloc{\reg_1}{\ind_1}{} , \; \ldots \; , \locreg{\loc}{\reg_n}_n \mapsto \concreteloc{\reg_n}{\ind_n}{}}
  \end{aligned}
\end{displaymath}

  \caption{Extended grammar of \ourcalc{} for dynamic semantics}
  \label{fig:opergram}
\end{figure}

\begin{figure}
  \small
  \begin{mathpar}
    \mprset{flushleft}
    \rddatacon{}
    
    \rdletlocstart{}\\
    \rdletloctag{}
    
    \rdletlocafter{}\\

    \rdletexp{}\\
    \rdletval{}
    
    \rdletregion{}\\
    \rdapp{}\\
    \rdcase{}\\
  \end{mathpar}
  \normalsize
  \caption{Dynamic semantics rules for \ourcalc{}}
  \label{fig:dynamic}
\end{figure}
  

The dynamic semantics for expressions in \ourcalc{} are given in
 \figref{fig:dynamic1} and \figref{fig:dynamic2}, where the transition rule is as follows.
%
\begin{displaymath}
\STOR;\MENV;\EXPR \stepsto \STOR';\MENV';\EXPR'
\end{displaymath}
%
To model the behavior of reading and writing from an indexed
memory, we introduce the \emph{store}, $\STOR$.
%
The store is a map from regions to \emph{heaps}, where each heap
consists of an array of \emph{cells}, which contain store values
(data constructor tags).
%
To bridge from symbolic to concrete locations, we use the
\emph{location map}, $\MENV$, which is a map from symbolic
to concrete locations.
%

Case expressions are treated by the \textsc{\dcase{}} rule.
%
The objective of the rule is to load the tag of the constructor $\DC$
located at $\concreteloc{\reg}{\ind}{}$ in the store and dispatch
the corresponding case.
%
%% For simplicity, \ourcalc{} requires that all patterns are exhaustive,
%% and our type system ensures that the there always exists a constructor
%% tag at the specified location at the expected type.
%
The expression produced by the right-hand side of the rule is the body
of the pattern, in which all pattern-bound variables are replaced by
the concrete locations of the fields of the constructor $\DC$.

The concrete locations of the fields are obtained by the following
process.
%
If there is at least one field, then its starting address is the
position one cell after the constructor tag.
%
The starting addresses of subsequent fields depend on the sizes of the
trees stored in previous fields.

A feature of \ourcalc{} is the flexibility it provides to pick the
serialization layout.
%
Our formalism uses our \emph{end-witness rule} to abstract from
different layout decisions.
%
Given a type $\TYP$, a starting address
$\concreteloc{\reg}{\ind_{s}}{}$, and store $\STOR$, the rule below
asserts that address of the end witness is
$\concreteloc{\reg}{\ind_{e}}{}$.
%
\begin{displaymath}
  \ewitness{\TYP}{\concreteloc{\reg}{\ind_{s}}{}}{\STOR}{\concreteloc{\reg}{\ind_{e}}{}}
\end{displaymath}
%
Using this rule, the starting address of the second field is obtained
from the end witness of the first, the starting address of the
third from the end witness of the second, and so on.

The allocation and finalization of a new constructor is achieved by
some sequence of transitions, starting with the \textsc{\dletloctag{}} rule, then
involving some number of transitions of the \textsc{\dletlocafter{}} rule,
depending on the number of fields of the constructor, and finally
ending with the \textsc{\ddatacon{}} transition.
%
The \textsc{\dletloctag{}} rule allocates one cell for the tag of some new
constructor of a yet-to-be determined type, leaving it to later to
write to the new location.
%
The resulting configuration binds its $\loc$ to the address
$\concreteloc{\reg}{\ind+1}{}$, that is, the address one cell past
given location $\loc'$ at $\concreteloc{\reg}{\ind}{}$.
%
Fields that occur after the first are allocated by the
\textsc{\dletlocafter{}} rule.
%
Here, its $\loc$ is bound to the address \concreteloc{\reg}{j}{} one
past the last cell of the constructor represented by its given
symbolic location $\loc_1$.
%
Like the \textsc{\dcase{}} rule, the required address is obtained by
application of end-witness rule to the starting address of the given
$\loc_1$ at the type of the corresponding field $\TYP$.
%
The final step in creating a new data constructor instance
is the \textsc{\ddatacon{}} rule.
%
It writes the specified constructor tag $\DC$ at the address in the
store represented by the symbolic location $\loc$.

The \textsc{\dletlocstart{}} rule for the \sgramwd{letloc} with \sgramwd{(start r)}.
expression binds the location to the starting address in the region
and starts running the body.

The \textsc{\dletexp{}} rule for let-expressions evaluates the let-bound
expression to a value and the \textsc{\dletval{}} rule substitutes the value for
the let-bound variable in the body.
%
The \textsc{\dapp{}} rule for function applications looks up the function by name
in the top-level environment and substitutes arguments for parameters
in the function body, substitutes argument symbolic locations for
parameter symbolic locations, then starts the resulting function body
running.
%
The \textsc{\dletregion{}} rule for the \sgramwd{letregion} expression binds the
new region and starts running the body.
%

The driver which runs an \ourcalc{} program initially loads all data
types, functions, type checks them, and if successful, then seeds the
$Function$, $\typeofcon$, and $\typeoffield$ environments.
%
Let $\EXPR_0$ be the main expression.
%
If $\EXPR_0$ type checks with respect to the \textsc{\tprogram{}} rule, then
the main program is safe to run.
%
The initial configuration for the machine with an empty store is
\begin{displaymath}
\emptyset; \set{\loc \mapsto \concreteloc{\reg}{0}{}}; \EXPR_0,
\end{displaymath}
which is, by itself, not particularly interesting or useful.
%
It is, however, straightforward to construct a type-safe initial configuration
whose store is nonempty, as long as the initial configuration
has a store that is well formed, as described in \Secref{sec:well-formedness}.
%
The program can start taking evaluation steps from this configuration.

\subsection{Example: Allocating a Binary Tree.}
\label{sec:dynsem-example}
%
Consider this code snippet of \ourcalc{}.
%
\begin{code}
letloc @$\locreg{l_1}{\reg}$@ = @$\locreg{l_0}{\reg}$@ + 1 in
let a : @\tyatlocreg{Tree}{l_1}{r}@ = (Leaf @$\locreg{l_1}{\reg}$@) in
letloc @$\locreg{l_2}{\reg}$@ = (after (@\tyatlocreg{Tree}{l_1}{r}@)) in
let b : @\tyatlocreg{Tree}{l_2}{r}@ = (Leaf @$\locreg{l_2}{\reg}$@) in
Node @$\locreg{l_0}{\reg}$@ a b
\end{code}
%
Assume that the store starts out with a fresh heap, $\STOR = \set{\reg
  \mapsto \emptyset}$ and the location $\locreg{l_0}{\reg}$ maps to
$\concreteloc{\reg}{0}{}$ in the location map.
%
After stepping past the first line, the \textsc{\dletloctag{}} step has
allocated a cell for the tag of the interior node and bound the
location $\locreg{\loc_1}{\reg}$ to $\concreteloc{\reg}{1}{}$.
%
After the next line, the \textsc{\ddatacon{}} transition writes a leaf node to
the store at the address represented by $\locreg{\loc_1}{\reg}$:
$\STOR = \set{\reg \mapsto \set{1 \mapsto \mathtt{Leaf}}}$.
%
The second \gramwd{letloc} obtains the starting address for the second
leaf node by using end witness of the previous leaf node.
%
The write of the second leaf node appears in the store after
the next line, leaving the following store:
$
\STOR = \set{\reg \mapsto \set{1 \mapsto \mathtt{Leaf}, 2 \mapsto \mathtt{Leaf}}}$.
%
Finally, after the \textsc{\ddatacon{}} step taken for the last line, the store
contains the finalized allocation:
$\STOR = \set{\reg \mapsto \set{0 \mapsto \mathtt{Node}, 1 \mapsto \mathtt{Leaf}, 2 \mapsto \mathtt{Leaf}}}$.
%

The end-witness judgement of the new data constructor is the
following:
$\ewitness{\mathtt{Tree}}{\concreteloc{\reg}{0}{}}{\STOR}{\concreteloc{\reg}{3}{}}$
%
The judgement applies, in part, because, as expected, the tag at the
address $\concreteloc{\reg}{0}{}$ is a tag of type $\mathtt{Tree}$.
%
In addition, because the tag indicates an interior node with two
subtrees for fields, the judgement obligation extends to recursively
showing (1) that the end witness of the first leaf node (also at type
$\mathtt{Tree}$) at $\concreteloc{\reg}{1}{}$ has an end witness
(which is $\concreteloc{\reg}{2}{}$), (2) that the second field has an
end witness starting at the end witness of the first field, namely
$\concreteloc{\reg}{2}{}$, and ending at some higher address (which in
this case is $\concreteloc{\reg}{3}{}$), and (3) finally that the end
witness of the second field is the end witness of the entire
constructor, as is the case here.

\subsection{Type Safety}

\fixme{The key to proving type safety is our store-typing rule will be written here!}
%
\begin{displaymath}
  \storewf{\SENV}{\CENV}{\AENV}{\NENV}{\MENV}{\STOR}
\end{displaymath}
%
The store typing specifies three categories of invariants.
%
The first enforces that allocations occur in the sequence
specified by the constraint environment $\CENV$.
%
In particular, if there is some location $\loc$ in the domain of
$\CENV$, then the location map and store have the expected
allocations at the expected types.
%
For instance, if $(\loc \mapsto
\afterl{\tyatlocreg{\TYP}{\loc'}{\reg}}) \in \CENV$, then $\loc'$ maps
to $\concreteloc{\reg}{\ind_1}{}$ and $\loc$ to
$\concreteloc{\reg}{\ind_2}{}$ in the location map, and $\ind_2$ is
the end witness of $\ind_1$ at type $\TYP$ in the store, at region
$\reg$.
%
The second category enforces that, for each symbolic location such
that $(\loc \mapsto \TYP) \in \SENV$, there is some
$\concreteloc{\reg}{\ind_1}{}$ for $\loc$ in the location map and
$\ind_1$ has some end witness $\ind_2$ at type $\TYP$.
%
The final category enforces that each address in the store is written
once.
%
This property is asserted by insisting that, if $\loc \in \NENV$, then
there is some $\concreteloc{\reg}{\ind}{}$ for $\loc$ in the location
map, but there is no write to for $\ind$ at $\reg$ in the store.
%
To support this property, there are two additional conditions which
require that the most recently allocated location (tracked by
to $\AENV, \NENV$) is at the end of its respective region.

To support this behavior, there are two additional conditions in this
category that enforce linearity of allocation.
%
The first one captures the situation when there is an allocation in
flight, such as is the case in the postcondition of allocating
a tag (and the corresponding binding the cell to $\loc$).
%
In this case, the typing rule specifies that $(\reg \mapsto \loc) \in
A$ and $\loc \in \NENV$.
%
The requirement is that the address bound to $\loc$ is the maximum
position in the location map for the region $\reg$ and that this
position is past the end of any address already written in the store
at $\reg$.
The second condition captures the other situation, just after
a field has been allocated and finalized, which is occurs
when $(\reg \mapsto \loc) \in A$ and $(\loc \mapsto \TYP) \in \SENV$
for the location $\loc$ representing the field.
%
In this case, the condition requires that the address of the end
witness of the field is the maximum of all other addresses in the
store at region $\reg$.

The type safety of \ourcalc{} follows from the following result.

\begin{theorem}[Type safety]
  \label{theorem:type-safety}
\begin{displaymath}
  \begin{aligned}
  \text{If} \;\; & (\emptyset;\SENV;\CENV;\AENV;\NENV \vdash \AENV';\NENV';\EXPR : \hTYP) \wedge
                   (\storewf{\SENV}{\CENV}{\AENV}{\NENV}{\MENV}{\STOR}) \\
  \text{and} \;\; & \STOR;\MENV;\EXPR \stepsto^n \STOR';\MENV';\EXPR' \\
  \text{then} \;\; & (\EXPR' \; \mathit{value}) \vee 
                     (\exists \STOR'', \MENV'', \EXPR'' . \; \STOR';\MENV';\EXPR' \stepsto \STOR'';\MENV'';\EXPR'')
  \end{aligned}
  \end{displaymath}  
\end{theorem}

\begin{nproof}
  The type safety follows from an induction with
  the progress and preservation lemmas \fixme{which will be explained below}
\end{nproof}

\section{Offsets and Indirections}\label{sec:indirections}

As motivated in \secref{sec:bg}, it is sometimes desirable to be
able to ``jump over'' part of a serialized tree.
%
As presented so far, \ourcalc{} makes use of an end witness judgment
to determine the end of a particular data structure in memory.
%
The simplest computational interpretation of this technique is,
however, a linear scan through the store.
%
Luckily, extending the language to account for storing and making use
of \emph{offset} information for certain datatypes is
straightforward, and does not add conceptual difficulty to
neither the formalism nor type-safety proof.

Such an extension may use annotations on datatype declarations
that identify which fields of a given constructor are provided \emph{offsets}
and to permit cells in the store to hold offset values.
%
Because the offsets of a given constructor are known from its type,
the \textsc{\dletloctag{}} rule can allocate space for offsets when it
allocates space for the tag.
%
It is straightforward to fill the offset fields because \textsc{\ddatacon{}}
rule already has in hand the required offsets, which are provided in
the arguments of the constructor.
%
Finally, the \textsc{\dcase{}} rule can use offsets instead of 
the end-witness rule.

\emph{Indirections} permit fields of data constructors to point across
regions, and thus require adding an annotation form (e.g., an
annotation on the type of a constructor field to indicate an
indirection) and extending the store to hold pointers.
%
Fortunately, as discussed later, regions in \ourcalc{} are never
collected; they are garbage collected in our implementation.
% and thus the extension adds conceptual difficulty to
% neither the formalism nor proof.
%
Every time an indirection field is constructed, space for the pointer
is allocated using a transition rule similar to the \textsc{\dletloctag{}}
rule.
%
The \textsc{\ddatacon{}} rule receives the address of the indirection in the
argument list, just like any other location and writes the indirection
pointer to the address of the destination field.
%

To type check, the type system extends with two new typing rules and a
new constraint form to indicate indirections.
%
To maintain type safety in the presence of offsets and indirections,
the store typing rule needs to be extended to include them.
%
Because the programmer is not manually managing the creation or use of offsets
or indirections (they are below the level of abstraction, indicated by
annotating the datatype, but not changing the code), the store-typing rule
generalizes straightforwardly and the changes preserve type safety.
%\rn{Still might be good to show some kind of concrete example of these annotations.}

In datatype annotations each field can be marked to store its offset in the constructor {\em
  or} be represented by an indirection pointer (currently not both):
\begin{code}
data T = K1 T (Ind T) | K2 T (Offset T) | K3 T
\end{code}
Type annotations would also be the place to express {\em permutations} on fields
that should be serialized in a different order, (e.g., postorder).  But it is
equivalent to generating \ourcalc with reordered fields in the source program.


\chapter{The Gibbon Compiler}

Words.

\section{Converting Functional Programs to the Location Calculus}\label{sec:infer-local}

\section{Compiling the Location Calculus} \label{sec:impl-local}


\chapter{Applications and Evaluation}

\note{General overview of the kinds of things we tested in Gibbon, how fast they were, etc.}

\section{Microbenchmarks}

See \tabref{tab:litmus-table}.

We are
% gibbon pointer cnf capn
202 / 2.6 / 3.2 / 9.4  $\times$ geomean faster than
Gibbon/NonPacked/CNF/CapNP respectively, and
0.96 / 2.6 / 3.2 / 7.3 $\times$ faster for only apples-to-apples asymptotics.
%% $3.2\times$,
%% $9.4\times$, and
%% $202\times$


%% \tabref{tab:litmus-table} shows the results.

\note{Re-write to either remove or better explain difference between Gibbon1 and Gibbon2.}
      
The column labeled ``Gibbon2'' shows performance of \lamadt programs
(low-level \ourcalc control was not needed for any of these)
using indirections and offsets, automatically.
%
``Gibbon1'' shows the approach described in \cite{ecoop17-gibbon}.
%
There are two major sources of overhead for our new approach versus Gibbon1:
 
\begin{enumerate}
\item Growable regions:
In each case, our compiler starts with smaller, growable regions\footnote{starting at 64K bytes}, 
which we require to create small output regions as
in {\bf id} or {\bf treeInsert},
but we suffer the overhead of bounds-checking. On the other hand,
Gibbon always stores fully serialized data in huge regions.

\item Likewise, we have found that the backend C compiler is sensitive to the number
of cases in switch statements on data constructor tags (for instance, triggering
the jump table heuristic).  By including the possibility that each tag we read
may be a tagged indirection,
% or end-of-chunk,
we increase
code size and increase the number of cases in our switch statements.
\end{enumerate}


However, the benchmarks where indirections and random-access offsets are important (id,
rightmost, treeInsert, findMax) show a huge difference between Gibbon1 and Gibbon2,
as we would expect based on Gibbon1
requiring additional traversals to compile those functions.

\paragraph{Versus pointer-based representations}
The ``NonPacked'' approach is \ourcalc configured to always insert indirections
and thus emulate a traditional object representation.
% example of a traditional compilation approach
% that use one heap object per data constructor.
%
In this case, we are being overly friendly to this pointer-based representation by
allowing it to read its input (for example, the input tree to {\bf treeInsert})
in an already-deserialized, pointer-based form.  A full apples-to-apples
comparison would force the pointer-based version to deserialize the input
message and reserialize the output; but we omit that here to focus only on the
cost of the tree traversals themselves.
%
%% Similarly, the earlier Gibbon work compared a subset of these programs against a
%% larger set of traditional compilers (Java, gcc, GHC, OCaml, MLton, Racket, Chez),
%% finding those compilers performance on
%% pointer-based inputs worse than the performance of lifted functions on
%% serialized inputs.

\paragraph{Versus competing libraries}

%% The CNF representation is just the traditional Haskell heap representation of
%% data constructors.  It wastes a word of header space for garbage collection
%% purposes, and it uses full 64-bit absolute pointers between heap objects.
%% The result is fast to read but not space efficient.
%% This is visible on benchmarks such as {\bf sumTree}, where CNF out-performs
%% Cap'N Proto by $3.5\times$.
%% % (/ 0.96 0.27)
%% Conversely, the Cap'N Proto representation is space efficient---using 40\% fewer
%% bytes for the binary tree---and faster to build.
%% % (- 1 (/ 805 1340.0))
%% %
%% CNF results are slow to build because they involve an extra copy: first to
%% create the data on the normal heap, second to copy it into the compact region.
%% This is why CNF's {\bf copyTree} is twice as fast as {\bf add1Leaves}, even
%% though the both computations walk the tree and build a new output tree, copy is
%% able to use a runtime system function to walk the data and copy directly from
%% input message to output message, without allocating on the regular (non-compact)
%% Haskell heap.

The biggest differences in \tabref{tab:litmus-table} are due
asymptotic complexity.
However, for constant factor performance, we see the expected
relationship---that our approach and Gibbon are faster than CNF and Cap'N Proto,
sometimes by an order of magnitude, \eg\,{\bf add1Leaves}.

CNF and Cap'N Proto encode some metadata in their serialization, to
support the GHC runtime, and protocol evolution, respectively.
On the other hand, 
our compiler only uses offsets and tagged indirections
when needed, and the size ratio of the encodings depends on how much these features are used.
%
For example, {\bf rightmost} uses a data-encoding that includes random-access
offsets, and {\bf treeInsert} creates an output with a logarithmic number of
tagged indirections.  Thus while our size advantage over CNF is
%
$4\times$ smaller
% (/ 1340.0 335)
%
on {\bf buildTree}, it is only
%
$2.22\times$
% (/ 1340.0 603)
% (/ 1340000000.0 (+ 334000000 (* (- (expt 2 25) 1) 8)))
%
for {\bf rightmost}.
% , and $XYZ\times$ for {\bf treeInsert}.

CNF results are slow to build because they involve an extra copy: first to
create the data on the normal heap, second to copy it into the compact region.
This is why CNF's {\bf copyTree} is twice as fast as {\bf add1Leaves}, even
though the both computations walk the tree and build a new output tree, copy is
able to use a runtime system function to walk the data and copy directly from
input message to output message, without allocating on the regular (non-compact)
Haskell heap.

\paragraph{Composing traversals}

For offset-insertion, we allow the whole-program compiler to select the data
representation based on what consuming functions are applied to it.
%% In a more general setting (say, modular compilation, or messages received over
%% the network from program in another language), we would need to adopt
%% random-access nodes uniformly, which would, e.g., give {\bf leftmost} the same
%% small amount of overhead in its input tree as {\bf rightmost}.
In the presence of multiple functions traversing a single data structure,
any function demanding random access changes the representation for all of them.
% the ``weakest link'' determines how many random-access nodes are needed.
% compiler uses a data representation required to optimize the slowest one.
{\bf repMax} is one such example:
%\begin{code}[mathescape=true]
\il{ repMax t = propagateConst (findMax t) t}.
%\end{code} \vspace{-5mm}
%  repMax = propagateConst . findMax -- almost correct!
% Consider the {\bf repMax} program in which
Here {\bf findMax} only requires a partial scan (random access), but propagating
that value requires a full traversal.  In this case, the compiler would add
offsets to the datatype to ensure that `findMax' remains
logarithmic.
%
However, this causes the subsequent traversal (propagateConst) to slow
down, as it now has to unnecessarily skip over some extra bytes.  Likewise, if
we do not include findMax in the whole program, the data remains fully
serialized, which is why {\bf propagateConst} and {\bf findMax} run separately
take less than 440ms, but run together take 480ms.  Yet the latter time is still
6$\times$ and 9$\times$ faster, respectively, than CNF and Cap'N Proto!
      


 \begin{table*}
  \begin{center}
    \small
      \begin{tabular}{ |c|c|c|c|c|c| }
        \hline
        Benchmark & Gibbon2 & Gibbon1 & NonPacked & CNF & CapnProto\hspace{-1mm} \\
        \hline
        % CNF: 100M iterations, calling NOINLINE id' function from NOINLINE rep.
        %      That gives 8.0ns (Allowing rep to inline, but not ID, makes it 2.15ns)
        % capnp: iter=20M median of 9 trials
        % Gibbon2: 100M iterations

        % Speedup:
        % CNF: (/ 2.1 2.1) = 1
        % CapnProto: (/ 129 2.1) = 61.42
        % Gibbon1: 152380952
        % Pointer: (/ 0.93 2.1) = 0.44
        {\bf id}:
        time, & 2.1ns &  0.32s  &  0.93ns &  2.1ns   & 129ns  \\
        complexity   & $O(1)$   &  $O(N)$ & $O(1)$    &  $O(1)$  & $O(1)$  \\
        %  output size (bytes)  &        &         &           &  8       &   8     \\
        \hline

        % CNF: 100M
        % capnp: iters = 20M median of 9

        % Speedup:
        % CNF: (/ 44 17.0) = 2.58
        % CapnProto: (/ 376 17.0) = 22.11
        % Gibbon1: (/ 18.0 17.0) = 1.05
        % Pointer: (/ 26 17.0) = 1.52
        {\bf leftmost}:
        time,        & 17ns         & 18ns        & 26ns        &    44ns      & 376ns            \\
        complexity   & $O(log(N))$  & $O(log(N))$ & $O(log(N))$ & $O(log(N))$  & $O(log(N))$ \\
        input size (bytes) & 335MB  & 335MB       & 335MB       &    1.34GB    & 805MB             \\
        \hline

        % CNF: 100M iterations
        % capnpL iter=20M, median of 9 trials
        % capnpL iter=20M, median of 9 trials
        % Speedup:
        % CNF: (/ 47 175.0) = 0.26
        % CapnProto: (/ 482 175.0) = 2.75
        % Gibbon1: 297142
        % Pointer: (/ 19 175.0) = 0.108
        {\bf rightmost}:
        time,        & 175ns        & 56ms    & 19ns        &    47ns      & 482ns \\
        complexity   & $O(log(N))$  & $O(N)$  & $O(log(N))$ & $O(log(N))$  & $O(log(N))$  \\
        input size (bytes) & 603MB & 335MB    & 335MB       &    1.34GB    & 805MB            \\
        \hline

        %capn: median of 9 trails iter=1
        % Speedup
        % CNF: (/ 4.5 0.27) = 16.66
        % CapnProto: (/ 1.8 0.27) = 6.66
        % Gibbon1: (/ 0.24 0.27) = 0.88
        % Pointer: (/ 2.7 0.27) = 10
        {\bf buildTree}:
        time,               & 0.27s     &  0.24s    & 2.7s &  4.5s  & 1.8s   \\
        complexity,         & $O(N)$    &  $O(N)$   &  $O(N)$ & $O(N)$  &  $O(N)$ \\
        output size (bytes) & 335MB     &  335MB    &  1.34GB &  1.34GB &  805MB       \\
        \hline

        % CNF: median of 9 trials:
        % capnp: median of 9 trails iter=1
        % Speedup:
        % CapnProto: (/ 3.8 0.25) = 15.2
        % CNF: (/ 2.7 0.25) = 10.8
        % Gibbon1: (/ 0.24 0.25) = 0.96
        % Pointer: (/ 3.1 0.25) = 12.4
        {\bf add1Leaves}:
        time,               & 0.25s     &  0.24s  & 3.1s     & 2.7s    &  3.8s  \\
        complexity,         & $O(N)$    &  $O(N)$ & $O(N)$   & $O(N)$  &  $O(N)$ \\
        \hline
        % CNF: median of 9 trials:
        % capn: median of 9 trails iter=1
        % Speedup:
        % CNF: (/ 0.27 0.095) = 2.84
        % CapnProto: (/ 0.96 0.095) = 10.10
        % Gibbon1: (/ 67.0 95) = 0.70
        % Pointer: 8.5
            {\bf sumTree}:
            time,               & 95ms     & 67ms     & 0.81s   &  0.27s  &  0.96s  \\
            complexity,         & $O(N)$   & $O(N)$   & $O(N)$  & $O(N)$  &  $O(N)$ \\
            \hline
            %capn: median of 9 trials iter =1
            % Speedup:
            % CNF: (/ 1.1 0.2) = 5.5
            % CapnProto: (/ 1.9 0.2) = 9.4
            % Gibbon1: (/ 0.24 0.2) = 1.2
            % pointer: (/ 3.5 0.2) = 17.5
                {\bf copyTree}:

                time,               & 0.2s      & 0.24s   & 3.5s   &  1.1s   &  1.9s       \\
                complexity,         & $O(N)$    & $O(N)$  & $O(N)$ & $O(N)$  &  $O(N)$ \\

                \hline
                \hline
                % CNF: median of 9 iters
                %capn: median of 9 trils iter=1
                % Gibbon numbers: median of 9 trials
                % Speedup:
                % CNF: (/ 4.27 0.5) = 8.54
                % CapnProto: (/ 2.1 0.5) = 4.2
                % Gibbon1: (/ 0.49 0.5) = 0.98
                % Pointer: (/ 2.96 0.5) = 5.92
                    {\bf buildSearchTree}:

                                    & 0.5s      & 0.49s     & 2.96s   &  4.27s   &  2.1s         \\
                    complexity,          & $O(N)$    & $O(N)$    & $O(N)$  & $O(N)$   &  $O(N)$ \\
                    output size (bytes)  & 603MB     & 603MB     & 1.61GB  &  1.61GB  &  805MB      \\

                    \hline

                    % CNF iters=1M
                    % This one is 2.5us with iters=1000
                    %            1.24us with iters=200K
                    %            1.04us with iters=1M
                    % NonPacked iters=1M
                    %
                    % capnp iters=20M median of 9
                    % Speedup:
                    % CNF: (/ 1 0.69) = 1.44
                    % CapnProto: (/ 1.3 0.69) = 1.88
                    % Gibbon: 144927
                    % Pointer: (/ 0.92 0.69) = 1.33
                    {\bf treeContains}:
                    time,                & 0.69$\mu$s   & 0.1s  & 0.92$\mu$s  &  1$\mu$s    & 1.3$\mu$s \\
                    complexity,          & $O(log(N))$ & $O(N)$ & $O(log(N))$ & $O(log(N))$ & $O(log(N))$ \\
                    \hline

                    % CNF iters=200K
                    % Capn median of 9 , iter=1
                    % Gibbon2 iters = 3500000
                    % Speedup:
                    % CNF: (/ 3.5 0.87) = 4.02
                    % CapnProto: (/ 150 0.87) = 172.4
                    % Gibbon: 436781
                    % Pointer: (/ 2.5 0.87) = 2.87
                    {\bf treeInsert}:

                    time,                & 0.87$\mu$s  & 0.38s  & 2.5$\mu$s   &  3.5$\mu$s   &  150$\mu$s  \\
                    complexity,          & $O(log(N))$ & $O(N)$ & $O(log(N))$ & $O(log(N))$  & $O(N)$  \\
                    avg bytes added      & 677 bytes        &  603MB  & 856 bytes   &  848 bytes  & 805MB  \\
                    \hline

                    %capnp: median of 9 trails iters=1M
                    {\bf InsertDestructive}:
                                         &  NA   & NA   & NA   & NA  &  1.37$\mu$s       \\
                    complexity,          &       &      &      &     & $O(log(N))$ \\

                    \hline

                    % Gibbon2:   100M iters
                    % Gibbon1:   20 iters
                    % Pointer:   100M iters
                    % CNF:       100M iters
                    % CapnProto: 1M iters
                    % Speedup:
                    % CNF: (/ 75 206.0) = 0.36
                    % CapnProto: (/ 597 206.0) = 2.89
                    % Gibbon: 427184
                    % Pointer: (/ 41 206.0) = 0.199
                    {\bf findMax}:
                    time,                & 206ns        & 88ms    & 41ns        & 75ns         & 597ns        \\
                    complexity           & $O(log(N))$  & $O(N)$  & $O(log(N))$ & $O(log(N))$  & $O(log(N))$  \\
                    \hline

                    % Speedup:
                    % CNF: (/ 4.2 0.43) = 9.76
                    % CapnProto: (/ 2.8 0.43) = 6.51
                    % Gibbon: (/ 0.42 0.43) = 0.97
                    % Pointer: (/ 3.3 0.43) = 7.67
                    {\bf propagateConst}:
                                         & 0.43s  &  0.42s  &  3.3s   & 4.2s   & 2.8s   \\
                    complexity,          & $O(N)$ &  $O(N)$ &  $O(N)$ & $O(N)$ & $O(N)$  \\
                    \hline

                    % Speedup:
                    % CNF: (/ 4.3 0.48) = 8.9
                    % CapnProto: (/ 2.9 0.48) = 6.04
                    % Gibbon: (/ 0.51 0.48) = 1.06
                    % Pointer: (/ 3.2 0.48) = 6.67
                    {\bf repMax}:
                    time,                & 0.48s  & 0.51s   & 3.2s    & 4.3s    & 2.9s   \\
                    complexity,          & $O(N)$ & $O(N)$  & $O(N)$  & $O(N)$  & $O(N)$  \\

                    \hline
      \end{tabular}
    \end{center}
  \vspace{-3mm}
    \caption{Tree-processing functions operating on serialized data.
      %% We are
      %% % gibbon pointer cnf capn
      %% 202 / 2.6 / 3.2 / 9.4  $\times$ geomean faster than
      %% Gibbon/NonPacked/CNF/CapNP, and
      %% 0.96 / 2.6 / 3.2 / 7.3 $\times$ faster for only apples-to-apples asymptotics.
      %% %% $3.2\times$,
      %% %% $9.4\times$, and
      %% %% $202\times$
      %% \captionscrunch{}
    }
    \label{tab:litmus-table}

\end{table*}


\section{Data Processing Benchmarks}

\note{Overview of how this technique can be applied to data processing programs.}
\note{If data to be processed is already serialized then this avoids marshaling cost.}
\note{If data to be processed is in some other format then we can still be faster by
  transforming the data into a format that is more efficient to process (like the JSON
  to byte array transformation necessary for the Twitter benchmark). And in these cases
  it is often still faster than libraries that are meant to process the data in its
  original form.}
 

\subsection{Twitter JSON Benchmark}

For a real data set, we
use Twitter metadata consisting of user ID's and hashtags for all tweets posted
in 1 month, and count the occurrences of the hashtag ``cats'' in this dataset.
Here we seek to replicate and extend the CNF experiment reported by
\cite{cnf-icfp15}.

The dataset is stored on disk in JSON format, and we use RapidJSON
v1.1.0 ({\footnotesize\url{http://rapidjson.org/}}) as a performance baseline: a widely
recognized fast C++ JSON library.
%
In \figref{fig:twitter_slowdown_plot}, we vary the amount of data processed,
up to 1GB.  (For each data-point, taking the median of 9 trials
ensures the data is already in the Linux disk cache.)
%
For fairness, all versions read the data via a single \il{mmap} call, plus
demand paging.

There are two RapidJSON versions. The ``lexer'' version never constructs an
object representing a parsed tweet, rather, it is a state-machine
that is able to count ``cats'' while tokenizing, {\em without parsing}.  It is
optimized to be as fast as possible for this particular JSON schema, with no
error detection (a non-compliant input would give silent failures and wrong
answers).
%
The ``parser'' version represents a more traditional and idiomatic situation use
of the library: calling the \il{.Parse()} method to produce a DOM object, and
then accessing its fields.
%
We have structured this benchmark to maximally advantage this parsing approach:
the 9,111,741 tweets processed in the rightmost data points of \figref{fig:twitter_slowdown_plot} are stored as one JSON object each, on each line of the input file.
%
Thus the data only needs to be read into memory once, and in a single pass the RapidJSON benchmark reads, parses, discards, and repeats.
%
Conversely, if the tweets were instead stored as a single JSON array, filling
the entire input file, then RapidJSON would have to parse the entire file
(writing the DOM tree out to memory, overflowing last level cache), then read
that same tree back into memory in a second pass to count hashtags.
%
Nevertheless, in spite of this single-pass advantage, our compiler achieves
6$\times$ and 12$\times$ speedup over RapidJSON lexer/parser.
%
We process the 9.1M tweets in 0.39s.


\begin{figure}
  \centering
  \input{twitter_slowdown_plot}
  \caption{Twitter data processing benchmark results}
  \label{fig:twitter_slowdown_plot}
\end{figure}

\subsection{Point Correlation}

\note{Summarize Laith's point correlation benchmark (from ECOOP
  paper), updating the language to match the new terminology from
  LoCal.}
\note{Update plot (ideally use gnuplot and eps latex output, to look
  nice and consistent with other plots).}

\begin{figure}
  \centering
  \includegraphics{data/point_corr/point_corr_perf}
  \label{fig:point_corr_plot}
  \caption{Speedup of packed implementation of point correlation over
    pointer-based implementation. X axis shows varying tree sizes
    (represented in number of nodes).}
\end{figure}

\section{Abstract Syntax Trees}

\note{This is where my results for ASTs will eventually go.}
\note{I can maybe put the Racket count nodes benchmark in here too, as a kind of
  AST traversal.}



\addcontentsline{toc}{chapter}{Bibliography}

\bibliographystyle{acm}
%\include{your_bibliography_name.tex}
\bibliography{refs}

\newpage

% \appendix command is necessary to change chapter numbering.
% Appendices are optional

\appendix




% Adds a line for your CV without a page number

%% \addtocontents{toc}{%
%%   \protect\contentsline{chapter}{Curriculum Vitae}{}}
\end{document}

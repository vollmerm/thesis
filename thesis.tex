\documentclass[showabstract,showacknowledgments,showpreface,showdedication]{iuphd} 

\PassOptionsToPackage{dvipsnames}{xcolor}
\usepackage{marvosym}
\usepackage{listings}
\usepackage{etoolbox}
\newcommand\hmmax{0}
\newcommand\bmmax{0}
\usepackage{amsmath}
% \usepackage{txfonts}
% \usepackage{mathtools}
\usepackage{MnSymbol}
\usepackage{xspace}
\usepackage{mathpartir}
\usepackage{stmaryrd}
\usepackage{hyperref}
\usepackage{graphicx}
\usepackage{subcaption}
\usepackage{float}
\usepackage{makecell}
\usepackage{adjustbox}
\usepackage{balance}
\usepackage{booktabs}
\makeatletter
\let\th@plain\relax
\makeatother
\usepackage[thmmarks,thref,amsmath]{ntheorem}
\usepackage[noabbrev]{cleveref}

\setcounter{secnumdepth}{3}

\newtheorem{theorem}{Theorem}[section]
\newtheorem{lemma}[theorem]{Lemma}

% Cases in a proof.
\theorempreskip{0pt}
\theorempostskip{10pt}
\theoremheaderfont{\scshape}
\theorembodyfont{\upshape}
\theoremindent10pt
\newtheorem*{goal}{Obl.}

%\theorempreskip{5pt}
%\theorempostwork{\setcounter{goal}{0}}
\theoremindent10pt
\newtheorem{subcase}{SubCase}
\theorempostwork{\setcounter{subcase}{0}}
%\theoremindent0pt
\newtheorem{ncase}{Case}
\newtheorem*{bcase}{Case}


% Proof sketches.
\theoremstyle{nonumberplain}
\theorempreskip{0pt}
\theorempostskip{10pt}
\theoremheaderfont{\scshape}
\theorembodyfont{\upshape}

\theorempostwork{\setcounter{ncase}{0}}
\newtheorem{proofsketch}{Proof Sketch}

\theorempostwork{\setcounter{ncase}{0}}
\newtheorem{nproof}{Proof}



% For title and abstract page

\title{A Language-based Approach to Programming with Serialized Data}
\author{Michael Vollmer}
\date{Month 2020} % Completion date of Dissertation
\department{Computer Science} % Change this to your department if not Mathematics

% For acceptance and abstract page

\committeechair{Ryan Newton, PhD}
\readertwo{Jeremy Siek, PhD}
\readerthree{Sam Tobin-Hochstadt, PhD}
\readerfour{Larry Moss, PhD}
\defensedate{November 13, 2020} % Date of PhD defense

% For Copyright Page
\cryear{2020} % Copyright year

% ================================================================================
% extra \newcommand's specific to this paper.
% ================================================================================

%% Oh no, we don't actually HAVE lambda!
%% Later we'll have to make this a higher-order calculus!
%% \newcommand{\ourcalc}[0]{\ensuremath{\lambda^{loc}}}
%% \newcommand{\lamadt}{\ensuremath{\lambda^{adt}}}
%% \newcommand{\lamcur}{\ensuremath{\lambda^{cur}}}

%% Proposal 1 - hey, we have capital lambda, and it's even the thing that binds
%% our lovely locations.
%% \newcommand{\ourcalc}[0]{\ensuremath{\Lambda^{loc}}}
%% \newcommand{\lamadt}{\ensuremath{\Lambda^{adt}}}
%% \newcommand{\lamcur}{\ensuremath{\Lambda^{cur}}}

% TODO: coming up with a new name.  RECONCILE usages with \ourcalc below.
% Ones using this macro have already been audited/reconciled.
\newcommand{\sysname}[0]{LoCal\xspace}

%% %% Proposal 2 - Core, like GHC Core.
% \newcommand{\ourcalc}[0]{\ensuremath{Core^{loc}}}

%% Helpers
\newcommand{\ourcalc}[0]{\sysname}
% \newcommand{\lamadt} [0]{\ensuremath{Core^{adt}}}
\newcommand{\lamadt} [0]{HiCal\xspace}
% \newcommand{\lamcur} [0]{\ensuremath{Core^{cur}}}
% \newcommand{\lamcur} [0]{\textsf{Cursored}}
% \newcommand{\lamcur} [0]{\textsf{CursedCal}}

\newcommand{\lamcur}[0]{NoCal\xspace}

\newcommand{\gramdef}{\; ::= \;}
\newcommand{\gramor}{\; | \;}
\newcommand{\keywd}[1]{\mathit{#1}}
\newcommand{\gramwd}[1]{\texttt{#1}}
\newcommand{\sgramwd}[1]{\texttt{#1}}
\newcommand{\skeywd}[1]{\mathit{#1}}

%% Grammar
\newcommand{\PROG}{\keywd{top}}
\newcommand{\DD}{\keywd{dd}}
\newcommand{\VD}{\keywd{vd}}
\newcommand{\FD}{\keywd{fd}}
\newcommand{\DC}{\keywd{K}}
\newcommand{\sDC}{\skeywd{K}}
\newcommand{\TC}{\keywd{T}}
\newcommand{\EXPR}{\keywd{e}}
\newcommand{\sEXPR}{\skeywd{e}}
\newcommand{\DATA}{\gramwd{data}}
\newcommand{\TYP}{\keywd{\tau}}
\newcommand{\hTYP}{\keywd{\hat{\tau}}}
\newcommand{\sTYP}{\skeywd{\tau}}
\newcommand{\var}{\svar}
\newcommand{\svar}{x}
\newcommand{\fvar}{\sfvar}
\newcommand{\sfvar}{f}
\newcommand{\yvar}{y}
\newcommand{\num}{n}
\newcommand{\ARROW}{\rightarrow}
\newcommand{\RP}{\keywd{dl}}
\newcommand{\sRP}{\skeywd{dl}}
\newcommand{\loc}{\skeywd{l}}
\newcommand{\sloc}{\skeywd{l}}
\newcommand{\concreteloc}[3]{\ensuremath{\langle #1, #2 \rangle ^{#3}}}
\newcommand{\reg}{\skeywd{r}}
\newcommand{\sreg}{\skeywd{r}}
\newcommand{\LS}{\keywd{ls}}
\newcommand{\LC}{\keywd{lc}}
\newcommand{\LE}{\keywd{le}}
\newcommand{\letpack}[3]{\gramwd{let}\;#1=#2\;\gramwd{in}\;#3}
\newcommand{\letloc}[3]{\gramwd{letloc}\;#1 = #2\;\gramwd{in}\;#3}
\newcommand{\letreg}[2]{\gramwd{letregion}\;#1\;\gramwd{in}\;#2}
\newcommand{\fapp}[2]{\fvar \;[#1]\; #2}
\newcommand{\TS}{\keywd{ts}}
\newcommand{\ssetbool}[4]{#4 = #3[#2\mapsto \keywd{#1}=\keywd{True}]}
\newcommand{\sinlinesetbool}[3]{#3[#2\mapsto \keywd{#1}=\keywd{True}]}
\newcommand{\CS}{\keywd{cs}}
\newcommand{\pat}{\keywd{pat}}
\newcommand{\ptr}[2]{(\gramwd{ptr}\;\keywd{#1}\;\keywd{#2})}
\newcommand{\sind}{i}
\newcommand{\ind}{\keywd{i}}
\newcommand{\indj}{\keywd{j}}
\newcommand{\VAL}{\keywd{v}}
\newcommand{\ec}{\keywd{\mathcal{E}}}
\newcommand{\evalc}[1]{\ec \llbracket #1 \rrbracket }
\newcommand{\STOR}{\keywd{S}}
\newcommand{\stepsto}{\Rightarrow}
\newcommand{\Name}{\Red{Name }}
\newcommand{\q}[1]{\texttt{#1}}
\newcommand{\frl}[1]{{\ensuremath \mathit{frl}(#1)}}
\newcommand{\caseclause}[2]{#1 \;\rightarrow \;#2}
\newcommand{\case}[2]{\gramwd{case}\; #1 \;\gramwd{of}\;#2}
\newcommand{\spat}{\keywd{spat}}
\newcommand{\switch}[2]{\gramwd{switch}\; #1 \;\gramwd{of}\;#2}
\newcommand{\readInt}[1]{\gramwd{\text{readInt}} \; #1}
\newcommand{\writeInt}[2]{\gramwd{\text{writeInt}} \; #1 \; #2}
\newcommand{\readTag}[1]{\gramwd{\text{readTag}} \; #1}
\newcommand{\writeTag}[2]{\gramwd{\text{writeTag}} \; #1 \; #2}
\newcommand{\readCursor}[1]{\gramwd{\text{readCursor}} \; #1}
\newcommand{\writeCursor}[2]{\gramwd{\text{writeCursor}} \; #1 \; #2}
\newcommand{\datacon}[3]{\ensuremath{#1 \;#2 \;#3}}
\newcommand{\litcon}[2]{\datacon{\keywd{L}}{#1}{#2}}
\newcommand{\litnum}{\keywd{n}}
\newcommand{\CP}{\keywd{cp}}
\newcommand{\tptr}[3]{\langle \mathit{ptr}\;#2\;#3 \rangle_{#1}}
\newcommand{\LM}{\keywd{M}}
\newcommand{\locis}[2]{\ensuremath{#1:#2}}
\newcommand{\lvar}{\keywd{lx}}
\newcommand{\has}[3]{\ensuremath{#1(#2) = #3}}
\newcommand{\startr}[1]{(\gramwd{start}\; #1)}
\newcommand{\afterl}[1]{(\gramwd{after}\; #1)}
\newcommand{\tightoverset}[2]{%
  \mathop{#2}\limits^{\vbox to -.5ex{\kern-0.75ex\hbox{$#1$}\vss}}}
\newcommand\set[1]{\{ \, \ensuremath{#1} \,\}}
\newcommand{\ecdot}{\bullet}
\newcommand{\heap}{\keywd{h}}
\newcommand{\alphaequiv}{=_{\alpha}}

%% Environments
\newcommand{\LENV}{\keywd{L}}
\newcommand{\RENV}{\keywd{R}}
\newcommand{\CENV}{\keywd{C}}
\newcommand{\TENV}{\keywd{\Gamma}}
\newcommand{\EENV}{\keywd{E}}
\newcommand{\SENV}{\keywd{\Sigma}}
\newcommand{\MENV}{\keywd{M}}
\newcommand{\FENV}{\keywd{F}}
\newcommand{\AENV}{\keywd{A}}
\newcommand{\NENV}{\keywd{N}}

%% Loc helpers
\newcommand{\inloc}[0]{\ensuremath{\downarrow\sloc}}
\newcommand{\outloc}[0]{\ensuremath{\uparrow\sloc}}
\newcommand{\locreg}[2]{\ensuremath{{#1}^{#2}}}
\newcommand{\tyatlocreg}[3]{#1 \ensuremath{@} \locreg{#2}{#3}}
\newcommand{\Cursorize}[0]{Cursorize}
\newcommand{\fresh}[0]{\keywd{fresh}}

%% Meta functions

% Takes FROM,TO, i.e. \subst{e}{x}{v}.
\newcommand{\subst}[3]{\ensuremath{#1[#3/#2]}}

\newcommand{\typeofcon}{\keywd{TypeOfCon}}
\newcommand{\typeoffield}{\keywd{TypeOfField}}
\newcommand{\kargtys}{\keywd{ArgTysOfConstructor}}
\newcommand{\allocptr}[2]{\keywd{MaxIdx}(#1,#2)}

% Update(M,key,val), e.g. M[key>val]
%\newcommand{\update}[3]{\ensuremath{((#2 \mapsto #3)\; #1)}}
\newcommand{\update}[3]{\ensuremath{#1+\{#2 \mapsto #3\}}}

\newcommand{\ewitness}[4]{\ensuremath{#1;#2;#3 \vdash_{ew} #4}}
\newcommand{\storewf}[6]{\ensuremath{#1;#2;#3;#4 \vdash_{wf} #5;#6}}
\newcommand{\storewfcfa}[3]{\ensuremath{#1 \vdash_{wf_{cfc}} #2;#3}}
\newcommand{\storewfca}[4]{\ensuremath{#1;#2 \vdash_{wf_{ca}} #3;#4}}

\newcommand{\tcfun}{\ensuremath{\vdash_{fun}}}
\newcommand{\tcpat}{\ensuremath{\vdash_{pat}}}
\newcommand{\tcts}{\ensuremath{\vdash_{ts}}}

%% Special references for proofs.
\newcommand{\refwellformed}[2]{WF~\ref{#1};\ref{#2}}
\newcommand{\refendwitness}[2]{EW~\ref{#1};\ref{#2}}
\newcommand{\refts}[1]{T-#1}
\newcommand{\refcase}[1]{Case-\ref{#1}}
\newcommand{\elimexists}[0]{$\exists$ elim}
\newcommand{\refinst}[0]{Inst.}

\newcommand{\tfunctiondef}{T-Function-Definition}
\newcommand{\tdatacon}{T-DataConstructor}
\newcommand{\ddatacon}{D-DataConstructor}
\newcommand{\dletloctag}{D-LetLoc-Tag}
\newcommand{\dletlocafter}{D-LetLoc-After}
\newcommand{\dletlocstart}{D-LetLoc-Start}
\newcommand{\dapp}{D-App}
\newcommand{\dletregion}{D-LetRegion}
\newcommand{\tcase}{T-Case}
\newcommand{\dcase}{D-Case}
\newcommand{\tpat}{T-Pattern}
\newcommand{\tprogram}{T-Program}
\newcommand{\tllafter}{T-LetLoc-After}
\newcommand{\tlltag}{T-LetLoc-Tag}
\newcommand{\tllstart}{T-LetLoc-Start}
\newcommand{\tlregion}{T-LetRegion}
\newcommand{\tvar}{T-Var}
\newcommand{\tlet}{T-Let}
\newcommand{\dletexp}{D-Let-Expr}
\newcommand{\dletval}{D-Let-Val}
\newcommand{\tapp}{T-App}
\newcommand{\tconcreteloc}{T-Concrete-Loc}

\newcommand{\MPL}{\text{MaPLe}}

\newcommand{\vsmaple}[1]{$\inferrule{\MPL{}}{\text{Ours}}{#1}$}
\newcommand{\vsocaml}[1]{$\inferrule{\text{OCaml}}{\text{Ours}}{#1}$}
\newcommand{\vsghc}[1]{$\inferrule{\text{GHC}}{\text{Ours}}{#1}$}


\newcommand{\emptytenv}{\emptyset}


%% Region-Parallel transitions
\newcommand{\transformsto}{\longrightarrow}
\newcommand{\Indr}{I}
\newcommand{\trdatadecl}{Tr-DataDecl}
\newcommand{\dregparsteptask}{D-RegionPar-Step}
\newcommand{\dregparletexppar}{D-RegionPar-Let-Fork}
\newcommand{\dregparletexpseq}{D-RegionPar-Let}
\newcommand{\dregparletlocafter}{D-RegionPar-LetLoc-After}
\newcommand{\dregparletlocafternewreg}{D-LetLoc-After-NewReg}
\newcommand{\dregparcasejoin}{D-RegionPar-Case-Join}
\newcommand{\dregparcase}{D-RegionPar-Case}
\newcommand{\dregpardatacon}{D-RegionPar-DataConstructor}
\newcommand{\dregpardataconjoin}{D-RegionPar-DataConstructor-Join}
\newcommand{\dregparletlocstart}{D-RegionPar-LetLoc-Start}
\newcommand{\dregparletloctag}{D-RegionPar-LetLoc-Tag}
\newcommand{\dregparletexp}{D-RegionPar-Let-Expr}
\newcommand{\dregparletval}{D-RegionPar-Let-Val}
\newcommand{\dregparapp}{D-RegionPar-App}
\newcommand{\dregparletregion}{D-RegionPar-LetRegion}

%%
\newcommand{\wftask}{\ensuremath{\vdash_{wf_{task}}}}
%\newcommand{\update}[3]{\ensuremath{#1+\{#2 \mapsto #3\}}}

\newcommand{\nfofi}{\keywd{Nf}}
\newcommand{\incri}{\keywd{Incr}}
\newcommand{\followinm}{\keywd{Deref}}
\newcommand{\rorngi}[2]{[#1, #2)}
\newcommand{\rorngh}{\keywd{RORngH}}
\newcommand{\storelookupone}[3]{\hat{#1}(#2,#3)}
\newcommand{\maplookupone}[2]{\hat{#1}(#2)}
\newcommand{\advancei}{\keywd{After}}
\newcommand{\mergeh}{\keywd{MergeH}}
\newcommand{\mergem}{\keywd{MergeM}}
\newcommand{\merges}{\keywd{MergeS}}
\newcommand{\tie}{\keywd{Tie}}
\newcommand{\linkfields}{\keywd{LinkFields}}

%\newcommand{\ewitnesstext}{\ensuremath{\vdash_{\textit{ew}}}}
% \newcommand{\ewitness}[4]{\ensuremath{#1;#2;#3 \ewitnesstext #4}}
% \newcommand{\tietext}{\ensuremath{\vdash_{\textit{tie}}}}
% \newcommand{\storewf}[7]{\ensuremath{#1;#2;#3;#4,#5 \vdash_{wf} #6;#7}}
% \newcommand{\storewfcfa}[4]{\ensuremath{#1,#2 \vdash_{wf_{cfc}} #3;#4}}
% \newcommand{\storewfca}[5]{\ensuremath{#1;#2,#3 \vdash_{wf_{ca}} #4;#5}}
% \newcommand{\storewftasks}[1]{\ensuremath{\vdash_{wf_{tasks}} #1}}

% \newcommand{\tcfun}{\ensuremath{\vdash_{fun}}}
% \newcommand{\tcpat}{\ensuremath{\vdash_{pat}}}
% \newcommand{\tcts}{\ensuremath{\vdash_{ts}}}

\newcommand{\parmergemenv}[2]{\mergem(#1, #2)}
\newcommand{\parmergestor}[2]{\merges(#1, #2)}
\newcommand{\parmergetask}[2]{\keywd{MergeT}(#1, #2)}

\newcommand{\concretelocp}[3]{\ensuremath{\langle #1, #2 \rangle ^{#3}}}

% \newcommand{\ind}{\keywd{i}}
% \newcommand{\indj}{\keywd{j}}
%\newcommand{\heap}{\keywd{h}}
\newcommand{\heapval}{\keywd{hv}}
\newcommand{\TASKVAR}{\keywd{T}}
\newcommand{\TASKSET}{\mathbb{T}}
\newcommand{\SEQSTATE}{\keywd{t}}


\newcommand{\concreteind}[1]{#1}
\newcommand{\svind}[1]{#1\circ}
\newcommand{\svindplusoff}[2]{#1 + #2}
\newcommand{\indbef}[1]{#1\diamond}
\newcommand{\concretelocvar}{\keywd{cl}}
\newcommand{\fut}[1]{\text{fut}\;#1}
\newcommand{\concretelocbefore}[1]{\gramwd{before}\; #1}
\newcommand{\indivar}[1]{\gramwd{i-var}\; #1}
\newcommand{\concretelocvardiamond}{\keywd{\concretelocvar\diamond}}
\newcommand{\indirection}[2]{\ensuremath{\text{\&}(#1,#2)}}
\newcommand{\indirectionvar}{hr}


%% Formatting

%% fdtools, harpoon are packages that provide this feature, but
%% have their own set of problems. We do this by hand instead:
%% https://tex.stackexchange.com/questions/304622
%% \makeatletter
%% \newcommand*\MY@rightharpoonupfill@{%
%%   \arrowfill@\relbar\relbar\rightharpoonup
%% }
%% \newcommand*\overrightharpoon{%
%%   \mathpalette{\overarrow@\MY@rightharpoonupfill@}%
%% }
%% \makeatother
\newcommand{\overharpoon}[1]{\overrightharpoon{#1}}


\newcommand{\rtvar}{
  \inferrule*[lab={\;\text{[\tvar]}}]{\TENV(\var) = \tyatlocreg{\TYP}{\loc}{\reg} \\ \SENV(\locreg{\loc}{\reg})=\TYP}{\TENV;\SENV;\CENV;\AENV;\NENV \vdash \AENV; \NENV; \var : \tyatlocreg{\TYP}{\loc}{\reg}}
}

\newcommand{\rtconcreteloc}{
  \inferrule*[lab={\;\text{[\tconcreteloc]}}]{\SENV(\locreg{\loc}{\reg})=\TYP}{\TENV;\SENV;\CENV;\AENV;\NENV \vdash \AENV; \NENV; \concreteloc{r}{i}{l} : \tyatlocreg{\TYP}{\loc}{\reg}}
}

\newcommand{\rtlet}{
  \inferrule*[lab={\;\text{[\tlet]}}]{\TENV;\SENV;\CENV;\AENV;\NENV \vdash \AENV';\NENV';\EXPR_1 : \tyatlocreg{\TYP_1}{\loc_1}{\reg_1} \\\\
    \TENV';\SENV';\CENV;\AENV';\NENV' \vdash \AENV'';\NENV'';\EXPR_2 : \tyatlocreg{\TYP_2}{\loc_2}{\reg_2}}
                   {\TENV;\SENV;\CENV;\AENV;\NENV \vdash \AENV''; \NENV''; \letpack{\var : \tyatlocreg{\TYP_1}{\loc_1}{\reg_1}}{\EXPR_1}{\EXPR_2} : \tyatlocreg{\TYP_2}{\loc_2}{\reg_2} 
    \\\\{ \begin{aligned}
          \text{where} \;
          & \; \TENV' = \TENV \cup \set{\var \mapsto \tyatlocreg{\TYP_1}{\loc_1}{\reg_1}} ;
          \;   \SENV' = \SENV \cup \set{\locreg{\loc_1}{\reg_1} \mapsto \TYP_1}
          \end{aligned}
        }
   }
}

\newcommand{\rtlregion}{
  \inferrule*[lab={\;\text{[\tlregion]}}]{\TENV;\SENV;\CENV;\AENV';\NENV \vdash \AENV''; \NENV'; \EXPR : \hTYP}{\TENV;\SENV;\CENV;\AENV;\NENV \vdash \AENV''; \NENV'; \letreg{\sreg}{\sEXPR} : \hTYP
    \\\\{ \begin{aligned}
          \text{where} \;
          \AENV'= \AENV \cup \set{\reg \mapsto \emptyset}
          \end{aligned}
        }
    }
}

\newcommand{\rtllstart}{
  \inferrule*[lab={\;\text{[\tllstart]}}]{\AENV(\reg)=\emptyset \\ \locreg{\loc}{\reg} \not \in \NENV'' \\ \locreg{\loc'}{\reg'} \neq \locreg{\loc}{\reg} \\\\ \TENV;\SENV;\CENV';\AENV';\NENV' \vdash \AENV'';\NENV'';\EXPR : \tyatlocreg{\TYP'}{\loc'}{\reg'}}{\TENV;\SENV;\CENV;\AENV;\NENV \vdash \AENV''; \NENV''; \letloc{\locreg{\loc}{\reg}}{\startr{\reg}}{\EXPR} : \tyatlocreg{\TYP'}{\loc'}{\reg'}
    \\\\ {\begin{aligned}
            \text{where} \; 
            \; \CENV' &= \CENV \cup \set{\locreg{\loc}{\reg} \mapsto \startr{r}} \\[-1em]
            \; \AENV' &= \AENV \cup \set{\reg \mapsto \locreg{\loc}{\reg}}  \\[-1em]
            \; \NENV' &= \NENV \cup \set{\locreg{\loc}{\reg}}
          \end{aligned}}
    }
}

\newcommand{\rtlltag}{
  \inferrule*[lab={\;\text{[\tlltag]}}]{\AENV(\reg)=\locreg{\loc'}{\reg} \\ \locreg{\loc'}{\reg} \in \NENV \\ \locreg{\loc}{\reg} \not \in \NENV'' \\ \locreg{\loc}{\reg} \neq \locreg{\loc''}{\reg''} \\\\ \TENV;\SENV;\CENV';\AENV';\NENV' \vdash \AENV'';\NENV'';\EXPR : \tyatlocreg{\TYP''}{\loc''}{\reg''}}{\TENV;\SENV;\CENV;\AENV;\NENV \vdash \AENV''; \NENV''; \letloc{\locreg{\loc}{\reg}}{(\locreg{\loc'}{r} + 1)}{\EXPR} : \tyatlocreg{\TYP''}{\loc''}{\reg''}
  \\\\{\begin{aligned}
         \text{where} \;
         \; \CENV' &= \CENV \cup \set{\locreg{\loc}{\reg} \mapsto (\locreg{\loc'}{\reg} + 1)} \\[-1em]
         \; \AENV' &= \AENV \cup \set{\reg \mapsto \locreg{\loc}{\reg}}  \\[-1em]
         \; \NENV' &= \NENV \cup \set{\locreg{\loc}{\reg}}
       \end{aligned}
      }
  }
}


\newcommand{\rtllafter}{
  \inferrule*[lab={\;\text{[\tllafter]}}]{\AENV(\reg)=\locreg{\loc_1}{\reg} \\ \SENV(\locreg{\loc_1}{\reg}) = \TYP' \\ \locreg{\loc_1}{\reg} \not \in \NENV \\ \locreg{\loc}{\reg} \not \in \NENV'' \\ \locreg{\loc}{\reg} \neq \locreg{\loc'}{\reg'} \\\\ \TENV;\SENV;\CENV';\AENV';\NENV' \vdash \AENV'';\NENV'';\EXPR : \tyatlocreg{\TYP'}{\loc'}{\reg'}}{\TENV;\SENV;\CENV;\AENV;\NENV \vdash \AENV''; \NENV''; \letloc{\locreg{\loc}{\reg}}{\afterl{\tyatlocreg{\TYP'}{\loc_{1}}{\reg}}}{\EXPR} : \tyatlocreg{\TYP'}{\loc'}{\reg'}
  \\\\{\begin{aligned}
         \text{where} \;
         \; \CENV' &= \CENV \cup \set{\locreg{\loc}{\reg} \mapsto \afterl{\tyatlocreg{\TYP'}{\loc_{1}}{\reg}}} \\[-1em]
         \; \AENV' &= \AENV \cup \set{\reg \mapsto \locreg{\loc}{\reg}}  \\[-1em]
         \; \NENV' &= \NENV \cup \set{\locreg{\loc}{\reg}}
       \end{aligned}
      }
  }
}

\newcommand{\rtdatacon}{
   \inferrule*[lab={\;\text{[\tdatacon]}}]{\typeofcon(\DC)=\TYP \\ \typeoffield(\DC,\ind)=\overharpoon{\hTYP_{\ind}} \\\\ \locreg{\loc}{\reg} \in \NENV \\
 \AENV(\reg) = \overharpoon{\locreg{\loc_n}{\reg}} \quad \text{if}\;n \neq 0 \quad \text{else} \; \locreg{\loc}{\reg} \\\\
 \CENV(\overharpoon{\locreg{\loc_1}{\reg}}) = \locreg{\loc}{\reg} + 1 \\ \CENV(\overharpoon{\locreg{\loc_{\indj + 1}}{\reg}}) = \afterl{(\overharpoon{\TYP'_{\indj}} \ensuremath{@} \overharpoon{\locreg{\loc'_{\indj}}{\reg}})} \\\\
%   \tyatlocreg{\overharpoon{\TYP'_{\indj}}}{\overharpoon{\loc_{\indj}}}{\reg})} \\\\
 \TENV;\SENV;\CENV;\AENV;\NENV \vdash \AENV;\NENV;\overharpoon{\VAL_{\ind}} : \overharpoon{\tyatlocreg{\TYP_{\ind}'}{\loc_{\ind}}{\reg}}
 }{\TENV;\SENV;\CENV;\AENV;\NENV \vdash \AENV';\NENV'; \datacon{\DC}{\locreg{\loc}{\reg}}{\overharpoon{\VAL}} : \tyatlocreg{\TYP}{\loc}{\reg}
 \\\\{\begin{aligned}
        \text{where} \;
          \; \AENV' & = \AENV \cup \set{\reg \mapsto \locreg{\loc}{\reg}} ;
          \; \NENV' = \NENV - \set{\locreg{\loc}{\reg}} \\[-1em]
          \; \litnum &= |\overharpoon{\VAL}| ;
          \; \ind \in \keywd{I} = \set{1, \; \ldots \; , \litnum} ;
          \; \indj \in \keywd{I} - \set{\litnum}
       \end{aligned}
     }
   }
}

\newcommand{\rtfunctiondef}{
  \inferrule*
   [lab={\;\text{[\tfunctiondef]}}]
   {\TENV;\SENV;\CENV;\AENV;\NENV \vdash \AENV;\NENV';
    \EXPR : \tyatlocreg{\TYP}{\loc}{\reg} \qquad \locreg{\loc}{\reg} \not \in \NENV'\\\\
    \forall_{i \in \set{1, \ldots, n}} . \exists_j . \overharpoon{\locreg{\loc_i}{\reg_i}} = \overharpoon{\locreg{\loc'_j}{\reg'_j}} \\ \exists_j . \locreg{\loc}{\reg} = \overharpoon{\locreg{\loc'_j}{\reg'_j}}
   }
   {\tcfun 
    \fvar : \forall _{\overharpoon{\locreg{l'}{r'}}}.
            \overharpoon{\tyatlocreg{\TYP}{\loc}{\reg}}
            \ARROW \tyatlocreg{\TYP}{\loc}{\reg};
    \fvar \overharpoon{\var} = \EXPR
    \\\\{\begin{aligned}
            \text{where} \;
             \; \TENV &= \set{\overharpoon{\var_1} \mapsto \overharpoon{{\tyatlocreg{\TYP_1}{\loc_1}{\reg_1}}}, \; \ldots \; , \overharpoon{\var_n} \mapsto \overharpoon{{\tyatlocreg{\TYP_n}{\loc_n}{\reg_n}}}}\\[-1em]
             \; \SENV &= \set{\overharpoon{\locreg{\loc_1}{\reg_1}} \mapsto \overharpoon{\TYP_1}, \; \ldots \; , \overharpoon{\locreg{\loc_n}{\reg_n}} \mapsto \overharpoon{\TYP_n}} \\[-1em]
             \; \CENV &= \emptyset; \; \AENV = \set{\reg \mapsto \locreg{\loc}{\reg}}; \; \NENV =  \set{\locreg{\loc}{\reg}} \\[-1em]
             \; n &= |\overharpoon{\var}| = |\overharpoon{\tyatlocreg{\TYP}{\loc}{\reg}}|\\[-1em]
         \end{aligned}
        }
   }
}

\newcommand{\rtprogram}{
  \inferrule*
   [lab={\;\text{[\tprogram]}}]
   { \tcfun \overharpoon{\FD} \\
    \TENV;\SENV;\CENV;\AENV;\NENV \vdash \AENV';\NENV';\EXPR:\tyatlocreg{\TYP}{\loc}{\reg}}
   {\vdash_{prog} \AENV';\NENV';\overharpoon{\DD} \;; \overharpoon{\FD} \;; \EXPR : \tyatlocreg{\TYP}{\loc}{\reg} \\\\{\begin{aligned}
            \text{where} \;
            & \; \TENV = \emptyset ;
              \; \SENV = \emptyset \\[-2mm]
            & \; \CENV = \set{\locreg{\loc}{\reg} \mapsto \startr{\reg}} ;
              \; \AENV = \set{\reg \mapsto \locreg{\loc}{\reg}} ;
              \; \NENV = \set{\locreg{\loc}{\reg}}
          \end{aligned}
         }
   }
}

\newcommand{\rtapp}{
  \inferrule*[lab={\;\text{[\tapp]}}]{
    |\overharpoon{\locreg{\loc'}{\reg'}}| = |\overharpoon{\locreg{\loc''}{\reg''}}| \quad |\overharpoon{\VAL}| = |\overharpoon{\var}|\\\\
    \TENV;\SENV;\CENV;\AENV;\NENV \vdash \AENV;\NENV;\overharpoon{\VAL_i} : \overharpoon{\tyatlocreg{\TYP_i}{\loc_i}{\reg_i}} \\
    \locreg{\loc}{\reg} \in \NENV \\ \AENV(\reg) = \locreg{\loc}{\reg}\\\\
    \forall_i . \exists_j . \overharpoon{\locreg{\loc'''_i}{\reg'''_i}} = \overharpoon{\locreg{\loc''_j}{\reg''_j}} \wedge \overharpoon{\locreg{\loc_i}{\reg_i}} = \overharpoon{\locreg{\loc'_j}{\reg'_j}}  \quad \exists_j . \locreg{\loc'''}{\reg'''} = \overharpoon{\locreg{\loc''_j}{\reg''_j}} \wedge \locreg{\loc}{\reg} = \overharpoon{\locreg{\loc'_j}{\reg'_j}}
    }{\TENV;\SENV;\CENV;\AENV;\NENV \vdash \AENV;\NENV';
      \fapp{\overharpoon {\locreg{\loc'}{\reg'}}}{\overharpoon{\VAL} : \tyatlocreg{\TYP}{\loc}{\reg}}
     \\\\{\begin{aligned}
            \text{where} \;
            & \; \fvar : \forall _{\overharpoon{\locreg{\loc''}{r''}}}.
            \overharpoon{\tyatlocreg{\TYP_i}{\loc'''_i}{\reg'''_i}} \ARROW \tyatlocreg{\TYP}{\loc'''}{\reg'''} ; (\fvar \overharpoon{\var} = \EXPR) = Function(f)\\[-2mm]
%            & \; \overharpoon{\hTYP_{\fvar_{i}}} = \subst{\overharpoon{\TYP_{\ind}}}{\overharpoon{\locreg{\loc_{\fvar}}{\reg}}}{\overharpoon{\loc_{\reg}}} \\[-2mm]
%              \; \tyatlocreg{\TYP}{\loc}{\reg} = \subst{\hTYP_{\fvar}}{\overharpoon{\locreg{\loc_{\fvar}}{\reg}}}{\overharpoon{\loc_{\reg}}} \\[-2mm]
%            & \; \AENV' = \AENV \cup \set{\reg \mapsto \locreg{\loc}{\reg}} ;
             & \; \NENV' = \NENV - \set{\locreg{\loc}{\reg}}; 
              \; n = |{\overharpoon{\VAL}}| ;
              \; \ind \in \set{1, \; \ldots \;, n} 
          \end{aligned}
         }
     }
}

\newcommand{\rtcase}{
  \inferrule*[lab={\;\text{[\tcase]}}]{
     \TENV;\SENV;\CENV;\AENV;\NENV \vdash \AENV;\NENV; \VAL : \tyatlocreg{\TYP'}{\loc'}{\reg'} \\\\
     \TYP';\TENV;\SENV;\CENV;\AENV;\NENV \vdash_{\pat} \AENV';\NENV'; \overharpoon{pat_{\ind}} : \hTYP}
     {\TENV;\SENV;\CENV;\AENV;\NENV \vdash \AENV';\NENV';
      \case{\VAL}{\overharpoon{\pat} : \hTYP}
      \\\\{\begin{aligned}
             \text{where} \;
               \; \litnum = |\overharpoon{\pat}| ;
               \; \ind \in \set{1, \; \ldots \; , \litnum}
             \end{aligned}
           }
      }
}

\newcommand{\rtpat}{
  \inferrule*[lab={\;\text{[\tpat]}}]{
     \typeofcon(\DC) = \TYP'' \\ \kargtys(\DC) = \overharpoon{\TYP'} \\ \SENV(\locreg{\loc}{\reg}) = \TYP \\\\
     \locreg{\loc}{\reg} \neq \overharpoon{\locreg{\loc'_i}{\reg'}} \\ \TENV';\SENV';\CENV;\AENV;\NENV \vdash \AENV';\NENV';\EXPR : \tyatlocreg{\TYP}{\loc}{\reg}}
     {\TYP'';\TENV;\SENV;\CENV;\AENV;\NENV \tcpat \AENV';\NENV';
     \caseclause{\datacon{\DC}{}{(\overharpoon{\var : \tyatlocreg{\TYP'}{\loc'}{\reg'}})}}{\EXPR} : \tyatlocreg{\TYP}{\loc}{\reg}
     \\\\{\begin{aligned}
            \text{where} \;
            & \; \TENV' = \TENV \cup \set{\overharpoon{\var_1} \mapsto \overharpoon{\TYP'_1} \ensuremath{@} \overharpoon{{\loc_1'}^{\reg'}}, \; \ldots \; , \overharpoon{\var_n} \mapsto \overharpoon{\TYP'_n} \ensuremath{@} \overharpoon{{\loc_n'}^{\reg'}}}\\[-2mm]
            & \; \SENV' = \SENV \cup \set{\overharpoon{\locreg{\loc_1'}{\reg'}} \mapsto \overharpoon{\TYP'_1}, \; \ldots \; , \overharpoon{\locreg{\loc_n'}{\reg'}} \mapsto \overharpoon{\TYP'_n}} \\[-2mm]
            & \; \ind \in \set{1, \ldots , n} ; 
              \; \litnum = |\overharpoon{\TYP'}| = |\overharpoon{\var : \tyatlocreg{\TYP'}{\loc'}{\reg}}|
         \end{aligned}
        }
     }
}

\newcommand{\rddatacon}{
  \inferrule*[lab={\;\text{[\ddatacon]}}]{}{
    \STOR;\MENV;
    \datacon{\DC}{\locreg{\loc}{\reg}}{\overharpoon{\VAL}}
    \stepsto
    \STOR';\MENV;\concreteloc{\reg}{\ind}{\locreg{\loc}{\reg}}
    \\\\ {
          \begin{aligned}
            \text{where} \;
            \; \STOR' = \STOR \cup \set{\reg \mapsto (\ind \mapsto \DC)} ;
            \; \concreteloc{\reg}{\ind}{} = \MENV(\locreg{\loc}{\reg})
          \end{aligned}
    }
    }}

\newcommand{\rdletlocstart}{
  \inferrule*[lab={\;\text{[\dletlocstart]}}]
  {}{\STOR;\MENV;\letloc{\locreg{\loc}{\reg}}{\startr{\reg}}{\EXPR} \stepsto \STOR;\MENV';\EXPR
    \\\\{
         \begin{aligned}
           \text{where} \;
%           & \; \locreg{\loc_f}{\reg} \; \fresh \\[-2mm]
%             \; \EXPR' = \subst{\EXPR}{\locreg{\loc}{\reg}}{\locreg{\loc_f}{\reg}} \\[-2mm]
           & \; \MENV' = \MENV \cup \set{\locreg{\loc}{\reg} \mapsto \concreteloc{\reg}{0}{}}
         \end{aligned}
        }
  }
}

\newcommand{\rdletloctag}{
  \inferrule*[lab={\;\text{[\dletloctag]}}]
  {}{\STOR;\MENV;\letloc{\locreg{\loc}{\reg}}{\locreg{\loc'}{\reg} + 1}{\EXPR} \stepsto \STOR;\MENV';\EXPR
    \\\\ {
          \begin{aligned}
            \text{where} \;
%            & \; \locreg{\loc_f}{\reg} \; \fresh ;
%              \; \EXPR' = \subst{\EXPR}{\locreg{\loc}{\reg}}{\locreg{\loc_f}{\reg}} \\[-2mm]
            & \; \MENV' = \MENV \cup \set{\locreg{\loc}{\reg} \mapsto \concreteloc{\reg}{i+1}{}} ;
              \; \concreteloc{\reg}{\ind}{} = \MENV(\locreg{\loc'}{\reg})
           \end{aligned}
         }
}}

\newcommand{\rdletlocafter}{
  \inferrule*[lab={\;\text{[\dletlocafter]}}]
  {}{\STOR;\MENV;\letloc{\locreg{\loc}{\reg}}{\afterl{\tyatlocreg{\TYP}{\loc_1}{\reg}}}{\EXPR} \stepsto \STOR;\MENV';\EXPR
    \\\\ {
          \begin{aligned}
            \text{where} \;
%            & \; \locreg{\loc_f}{\reg} \; \fresh ;
%              \; \EXPR' = \subst{\EXPR}{\locreg{\loc}{\reg}}{\locreg{\loc_f}{\reg}} \\[-2mm]
            & \; \MENV' = \MENV \cup \set{\locreg{\loc}{\reg} \mapsto \concreteloc{\reg}{j}{}} ;
              \; \concreteloc{\reg}{\ind}{} = \MENV(\locreg{\loc_1}{\reg}) \\[-2mm]
            & \; \ewitness{\TYP}{\concreteloc{\reg}{\ind}{}}{\STOR}{\concreteloc{\reg}{\indj}{}}
           \end{aligned}
         }
  }
}

\newcommand{\rdcase}{
  \inferrule*[lab={\;\text{[\dcase]}}]
  {}{
    \STOR;\MENV;\case{\concreteloc{\reg}{\ind}{\locreg{\loc}{\reg}}}{[\ldots, \caseclause{\datacon{\DC}{}{(\overharpoon{\var : \tyatlocreg{\TYP}{\loc}{\reg}})}}{\EXPR}, \ldots]}
    \stepsto
    \\\\
    \STOR;\MENV';\subst{\EXPR}{\overharpoon{\var}}{\concreteloc{\reg}{\overharpoon{w}}{\overharpoon{\locreg{\loc}{\reg}}}}
    \\\\ {
         \begin{aligned}
           \text{where} \;
%           & \; \overharpoon{\locreg{\loc_f}{\reg'}} \; \fresh ;
%             \; \EXPR'' = \subst{\EXPR'}{\overharpoon{\locreg{\loc'}{\reg'}}}{\overharpoon{\locreg{\loc_f}{\reg'}}} \\[-2mm]
           & \; \MENV' = \MENV \cup \set{\overharpoon{\locreg{\loc}{\reg}_1} \mapsto \concreteloc{\reg}{\ind + 1}{}, \ldots , \overharpoon{\locreg{\loc}{\reg}_{j+1}} \mapsto \concreteloc{\reg}{\overharpoon{w_{j+1}}}{}} \\[-2mm]
%           & \; \EXPR' = \subst{\EXPR}{\overharpoon{\var}}{\concreteloc{\reg}{\overharpoon{w}}{\overharpoon{\locreg{\loc}{\reg}}}}\\[-2mm]
           & \; \ewitness{\overharpoon{\TYP_1}}{\concreteloc{\reg}{\ind + 1}{}}{\STOR}{\concreteloc{\reg}{\overharpoon{w_1}}{}} \\[-1mm]
           & \; \ewitness{\overharpoon{\TYP_{j+1}}}{\concreteloc{\reg}{\overharpoon{w_j}}{}}{\STOR}{\concreteloc{\reg}{\overharpoon{w_{j+1}}}{}} \\[-1mm]
           & \; \DC = \STOR(\reg)(\ind); \; \indj \in \set{1 , \ldots , \litnum - 1} ;
             \; \litnum = | \overharpoon{\var : \hTYP} | \\
%           & \; \DC = \STOR(\reg')(\ind) \\[-2mm]
         \end{aligned}
       }
  }
}

\newcommand{\rdletexp}{
  \inferrule*[lab={\;\text{[\dletexp]}}]
              {\STOR;\MENV;\EXPR_1 \stepsto \STOR';\MENV';\EXPR'_1 \\ \EXPR_1 \neq \VAL}
              {\STOR;\MENV;\letpack{\var : \hTYP}{\EXPR_1}{\EXPR_2} \stepsto
               \STOR';\MENV';\letpack{\var : \hTYP}{\EXPR'_1}{\EXPR_2}}
}

\newcommand{\rdletval}{
  \inferrule*[lab={\;\text{[\dletval]}}]
    {}{\STOR;\MENV;\letpack{\var : \hTYP}{\VAL_1}{\EXPR_2} \stepsto \STOR;\MENV;\subst{\EXPR_2}{\var}{\VAL_1}} 
}

\newcommand{\rdapp}{
  \inferrule*[lab={\;\text{[\dapp]}}]
    {}{\STOR;
      \MENV;
      \fapp{\overharpoon{\locreg{\loc}{\reg}}}{\overharpoon{\VAL}}
      \stepsto \STOR;
      \MENV;
      \subst{\EXPR}{\overharpoon{\var}}{\overharpoon{\VAL}} \subst{}{\overharpoon{\locreg{\loc'}{\reg'}}}{\overharpoon{\locreg{\loc}{\reg}}}
      \\\\{
        \begin{aligned}
          \text{where} \;
%          & \; \EXPR' =  \\[-2mm]
          & \; \FD = Function(f) \\[-2mm]
          & \; \fvar : \forall _{\overharpoon{\locreg{\loc'}{\reg'}}}.
          \overharpoon{\hTYP_{\fvar}} \ARROW \hTYP_{\fvar} ; (\fvar \overharpoon{\var} = \EXPR) = Freshen(\FD)
        \end{aligned}
      }
    }
}

\newcommand{\rdletregion}{
  \inferrule*[lab={\;\text{[\dletregion]}}]
    {}{\STOR;\MENV;\letreg{\reg}{\EXPR} \stepsto \STOR;\MENV;\EXPR}
}


%\usepackage[usenames,dvipsnames]{xcolor}
%\usepackage[dvipsnames]{xcolor}
%\usepackage[]{xcolor}

% [2016.10.23] Switching to margin notes for peanut gallery:
\usepackage{todonotes}
\presetkeys{todonotes}{inline}{}  % Activate to turn-off margin notes.

% [2016.05.26] This can just be a growing list from paper to paper:
\newcommand{\rn}[1]{\pgwrapper{RRN}{#1}}  % Ryan Newton

\newcommand{\tlm}[1]{\pgwrapper{TLM}{#1}} % Trevor
\newcommand{\bjs}[1]{\pgwrapper{BJS}{#1}} % Bo Joel Svenson
\newcommand{\chc}[1]{\pgwrapper{CHC}{#1}} % Chao-Hong Chen
\newcommand{\vc}[1]{\pgwrapper{VC}{#1}} % Vikraman Choudhury
\newcommand{\bc}[1]{\pgwrapper{BC}{#1}} % Buddhika Chamith
\newcommand{\mv}[1]{\pgwrapper{MV}{#1}} % Michael Vollmer
\newcommand{\mav}[1]{\pgwrapper{MV}{#1}} % Michael Vollmer
\newcommand{\osa}[1]{\pgwrapper{OSA}{#1}} % Omer Agacan
\newcommand{\mm}[1]{\pgwrapper{MM}{#1}} % Madan Musuvathi
\newcommand{\rgs}[1]{\pgwrapper{RGS}{#1}} % Ryan Scott
\newcommand{\csk}[1]{\pgwrapper{CSK}{#1}} % Chaitanya Koparkar

\newcommand{\mk}[1]{\pgwrapper{MK}{#1}} % Milind Kulkarni
\newcommand{\ls}[1]{\pgwrapper{LS}{#1}} % Laith Sakka

\newcommand{\spall}[1]{\pgwrapper{SS}{#1}} % Sarah Spall

\newcommand{\jd}[1]{\pgwrapper{JD}{#1}} % Joe Devietti
\newcommand{\onl}[1]{\pgwrapper{ONL}{#1}} % Omar Navarro-Leija

\newcommand{\mr}[1]{\pgwrapper{MR}{#1}} % Mike Rainey

% Add co-authors here or in a downstream file.

\InputIfFileExists{activateeditingmarks}{
}{
    \def\noeditingmarks{}
}

\definecolor{comment-red}{rgb}{0.8,0,0}
\definecolor{dark-green}{rgb}{0.0,0.4,0}
\definecolor{dark-blue}{rgb}{0.0,0.0,0.55}
\definecolor{very-dark-green}{rgb}{0.0,0.3,0}

\ifx\noeditingmarks\undefined
   %% This means someone else should take a look:
   \newcommand{\auditme}[1]{{\color{dark-green}{#1}}}

   %% This meants there's a serious problem that must be fixed before
   %% the paper goes out:
   \newcommand{\fixme}[1]{{\color{comment-red}{#1}}}

   %% This is for outlining, i.e., psuedotext that must be converted
   %% to real text:
   \newcommand{\note}[1]{{\begin{itemize}\item \textcolor{blue}{NOTE: #1} \end{itemize}}}

   %% A very poor-man's version of ``track changes'':
   \newcommand{\new}[1]{{\color{blue}{#1}}}

   \newcommand{\const}[1]{\textred{#1}}
   %\definecolor{mygrey}{rgb}{0.6,0.6,0.6}
   \definecolor{mygrey}{rgb}{0.7,0.7,0.7}

   \newcommand{\textred}[1]{{\color{comment-red}{#1}}}

   % Old version was inline:   
   % \newcommand{\pgwrapper}[2]{\textred{(#1: #2)}}
   % New version is in the margin:
   \setlength{\marginparwidth}{1.6cm}
   \newcommand{\pgwrapper}[2]{\todo[size=\scriptsize]{#1: #2}}

   %% RRN: Dim the background for Ryan's eyes:
   %% Hack: do this ONLY on Ryan's machine.
   %%  I don't know a way based on user or host name, so...
   %%  instead I just signal on a file, activategreybg.tex which I
   %%  don't then checkin.
   \InputIfFileExists{activategreybg}{
       \definecolor{comment-red}{rgb}{0.5,0,0}
       \pagecolor{mygrey}
   }{}

  % When they are their own para, todonotes annoyingly take up vertical space:
  \newcommand{\rnfloat}[1]{\pgwrapper{RRN}{#1}\vspace{-4mm}}  % Ryan Newton

\else
%   \newcommand{\textred}[1]{#1}
% Keep it on so we catch stuff to fix:
   \newcommand{\textred}[1]{{\color{comment-red}{#1}}}
   \newcommand{\auditme}[1]{#1}
   % This is more severe and should not go away:
   \newcommand{\fixme}[1]{\textred{#1}}
   \newcommand{\pgwrapper}[2]{}
%   \newcommand{\note}[1]{}
   \newcommand{\note}[1]{{\begin{itemize}\item {\textcolor{blue}{#1}} \end{itemize}}}
   % \newcommand{\new}[1]{#1}
   \newcommand{\new}[1]{{\color{blue}{#1}}}
   \newcommand{\const}[1]{#1}

   \newcommand{\rnfloat}[1]{}  % Ryan Newton
\fi

% Just an alias because sometimes we use this:
\newcommand{\Red}[1]{\textred{#1}}

% These probable should be elsewhere:
\newcommand{\ie}{{i.e.,}}
\newcommand{\eg}{{e.g.,}}
\newcommand{\etal}{\textit{et al.}}

\def\scrunchit{}
\ifx\scrunchit\undefined
  \newcommand{\captionscrunch}{}
\else
%  \newcommand{\captionscrunch}{\vspace{-4mm}}
  \newcommand{\captionscrunch}{\vspace{-3mm}}
\fi

%% \newcommand{\secref}[1]{section~\ref{#1}}
%% \newcommand{\Secref}[1]{Section~\ref{#1}}
\newcommand{\secref}[1]{\S\ref{#1}}
\newcommand{\Secref}[1]{\secref{#1}}

%% \newcommand{\Figref}[1]{Figure~\ref{#1}}
%% \newcommand{\figref}[1]{figure~\ref{#1}}
\newcommand{\Figref}[1]{Fig.~\ref{#1}}
\newcommand{\figref}[1]{Fig.~\ref{#1}}
\newcommand{\tabref}[1]{Table~\ref{#1}}


\newcommand{\appendixref}[1]{\ref{#1}}
\newcommand{\Appendixref}[1]{\ref{#1}}


%% \input{haskell_style}
%% \input{lsthaskell.sty}
\usepackage{upquote}

\lstnewenvironment{code}
    {%\lstset{style=haskell}%
      \lstset{escapechar={\@},mathescape=true}
      \lstset{language=Haskell}
      \lstset{
        morekeywords={letloc},
        deletekeywords={Int,read},
        literate=
        {\\\\}{{\char`\\\char`\\}}1
        {>->}{>->}3
        {>>=}{>>=}3
        {->}{{$\rightarrow$}}2
        {->.}{{$\multimap$}}2
        {>=}{{$\geq$}}2
        {<-}{{$\leftarrow$}}2
        {<=}{{$\leq$}}2
        {=>}{{$\Rightarrow$}}2
        {|}{{$\mid$}}1
        {forall}{{$\forall$}}1
        {exists}{{$\exists$}}1
        {...}{{$\cdots$}}3
      }
      \csname lst@SetFirstLabel\endcsname}
    {\csname lst@SaveFirstLabel\endcsname}
    {}
%% \lstset{style=haskell}
\lstdefinestyle{inline}{%
  basicstyle=\footnotesize\ttfamily,
  upquote=true,
    keywordstyle=[1],
    keywordstyle=[2],
    keywordstyle=[3],
    keywordstyle=[4],
    escapechar={\@},
    mathescape=true,
    literate=
        {\\}{{$\lambda$}}1
        {Lambda}{{$\Lambda$}}1        
        {\\\\}{{\char`\\\char`\\}}1
        {>->}{>->}3
        {>>=}{>>=}3
        {->}{{$\rightarrow$\space}}3    % include forced space
        {>=}{{$\geq$}}2
        {<-}{{$\leftarrow$}}2
        {<=}{{$\leq$}}2
        {=>}{{$\Rightarrow$}}2
        {|}{{$\mid$}}1
%        {~}{{$\sim$}}1
        {forall}{{$\forall$}}1
        {exists}{{$\exists$}}1
        {...}{{$\cdots$}}3
}

% \newcommand{\inl}[1]{\lstinline[style=inline];#1;}
\newcommand{\il}[1]{\lstinline[style=inline,mathescape=true];#1;}

\newcommand{\makeatcode}{\lstMakeShortInline[style=inline]@}
\newcommand{\makeatchar}{\lstDeleteShortInline@}

\begin{document}
\maketitle
\acceptancepage

% This page is optional
%\copyrightpage


% This page is optional but generally included

\begin{acknowledgments}
  Need to thank lots of people.
  Thank Victoria Lewis (of course!), family.
  Definitely thank committee (Ryan Newton, Jeremy Siek, Sam Tobin-Hochstadt, Larry Moss).
  PL Wonks crowd at IU, especially Dan.
  Thank people at Cal State Sacramento.
  Mention NSF funding?

This is the (optional) acknowledgments page, which is designed to recognize people or agencies to whom you feel grateful for any academic, technical, financial, or personal aid in the preparation of your thesis. As a matter of courtesy, you should ordinarily mention the members of your committee here, as well as institutions that provided funding.
\end{acknowledgments}

% This page is optional

\begin{dedication}
This is the (optional) dedication page. Per Graduate School standards, this page should appear with no title and should be centered horizontally and vertically.
\end{dedication}

% This page is optional

\begin{preface}
This is the (optional) preface page which can be used if you wish. This page should appear after the dedication (or acknowledgements page if there is no dedication page) and before the abstract page.
\end{preface}

% % This page is required

\begin{abstract}
This abstract page is required. Make sure to adhere to the word count and other limits set by ProQuest. It should appear in your dissertation that is submitted to the Graduate School \emph{without} signatures.
\end{abstract}

\newpage

% This page is required

\tableofcontents

\chapter{Introduction}\label{chapter:intro}

Many programs running today use heap object representations that are {\em fixed}
by the language runtime system.
% , which are not free to vary based on the workload.
For instance, the Java or Haskell runtimes dictate an object layout, and the
compiler must stick to it for all programs.
%
In contrast, when humans optimize a program, one of their primary {\em levers on
performance} is changing data representation.  For example, an HPC programmer
knows how to pack a regular tree into a byte array for more efficient
access~\cite{makino90,goldfarb13sc,Meyerovich2011}.

Furthermore, whenever a program receives data from the network or
disk, rigid insistence on a particular heap layout causes an impedance mismatch
we know as {\em deserialization}.
%
%% \note{Compilers normally dictate a standard heap representation for data
%%   manipulated by programs.  Yet substantial performance gains can be had by
%%   operating on data directly in a denser format: the serialization format
%%   already used by incoming data~\cite{}, or a custom format designed by the
%%   programmer, as in HPC applications~\cite{}.}
%
At first glance, the only alternative would seem to be writing low-level code to
deal directly with specialized or serialized data layouts, an error-prone way to
achieve performance optimization at the expense of safety and readability.

In my thesis I propose a way to ameliorate both of these concerns: reify {\em
data layout} as an explicit part of the program. I will introduce a language,
\ourcalc (which stands for {\em location calculus}), whose type system directly
encodes a byte-level layout for algebraic datatypes manipulated by the language.
%
A well-typed program consists of functions, data definitions, {\em and}
data representation choices, which
% {Well-typed programs are self-describing with respect to their data-layout},
% and that layout
can then be tailored to an application.
%
This means that programs can operate over {\em densely}
encoded (serialized) data in a type-safe way.
%
%% \rn{Somewhat related work: high-level synthesis (hardware) languages that
%%    describe bit-level data using advanced type systems.}

{If data resides on disk in a \ourcalc-compatible format}, it becomes
possible to {\em bring the program to the data} rather than the traditional
approach of bending the data to the code: deserializing it to match the rigid
heap format of the language runtime.
% Its type system ensures that \ourcalc working directly with densely encoded data,
{This effort contrasts with earlier work on persistent
languages~\cite{persistent-java,persistent-objects-thor} and object databases~\cite{object-fault-handling},
which sought to expand the mutable heap to encompass disk as well memory,
translating (swizzling) between persistent pointers and in-memory pointers.  Instead, the
emphasis here is on processing immutable data, and eschewing pointers entirely
wherever possible.}

%% \auditme{Type-safe handling of encoded data applies to \ourcalc programs written by hand,
%% but we also explore synthesizing \ourcalc programs {\em automatically} from
%% vanilla functional programs that say nothing about data layout.}
%% \rn{I think we should perhaps make this less LoCal centric since it is more
%%   HiCal centric once we call it ``Gibbon2''.}

The layout of a \ourcalc data constructor by default takes only one (unaligned)
byte in memory and fields may be referred to \emph{either} by pointer
indirections or unboxed into the parent object (serialized). This allows
programmers to interpolate between fully serialized and fully pointer-based
representations.
%
\ourcalc can thus serve as a flexible intermediate representation for compilers
or synthesis tools.
% , and I will show how it was used as the intermediate language
% for the Gibbon compiler.

Gibbon is one such compiler. Gibbon is an experimental compiler that
automatically transforms high-level functional programs that operate on
tree-like data into low-level C code that operates directly on serialized data.
Internally, Gibbon represents programs in \ourcalc, so it is able to tune the
memory layout of particular data structures to control precisely how and how
much the data will be serialized. In general, Gibbon produces code that is often
significantly faster than equivalent pointer-based code, with speedups of over
$2\times$ compared to hand-optimized, pointer-based C code, and often much more
than that compared to existing compilers for functional languages like Haskell
and OCaml. In addition to presenting \ourcalc, in this thesis I will cover
the design and implementation of the Gibbon compiler, as well as an evaluation
of the compiler on various benchmarks and applications.

\mav{TODO summarize some applications of Gibbon here}

This thesis will be broken down into three parts. \Cref{chapter:intro} (this
chapter) gives an overview of \ourcalc, with background, motivation, and related
work. \Cref{chapter:local} describes the language and presents a formal
semantics for \ourcalc, as well as some extensions. Finally,
\Cref{chapter:gibbon} presents the Gibbon compiler, which uses \ourcalc
internally, and presents an evaluation of Gibbon's performance on various tasks.
A full proof of type safety for the \ourcalc language is included in \Cref{appendix:proof}.


%% \section{Background}\label{sec:bg}
% ================================================================================

%% \mav{Many programs batch-process input data from disk or network---parsing
%% logs and reading serialized data formats such as JSON or protobufs.  These
%% programs spend significant time simply \emph{deserializing} data into a
%% memory format that the program can directly work with: for example,
%% pointer-based object graphs in the Java heap. Recently, researchers and
%% industry practitioners have begun to look (again) at unifying internal and
%% external memory representations \cite{cnf-icfp15,capnproto} and}

\section{Motivation}\label{sec:intro-motivation}

Consider the simple tree data structure in \Figref{fig:haskell_tree}, written in
a language that supports algebraic datatypes, where a tree is either a leaf with
an integer or a node with two trees.

\begin{figure}
\begin{subfigure}{\textwidth}
\begin{code}
-- A Tree is either:
--  - a Leaf with an Int, or
--  - a Node with a Tree and a Tree
data Tree = Leaf Int | Node Tree Tree
\end{code}
\caption{Haskell data type representing a binary tree}
\label{fig:haskell_tree}
\end{subfigure}
\begin{subfigure}{\textwidth}
\begin{code}
sum :: Tree -> Int
sum t = case t of
          Leaf n   -> n
          Node x y -> (sum x) + (sum y)
\end{code}
\caption{Function that sums leaf values in a binary tree}
\label{fig:haskell_sumtree}
\end{subfigure}
\begin{subfigure}{\textwidth}
\begin{lstlisting}[language=C++]
int sumPacked (byte * &ptr) {
  int ret = 0;
  if (* ptr == LEAF) {
    ptr++; // skip past leaf tag
    ret = * (int*)ptr; // retrieve integer from leaf
    ptr += sizeof(int); //skip past integer
  } else { // tag must be node
    ptr++; // skip past node tag
    ret += sumPacked(ptr); // sum left sub-tree
    ret += sumPacked(ptr); // sum right sub-tree
  }
  return ret;
}
\end{lstlisting}
\caption{A low-level traversal of serialized tree data written in C++}
\label{fig:cpp-example}
\end{subfigure}

\end{figure}

In memory, each node in this tree is either a \il{Leaf} node, typically
consisting of a header word (denoting that it is a \il{Leaf}) and another word
holding the integer data, or an interior \il{Node}, consisting of a header word
and {\em two} double words (on a 64-bit system) holding pointers to its
children. A tree with 2 internal nodes and 3 leaf nodes, then, occupies 64 bytes
of space (20 bytes per internal node and 8 bytes per leaf node), even though it
contains only 12 bytes of ``useful'' data.  Storing the pointers that maintain
the internal structure of the tree represents a significant storage overhead.

\mav{Include a picture, maybe from previous papers, of a sample tree vs a sample
  byte array. Also use that particular as an ongoing example.}

When relying on the usual pointer-based representation, this data
%In memory, this data structure can easily be represented using pointers from
%\il{Node}s to their children, and
can be readily traversed using standard idioms to perform computations such as summing all the leaf values, as shown in \Figref{fig:haskell_sumtree}.


\mav{TODO: include C++ pointer-based sum function}

But when represented on disk or sent over the wire, the same tree structure
would not preserve pointers from a node to its children. Instead, the tree
would be {\em serialized}, with the \il{Node}s and \il{Leaf}s of the tree laid
out in a buffer in some sequential order. For example, the tree could be
linearized in a left-to-right preorder, containing {\em tags} to mark data
constructors and atomic fields such as integers, but ditching the pointers.
Because it contains no pointers, this serialized representation is
significantly more compact.
%
But without this structural information, in most settings the
pre-order serialization would be deserialized prior to processing,
requiring more code than the simple \il{sum} function above.

However, this deserialization
is not necessary---it is perfectly possible to write code that performs
the same \il{sum} operation directly on the serialized representation.
All that is necessary is for the code to visit every node in the tree, skipping over
tags and \il{Node} data, and accumulating leaves into the \il{sum}.
%
This traversal can be accomplished in existing languages, writing low-level
buffer-processing code as in the C++ code given in \Figref{fig:cpp-example}.

Essentially, this code operates as follows: \il{ptr} scans along the packed
data structure. For each node type it encounters, it continues scanning
through the node, retrieving the data it needs from the packed representation
(in the case of \il{Leaf}s, the integer, in the case of \il{Node}s,
nothing) and performing the necessary computation. Because this serialized
representation is already in left-to-right preorder, no pointer-like
accesses are necessary: scanning sequentially through the buffer suffices to
access all the nodes of the tree. Note that the \il{sumPacked} function is
still recursive; the program stack helps capture the tree structure of the
data.
%While this isn't strictly necessary for this function, other, more
%complicated traversals may rely on reconstructing the structure.

\begin{figure}
  \begin{subfigure}{\textwidth}
\begin{code}
add1 :: Tree -> Tree
add1 tr =
  case tr of
    Leaf n -> Leaf (n + 1)
    Node a b -> Node (add1 a) (add1 b)
\end{code}
    \caption{add1 in haskell}\label{fig:add1-haskell}
  \end{subfigure}

  \begin{subfigure}{\textwidth}
\begin{lstlisting}[language=C++]
char* add1(char* tin, char* tout) {
  if (*tin == Leaf) {
    *tout = Leaf;
    tin++; tout++;
    *(int*)tout = *(int*)tin + 1;
    return (tin + sizeof(int));
  } else {
    *tout = Node;
    tin++; tout++;
    char* t2 = add1(tin,tout);
    tout += (t2 - t);
    return add1(t2,tout);
  }
}
\end{lstlisting}
    \caption{add1 in c++}\label{fig:add1-cpp}
  \end{subfigure}
\caption{add1 example}
\end{figure}

% \mav{TODO: fix up and include add1 tree example here}
As another example, consider a function that traverses a binary tree as defined
previously, returns a new binary tree such that each integer has been incremented
by one. This is again straightforward to describe recursively in a language
like Haskell. \Figref{fig:add1-haskell}.

Like before, we can write a C++ function to do this same computation, only
operating on and returning serialized binary trees. This new function will
have to accept two arguments (pointers to the input and output byte arrays,
respectively), and will additionally return a pointer. \Figref{fig:add1-cpp}


% \begin{comment}
There are several advantages to working directly on serialized data: the
serialized representation can take many times fewer bytes to represent than
a normal pointer-based representation; data can be traversed faster once in
memory due to predictable memory accesses; \emph{and} data can be read from
disk without deserialization (e.g. via \texttt{mmap}).
%
%% This motivates contemporary efforts to unify internal and
%% external memory representations like Cap'N Proto~\cite{capnproto}
%% and GHC CNF~\cite{cnf-icfp15}.

However, working directly with serialized data is not always easy. First,
programs written with typical pointer-based representations benefit from
standard techniques, such as type checking, to help programmers avoid errors
while constructing traversals of their data structures (so, e.g., type
checking can prevent a programmer from reading an integer value out of an
interior node of the tree, or from visiting the children of a leaf node). But
operations on serialized representations provide no such protection: all of the
data in the tree is packed into a flat buffer that is traversed using cursors.
Cursors need to be manipulated carefully to visit
the necessary portions of the buffer---skipping over the sections that are not
needed---and read out the appropriate data, all without the safety net of a
type checker. Hence, writing code to work directly on the serialized data can
be tedious and error-prone.

I propose instead to write the above example in a language, \emph{\ourcalc},
expressly designed to use dense serializations for its values. The \ourcalc
\il{sum} function in \Figref{fig:gibbon_sumtree} extends the simple functional
one above with region and location annotations.

\begin{figure}
\begin{code}
sum : forall @\locreg{l}{r}@ . @\tyatlocreg{Tree}{l}{r}@ -> Int
sum [@\locreg{l}{r}@] t = case t of
              Leaf (n : @\tyatlocreg{Int}{l_n}{r}@ ) -> n
              Node (a : @\tyatlocreg{Tree}{l_a}{r}@) (b : @\tyatlocreg{Tree}{l_b}{r}@)
               -> (sum [@\locreg{l_a}{r}@] a) + (sum [@\locreg{l_b}{r}@] b)
\end{code}
\caption{Function that sums the leaf values of a serialized binary tree}
\label{fig:gibbon_sumtree}
\end{figure}

\mav{TODO: include packed add1 gibbon function}

This code operates on serialized data, taking locations of that data (input and
output) as additional function arguments.  It is a {\em region-polymorphic}
function that performs a traversal within region $r$ that contains serialized
data.  Well-typedness ensures that it only reads memory in a type-safe way.
%
Location variables ($\locreg{l}{r}$) have lexical scopes and are introduced as
function arguments and pattern matches.
%
For instance, in the above program, we cannot
access child node locations ($\locreg{l_a}{r}$,$\locreg{l_b}{r}$) until we correctly parse the input
data at $l^r$ and ascertain that represents an intermediate node.
%
Conversely, as I will show later, to \emph{construct} data the type system must
enforce that adjacent fields be serialized consecutively.


\section{Background and Related Work}\label{sec:bg}

Many libraries exist for working with serialized data, and a few
% The library approach is more common, with
% CNF Cap'N Proto
% But compiler-based approaches like Gibbon are not the only option.  There are also
make it easier to use serialized data as in-memory data, or to export the
host-language's pre-existing in-memory format as external data.
% Specifically, we compare our approach to
Cap'N Proto\footnote{An ``insanely fast data interchange format,''
  \url{https://capnproto.org/}, \cite{capnproto}},
is designed to eliminate encoding/decoding by
standardizing on a new binary format for use in memory as well on disk/network.
%% Cap'N Proto allows intra-message pointers, represented as 30-bit relative offsets, but
%% without cycles, and with a unique parent for each node (trees, not DAGs).
%
% Additionally, we compare against
%
{\em Compact Normal Forms} (CNF)~\cite{cnf-icfp15} is
a feature provided by the Glasgow Haskell Compiler since release 8.2.
%
%% This is a
%% feature reminiscent of Lisp and Self systems which save heap snapshots\footnote{For
%%   instance, you can save the Self ``world'' to a \il{.snap} file (\url{selflanguage.org}).  In the Lisp lineage, Chez
%%   Scheme, before version 9, is an example of a system that could save/restore heap
%%   files (\url{www.scheme.com/csug8}).},
%% %
%% except more targeted.
%
The idea is that any purely functional value, once fully evaluated, can be {\em compacted}
into its own region of the heap
--- capturing a transitive closure of its reachable heap.
%% objects, and turning a value of type \il{T} into a \il{Compact T}.
After compaction, the CNF can be stored externally and loaded
back into the heap later.
%
{Persistent languages tackle the problem of automatically moving data between disk
and in-memory representations~\cite{persistent-java,persistent-object-systems,persistent-objects-thor}, and can swizzle pointers as part of this
translation to create more efficient representations. However, like CNF, these
representations still maintain pointers, so cannot realize the full advantage of
our system.}

If we look instead at compiler support for computing with data in dense or
external forms, there are many compilers for stream processing
languages~\cite{streamit,wavescript-nsdi}---or restricted languages such as
XPath~\cite{xpath-streams}---that generate efficient computations over data
streams.
These are somewhat related, but \ourcalc differs in targeting general
purpose recursive functions over algebraic datatypes.
%
%% In this category, the main published approach is our prior work on Gibbon~\cite{ecoop17-gibbon},
%% which compiles idiomatic programs to operate
%% on serialized data.  However, the Gibbon approach described previously can only handle fully
%% serialized data and thus introduces asymptotic slowdowns as we'll see in the
%% next section.  (At the time, we considered adding indirections as future work but
%% they were not part of the compiler.)
%% %
%% Also, the prior Gibbon compiler had no analogue to our location calculus: no way for a
%% type-system to enforce correct handling of regions and locations within
%% serialized data---which provides a much stronger foundation for building such compilers.

The problem of computing \emph{without} deserializing can be viewed as a
\emph{{fusion/deforestation}} problem: to fuse the compute loop with the
deserialize loop.  But traditional deforestation
approaches~\cite{wadler-deforestation}, don't rise to being able to handle
a full deserializer, and popular approaches based on more restrictive {\em combinator
libraries}~\cite{stream-fusion} are less expressive than \lamadt and \ourcalc.

\emph{Ornaments} are a body of theory regarding connections between related
data structure that differ based on additions or
reorganization~\cite{ornaments}.
%
Indeed, \ourcalc's addition of offset fields to data {\em is} ornamentation.
%
Practical implementations of ornaments~\cite{ornament-ml} provide support for
lifting functions across types related by ornaments, transforming the code.
%
However, the isomorphism between a datatype and its serialized form is not an
ornament, and thus lifting functions across that isomorphism is not supported.

%% \ourcalc does not allow construction of cyclic values (only DAGs), so it is less
%% related to graph-processing systems.


Finally, \ourcalc relates to a broader literature on optimizing
tree-traversing programs and heap representations.
%
There has been significant work in optimizing the layout and
traversal patterns of tree data structures for performance reasons.
%
The most closely related line of work is \emph{cache-conscious structure
  layout}~\cite{chilimbi1999}, which proposes a data layout scheme that lays out
the nodes of a tree according to an order determined by a provided traversal
function. Because this layout is determined by a specific traversal function,
it serves a similar purpose to the linearization of data in our packed layout:
when trees are traversed in the same manner as the layout order, spatial
locality is improved. Note, however, that Chilimbi et al.'s approach does not
change the internal structure of objects, nor the code that traverses those
data structures. Hence, all pointers are preserved, and this approach does not
offer the additional benefits of our packed layout such as denser accesses and
avoidance of pointer indirection. Other spatial locality
work~\cite{Truong1998,Lattner2005,Chilimbi1999a} has similar effects and
limitations to cache-conscious structure layout.

\emph{Automatic pool allocation}~\cite{Lattner2005} proposes
allocating disjoint data structures into disjoint partitions of
memory, and this approach can be extended with compression
optimization~\cite{Lattner2005mspc}. Because a data structure is
allocated into an isolated pool, ``internal'' pointers that connect
nodes of that data structure definitely do not access arbitrary memory
locations, and hence can use narrower bit widths to save space. Unlike
the spatial locality work discusses in the previous paragraph, this
compression optimization both shrinks the overall representation of
the data structure (as in our packed representation) and utilizes
compiler rewrites to do so (as in our compiler transformations).

Tree linearization, and the attendant changes required to traversal code, are
common in high-performance computing settings, especially for
vectorization~\cite{makino90,goldfarb13sc,Meyerovich2011,ren13cgo,ren14taco}.
These approaches, by closely matching data structure layout to the traversal
behavior of specific applications, can eliminate many pointer dereferences,
compress data structures, and simplify traversal implementations. However, all
of these approaches are programmer-directed: either \emph{ad hoc},
application-specific
implementations~\cite{makino90,goldfarb13sc,Meyerovich2011}, or driven by
library functions that the programmer must exploit~\cite{ren13cgo,ren14taco}.




\chapter{The Location Calculus}\label{chapter:local}

\section{Overview}

This section describes \ourcalc{}, which is a programming language
for programming with serialized data. A primary use case of \ourcalc{}
is as an \emph{intermediate language} for a compiler.
%
Traditional compilers are built on a number of well-defined intermediate
abstractions and translations that close the semantic gap between source and
target. If we are building a compiler to generate code that operates on
serialized data, what we need are analogous way-points to structure compilers
that target serialized-data traversals (stream-processors, essentially). Indeed,
there is quite a semantic gap between the low-level, buffer-mutating,
pointer-bumping programs, and a source language of high-level, pure, recursive
functions on algebraic datatypes. \ourcalc{} is designed to structure the space
between, where types are augmented to track \emph{locations} within regions
(\eg{} byte offsets).

\ourcalc{} follows in the tradition of typed assembly language~\cite{TAL},
region calculi~\cite{mlkit-retrospective}, and Cyclone~\cite{cyclone-pldi}
in that it uses types to both
expose and make safe low-level implementation mechanisms.
%
%% Ultimately, our compiler output will increment pointers into memory buffers, as
%% seen in code of the previous section.
%
The basic idea of \ourcalc{} is to first establish what data \emph{share} which
logical memory regions (essentially, buffers), and in what \emph{order} those data reside,
abstracting the details of computing exact addresses.
%
For example, data constructor applications, such as \lstinline{Leaf 3}, take an extra
location argument in \ourcalc{}, specifying where the data constructor
should place the resulting value in memory:
% starts in memory, that is, where the data
% constructor itself is written:
\lstinline[mathescape]{Leaf $\;\locreg{l}{}\;$ 3}.
%
This location becomes part of type of the value: \lstinline[mathescape]{$\,\tyatlocreg{Tree}{\,l}{}$}.
Every location resides in a region, and when we want to
name that region, we write $\locreg{l}{r}$.


Locations represent information about where values are in a store, but
are less flexible than pointers. They are introduced {relative} to other
locations. A location variable is either \emph{after} another
variable, or it is at the beginning of a region,
thus specifying a serial order.
%so they essentially describe what \emph{order} values are in within a region.
If location $l_2$ is declared as
\lstinline[mathescape]{$l_2$ = after($\tyatlocreg{Tree}{l_1}{\reg}$)},
%% where $x$ is a tree at $l_1$,
then $l_2$ is \emph{after} every element of the tree
rooted at $l_1$.

{\emph{Regions} in \ourcalc{} represent the memory buffers containing serialized data
  structures.  Unlike some other region calculi, in \ourcalc{}, values in a
  region \emph{may} escape the static scope which binds and allocates that region. In
  fact, an extension introduced later in~\secref{subsec:indirections}
  specifically relies on inter-region pointers and coarse-grained garbage
  collection of regions.}

%% In this paper, we present a new method for building compilers to lift functions
%% over serialized data.
%% We also present a new calculus, \ourcalc{}, which corresponds to a simplified
%% presentation of the core intermediate language of our compiler.
%% %
%% It was designed both to form a foundation for the optimizations we describe in
%% this paper, generalizing previous work on region calculi to represent recursive
%% programs operating on serialized data. Crucially, in addition to associating
%% values with \emph{regions}, in \ourcalc{} values are associated with a

% As shown in the grammar in \figref{fig:grammar},
\ourcalc{} is a first-order,
call-by-value functional language with algebraic datatypes and pattern
matching. Programs consist of a series of datatype definitions, function
definitions, and a main expression.
%
%% The algebraic data types work essentially like one would expect, but with
%% additional machinery to handle \emph{where} a particular value is in the store
%% and the \emph{relative position} of its fields in the store.  Every invocation
%% of a data constructor is annotated with a location variable ---
%
\ourcalc{} programs can be written directly by hand, and \ourcalc{} also serves
as a practical {\em intermediate language} for other tools or front-ends that
want to convert computations to run on serialized data (essentially fusing a
consuming recursion with the deserialization loop).
%
%% We return to this use-case in \secref{sec:impl-hical}.

\paragraph{Allocating to output regions}

Now that we have seen how data constructor applications are parameterized by
locations, let us look at a more complex example than those of the prior section.
Consider \lstinline[mathescape]{buildtree}, which constructs the same trees consumed by \lstinline[mathescape]{sum}
and \lstinline[mathescape]{rightmost} above.  First, in the source language without locations:

%% \floatstyle{plain}\restylefloat{figure}
%% \begin{figure}
%% \centering
\begin{code}
buildtree : Int -> Tree
buildtree n = if n == 0
              then Leaf 1
              else Node (buildtree (n - 1))
                        (buildtree (n - 1))
\end{code}
%% \vspace{-1mm}
%% \caption{Example: Building an output data structure}
%% \label{fig:buildtree}
%% \vspace{-5mm}
% \end{figure}
% \mav{TODO: vertically align the stmts}
%
Then in {\ourcalc}, where the type scheme binds an output rather than input location:
% \floatstyle{plain}\restylefloat{figure}
%\vspace{-1mm}
%\floatstyle{boxed}\restylefloat{figure}
%\begin{figure}[h]
%\centering
\begin{code}
buildtree : forall @\locreg{l}{r}@ . Int -> @\tyatlocreg{Tree}{l}{r}@
buildtree [@\locreg{l}{r}@] n =
  if n == 0 then (Leaf @\locreg{l}{r}@ 1) -- write tag + int to output
  else -- skip past tag:
       letloc $\locreg{l_a}{r}$ = $\locreg{l}{r}$ + 1 in
       -- build left in place:
       let left : @\tyatlocreg{Tree}{l_a}{\reg}@ =
           buildtree [@\locreg{l_a}{r}@] (n - 1) in
       -- find start of right:
       letloc $\locreg{l_b}{r}$ = after(@\tyatlocreg{Tree}{l_a}{\reg}@) in
       -- build right in place:
       let right : @\tyatlocreg{Tree}{l_b}{\reg}@ =
           buildtree [@\locreg{l_b}{r}@] (n - 1) in
       -- write datacon tag, connecting things together:
       (Node @\locreg{l}{r}@ left right)
  \end{code}
%\end{figure}
%\vspace{-3mm}

\noindent
Here, we see that \ourcalc{} must represent locations that have {\em not yet
  been written}, \ie{} they are output destinations.  Nevertheless, in the
recursive calls of \lstinline[mathescape]{buildtree} this location is passed as an argument: a form
of destination-passing style~\cite{destination-passing}.
The type system guarantees that memory will be initialized and written exactly once.
%
The output location is threaded through the recursion to build the left subtree,
and then offset to compute the starting location of the right subtree.
%
%% Computing \lstinline[mathescape]{after($\tyatlocreg{Tree}{l_a}{\reg}$)} is a potentially expensive operation, and
%% implementing it---or rearranging the program to avoid needing it---is the
%% responsibility of the \ourcalc implementation (\secref{sec:impl-local}).
{It might appear that computing
\lstinline[mathescape]{after($\tyatlocreg{Tree}{l_a}{\reg}$)} could be quite expensive,
if there is a large tree at that location. This does not need to be the case.
In~\secref{sec:impl-local} I will present different techniques for efficiently
compiling \ourcalc programs without requiring linear walks through serialized data.}

%% Later
%% (in~\secref{sec:route-ends}), we describe a compilation technique that
%% makes obtaining the \lstinline[mathescape]{after} of a location in cases like this
%% straightforward by having computations return the ending locations of
%% whatever trees they traverse.
%
%% Furthermore, while the above \lstinline[mathescape]{buildTree} code writes the left and right subtrees {\em
%%   before} writing the data constructor tag into the output, we will want the
%% compiler to eventually reorder these writes to achieve a linear memory access
%% pattern on the output region.

%% \rn{May want to modify the discussion of ``size information'' below.}
{One of the goals of \ourcalc{} is to support several compilation
  strategies. One extreme is compiling programs to work with a representation of
  data structures that do not include \emph{any} pointers or indirections at
  run-time---within such a representation, the size of a value can be observed
  by threading through ``end witnesses'' while consuming packed values: for
  example, \lstinline[mathescape]{buildtree} above would \emph{return} $\locreg{l_b}{r}$, rather than computing
  it with an \lstinline[mathescape]{after} operation.
  %
  (The end-witness strategy was first used in
  {\em Gibbon}~\cite{ecoop17-gibbon} prior to the design of \ourcalc{},
  which previously compiled functions on fully serialized data,
  while not preserving asymptotic complexity.)
  %% %% Our language allows for this by ensuring that
  %% %% when location variables are introduced after values, a variable binding the
  %% %% preceeding value must be in scope.
  %% %
  %% In \secref{sec:compiler}, we will discuss in detail how the invariants
  %% enforced by \ourcalc{} make subsequent transformations and optimizations
  %% more tractable.
}
%
%% In practice, as a whole-program compiler, our compiler will decide
%% whether or not particular data constructor needs {\em random-access}
%% based on how it is used in the program. The analyses required for this
%% are described in more detail in \secref{sec:compiler}.
%
%% \mav{Forward reference whatever section where we add back limited random-access
%%   capabilities.}
%
%% Furthermore, while our compiler does translate every program into
%% \ourcalc{}, later stages in the compiler will relax or violate the
%% invariants enforced by \ourcalc{}, such as the absolute ordering of
%% values in a region.
%% %
%% In fact, by code generation time the compiler will
%% have likely transformed the program to include explicit size
%% information and inter-region pointers.
%
Next, I will present a formalized core subset of \ourcalc,
%% in more detail
%% (\secref{subsec:grammar}),
its type system (\secref{subsec:static}),
and its operational semantics (\secref{subsec:dynamic}).
% which uses a store to  represent regions and locations.
%% before moving on to implementation (\secref{sec:impl-local}, \secref{sec:impl-hical})
%% and evaluation (\secref{sec:eval}).

\section{Formal Language and Grammar}
\label{subsec:grammar}

\begin{figure}
  \begin{displaymath}
  \begin{aligned}
  &\DC \in \; \textup{Data Constructors},\:\: \TYP \in \; \textup{Type Constructors},\\
  &\var, \yvar,\fvar \in \; \textup{Variables},\:\:
  \loc, \locreg{l}{r} \in \; \textup{Symbolic Locations},\\
  &\reg \in \; \textup{Regions},\:\: \ind, \indj \in \; \textup{Region Indices},\\
  &\concreteloc{r}{i}{l} \in \; \textup{Concrete Locations}
  \end{aligned}
\end{displaymath}
\begin{displaymath}
  \begin{aligned}
    \textup{Top-Level Programs} && \PROG && \gramdef & \overharpoon{\DD} \;; \overharpoon{\FD} \;; \EXPR \\
    \textup{Datatype Declarations} && \DD && \gramdef & \DATA\;\TYP = \overharpoon{\DC \; \overharpoon{\sTYP}\;} \\
    \textup{Function Declarations} && \FD && \gramdef & \fvar : \TS ; \fvar \overharpoon{\var} = \EXPR \\
    %\textup{Data Location} && \RP && \gramdef & \loc^{\reg} \\
%    \textup{Types} && \TYP && \gramdef & \ldots \gramor \TC \gramor \ldots \\
    %\gramor \ldots \\ % maybe don't even talk about int, is it confusing to have non-packed data?
    \textup{Located Types} && \hTYP && \gramdef & \tyatlocreg{\TYP}{\loc}{\reg} \\
    \textup{Type Scheme} && \TS && \gramdef &
      \forall _{\overharpoon{\locreg{l}{r}}}.
      \overharpoon{\hTYP} \ARROW \hTYP \\
    \textup{Values} && \VAL && \gramdef & \var \gramor \concreteloc{\reg}{\ind}{\locreg{\loc}{\reg}}\\
    \textup{Expressions} && \EXPR && \gramdef & \VAL\\[-5pt]
    && && \gramor & \fapp{\overharpoon{\locreg{l}{r}}}{\overharpoon{\VAL}} \\
    && && \gramor & \datacon{\DC}{\keywd{\locreg{\loc}{\reg}}}{\overharpoon{\VAL}}\\
    %% && && \gramor & \datacon{\DC}{\keywd{\locreg{\loc}{\reg}}}{\overline{\var}} \gramor \litcon{\locreg{\loc}{\reg}}{\EXPR} \\
    && && \gramor & \letpack{\var:\hTYP}{\EXPR}{\EXPR} \\
    && && \gramor & \letloc{\locreg{\loc}{\reg}}{\LE}{\EXPR} \\
    && && \gramor & \letreg{\reg}{\EXPR} \\
    && && \gramor & \case{\VAL}{\overharpoon{\pat}} \\
    \textup{Pattern} && \pat && \gramdef &
    \caseclause{\datacon{\DC}{}{(\overharpoon{\var : \hTYP})}}{\EXPR} \\
    \textup{Location Expressions} && \LE && \gramdef
    & \startr{\reg} \\
    && && \gramor & (\locreg{\loc}{r} + 1) \\
    && && \gramor & \afterl{\hTYP}
  \end{aligned}
\end{displaymath}

  \caption{Grammar of \ourcalc{}}
  \label{fig:grammar}
\end{figure}

\Figref{fig:grammar} gives the grammar for a formalized core of \ourcalc{}.
%
Iuse the notation $\overharpoon{x}$ to denote a vector $[
x_1, \ldots, x_n]$, and $\overharpoon{x_{\ind}}$ the item at position
$\ind$.
%
To simplify presentation, the language supports
algebraic datatypes without any base primitive types, but could be extended in a straightforward
manner to represent primitives such as an $\sgramwd{Int}$ type or tuples.
%
The expression language is based on the first-order lambda calculus,
using A-normal form.
%
The use of A-normal form simplifies our formalism and proofs
without loss of generality.
%in ways that do not limit generality.
%It would be straightforward to generalize to direct style.

Like previous work on region-based memory~\cite{regioncalcs},
\ourcalc{} has a special binding
form for introducing region variables, written as
$\sgramwd{letregion}$.
%
Location variables are similarly introduced by $\sgramwd{letloc}$.
%
The pattern-matching form $\sgramwd{case}$ binds variables to
serialized values, as well as binding the location for each variable.
%
% To simplify the formalism,
It is required that each bound location in
a source program is unique.
%
% This convention rules out programs with space leaks caused by
%shadowing of \gramwd{letloc}-bound names.
% \rn{I can't easily construct hypothetical example...}

The $\sgramwd{letloc}$ expression binds locations in only three ways:
a location is either the \emph{start} of a region (meaning, the
location corresponds to the very beginning of that region), is
immediately after another location, or it occurs \emph{after} the last
position occupied by some previously allocated data constructor.
%
For the last case, the location is written to exist at
$\afterl{\tyatlocreg{\TYP}{\loc}{\reg}}$, where $\loc$ is already
bound in a region, and has a value written to it.

%% \begin{comment}
%% Second, the convention enables the dynamic semantics to treat symbolic
%% locations at runtime as a source of dynamically fresh names (i.e., a
%% gensym-like mechanism), which enables a simpler environment structure
%% for tracking locations.
%% %
%% It is straightforward to relax these requirements.
%% \end{comment}

Values in \ourcalc{} are either (non-location) variables or
\emph{concrete locations}.
%
In contrast to bound location variables, concrete locations
do not occur in source programs; rather, they appear at runtime, created by the
application of a data constructor, which has the effect of
extending the store.
%
Every application of a data constructor writes a \emph{tag} to the store, and
concrete locations allow the program to navigate through it.
%
To distinguish between concrete locations and location variables in
the formalism, I refer to the latter as \emph{symbolic locations}.
%
A concrete location is a tuple
$\concreteloc{\reg}{\ind}{\loc}$ consisting of a region, an index, and
symbolic location corresponding to its binding site.
%
The first two components are sufficient to fully describe
an \emph{address} in the store.

%% \begin{comment}
%% Rather than annotating expressions with their region (such as
%% $\sEXPR\; @ \;\sreg$) like standard region calculi, \emph{location
%% annotations} appear at various points in the program source, such as
%% in types, in function declarations, and at the site of constructor
%% application.
%% %
%% %% As a first-order language, functions are only defined at the top
%% %% level. Each function has a name, a \emph{type scheme}, an argument,
%% %% and an expression (the body of the function).  The type scheme
%% %% consists of the types of the input and output of the function,
%% %% parameterized by \emph{data locations} (location and region
%% %% pair).
%% %
%% %% {Function application similarly requires a list of data
%% %% locations.}
%% %
%% Function types are only introduced at the top level, and these
%% functions may only be polymorphic with respect to data locations.
%% \end{comment}
%
%% \mv{Explain that locations of inputs to functions must have been written,
%% and locations of outputs to functions cannot have been written.}
%
%% Data locations are annotated with an arrow indicating whether they are
%% input ($\inloc$) or output ($\outloc$) locations, which essentially is intended to represent
%% whether the location will be associated with a value that is coming
%% into the function as an input or a value that will be computed and
%% returned from the function as an output. These annotations occur both
%% in the declarations of functions and in the application of functions;
%% in the former case, as parameters for the function, and in the latter
%% case, as arguments to the function. In the case where data locations
%% occur as arguments, the locations must be in scope, and their
%% annotation of input or output must match their state at that point
%% (so, input locations have been already written to, output locations have not).

%% We elide the superscript $r$ on location variables ($\sRP$ in the
%% grammar) when that information is not relevant.
%

%% \begin{comment}
%% Naturally, a location's region matches the one it's derived from: in
%% \il{letloc $\;l^r\;$ = start $\;r$}, the locations $r$
%% match.
%% %
%% These regions model logically continuguous memory buffers that are
%% growable at one end.  This is similar to traditional region
%% systems, except that allocating to, \eg{} an MLKit region happens in
%% units of complete objects (with pointers), whereas we permit leaving
%% objects ``partially written'' for arbitrarily long periods of time ---
%% separating writing the tag of a data constructor from serializing out
%% its fields.
%% %
%% Indeed, the entire purpose of using a region system in \ourcalc{} is
%% to group objects together so that their representations can be merged
%% and compressed by ``inlining'' pointers.
%% \end{comment}

%% Pattern matching, in the form of $\sgramwd{case}$ statements, are
%% familiar from other functional languages. Each alternative in the
%% pattern match statement matches on a constructor, its location, and a
%% series of variable and annotated type pairs corresponding to each
%% field of that constructor.

%% For this presentation of \ourcalc{}, we will not be too concerned with
%% including primitive operations on, e.g., numbers, strings, etc though
%% our full implementation obviously includes these things.  At certain
%% points in this paper we will give examples of \ourcalc{} programs that
%% make use of these extensions. Extending the presentation of \ourcalc{}
%% to include such types and operations is straightforward.
%% %
%% \rn{Doesn't need that much discussion, but could combine this point
%% with a bigger point about what syntactic sugar we will allow ourselves
%% in examples.}

\subsection{Static Semantics}
\label{subsec:static}


\begin{figure}
  \begin{displaymath}
  \begin{aligned}
    \textup{Typing Env.} && \TENV && \gramdef & \; \set{\var_1 \mapsto \hTYP_1, \; \ldots \; , \var_n \mapsto \hTYP_n} \\
    \textup{Store Typing} && \SENV && \gramdef & \; \set{\locreg{\loc}{r_1}_1 \mapsto \TYP_1, \; \ldots \; , \locreg{\loc}{r_n}_n \mapsto \TYP_n} \\      
    \textup{Constraint Env.} && \CENV && \gramdef & \;
      \set{\locreg{\loc}{r_1}_1 \mapsto \LE_1, \; \ldots \; , \locreg{\loc}{r_n}_n \mapsto \LE_n} \\
    \textup{Allocation Pointers} && \AENV && \gramdef & \; \set{r_1 \mapsto ap_1, \; \ldots \; , r_n \mapsto ap_n} \\
    && && & {\text{where} \; ap = \locreg{\loc}{\reg} \gramor \emptyset} \\
    \textup{Nursery} && \NENV && \gramdef & \; \set{\locreg{\loc}{r_1}_1, \; \ldots \; , \locreg{\loc}{r_n}_n}
  \end{aligned}
\end{displaymath}

  \caption{Extended grammar of \ourcalc{} for static semantics}
  \label{fig:typegrammar}
\end{figure}
\begin{figure}
  \footnotesize
  \begin{mathpar}
    \rtvar{}\hspace{1em}
    \rtconcreteloc{}\\
    \rtlet{}\hspace{1em}
    \rtlregion{}\\
    \rtlltag{}\hspace{1em}
    \rtllstart{}\\
    \rtllafter{}\hspace{1em}
    \rtdatacon{}
  \end{mathpar}
  \normalsize
  \caption{Typing judgments for \ourcalc{} (1)}
  \label{fig:types1}
\end{figure}

%\floatstyle{boxed}\restylefloat{figure}
\begin{figure}
  \footnotesize
  \begin{mathpar}
    \rtapp{}\hspace{1em}
    \rtfunctiondef{}\\
    \rtpat{}\\
    \rtcase{}\hspace{1em}
    \rtprogram{}
  \end{mathpar}
  \normalsize
   \caption{Typing judgments for \ourcalc{} (2)}
   \label{fig:types2}

\end{figure}



In \figref{fig:typegrammar}, I extend the grammar with some extra
details necessary for describing the type system.
The typing rules for expressions in \ourcalc{} are given in
\figref{fig:types1} and \figref{fig:types2}, where the rule form is as follows:
%% $\RENV;\EENV;\CENV;\SENV;\TENV \vdash \EXPR : \TYP \tyatlocreg{\loc}{\reg} ; \RENV';\EENV'$.

\[ \TENV;\SENV;\CENV;\AENV;\NENV \vdash \AENV'; \NENV'; \EXPR : \hTYP \]

%

The five letters to the left of the turnstile are different environments.
$\TENV$ is a standard typing environment.
$\SENV$ is a store-typing environment, which maps all
\emph{materialized} symbolic
locations to their types. That is, every location in $\SENV$ {\em has been written}
and contains a value of type $\SENV(l^r)$.
$\CENV$ is a constraint environment, which keeps
track of how symbolic locations relate to each other.
$\AENV$ maps each region in scope to a location, and is used to symbolically
track the allocation and incremental construction of data structures;
$\AENV$ can be thought of as representing the
\emph{focus} within a region of the computation.
% (the exact usage of $\AENV$ will be expanded on below).
$\NENV$ is a nursery of all symbolic locations that have been allocated,
but not yet written to.
Locations are removed from $\NENV$
upon being written to, as the purpose is to prevent multiple writes to
a location.
Both $\AENV$ and $\NENV$ are threaded through the typing
rules, also occuring in the output (to the right of the
turnstile).

%% To give the typing rules, we first extend the grammar with some extra
%% details that are necessary for keeping track of location and region
%% variables. A \emph{location state} is a three-tuple of boolean values,
%% representing whether a location has been written to, whether the
%% location was introduced after some other location, and whether a
%% different location has been introduced after the location. An
%% environment maps locations to their location states.
%% %
%% The location state environment is \emph{threaded through} the
%% typing rules: it occurs as a premise (before the turnstile)
%% and alongside the type after the turnstile. This allows the typing
%% rules to encode \emph{changes} in the state of locations (for
%% example, recording that they have been written to).

The \textsc{\tvar} rule ensures that the variable is in scope, and
the symbolic location of the variable has been written to.
%
\textsc{\tconcreteloc} is very similar, and also just ensures that
the symbolic location has been written to.
%
\textsc{\tlet} is straightforward, but note that along with $\TENV$,
it also extends $\SENV$ to signify that the location $\loc$ has
materialized.

In \textsc{\tlregion}, extending $\AENV$ with an empty allocation pointer
brings the region $\reg$ in scope, and also indicates that
a symbolic location has not yet been allocated in this region.

There are three rules for introducing locations
(\textsc{\tllstart}, \textsc{\tlltag} and \textsc{\tllafter}, all shown in \Figref{fig:types1}),
corresponding to three ways of allocating a new location in a
region. A new location is either: at the start of a region, one cell
after an existing location, or after the data structure rooted at an
existing location. Introducing locations in this fashion sets up an
ordering on locations, and the typing rules must ensure that the
locations are used in a way that is consistent with this intended
ordering.
%
To this end, each such rule extends the constraint environment $\CENV$
with a constraint that is based on how the location was introduced,
and $\NENV$ is extended to indicate that the new location is in scope
and unwritten.

Additionally, the location-introduction rules use $\AENV$ to ensure that a program
must introduce locations in a certain pattern (corresponding to the
left-to-right allocation and computation of fields, as explained
in~\secref{subsec:dynamic}).
%% Recall from earlier, constructing a data structure proceeds through a
%% series of steps: allocating a location for a tag, allocating a
%% location after the tag for the first field (if there are fields), then
%% for each field alternating materializing fields and allocating after
%% them, until finally writing the initial tag.
%
In $\AENV$, each region is mapped to either the right-most allocated
symbolic location in that region (if it is unwritten), or to the
symbolic location of the most recently materialized data structure.
This mapping in $\AENV$ is used by the typing rules to ensure that:
%
(1) \textsc{\tllstart} may only introduce a location at the start
of a region once;
%
(2) \textsc{\tlltag} may only introduce a location if an unwritten location has
just been allocated in that region (to correspond to the tag of some
soon-to-be-built data structure); and
%
(3) \textsc{\tllafter} may only introduce a location if a data structure has
just been materialized at the end of the region, and the programmer wants to
allocate \emph{after} it. To attempt, for example, to allocate the location of
the right sub-tree of a binary tree \emph{before} materializing the left sub-tree would
be a type error.
%
{Each location-introduction rule also ensures that the
introduced location must be written to at some point, by checking that
it's absent from the nursery after evaluating the expression.}

%% When a fresh
%% region $\reg$ is bound, $\AENV$ is extended to map that region
%% to $\emptyset$, indicating no locations have been allocated in it,
%% and so \textsc{\tllstart} may introduce a location at the start of $\reg$
%% and update $\AENV$ to note that a location has been allocated there.
%% For allocating a location intended as the first field after a tag,
%% \textsc{\tlltag} expects $\AENV$ to map to some location $\loc'$
%% which is also in $N$, because the tag of a dat

%% and each enforces that the location will be allocated in a region in a
%% specific way, extending the location constraint environment with
%% details of where the location was allocated relative to the rest of
%% the region. $\AENV$ and $\NENV$ are also extended to reflect this allocation.

In order to type an application of a data constructor, \textsc{\tdatacon} starts by
ensuring that the tag being written and all the fields have the correct type.
Along with that, the locations of all the fields of the constructor must
also match the expected constraints. That is, the location of the first field
should be immediately after the constructor tag, and there should be
appropriate $\mathit{after}$ constraints for other fields in the location constraint environment.
After the tag has been written, the location $\loc$ is removed from the
nursery to prevent multiple writes to a location.
%
{
%
As mentioned earlier, LoCal uses destination-passing style. To guarantee destination-passing
style, it suffices to ensure that a function returns its value in a location passed from its caller.
The LoCal type system enforces this property by using constraints of the form $l' \neq l$ in the premises
of the typing rules of the operations that introduce new locations
}

{As demonstrated by \textsc{\tdatacon}, the type system enforces a
  particular ordering of writes to ensure the resulting tree is
  serialized in a certain order. Some interesting patterns are
  expressible with this restriction (for example, writing or reading
  multiple serialized trees in one function), and, as I will address
  shortly in \secref{subsec:indirections}, \ourcalc is flexible enough
  to admit extensions that soften this restriction and allow for
  programmers to make use of more complicated memory layouts.}

{A simple demonstration of the type system is shown in
Table~\ref{table:types-example}, which tracks how $\AENV$, $\CENV$,
and $\NENV$
% (allocation point, constraints, nursery)
change after each line in a simple expression that builds
a binary tree with leaf children. Introducing $\loc$ at the top
establishes that it is at the beginning of $\reg$,
$\AENV$ maps $\reg$ to $\loc$, and $\NENV$ contains $\loc$.
The location for the left sub-tree,
$\loc_a$, is defined to be $+ 1$ after it, which updates $\reg$ to
point to $\loc_a$ in $\AENV$ and adds a constraint to $\CENV$ for
$\loc_a$. Actually constructing the \texttt{Leaf} in the next line
removes $\loc_a$ to $\NENV$, because it has been written to. Once $\loc_a$
has been written, the next line can introduce a new location $\loc_b$
\emph{after} it, which updates the mapping in $\AENV$ and adds a
new constraint to $\CENV$. Once $\loc_b$ has been written and removed
from $\NENV$ in the next line, the final \texttt{Node} can be constructed,
which expects the constraints to establish that $\loc$ is before $\loc_a$,
which is before $\loc_b$.}

%% Additional rules (such as for function application, pattern matching)
%% are conventional%
%% \iftoggle{EXTND}{
%% and are in the Appendix,~\appendixref{subsec:types2}.
%% }{
%% and are available in the Appendix of the extended version~\cite{LoCal-tr}
%% }

%% Two rules (\textsc{\tcase}, \textsc{\tpat}) cover pattern matching. \ldots

%% \mav{We could also potentially explain the $\loc \neq \loc_i$ bit in \tpat.}

%All the other typing judgements are straightforward.

To finish out the typing rules, \Figref{fig:types2} contains rules for
function application and definition, as well as pattern matching.
%
Function application in \textsc{\tapp} ensures the location of the result of
the application is initially unwritten, and is considered written afterward.
%
Types and locations for the function are pulled from the function signature.
%
Pattern matching is handled by \textsc{\tcase} and \textsc{\tpat}, which
are straightforward.
%
The final rule type checks a whole program, consisting of datatype
and function definitions.

To simplify the formalism and proofs, I restricted typing rules
somewhat so that, in effect, the rules restrict well-typed expressions
so that they can return only the the result of a freshly allocated
constructor application.
%
Consequently, it is not possible, for instance, to type the following
expression, because the right-hand side is a value and, as such, does
not allocate.
%
\begin{code}
let x : @\tyatlocreg{T}{l}{r}@ = y in ...
\end{code}
%
This restriction is enforced by there being an assertion of the form
$\locreg{\loc}{\reg} \in \NENV$ in the premise of the typing rules of
the non-value expressions, such that $\tyatlocreg{\TYP}{\loc}{\reg}$
is the result type of the given expression.
%
Lifting this restriction is conceptually straightforward, but would
require either added complexity to the substitution lemma or the use
of a different factoring of the grammar and typing rules.
%
Similarly, our formalism and proofs could be extended to treat
primitive types, such as ints, bools, tuples, etc., as well as with
offsets and indirections in data constructors, with some conceptually
straightforward extensions to the formalism.


%% \setlength{\tabcolsep}{2pt}
%% \floatstyle{plaintop}\restylefloat{table}

\begin{table}[]
  \centering
%% \begin{adjustbox}{width=0.75\linewidth,center}

\begin{tabular}{llll}
  Code & $\AENV$ & $\CENV$ & $\NENV$ \\ \hline
  \begin{code}
letloc $\locreg{l}{r}$ =
    start($r$)
  \end{code} & $\{r \mapsto \locreg{l}{r} \}$ & $\emptyset$ & $\{\locreg{l}{r}\}$ \\
  \begin{code}
letloc $\locreg{l_a}{r}$ = $\locreg{l}{r}$ + 1
  \end{code} &
  $\{r \mapsto \locreg{l_a}{r} \}$ &
  $\{ \locreg{l_a}{r} \mapsto \locreg{l}{r} + 1 \}$  &
  $\{\locreg{l}{r},\locreg{l_a}{r}\}$ \\
  \begin{code}
let x : @\tyatlocreg{T}{l_a}{r}@ =
    Leaf $\locreg{l_a}{r}$ 1
  \end{code} &
  $\{ r \mapsto \locreg{l_a}{r} \}$ &
  $\{ l_a \mapsto \locreg{l}{r} + 1 \}$ &
  $\{\locreg{l}{r}\}$ \\
  \begin{code}
letloc $\locreg{l_b}{r}$ =
    after(@\tyatlocreg{T}{l_a}{r}@)
  \end{code} &
  $\{ r \mapsto \locreg{l_b}{r} \}$ &
  \makecell[cl]{$\{ \locreg{l_a}{r} \mapsto \locreg{l}{r} + 1,$\\\;$\locreg{l_b}{r} \mapsto \mathit{after}(\tyatlocreg{T}{l_a}{r})\}$} &
  $\{ \locreg{l}{r}, \locreg{l_b}{r} \}$ \\
  \begin{code}
let y : @\tyatlocreg{T}{l_b}{r}@ =
    Leaf $\locreg{l_b}{r}$ 2
  \end{code} &
  $\{ r \mapsto \locreg{l_b}{r} \}$ &
  \makecell[cl]{$\{ \locreg{l_a}{r} \mapsto l + 1,$\\\;$\locreg{l_b}{r} \mapsto \mathit{after}(\tyatlocreg{T}{l_a}{r})\}$} &
  $\{ \locreg{l}{r} \}$ \\
  \begin{code}
Node $\locreg{l}{r}$ x y
  \end{code} &
  $\{ r \mapsto \locreg{l}{r} \}$ &
  \makecell[cl]{$\{ \locreg{l_a}{r} \mapsto \locreg{l}{r} + 1,$\\\;$\locreg{l_b}{r} \mapsto \mathit{after}(\tyatlocreg{T}{l_a}{r})\}$} &
  $\emptyset$ \\
&&& \\
\end{tabular}
%% \end{adjustbox}
\caption{Step-by-step example of type checking a simple expression.}
\label{table:types-example}
\end{table}


\subsection{Dynamic Semantics}
\label{subsec:dynamic}
%\floatstyle{boxed}\restylefloat{figure}
\begin{figure}
  \begin{displaymath}
  \begin{aligned}
    \textup{Store} && \STOR && \gramdef & \set{\reg_1 \mapsto \heap_1 , \; \ldots \; , \reg_{n} \mapsto \heap_{n}} \\
    \textup{Heap} && \heap && \gramdef & \set{\ind_1 \mapsto \DC_1 , \; \ldots \; , \ind_{n} \mapsto \DC_{n}}\\
    \textup{Location Map} && \MENV && \gramdef & \set{\locreg{\loc}{\reg_1}_1 \mapsto \concreteloc{\reg_1}{\ind_1}{} , \; \ldots \; , \locreg{\loc}{\reg_n}_n \mapsto \concreteloc{\reg_n}{\ind_n}{}}
  \end{aligned}
\end{displaymath}

  \caption{Extended grammar of \ourcalc{} for dynamic semantics}
  \label{fig:opergram}
\end{figure}

\begin{figure}
  \small
  \begin{mathpar}
    \mprset{flushleft}
    \rddatacon{}

    \rdletlocstart{}\\
    \rdletloctag{}

    \rdletlocafter{}\\

    \rdletexp{}\\
    \rdletval{}

    \rdletregion{}\\
    \rdapp{}\\
    \rdcase{}\\
  \end{mathpar}
  \normalsize
  \caption{Dynamic semantics rules for \ourcalc{}}
  \label{fig:dynamic}
\end{figure}


The dynamic semantics for expressions in \ourcalc{} are given in
 \figref{fig:dynamic}, where the transition rule is as follows.
%
\begin{displaymath}
\STOR;\MENV;\EXPR \stepsto \STOR';\MENV';\EXPR'
\end{displaymath}
%
To model the behavior of reading and writing from an indexed
memory, the semantics make use of a \emph{store}, $\STOR$.
%
The store is a map from regions to \emph{heaps}, where each heap
consists of an array of \emph{cells}, which contain store values
(data constructor tags).
%
To bridge from symbolic to concrete locations, I use the
\emph{location map}, $\MENV$, to map symbolic locations
to concrete locations.
%

Case expressions are treated by the \textsc{\dcase{}} rule.
%
The objective of the rule is to load the tag of the constructor $\DC$
located at $\concreteloc{\reg}{\ind}{}$ in the store and dispatch
the corresponding case.
%
%% For simplicity, \ourcalc{} requires that all patterns are exhaustive,
%% and our type system ensures that the there always exists a constructor
%% tag at the specified location at the expected type.
%
The expression produced by the right-hand side of the rule is the body
of the pattern, in which all pattern-bound variables are replaced by
the concrete locations of the fields of the constructor $\DC$.

The concrete locations of the fields are obtained by the following
process.
%
If there is at least one field, then its starting address is the
position one cell after the constructor tag.
%
The starting addresses of subsequent fields depend on the sizes of the
trees stored in previous fields.

A feature of \ourcalc{} is the flexibility it provides to pick the
serialization layout.
%
Our formalism uses our \emph{end-witness rule} to abstract from different layout
decisions (for a more thorough explanation, see \secref{subsec:typesafety}).
%
Given a type $\TYP$, a starting address
$\concreteloc{\reg}{\ind_{s}}{}$, and store $\STOR$, the rule below
asserts that address of the end witness is
$\concreteloc{\reg}{\ind_{e}}{}$.
%
\begin{displaymath}
  \ewitness{\TYP}{\concreteloc{\reg}{\ind_{s}}{}}{\STOR}{\concreteloc{\reg}{\ind_{e}}{}}
\end{displaymath}
%
Using this rule, the starting address of the second field is obtained
from the end witness of the first, the starting address of the
third from the end witness of the second, and so on.

The allocation and finalization of a new constructor is achieved by
some sequence of transitions, starting with the \textsc{\dletloctag{}} rule, then
involving some number of transitions of the \textsc{\dletlocafter{}} rule,
depending on the number of fields of the constructor, and finally
ending with the \textsc{\ddatacon{}} transition.
%
The \textsc{\dletloctag{}} rule allocates one cell for the tag of some new
constructor of a yet-to-be determined type, leaving it to later to
write to the new location.
%
The resulting configuration binds its $\loc$ to the address
$\concreteloc{\reg}{\ind+1}{}$, that is, the address one cell past
given location $\loc'$ at $\concreteloc{\reg}{\ind}{}$.
%
Fields that occur after the first are allocated by the
\textsc{\dletlocafter{}} rule.
%
Here, its $\loc$ is bound to the address \concreteloc{\reg}{j}{} one
past the last cell of the constructor represented by its given
symbolic location $\loc_1$.
%
Like the \textsc{\dcase{}} rule, the required address is obtained by
application of end-witness rule to the starting address of the given
$\loc_1$ at the type of the corresponding field $\TYP$.
%
The final step in creating a new data constructor instance
is the \textsc{\ddatacon{}} rule.
%
It writes the specified constructor tag $\DC$ at the address in the
store represented by the symbolic location $\loc$.

The \textsc{\dletlocstart{}} rule for the \sgramwd{letloc} with \sgramwd{(start r)}.
expression binds the location to the starting address in the region
and starts running the body.

The \textsc{\dletexp{}} rule for let-expressions evaluates the let-bound
expression to a value and the \textsc{\dletval{}} rule substitutes the value for
the let-bound variable in the body.
%
The \textsc{\dapp{}} rule for function applications looks up the function by name
in the top-level environment and substitutes arguments for parameters
in the function body, substitutes argument symbolic locations for
parameter symbolic locations, then starts the resulting function body
running.
%
The \textsc{\dletregion{}} rule for the \sgramwd{letregion} expression binds the
new region and starts running the body.
%

The driver which runs an \ourcalc{} program initially loads all data
types, functions, type checks them, and if successful, then seeds the
$Function$, $\typeofcon$, and $\typeoffield$ environments.
%
Let $\EXPR_0$ be the main expression.
%
If $\EXPR_0$ type checks with respect to the \textsc{\tprogram{}} rule, then
the main program is safe to run.
%
The initial configuration for the machine with an empty store is
\begin{displaymath}
\emptyset; \set{\loc \mapsto \concreteloc{\reg}{0}{}}; \EXPR_0,
\end{displaymath}
which is, by itself, not particularly interesting or useful.
%
It is, however, straightforward to construct a type-safe initial configuration
whose store is nonempty, as long as the initial configuration
has a store that is well formed, as described in \Secref{sec:well-formedness}.
%
The program can start taking evaluation steps from this configuration.

\paragraph{Example} %: Allocating a Binary Tree.}
%
Consider this code snippet of \ourcalc{}.
%
\begin{code}
letloc @$\locreg{l_1}{\reg}$@ = @$\locreg{l_0}{\reg}$@ + 1 in
let a : @\tyatlocreg{Tree}{l_1}{r}@ = (Leaf @$\locreg{l_1}{\reg}$@) in
letloc @$\locreg{l_2}{\reg}$@ = (after (@\tyatlocreg{Tree}{l_1}{r}@)) in
let b : @\tyatlocreg{Tree}{l_2}{r}@ = (Leaf @$\locreg{l_2}{\reg}$@) in
Node @$\locreg{l_0}{\reg}$@ a b
\end{code}
%
Assume that the store starts out with a fresh heap, $\STOR = \set{\reg
  \mapsto \emptyset}$ and the location $\locreg{l_0}{\reg}$ maps to
$\concreteloc{\reg}{0}{}$ in the location map.
%
After stepping past the first line, the \textsc{\dletloctag{}} step has
allocated a cell for the tag of the interior node and bound the
location $\locreg{\loc_1}{\reg}$ to $\concreteloc{\reg}{1}{}$.
%
After the next line, the \textsc{\ddatacon{}} transition writes a leaf node to
the store at the address represented by $\locreg{\loc_1}{\reg}$:
$\STOR = \set{\reg \mapsto \set{1 \mapsto \mathtt{Leaf}}}$.
%
The second \gramwd{letloc} obtains the starting address for the second
leaf node by using end witness of the previous leaf node.
%
The write of the second leaf node appears in the store after
the next line, leaving the following store:
$
\STOR = \set{\reg \mapsto \set{1 \mapsto \mathtt{Leaf}, 2 \mapsto \mathtt{Leaf}}}$.
%
Finally, after the \textsc{\ddatacon{}} step taken for the last line, the store
contains the finalized allocation:
$\STOR = \set{\reg \mapsto \set{0 \mapsto \mathtt{Node}, 1 \mapsto \mathtt{Leaf}, 2 \mapsto \mathtt{Leaf}}}$.
%

The end-witness judgement of the new data constructor is the
following:
$\ewitness{\mathtt{Tree}}{\concreteloc{\reg}{0}{}}{\STOR}{\concreteloc{\reg}{3}{}}$
%
The judgement applies, in part, because, as expected, the tag at the
address $\concreteloc{\reg}{0}{}$ is a tag of type $\mathtt{Tree}$.
%
In addition, because the tag indicates an interior node with two
subtrees for fields, the judgement obligation extends to recursively
showing (1) that the end witness of the first leaf node (also at type
$\mathtt{Tree}$) at $\concreteloc{\reg}{1}{}$ has an end witness
(which is $\concreteloc{\reg}{2}{}$), (2) that the second field has an
end witness starting at the end witness of the first field, namely
$\concreteloc{\reg}{2}{}$, and ending at some higher address (which in
this case is $\concreteloc{\reg}{3}{}$), and (3) finally that the end
witness of the second field is the end witness of the entire
constructor, as is the case here.

\subsection{Type Safety}
\label{subsec:typesafety}

\mv{Overview of type safety of LoCal and why we should care.}
% Full type safety proof is give in the appendix of \cite{local-extended}.

\ourcalc{} is a \emph{type safe} language, and I will demonstrate that property
in this section. Some details of the proof are listed in \secref{appendix:proof},
such as an overview of notation and the complete cases for the lemmas and theorems.



% \subsubsection{Global environment and metafunctions}
% \begin{itemize}
% \item $Function(f)$: An environment that maps a function $f$ to its definition $\FD$.

% \item $Freshen(\FD)$: A metafunction that freshens all bound variables in function definition
% $\FD$ and returns the resulting function definition.

% \item $TypeOfCon(\DC):$ An environment that maps a data constructor to its type.

% \item $TypeOfField(\DC,i)$: A metafunction that returns the type of the \il{i}'th field
% of data constructor $\DC$.

% \item $ArgTysOfConstructor(\DC)$: An environment that maps a data constructor to its field types.

% \item $\allocptr{\reg}{\STOR}$: $\max \set{-1} \cup \set{ \indj \; | \; {(\reg \mapsto (\indj \mapsto \DC)) \in \STOR}}$.
% \end{itemize}


\subsubsection{Store typing}
\label{sec:well-formedness}
A key part of the safety of \ourcalc{} programs is the following property: if a
term $e$ is type $\tyatlocreg{\TYP}{\loc}{\reg}$, then if we look in the store
under region $\reg$ at the location represented by $\loc$, we will find the
start of a serialized data structure which is a valid serialization of a value
of type $\TYP$. In other words, the types tell us the truth about the values in
the store.
%
To achieve this, we rely on the store being \emph{well-formed}, whose definition
itself uses three other judgements (shown in Table~\ref{tbl:swf-judgements}).
%
% \begin{displaymath}
%   \storewf{\SENV}{\CENV}{\AENV}{\NENV}{\MENV}{\STOR}
% \end{displaymath}

\begin{table}
\bgroup
% \def\arraystretch{1.2}
% \setlength\tabcolsep{0.5cm}
\begin{tabular}{llp{8cm}}
 & \textbf{Judgement form} & \textbf{Summary}
 \\\\
\parbox[t]{3.5cm}{Store \\ well formedness} & $\storewf{\SENV}{\CENV}{\AENV}{\NENV}{\MENV}{\STOR}$ &

The store $\STOR$ along with location map $\MENV$ are well formed with respect to
typing environments $\SENV$, $\CENV$, and $\AENV$.
\\\\
End witness & $\ewitness{\TYP}{\concreteloc{\reg}{\ind_{s}}{}}{\STOR}{\concreteloc{\reg}{\ind_{e}}{}}$ &

The store address $\concreteloc{\reg}{\ind_{e}}{}$ is the position one
after the last cell of the tree of type $\TYP$ starting at
$\concreteloc{\reg}{\ind_{s}}{}$ in store $\STOR$.
\\\\
\parbox[t]{3.5cm}{Constructor-application \\ well formedness}
 & $\storewfcfa{\CENV}{\MENV}{\STOR}$ &

All in-flight data-constructor applications in store $\STOR$ along with location map $\MENV$
are well formed with respect to constructor-progress typing environment $\CENV$.
\\\\
\parbox[t]{3.5cm}{Allocation \\ well formedness} & $\storewfca{\AENV}{\NENV}{\MENV}{\STOR}$ &

Allocation in store $\STOR$ along with location map $\MENV$ is well formed
with respect to allocation-typing environments $\AENV$ and $\NENV$.
\end{tabular}
\egroup
\caption{Summary of judgements used to establish well formedness of the store.}
\label{tbl:swf-judgements}
\end{table}


% The store typing specifies three categories of invariants.
% %
% \begin{enumerate}
% \item The first enforces that allocations occur in the sequence
% specified by the constraint environment $\CENV$.
% %
% In particular, if there is some location $\loc$ in the domain of
% $\CENV$, then the location map and store have the expected
% allocations at the expected types.
% %
% For instance, if $(\loc \mapsto
% \afterl{\tyatlocreg{\TYP}{\loc'}{\reg}}) \in \CENV$, then $\loc'$ maps
% to $\concreteloc{\reg}{\ind_1}{}$ and $\loc$ to
% $\concreteloc{\reg}{\ind_2}{}$ in the location map, and $\ind_2$ is
% the end witness of $\ind_1$ at type $\TYP$ in the store, at region
% $\reg$.
% %
% \item The second category enforces that, for each symbolic location such
% that $(\loc \mapsto \TYP) \in \SENV$, there is some
% $\concreteloc{\reg}{\ind_1}{}$ for $\loc$ in the location map and
% $\ind_1$ has some end witness $\ind_2$ at type $\TYP$.
% %
% \item The final category enforces that each address in the store is written
% once.
% %
% This property is asserted by insisting that, if $\loc \in \NENV$, then
% there is some $\concreteloc{\reg}{\ind}{}$ for $\loc$ in the location
% map, but there is no write to for $\ind$ at $\reg$ in the store.
% %
% To support this property, there are two additional conditions which
% require that the most recently allocated location (tracked by
% to $\AENV, \NENV$) is at the end of its respective region.
% \end{enumerate}

% To support this behavior, there are two additional conditions in this
% category that enforce linearity of allocation.
% %
% The first one captures the situation when there is an allocation in
% flight, such as is the case in the postcondition of allocating
% a tag (and the corresponding binding the cell to $\loc$).
% %
% In this case, the typing rule specifies that $(\reg \mapsto \loc) \in
% A$ and $\loc \in \NENV$.
% %
% The requirement is that the address bound to $\loc$ is the maximum
% position in the location map for the region $\reg$ and that this
% position is past the end of any address already written in the store
% at $\reg$.
% The second condition captures the other situation, just after
% a field has been allocated and finalized, which is occurs
% when $(\reg \mapsto \loc) \in A$ and $(\loc \mapsto \TYP) \in \SENV$
% for the location $\loc$ representing the field.
% %
% In this case, the condition requires that the address of the end
% witness of the field is the maximum of all other addresses in the
% store at region $\reg$.




The definition of store well formedness follows.

\paragraph{Judgement form}

$\storewf{\SENV}{\CENV}{\AENV}{\NENV}{\MENV}{\STOR}$

The well-formedness judgement specifies the valid layouts of the store by using the location
map and the various environments from the typing judgement.
%
Rule~\ref{wf:map-store-consistency} specifies that, for each location in the store-typing environment,
there is a corresponding concrete location in the location map and that concrete location satisfies
a corresponding end-witness judgement.
%
Rules~\ref{wf:cfc} and~\ref{wf:ca} reference the judgements for well formedness concerning
in-flight constructor applications (\secref{sec:end-witness}) and correct allocation in
regions (\secref{sec:well-formedness-allocation}), respectively.
%
Finally, Rule~\ref{wf:impl1} specifies that the nursery and store-typing environments reference
no common locations, which is a way of reflecting that each location is either in the process
of being constructed and in the nursery, or allocated and in the store-typing environment, but
never both.

\paragraph{Definition}

\begin{enumerate}

    \item \label{wf:map-store-consistency} $ (\locreg{\loc}{\reg} \mapsto \TYP) \in \SENV \Rightarrow \\
            ((\locreg{\loc}{\reg} \mapsto \concreteloc{\reg}{\ind_1}{}) \in \MENV \wedge \\
            \ewitness{\TYP}{\concreteloc{\reg}{\ind_1}{}}{\STOR}{\concreteloc{\reg}{\ind_2}{}})
          $

    \item \label{wf:cfc} $\storewfcfa{\CENV}{\MENV}{\STOR}$

    \item \label{wf:ca} $\storewfca{\AENV}{\NENV}{\MENV}{\STOR}$

    \item \label{wf:impl1} $dom(\SENV) \cap \NENV = \emptyset $
\end{enumerate}


\subsubsection{End-Witness judgement}
\label{sec:end-witness}

\paragraph{Judgement form}

$\ewitness{\TYP}{\concreteloc{\reg}{\ind_{s}}{}}{\STOR}{\concreteloc{\reg}{\ind_{e}}{}}$

The end-witness judgement specifies the expected layout in the store of a fully
allocated data constructor.
%
Rule~\ref{ewitness:impl1} requires that the first cell store a constructor
tag of the appropriate type.
%
Rule~\ref{ewitness:impl2} specifies the address of the cell one past the tag.
%
Rule~\ref{ewitness:impl3} recursively specifies the positions of the constructor
fields.
%
Finally, Rule~\ref{ewitness:impl4} specifies that the end witness of
the overall constructor is the address one past the end of either the
tag, if the constructor has zero fields, or the final field,
otherwise.

\paragraph{Definition}

\begin{enumerate}
\item \label{ewitness:impl1} $\STOR(\reg)(\ind_s) = \DC'$ \; \text{such that} \\
      $\; \DATA\;\TYP = \overharpoon{\DC_1 \; \overharpoon{\sTYP}_1\;} \; | \; \ldots \; | \; \DC' \; \overharpoon{\sTYP}' \; | \; \ldots \; | \; \overharpoon{\DC_m \; \overharpoon{\sTYP}_m\;}$

\item \label{ewitness:impl4} $\overharpoon{w_0} = \ind_s + 1$

\item \label{ewitness:impl2}
  $\ewitness{\overharpoon{\TYP'_1}}{\concreteloc{\reg}{\overharpoon{w_0}}{}}{\STOR}{\concreteloc{\reg}{\overharpoon{w_1}}{}} \wedge$ \\
  $\ewitness{\overharpoon{\TYP'_{j+1}}}{\concreteloc{\reg}{\overharpoon{w_j}}{}}{\STOR}{\concreteloc{\reg}{\overharpoon{w_{j+1}}}{}}$
  \\ where $\indj \in \set{1,\ldots,n-1} ; n = | \overharpoon{\TYP'} |$

\item \label{ewitness:impl3}
  $\ind_e = \overharpoon{w_n}$
\end{enumerate}

\subsubsection{Well-formedness of constructor application}
\label{sec:well-formedness-constructors}

\paragraph{Judgement form}

$\storewfcfa{\CENV}{\MENV}{\STOR}$

The well-formedness judgement for constructor application specifies the various constraints
that are necessary for ensuring correct formation of constructors, dealing with constructor
application being an incremental process that spans multiple \ourcalc{} instructions.
%
Rule~\ref{wfconstr:constraint-start} specifies that, if a location corresponding to the
first address in a region is in the constraint environment, then there is a corresponding
entry for that location in the location map.
%
Rule~\ref{wfconstr:constraint-tag} specifies that, if a location corresponding to the address one past a constructor
tag is in the constraint environment, then there are corresponding locations for the address
of the tag and the address after in the location map.
%
Rule~\ref{wfconstr:constraint-after} specifies that, if a location corresponding to the address
one past after a fully allocated constructor application is in the constraint environment,
then there are corresponding locations for the address one past the constructor application
and for the address of the start of that constructor application in the location map, as well as the existence
of an end witness for that fully allocated location.

\paragraph{Definition}

\begin{enumerate}
    \item \label{wfconstr:constraint-start} $ (\locreg{\loc}{\reg} \mapsto \startr{\reg}) \in \CENV \Rightarrow \\
            (\locreg{\loc}{\reg} \mapsto \concreteloc{\reg}{0}{}) \in \MENV $

    \item \label{wfconstr:constraint-tag} $ (\locreg{\loc}{\reg} \mapsto (\locreg{\loc'}{\reg} + 1)) \in \CENV \Rightarrow
            \\
            (\locreg{\loc'}{\reg} \mapsto \concreteloc{\reg}{\ind_l}{})  \in \MENV \wedge \\
            (\locreg{\loc}{\reg} \mapsto \concreteloc{\reg}{\ind_l + 1}{})  \in \MENV
            $

    \item \label{wfconstr:constraint-after} $ (\locreg{\loc}{\reg} \mapsto \afterl{\tyatlocreg{\TYP}{\locreg{\loc'}{\reg}}{\reg}}) \in \CENV \Rightarrow \\
            ((\locreg{\loc'}{\reg} \mapsto \concreteloc{\reg}{\ind_1}{}) \in \MENV \wedge \\
            \ewitness{\TYP}{\concreteloc{\reg}{\ind_1}{}}{\STOR}{\concreteloc{\reg}{\ind_2}{}} \wedge \\
            (\locreg{\loc}{\reg} \mapsto \concreteloc{\reg}{\ind_2}{}) \in \MENV)
            $

\end{enumerate}


\subsubsection{Well-formedness concerning allocation}
\label{sec:well-formedness-allocation}

\paragraph{Judgement form}

$\storewfca{\AENV}{\NENV}{\MENV}{\STOR}$

The well-formedness judgement for safe allocation specifies the various properties
of the location map and store that enable continued safe allocation, avoiding in particular
overwriting cells, which could, if possible, invalidate overall type safety.
%
Rule~\ref{wf:impl-linear-alloc} requires that, if a location is in both the allocation
and nursery environments, i.e., that address represents an in-flight
constructor application, then there is a corresponding location in the location
map and the address of that location is the highest address in the store.
%
Rule~\ref{wf:impl-linear-alloc2} requires that, if there is an address in the allocation
environment and that address is fully allocated, then the address of that location is the
highest address in the store.
%
Rule~\ref{wf:impl-write-once} requires that, if there is an address in the nursery, then
there is a corresponding location in the location map, but nothing at the corresponding
address in the store.
%
Finally, Rule~\ref{wf:impl-empty-region} requires that, if there is a region that has been
created but for which nothing has yet been allocated, then there can be no addresses
for that region in the store.

\paragraph{Definition}

\begin{enumerate}
    \item \label{wf:impl-linear-alloc} $ ((\reg \mapsto \locreg{\loc}{\reg}) \in \AENV \wedge \locreg{\loc}{\reg} \in \NENV) \Rightarrow \\
          ((\locreg{\loc}{\reg} \mapsto \concreteloc{\reg}{\ind}{}) \in \MENV \wedge
%          \ind = \max \set{0} \cup \set{\indj \; | \; \concreteloc{\reg}{\indj}{} \in \MENV} \wedge \\
          \ind > \allocptr{\reg}{\STOR})
          $

    \item \label{wf:impl-linear-alloc2} $ ((\reg \mapsto \locreg{\loc}{\reg}) \in \AENV \wedge
    \, (\locreg{\loc}{\reg} \mapsto \concreteloc{\reg}{\ind_s}{}) \in \MENV \wedge \locreg{\loc}{\reg} \not \in \NENV \wedge
    \, \ewitness{\TYP}{\concreteloc{\reg}{\ind_s}{}}{\STOR}{\concreteloc{\reg}{\ind_e}{}}) \Rightarrow \\
          \ind_e > \allocptr{\reg}{\STOR}
          $

    \item \label{wf:impl-write-once} $ \locreg{\loc}{\reg} \in \NENV \Rightarrow \\
          ((\locreg{\loc}{\reg} \mapsto \concreteloc{\reg}{\ind}{}) \in \MENV \wedge \\
          (\reg \mapsto (\ind \mapsto \DC)) \not \in \STOR)
          $

    \item \label{wf:impl-empty-region} $(\reg \mapsto \emptyset) \in \AENV \Rightarrow \\
    \reg \not \in dom(\STOR)$
\end{enumerate}



\subsubsection{Type safety theorem}

The key to proving type safety is the handling of the store above. After that has been
established, type safety for \ourcalc{} can be proven in the standard way using
progress and preservation. The full details of the proof are shown in~\ref{appendix:proof},
but the statements of the main theorems are written here, as well as a handful
of representative cases.

\newcommand{\substlemmasubsts}[1]{\subst{#1}{\overharpoon{\var}}{\overharpoon{\VAL}} \subst{}{\overharpoon{\locreg{\loc}{\reg}}}{\overharpoon{\locreg{\loc'}{\reg'}}} \subst{}{\locreg{\loc}{\reg}}{\locreg{\loc'}{\reg'}}}


% \begin{lemma}[Substitution lemma]
%   \label{lemma:substitution}
%   \begin{align*}
%   \text{If} \quad & \TENV \cup \set{\overharpoon{\var_1} \mapsto \overharpoon{\TYP_1} \ensuremath{@} \overharpoon{\locreg{\loc_1}{\reg_1}}, \ldots, \overharpoon{\var_n} \mapsto  \overharpoon{\TYP_n} \ensuremath{@} \overharpoon{\locreg{\loc_n}{\reg_n}}}; \SENV; \CENV; \AENV; \NENV \vdash \AENV'; \NENV'; \EXPR : \tyatlocreg{\TYP}{\loc}{\reg}\\
%   \text{and} \quad & \TENV; \SENV'; \CENV'; \AENV'; \NENV' \vdash \AENV'; \NENV'; \overharpoon{\VAL_i} : \overharpoon{\TYP_i} \ensuremath{@} \overharpoon{\locreg{\loc'_i}{\reg'_i}} \qquad \; i \in \set{1, \ldots, n}\\
%   \text{then} \quad & \TENV; \SENV'; \CENV'; \AENV'; \NENV' \vdash \AENV'''; \NENV'''; \substlemmasubsts{\EXPR} : \tyatlocreg{\TYP}{\loc'}{\reg'}\\
%    \text{where} \quad & \SENV = \SENV_0 \cup \set{\overharpoon{\locreg{\loc_1}{\reg_1}} \mapsto \overharpoon{\TYP_1}, \ldots, \overharpoon{\locreg{\loc_n}{\reg_n}} \mapsto \overharpoon{\TYP_n}}\\
%    \text{and} \quad & \forall_{(\var \mapsto \TYP'' \ensuremath{@} \locreg{\loc''}{\reg''}) \in \TENV} . (\locreg{\loc''}{\reg''} \mapsto \TYP'') \in \SENV_0 \\
%   \text{and} \quad & dom(\SENV) \cap \NENV = \emptyset\\
%   \text{and} \quad & \NENV = \NENV_0 \cup \locreg{\loc}{\reg}\\
%   \text{and} \quad & \SENV' = \SENV \cup \set{\overharpoon{\locreg{\loc'_1}{\reg'_1}} \mapsto \overharpoon{\TYP_1}, \ldots, \overharpoon{\locreg{\loc'_n}{\reg'_n}} \mapsto \overharpoon{\TYP_n}}\\
%   \text{and} \quad & \CENV' = \subst{\CENV}{\overharpoon{\locreg{\loc}{\reg}}}{\overharpoon{\locreg{\loc'}{\reg'}}} \subst{}{\locreg{\loc}{\reg}}{\locreg{\loc'}{\reg'}} \\
%   \text{and} \quad & \AENV' = \subst{\AENV}{\overharpoon{\locreg{\loc}{\reg}}}{\overharpoon{\locreg{\loc'}{\reg'}}} \subst{}{\locreg{\loc}{\reg}}{\locreg{\loc'}{\reg'}} \subst{}{\reg}{\reg'}\\
%   \text{and} \quad & \NENV' = \subst{\NENV}{\locreg{\loc}{\reg}}{\locreg{\loc'}{\reg'}}
%   \end{align*}
% \end{lemma}
% \begin{nproof}
% The proof is by rule induction on the given typing derivation.
% \end{nproof}

\begin{lemma}[Progress]
  \label{lemma:progress_brief}
  \begin{displaymath}
  \begin{aligned}
  \text{if} \;\; & \emptyset;\SENV;\CENV;\AENV;\NENV \vdash \AENV';\NENV';\EXPR : \hTYP \\
  \text{and} \;\; & \storewf{\SENV}{\CENV}{\AENV}{\NENV}{\MENV}{\STOR} \\
  \text{then} \;\; & \EXPR \; \mathit{value} \\
  \text{else} \;\; & \STOR;\MENV;\EXPR \stepsto \STOR';\MENV';\EXPR'
  \end{aligned}
  \end{displaymath}
\end{lemma}
\begin{nproof}
  The proof is by rule induction on the given typing derivation.

  A representative case to look at in more detail is the \tdatacon{} case.

    Because $\EXPR = \datacon{\DC}{\keywd{\locreg{\loc}{\reg}}}{\overharpoon{\VAL}}$ is not
    a value, the proof obligation is to show that there is a rule in the dynamic semantics whose
    left-hand side matches the machine configuration $\STOR;\MENV;\EXPR$.
    %
    The only rule that can match is \ddatacon{}, but to establish the
    match, there remains one obligation, which is obtained
    by inversion on \ddatacon{}.
    %
    The particular obligation is to establish that
    $\concreteloc{\reg}{\ind}{} = \MENV(\locreg{\loc}{\reg})$,
    for some $\ind$.
    %
    To obtain this result, we need to use the well formedness
    of the store, given by the premise of this lemma, and in particular rule
    \refwellformed{sec:well-formedness-allocation}{wf:impl-write-once}.
    %
    But a precondition for using
    \refwellformed{sec:well-formedness-allocation}{wf:impl-write-once} that
    the location is unwritten, i.e., $\locreg{\loc}{\reg} \in \NENV$.
    %
    This precondition is satisfied by inversion on \tdatacon{}.
    %
    The application of rule \refwellformed{sec:well-formedness-allocation}{wf:impl-write-once}
    therefore yields the desired result, thereby discharging this case.


    % In this case, where $\EXPR = \datacon{\DC}{\keywd{\locreg{\loc}{\reg}}}{\overharpoon{\VAL}}$,
    % we know $\datacon{\DC}{\keywd{\locreg{\loc}{\reg}}}{\overharpoon{\VAL}}$
    % is not a value.
    % From the premises of the \tdatacon{} rule, we know that
    % the location $\loc$ is unwritten ($\locreg{\loc}{\reg} \in \NENV$),
    % and by assumption of the lemma we know the store is well formed, which
    % means we know $\storewfca{\AENV}{\NENV}{\MENV}{\STOR}$.

    % To prove progress, we need to show that there is a rule in the
    % dynamic semantics whose left-hand side matches the configuration
    % $\STOR;\MENV;\datacon{\DC}{\keywd{\locreg{\loc}{\reg}}}{\overharpoon{\VAL}}$.
    % The only such rule is \ddatacon{}. That rule requires
    % $\concreteloc{\reg}{\ind}{} = \MENV(\locreg{\loc}{\reg})$,
    % for some $\ind$, and this can be demonstrated by
    % Rule~\ref{wf:impl-write-once} of
    % $\storewfca{\AENV}{\NENV}{\MENV}{\STOR}$, which is
    % $\locreg{\loc}{\reg} \in \NENV \Rightarrow
    %       ((\locreg{\loc}{\reg} \mapsto \concreteloc{\reg}{\ind}{}) \in \MENV \wedge
    %       (\reg \mapsto (\ind \mapsto \DC)) \not \in \STOR)$.
    % Because we have already established that $\locreg{\loc}{\reg} \in \NENV$,
    % we can conclude $(\locreg{\loc}{\reg} \mapsto \concreteloc{\reg}{\ind}{}) \in \MENV$,
    % and therefore the requirements for \ddatacon{} are satisfied:
    % $\STOR;\MENV;\datacon{\DC}{\keywd{\locreg{\loc}{\reg}}}{\overharpoon{\VAL}}$
    % steps to
    % $\STOR';\MENV;\concreteloc{\reg}{\ind}{\locreg{\loc}{\reg}}$,
    % where $\STOR' = \STOR \cup \set{\reg \mapsto (\ind \mapsto \DC)}$ and
    % $\concreteloc{\reg}{\ind}{} = \MENV(\locreg{\loc}{\reg})$.

\end{nproof}

\begin{lemma}[Preservation]
  \label{lemma:preservation_brief}
  \begin{displaymath}
    \begin{aligned}
      \text{If} \;\; & \emptytenv;\SENV;\CENV;\AENV;\NENV \vdash \AENV';\NENV';\EXPR : \hTYP \\
      \text{and} \;\; & \storewf{\SENV}{\CENV}{\AENV}{\NENV}{\MENV}{\STOR}\\
      \text{and} \;\; & \STOR;\MENV;\EXPR \stepsto \STOR';\MENV';\EXPR' \\
      \text{then for some} \;\; & \SENV' \supseteq \SENV, \CENV' \supseteq \CENV ,\\
      & \emptytenv;\SENV';\CENV';\AENV';\NENV' \vdash \AENV'';\NENV'';\EXPR' : \hTYP \\
      \text{and} \;\; & \storewf{\SENV'}{\CENV'}{\AENV'}{\NENV'}{\MENV'}{\STOR'}\\
    \end{aligned}
  \end{displaymath}
\end{lemma}
\begin{nproof}
  The proof is by rule induction on the given derivation of the dynamic semantics.

  A representative case to look at in more detail is the \dcase{} case.

  % For this case,
  % $\EXPR' = \subst{\EXPR}{\overharpoon{\var}}{\concreteloc{\reg}{\overharpoon{w}}{\overharpoon{\locreg{\loc}{\reg}}}}$.
  % To prove this case, we have to show two things: first, that $\EXPR'$ is
  % well typed; and second, that the store $\STOR'$ is well formed.

  % For the first part, the goal is to show that
  % $\emptytenv;\SENV';\CENV;\AENV;\NENV \vdash \AENV; \NENV; \EXPR' : \hTYP$,
  % where $\hTYP = \tyatlocreg{\TYP}{\loc}{\reg}$.
  % To demonstrate this, we need to do a few things. First, we have to show that
  % the body of the pattern match, $\EXPR$, is well typed, so we need to show
  % $\TENV';\SENV';\CENV;\AENV;\NENV \vdash \AENV;\NENV;\EXPR : \tyatlocreg{\TYP}{\loc}{\reg}$
  % where
  % $\TENV' = \set{\overharpoon{\var_1} \mapsto \overharpoon{\TYP_1} \ensuremath{@} \overharpoon{\locreg{\loc_1}{\reg}}, \ldots, \overharpoon{\var_1} \mapsto \overharpoon{\TYP_n} \ensuremath{@} \overharpoon{\locreg{\loc_n}{\reg}}}$
  % and
  % $\SENV' = \SENV \cup \set{\overharpoon{\locreg{\loc_1}{\reg}}\mapsto\overharpoon{\TYP}_1,\ldots,\overharpoon{\locreg{\loc_n}{\reg}}\mapsto\overharpoon{\TYP}_n}$.
  % This can be concluded by inversion on the typing rules \tcase{} and \tpattern{}.
  % Second, we

    The first of two proof obligations is to show that
    the result $\EXPR' = \subst{\EXPR}{\overharpoon{\var}}{\concreteloc{\reg}{\overharpoon{w}}{\overharpoon{\locreg{\loc}{\reg}}}}$ of
    the given step of evaluation is well typed, that is,
    $\emptytenv;\SENV';\CENV;\AENV;\NENV \vdash \AENV; \NENV; \EXPR' : \hTYP$
    %
    where $\hTYP = \tyatlocreg{\TYP}{\loc}{\reg}$.
    %
    To establish the above, we start by obtaining the type
    for the body of the pattern, then the types of the
    concrete locations being substituted into the body,
    and finally use these two results
    with the substitution lemma to discharge the case.
    %
    First, by inversion on the typing rules \tcase{} and \tpat{}, we
    establish that the body of the pattern, namely $\EXPR$, is well typed, i.e.,
    $\TENV';\SENV';\CENV;\AENV;\NENV \vdash \AENV;\NENV;\EXPR : \tyatlocreg{\TYP}{\loc}{\reg}$,
    where
    $
    \TENV' = \set{\overharpoon{\var_1} \mapsto \overharpoon{\TYP_1} \ensuremath{@} \overharpoon{\locreg{\loc_1}{\reg}}, \ldots, \overharpoon{\var_1} \mapsto \overharpoon{\TYP_n} \ensuremath{@} \overharpoon{\locreg{\loc_n}{\reg}}}$ and
    $\SENV' = \SENV \cup \set{\overharpoon{\locreg{\loc_1}{\reg}}\mapsto\overharpoon{\TYP}_1,\ldots,\overharpoon{\locreg{\loc_n}{\reg}}\mapsto\overharpoon{\TYP}_n}$.
    %
    Second, we establish that the concrete locations being substituted for the
    pattern variables $\overharpoon{x}$ are well typed.
    %
    The specific obligation is, for each $i \in \set{1, \ldots, n}$, to establish that
    $\emptyset;\SENV';\CENV;\AENV;\NENV \vdash \AENV;\NENV; \concreteloc{\reg}{\overharpoon{w_i}}{\overharpoon{\locreg{\loc_i}{\reg}}} : \overharpoon{\TYP}_i \ensuremath{@} \overharpoon{\locreg{\loc_i}{\reg}}$.
    %
    This holds because, by inversion on \tconcreteloc{}, the obligation is
    to show that, for each such $i$, $(\locreg{\overharpoon{\loc_i}}{\reg} \mapsto \overharpoon{\TYP_i}) \in \SENV'$,
    which is immediate by inspection on $\SENV'$ above.
    %
    Third, and finally, to establish the typing judgement for $\EXPR'$, we use the Substitution
    Lemma (given and proven in Appendix~\ref{lemma:substitution}), which yields
    %
    $\emptyset;\SENV';\CENV;\AENV;\NENV \vdash \AENV;\NENV; \subst{\EXPR}{\overharpoon{\var_1}}{\concreteloc{\reg}{\overharpoon{w_1}}{\overharpoon{\locreg{\loc_1}{\reg}}}}
    \ldots \subst{}{\overharpoon{\var_n}}{\concreteloc{\reg}{\overharpoon{w_1}}{\overharpoon{\locreg{\loc_n}{\reg}}}}: \hTYP$
    as needed, thereby discharging this obligation.
    \item The second obligation
    for this proof case is, given the affected environments, namely
    $\SENV'$ and $\MENV'$, to establish the well formedness
    of the resulting store, i.e.,
    %
    $\storewf{\SENV'}{\CENV}{\AENV}{\NENV}{\MENV'}{\STOR}$.
    I will omit most of the details of this proof obligation because they
    discharge straightforwardly.
    %
    The only part that requires attention is rule
    \refwellformed{sec:well-formedness}{wf:map-store-consistency},
    which is affected by the fresh locations in the location
    environment $\MENV'$.
    %
    This requirement discharges by inspection of \dcase{}, thereby
    discharging this obligation.

\end{nproof}


Finally, the type safety theorem follows:
\begin{theorem}[Type safety]
  \label{theorem:type-safety}
\begin{displaymath}
  \begin{aligned}
  \text{If} \;\; & (\emptyset;\SENV;\CENV;\AENV;\NENV \vdash \AENV';\NENV';\EXPR : \hTYP) \wedge
                   (\storewf{\SENV}{\CENV}{\AENV}{\NENV}{\MENV}{\STOR}) \\
  \text{and} \;\; & \STOR;\MENV;\EXPR \stepsto^n \STOR';\MENV';\EXPR' \\
  \text{then} \;\; & (\EXPR' \; \mathit{value}) \vee
                     (\exists \STOR'', \MENV'', \EXPR'' . \; \STOR';\MENV';\EXPR' \stepsto \STOR'';\MENV'';\EXPR'')
  \end{aligned}
  \end{displaymath}
\end{theorem}

\begin{nproof}
  The type safety follows from an induction with
  the progress and preservation lemmas.
\end{nproof}



\mv{Show some cases of the proof as examples!}


\section{Extensions}\label{sec:extensions}


\subsection{Offsets and Indirections}\label{subsec:indirections}

As motivated in \secref{sec:bg}, it is sometimes desirable to be
able to ``jump over'' part of a serialized tree.
%
As presented so far, \ourcalc{} makes use of an end witness judgment
to determine the end of a particular data structure in memory.
%
The simplest computational interpretation of this technique is,
however, a linear scan through the store.
%
Luckily, extending the language to account for storing and making use
of \emph{offset} information for certain datatypes is
straightforward, and does not add conceptual difficulty to
neither the formalism nor type-safety proof.

Such an extension may use annotations on datatype declarations
that identify which fields of a given constructor are provided \emph{offsets}
and to permit cells in the store to hold offset values.
%
Because the offsets of a given constructor are known from its type,
the \textsc{\dletloctag{}} rule can allocate space for offsets when it
allocates space for the tag.
%
It is straightforward to fill the offset fields because \textsc{\ddatacon{}}
rule already has in hand the required offsets, which are provided in
the arguments of the constructor.
%
Finally, the \textsc{\dcase{}} rule can use offsets instead of
the end-witness rule.

\emph{Indirections} permit fields of data constructors to point across
regions, and thus require adding an annotation form (e.g., an
annotation on the type of a constructor field to indicate an
indirection) and extending the store to hold pointers.
%
Fortunately, as discussed later, regions in \ourcalc{} are never
collected; they are garbage collected in our implementation.
% and thus the extension adds conceptual difficulty to
% neither the formalism nor proof.
%
Every time an indirection field is constructed, space for the pointer
is allocated using a transition rule similar to the \textsc{\dletloctag{}}
rule.
%
The \textsc{\ddatacon{}} rule receives the address of the indirection in the
argument list, just like any other location and writes the indirection
pointer to the address of the destination field.
%

To type check, the type system extends with two new typing rules and a
new constraint form to indicate indirections.
%
To maintain type safety in the presence of offsets and indirections,
the store typing rule needs to be extended to include them.
%
Because the programmer is not manually managing the creation or use of offsets
or indirections (they are below the level of abstraction, indicated by
annotating the datatype, but not changing the code), the store-typing rule
generalizes straightforwardly and the changes preserve type safety.
%\rn{Still might be good to show some kind of concrete example of these annotations.}

In datatype annotations each field can be marked to store its offset in the constructor {\em
  or} be represented by an indirection pointer (currently not both):
\begin{code}
data T = K1 T (Ind T) | K2 T (Offset T) | K3 T
\end{code}
Type annotations would also be the place to express {\em permutations} on fields
that should be serialized in a different order, (e.g., postorder).  But it is
equivalent to generating \ourcalc{} with reordered fields in the source program.

\subsection{Parallelism}\label{subsec:parallelism}

So far, this section has presented a language for performing tree traversals
over serialized data (with some potential indirection). While this data representation
strategy works well for sequential programs, there is an intrinsic tension if we
want to parallelize these tree traversals. As the name implies, efficiently
serialized data must often be read serially. To change that, first, enough
\emph{indexing} data must be left in the representation in order for parallel tasks to
``skip ahead'' and process multiple subtrees in parallel. Second, the allocation
areas must be bifurcated to allow allocation of outputs in parallel.

% Here I will outline how \ourcalc{} can be extended to support parallel
% computation. In this strategy, form follows function: data representation is
% random-access only insofar as parallelism is needed, and both data
% representation and control flow ``bottom out'' to sequential pieces of work.
% That is, granularity-control in the data mirrors traditional granularity-control
% in parallel task scheduling.

% We implemented this in the Gibbon compiler.
% % and found that it performed as well or better than parallel GHC and parallel
% % MLton on a variety of benchmarks. When utilizing 18 cores, our geomean speedup
% % is $1.87\times$ and $3.16\times$ over parallel MLton and GHC, respectively.
% A thorough evaluation of the performance of parallel \ourcalc{} is
% in~\secref{subsec:parallelbench}.

There are several opportunities for parallelism in \ourcalc{} programs.
%
The first kind of parallelism is available when \ourcalc{} programs access
the store in a read-only fashion, such as the program that calculates
the size of a binary tree (as shown in \Figref{fig:sizefunc}).
\begin{figure}
\begin{code}
size : forall @\locreg{l}{r}@ . @\tyatlocreg{Tree}{l}{r}@ -> Int
size [@\locreg{l}{r}@] t = case t of
              Leaf -> 1
              Node (a : @\tyatlocreg{Tree}{l_a}{r}@) (b : @\tyatlocreg{Tree}{l_b}{r}@)
               -> (size [@\locreg{l_a}{r}@] a) + (size [@\locreg{l_b}{r}@] b)
\end{code}
\caption{\ourcalc{} function computing the size of a binary tree}
\label{fig:sizefunc}
\end{figure}
%
However, even though the recursive calls in the \il{Node} case can
safely evaluate in parallel, there is a subtelty: parallel evaluation
is efficient only if the \il{Node} constructor stores offset
information for its child nodes.
%
If it does, then the address of \il{b} can be calculated in constant
time, thereby allowing the calls to proceed immediately in parallel.
%
If there is no offset information, then the overall tree traversal is
necessarily sequential, because the starting address of \il{b} can be
obtained only after a full traversal of \il{a}.
%
As such, there is a tradeoff between space and time, that is, the cost
of the space to store the offset in the \il{Node} objects versus the
time of the sequential traversal (e.g., of \il{a}) forced by the
absence of offsets.

Programs that write to the store also provide opportunities for
parallelism.
%
The most immediate such opportunity exists when the program performs
writes that affect different regions.
%
For example, the writes to construct the leaf nodes for \il{a} and
\il{b} can happen in parallel because different regions cannot overlap
in memory.
%
\begin{code}
letregion @ra@ in
letregion @rb@ in
letloc @\locreg{la}{ra}@ = start @ra@ in
letloc @\locreg{lb}{rb}@ = start @rb@ in
let a : @\tyatlocreg{Tree}{la}{ra}@ = Leaf @\locreg{la}{ra}@ in
let b : @\tyatlocreg{Tree}{lb}{ra}@ = Leaf @\locreg{lb}{rb}@ in
@\ldots@
\end{code}
%
There is another kind of parallelism that is more challenging
to exploit, but is at least as important as the others:
%
the parallelism that can be realized by allowing different fields
of the same constructor to be filled in parallel.
%
This is crucial in \ourcalc{} programs, where large, serialized data
frequently occupy only a small number of regions, and yet there are
opportunities to exploit parallelism in their construction.
%
Consider the \il{buildtree} function, mentioned earlier in this chapter, which
creates a binary tree of a given size \il{n} in a given region \il{r}.
%
If we want to access the parallelism between the recursive calls,
we need to break the data dependency that the right branch has on
the left.
%
The starting address of the right branch, namely $\locreg{l_b}{r}$, is
assigned to be end witness of the left branch by the \il{letloc}
instruction.
%
But the end witness of the left branch is, in general, known only
after the left branch is completely filled, which would effectively
sequentialize the computation.
%
One non-starter would be to ask the programmer to specify the
size of the left branch up front, which would make it possible to
calculate the starting address of the right branch.
%
Unfortunately, this approach would introduce safety issues, such as
incorrect size information, of exactly the kind that \ourcalc{}
is designed to prevent.
%
Instead, here I will explore an approach that is safe-by-construction and
efficient.
%
% Parallel \ourcalc{} overview continues here.
%
A full formalization and proof of type safety for parallel \ourcalc{}
is future work, so this section will outline a sketch of
how the operational semantics of \ourcalc{} can be extended
to safely allow parallel execution.



The expression syntax is the same, and no changes are necessary to the type system.
The operational semantics do require some changes, most notably the addition
of a richer form of indexing in regions.


% \begin{displaymath}
%   \begin{aligned}
%     \textup{Store} && \STOR && \gramdef & \set{\reg_1 \mapsto \heap_1 , \; \ldots \; , \reg_{n} \mapsto \heap_{n}} \\
%     \textup{\new{Heap Values}} && \heapval && \gramdef & \DC \gramor \indirection{\reg}{\concreteind{\ind}} \\
%     \textup{Heap} && \heap && \gramdef & \set{\concreteind{\ind}_1 \mapsto \heapval_1 , \; \ldots \; , \concreteind{\ind}_{n} \mapsto \heapval_{n}}\\
%     \textup{Location Map} && \MENV && \gramdef & \set{\locreg{\loc}{\reg_1}_1 \mapsto \concretelocvar{}_1 , \; \ldots \; , \locreg{\loc}{\reg_n}_n \mapsto \concretelocvar{}_n} \\
%     \textup{Sequential States} && \SEQSTATE && \gramdef & \STOR ; \MENV ; \EXPR \\
%     \textup{Parallel Tasks} && \TASKVAR && \gramdef & (\hTYP, \concretelocvar{}, \SEQSTATE) \\
%     \textup{Task Set} && \TASKSET && \gramdef & \set { \TASKVAR_1, \ldots, \TASKVAR_n } \\
%     \textup{Concrete Locations} && \concretelocvar{} && \gramdef & \concreteloc{\reg}{\indbef{\ind}}{\loc}\\
%     \textup{\new{Extended Region Indices}} && \indbef{\ind}, \indbef{\indj} && \gramdef & \concreteind{\ind} \gramor \indivar{\ind} \gramor \indirection{\reg}{\concreteind{\ind}}
%   \end{aligned}
% \end{displaymath}

Take, as an example, a \ourcalc{} computation involving a binary tree, where the
left child \il{a} at location $\locreg{\loc_a}{\reg}$ is to be computed in
parallel with the right child \il{b}.
Each task has its own private view of memory, which is realized by
giving the child and parent task copies of the store $\STOR$ and
location map $\MENV$.
%
These copies differ in one way, however: each sees a different mapping
for the starting location of \il{a}. %, namely $\locreg{\loc_a}{\reg}$.
%
The child task sees the mapping
$\locreg{\loc_a}{\reg} \mapsto \concreteloc{\loc_a}{1}{}$, which
is the ultimate starting address of \il{a} in the heap.

The parent task sees a different mapping for $\locreg{\loc_a}{\reg}$,
namely $\concreteloc{\reg}{\indivar{1}}{}$.
%
This location is an \emph{ivar index}: it behaves exactly like an I-Var~\cite{IStructures},
and, in our example, stands in for the completion of the memory being
filled for \il{a}, by the child task.
%
Any expression in the body of the let expression that tries to read
from this location blocks on the completion of the child task.
%
When \il{a} is used, it will force the parent task to join with
its child task. The ivar index $\concreteloc{\reg}{\indivar{1}}{}$
will be substituted with $\concreteloc{\loc_a}{1}{}$, and
all the new entries in the location map and store map of the
child task are merged into the corresponding environments in
the parent task. Finally, the results are wired together using
indirections, as covered in~\secref{subsec:indirections}.

To make parallel \ourcalc{} practical when adding it to Gibbon, it was
necessary to add \il{spawn} and \il{sync} forms to the language, so
that it is explicit where parallelism should be exploited, but these
annotations are not strictly necessary in the semantics of
parallel \ourcalc{}.
%
A thorough evaluation of the performance of parallel \ourcalc{} is
in~\secref{subsec:parallelbench}.


\chapter{The Gibbon Compiler}\label{chapter:gibbon}

\section{Converting Functional Programs to the Location Calculus}\label{sec:infer-local}

\ourcalc{} captures a notion of computation over (mostly) serialized data,
\emph{exposing} choices about representation. It provides the levers needed by a human
or another tool to explore the design space of optimization trade-offs above
this level, \ie{} for the human or tool to answer the question
\emph{``how do we globally optimize a pipeline of functions on serialized data?''}.

%\begin{itemize}
%\paragraph{How to globally optimize pipelines of functions on serialized data?}
%  \begin{itemize}
   First, if multiple functions use the same datatype, do they standardize on one
    representation?  Or does that datatype take different encodings at different
    points in the pipeline (implemented by cloning the datatype and
    presenting it to \ourcalc with different annotations)?
%
    Second, when up against the constraint of {\em already-serialized} data on
    disk, the compiler can't change the existing representation, if the
    external data \emph{lacks offsets}, is it better to \emph{force}
    the first consuming function to use that representation, or to insert an
    extra {\em reserialization} step to convert\footnote{Still faster
    than traditional deserialization: no object graph allocation.}?
%
    Third, can the compiler permute fields to improve performance or reduce the
    stored offsets needed?
%  \end{itemize}
%\end{itemize}

% This large space of future work is beyond the scope of this paper, but we nevertheless
% illustrate the process of
% integrating \ourcalc into Gibbon.

    All of these choices can be represented directly in \ourcalc, making it an
    ideal intermediate language for a compiler. This chapter will describe Gibbon,
    an experimental compiler that transforms functional programs to work on
    (mostly) serialized data. Gibbon represents its programs in \ourcalc,
    performing various analyses and transformations, then generates low-level C code.
%% building a front-end above \ourcalc with one example
%% tool: the {\em \lamadt} language and compiler.
%
The front-end language for Gibbon, \lamadt, is a vanilla purely functional
language without any region or location
annotations.  It hides data-layout from the programmer (and the low-level
control that comes with it).  It also facilitates comparison with
mature compilers, as \lamadt runs standard functional programs:
for example, the {\em unannotated} examples we've seen in this paper.

{The syntax for \lamadt is a subset of Haskell syntax, supporting
  algebraic data types and top-level function definitions. It is a
  monomorphic, strict functional programming language, and for
  simplicity it is first order, like \ourcalc.}
% \mv{Future work: closures?}
{In future-work, the plan is to add support for a higher-order, polymorphic
  front-end language through standard monomorphization and defunctionalization.
  (An interesting consequence of this will be that closures become regular
  datatypes, such that a list of closures could be serialized in a dense
  representation.)}


\paragraph{Implementing \lamadt}
The compiler must perform a variant of {\em region
  inference}~\cite{regioncalcs,mlkit-retrospective}, but differs from
previous approaches in some key ways.
%
The inference procedure uses a combination of standard techniques from the literature and specialized
approach for satisfying \ourcalc's particular needs\footnote{The {\em
    Directed Inference Engine for region Types}, if you will.}.
%
Because the inference must determine not only what region a value belongs to,
but \emph{where} in that region it will be, the inference procedure returns a
set of constraints for an expression similar to the constraint environment used
in the typing rules in \figref{fig:types1} and \figref{fig:types2}, which are
used to determine placement of symbolic location bindings. Additionally, certain
locations are marked as \emph{fixed} (function parameters, data constructor
arguments), and when two fixed locations attempt to unify it signals the need
for an indirection, and the program must be transformed accordingly.

Our current implementation adds an extra variant to every data type
\footnote{These indirections do double-duty in allowing the memory
  manager to use non-contiguous memory slabs for a region \secref{subsec:rts}.}
representing a single indirection (called \il{I}).  For example, a binary tree
\il{T} becomes
\begin{code}
data T = Leaf | Node T T | I (Ind T)
\end{code}
%
The identity function \il{id x = x}, when compiled to \ourcalc, is \il{id x = I x}.
Likewise, sharing demands indirections, and
\il{let x = _ in Node x x}
becomes
\il{let x = _ in Node x (I x)}.
%
%% For it to {synthesize} a well-typed \ourcalc term, it adds
%% indirections (not offsets) to satisfy two constraints: (1) a function
%% must write its output to distinct destination region(s) passed as arguments,
%% and (2) a data-constructor must have all its arguments within the same region,
%% consecutively ordered by location.
%
%% For example, these problems arise in compiling the following functions:
%% \begin{code}
%%   id x = x
%% shareTree 0 = Leaf 1
%% shareTree n = let x = shareTree (n-1) in
%%               Node x x
%% \end{code}

%% In the case of the identity function, the fixed locations of the
%% function type prevent the program from returning a value of the same
%% location as its input, so it must be transformed to instead return a
%% location in a new region that consists of an indirection to the
%% input. Similarly, in the second function, because the same subtree is
%% used twice (shared) as both fields of a node, the node constructor
%% will be annotated to expect indirections for both fields.
%For example, the identity function becomes:


\section{Compiling the Location Calculus} \label{sec:impl-local}

In this section I present a compiler for the \ourcalc language, which consists of
the formalized core from~\secref{subsec:grammar}, extended with various primitive types, tuples,
convenience features, and a standard library.
%
A well-typed \ourcalc program guarantees correct handling of regions, but the
implementation still has substantial leeway to further modify datatypes and
the functions that use them.
%
By default, the compiler inserts enough indirection in
datatypes to preserve the asymptotic complexity of the source functions (under
the assumption of $O(1)$ field access), but it also provides a mode---activated
globally or per-datatype---that leaves the data types
\emph{fixed} and instead introduces inefficient ``dummy traversals'' and copying
code into compiled functions.
%
%% (In this mode, our compiler produces a similar result to what Gibbon produced
%% previously~\cite{ecoop17-gibbon}---for comparison, we will distinguish between
%% Gibbon2 (with \ourcalc) and Gibbon1 (without \ourcalc) in~\secref{sec:eval}.)
%
%% Where the formal presentation of the language relied on the end witness judgment
%% to reach the end of sub-trees, a proper compiler need to actually generate
%% efficient code that avoids unnecessary traversals. To do this, we make use of
%% some of the compilation strategies described in~\cite{ecoop17-gibbon}, namely
%% making explicit the threading of end-witnesses through recursive programs and
%% the subsequent lowering of programs to a ``cursor-passing'' form.

{Note that this ``inflexible'' mode---which doesn't allow the compiler to
  insert indirections---is also used when reading in external data.  In our
  \ourcalc implementation, we provide a mechanism for any datatype to be read
  from a file (via \il{mmap}),whose contents are the pointer-free, full
  serialization.  It uses the same basic encoding as Haskell's
  \il{Data.Serialize} module derives by default, but we plan to extend it in the
  future.}

Ultimately, because \ourcalc is meant to be generated by tools as well as programmers,
its goal is to add value in both safety and performance, but to leave
\emph{open} the design space of broader optimization questions to a front-end that
targets \ourcalc{}. One example of such a front-end tool is the front-end of
Gibbon, as described previously in \secref{sec:infer-local}.

%\floatstyle{boxed}\restylefloat{figure}
\begin{figure}
  %% \small
  \begin{displaymath}
    \begin{aligned}
      &n \in \; \textup{Integers}
    \end{aligned}
  \end{displaymath}
  \begin{displaymath}
    \begin{aligned}
      \textup{Types} && \TYP && \gramdef & \ldots \gramor \gramwd{Cursor} \gramor \gramwd{Int} \\
      \textup{Pattern} && \spat && \gramdef & \caseclause{\datacon{\DC}{}{(\var : \gramwd{Cursor})}}{\EXPR} \\
      \textup{Expressions} && \EXPR && \gramdef & \dots \\
      && && \gramor & \switch{\var}{\spat}\\
      && && \gramor & \readInt{\var} \gramor \writeInt{\var}{n} \\
      && && \gramor & \readTag{\var} \gramor \writeTag{\var}{\DC} \\
      && && \gramor & \readCursor{\var} \\
      && && \gramor & \writeCursor{\var}{\concreteloc{r}{i}{l}} \\
    \end{aligned}
  \end{displaymath}
  \normalsize
  \caption{{Grammar of \lamcur{} (an extension of \ourcalc{}) \captionscrunch}}
  \label{fig:nocal-grammar}
\end{figure}

\subsection{Compiler Structure}\label{subsec:compiler_structure}
\ourcalc was implemented with a micropass framework. It is a whole-program
compiler that performs full region/location type checking
%% (and running test programs)
between every pair of passes on the \ourcalc intermediate representation (IR).
%
% The most important intermediate representations (IRs) are these three:
%\rn{Standardize on whether or not we'll use the term Cursor.}
% => WE ARE!
%
After a series of \ourcalc$\rightarrow$\ourcalc passes, we lower to a second IR,
\emph{\lamcur}.
As shown in \figref{fig:nocal-grammar},
\lamcur is not a calculus at all, but a low-level language where
memory operations are made explicit.  \lamcur{} functions closely resemble the C++
code shown early in Chapter~\ref{chapter:intro}.
%
Code in this form manipulates pointers into regions we call {\em cursors} because of
their (largely continuous) motion through regions.
%
{We represent \lamcur internally as a
  distinct AST type, with high level (non-cursor) operations excluded.}

Within this prototype compiler, tuples, and built-in scalar types like Int, Bool
etc. are \emph{unboxed} (never require indirections).
%
In the following subsections, I will describe the
% \ourcalc ($\rightarrow$ \ourcalc)* $\rightarrow$ \lamcur{}
compiler in four stages.
{Similar to \lamcur, our compiler represents
  programs at these stages with AST types that track changes in the
  grammar needed by each pass.
  After these four steps,}
  the final backend is completely standard.  It eliminates
tuples in the \emph{unariser},
% (for more easily generating efficient C code)
performs simple optimizations, and generates C code.
%
Because of inter-region indirections, a small \ourcalc runtime system is
necessary to support the generated code.

The \ourcalc runtime system is responsible for region-based memory management.
A detailed description of the memory management strategy is available in \secref{subsec:rts} %Appendix~\appendixref{chapter:rts}.
In brief, the compiler uses region-level reference counts.  Each region is implemented as
linked list of contiguous memory chunks, doubling in size.  This memory is write-once,
and immutability allows us to track reference counts \emph{only} at the region
level.  Exiting a \il{letregion} decrements the region's count, and it is freed
when no other regions point into it.

\begin{figure*}\center
  \includegraphics[width=0.6\textwidth]{compiler_arch}
  \caption{Compiler architecture.  Here I show the most important
    passes within the three phases of the compiler delimited by
    the three core IRs (front-end, middle-end, back-end).}
  \label{fig:compiler-arch}
\end{figure*}

\subsubsection{Finding Traversals}
\label{sec:find-traversal}

%% After location inference produces a valid \ourcalc{} program, the program is
%% still in a form that assumes access to any field of a data constructor.

%% In order to efficiently compile expressions with pattern matching on data
%% structures {with one or more fields which are not indirections}, the
%% compiler needs to generate code for computing the pointers of each field of the
%% structure in order to bind them to pattern variables.

Pattern matches in \ourcalc{} bind all constructor fields, including those that
occur at non-constant offsets, later in memory.
%
The compiler must determine which fields are reachable based on either (1)
constant offsets, (2) stored offsets/indirections present in the datatype, or
(3) by leveraging traversals already present in the code that scan past the
relevant data.
%
The third case corresponds to determining end witnesses in the formal semantics.
%% and in general \ourcalc{} allows for all fields in any
%% data structure to be bound by pattern matching.
%
%% That is, for a pattern match on \il{Node x y}, the program may assume that it
%% can reference \il{y} and therefore (implicitly) compute the end witness of \il{x} to reach
%% the address of \il{y} in memory.
%
% including those to the right of inline-serialized child structures.
Likewise, this compiler pass identifies data reached by the work the program
already performs.

%% %% we have an extremely simple metric for whether it is at risk of increased work
%% %% and degraded asymptotic complexity:
%% %% \begin{itemize}
%% %% \item Was a \il{copy} inserted?
%% %% \item Does the program refer to a location
%% %% \end{itemize}

%% %% After the location inference algorithm generates a \ourcalc{} program,
%% %% we run {a simple analysis to check if the asymptotic complexity of
%% %%   any of the functions has changed},
%% %% \rn{no that's not simple in general!}
%% %% and if it has, we add
%% %% layout information to the program to restore it.
%% %% Since the location inference algorithm applies some `known' program
%% %% repair strategies like inserting calls to copy functions,
%% %% adding dummy traversals etc. performing this analysis is a simple matter
%% %% identifying these repair strategies and
%% %% replacing them with more efficient ones.
%% %% All calls to copy functions can be eliminated just by inserting
%% %% an indirection node instead.
%% %% However, doing the same for traversal functions
%% %% is tricky and requires some further analysis.


To this end, I use a variation of a technique I previously proposed
in~\cite{ecoop17-gibbon}. Specifically, I assign \emph{traversal effects} to
functions. A function is said to \emph{traverse} it's input location if it
touches every part of it. In \ourcalc{}, a case expression is the only way to
induce a traversal effect. If all clauses of the case expression in turn
traverse the packed elements of the corresponding data constructors, the
expression traverses to the end of the scrutinee.
%
Traversing a location means witnessing the size of the value encoded at that
location, and thus computing the address of the \emph{next} value in memory.
% (This is the $\phi$ function from \secref{sec:lang}).
%
After this pass, the type schemes of top-level function definitions reflect
their traversal effects.

%% This analysis is very similar to the \emph{effect inference} process
%% described in~\citet{ecoop17-gibbon}.
%% % , and we leave the reader to find more details in that paper.
%% The primary difference in our approach, other than the overall purpose (their
%% effect inference procedure is used to infer something like location annotations,
%% though ``abstract locations'' in that paper are quite different than what we
%% call locations---and, in fact, unnecessary traversals and copies are inserted
%% into Gibbon programs after effect inference), is that no unification is
%% required.

\begin{code}
maplike : forall $\locreg{l_1}{r_1}$ $\locreg{l_2}{r_2}$. @\tyatlocreg{Tree}{l_1}{r_1}@ $\xrightarrow{\{ l_1, l_2 \}}$ @\tyatlocreg{Tree}{l_2}{r_2}@
rightmost : forall $\locreg{l}{r}$ . @\tyatlocreg{Tree}{l}{r}@ $\xrightarrow{\{\}}$ Int
\end{code}

\subsubsection{Implementing Random Access}
\label{sec:rand-access}
% ~~~~~~~~~~~~~~~~~~~~~~~~~~~~~~~~~~~~~~~~
%% (by changing data constructors to contain random-access skip-ahead pointers, and
%% thus avoiding any need for dummy traversals in these situations).

Once it is known what fields are traversed, it is possible to determine which fields are
used but \emph{not} naturally reachable by the program: e.g. the right subtree
read by \il{rightmost}.
%
% Compiled in the style of
% If we take no further action, \il{sizeof(left)}
In later stages of the compiler, all direct references to
pattern-matched fields {\em after} the first variable-sized one are eliminated.
%
This is where space/time optimization choices must be made:
bytes for offsets v.s. cycles for unnecessary traversals.

%% Through this compiler pass we determine which functions \emph{do not}
%% traverse their inputs, and hence if it is necessary to modify the
%% data structures those functions operate on to allow them to execute
%% efficiently.
%% For example, a function that returns the leftmost leaf of a tree does not
%% traverse its input, but it can be compiled without any modifications to use the
%% default serialization format.
%% %
%% Conversely, the \il{rightmost} function does not touch all input bytes, \emph{and}
%% needs additional support from the datatype to avoid asymptotically expensive
%% dummy traversals.
%% %
%% To avoid this, we enumerate all data-types where we need to ``skip over''
%% variable-sized fields and we modify the data type to enable random-access.

% \paragraph{Random-access nodes include offsets}
% \paragraph{Adding offsets}
%\paragraph{Layout Information}
%\label{sec:RAN}
% ~~~~~~~~~~~~~~~~~~~~~~~~~~~~~~~~~~~~~~~~

%% When the compiler knows for sure that a data constructor being allocated will
%% need random access to its fields, it can do much better than implementing all
%% fields as \emph{tagged} indirection nodes.
%% First of all, random access to field $N$ would
%% require that \emph{all} fields $1 \ldots N-1$ be indirection nodes in order to
%% make assumptions about the offset to reach field $N$, which negates the
%% flexibility of tagged indirections occurring anywhere.
%% Second, the aforementioned space and time overheads for dispatching on
%% ``\il{I}'' tags would add up.

To activate random-access for a particular field within a data constructor, the compiler
adds additional information following the tag.  Specifically, for a constructor
\il{K T1 T2}, if immediate access to \il{T2} is needed, the compiler includes a 4-byte
relative-offset after the constructor.
% \il{K' (Ptr T2) T1 T2}.
%
%% (We never do this for the leftmost field, which we can already reach in constant
%% time.)
%% applying this technique to the running example datatype yields:
%% \begin{code}
%%   data Tree = Leaf Int | ${\color{blue} {\q{Node'}\ \q{(Ptr Tree) Tree Tree}}}$
%% \end{code}
%%
%% Here we have eliminated the original \il{Node} constructor, but we could also
%% keep both variants; in future work we plan to explore this possibility.
%% %
%% The random-access constructor \il{Node'} takes one word more space,
%% saving on space compared to a traditional pointer-based
%% representation\footnote{It may look like \il{Node'} has three fields, but at
%%   runtime this does not imply three words.  Rather, the naked \il{Tree} fields
%%   are still serialized directly in the stream, and so they are more akin to
%%   promises of future data on the stream, than fields in traditional structs or
%%   tuples.},
%% yet it can still be consumed either in-order or via random-access.
%% %
%% If reading \il{Tree} in-order, the reader ignores the \il{Ptr Tree} field and
%% reads the left- and right-subtrees as usual.
%% %
%% %% The random-access technique is composable with the indirection technique,
%% %% increasing the space of possible encodings (and we could mix and match normal
%% %% and random-access constructors within an encoding as well, \il{Node} and
%% %% \il{Node'}).
%% %
%% An example with random-access nodes is pictured below, including
%% relative offsets, and representing the source expression

%% \lstinline{Node (Node (Leaf 1) (Leaf 2)) (Node (Leaf 3) (Leaf 4))}.

%% %
%% \begin{center}
%%   \includegraphics[width=0.4\textwidth]{figs/tree_random_access_nodes}
%% \end{center}
%
%% Unlike tagged indirections, random-access nodes use fewer bytes than a
%% traditional pointer-based representation.  In this tree example, we use
%% \emph{one} word for pointer data, rather than two words (both left and right
%% pointers).  Compared with the tagged-indirection version, this saves a couple of
%% bytes on tags as well as the runtime overhead of switching on them.

%% % In total, this encoding takes 45 bytes rather than the 47 with tagged
%% % Overall, the left encoding above takes 29 bytes, and the right takes 47 bytes.

% \subsubsection{Back-tracking to add random-access}
\paragraph{Back-tracking}
%
Unfortunately, when datatypes are modified to add offsets, it invalidates
previously computed location information.  Thus the compiler {\em backtracks},
rewinding in time to before find-traversals (and inserting extra \il{letloc}
expressions to skip-over the offset bytes themselves).
% InferLocations
% (as pictured in \figref{fig:compiler-arch}), restoring correct location information.
%
Adding random access to one datatype never \emph{increases} the set of
constructors needing random-access to maintain work-efficiency, so in fact it will
only backtrack at most once\footnote{The marked set of constructors
  is a conservative over-approximation; it would be possible in principle to
  construct a program with types \il{A} and \il{B}, both of which are marked for
  random access, but where \il{A} becoming random-access would obviate the need
  for \il{B}.  Further optimizations are possible.}.
%
%
After this is complete, the \il{rightmost} example becomes:
%
\begin{code}
rightmost : @\tyatlocreg{Tree}{l_{1}}{r_1}@ -> Int
rightmost tr =
 case tr of
   Leaf (n : @\tyatlocreg{Int}{l_n}{r_1}@ ) -> n
   Node' (ran : (Ptr (@\tyatlocreg{Tree}{l_b}{r_1}@)) @\tyatlocreg{}{l_{ran}}{r_1}@)
         (a : @\tyatlocreg{Tree}{l_a}{r_1}@) (b : @\tyatlocreg{Tree}{l_b}{r_1}@)
      -> rightmost [@\locreg{l_{b}}{r_1}@] *ran
\end{code}
\mav{TODO: Clarify notation for using RAN pointers}


%% That's because it never uses the right child of the tree, which it cannot reach
%% without a dummy traversal.  So we perform a simple occurs-check to look for uses
%% of such {\em unreachable} packed elements in the function body, and mark the
%% corresponding data constructor as needing random access if so.
%
In the default (offset-adding) mode, any function that demands random access to
a field will determine the representation for all functions using the
datatype.
%
Our current LoCal compiler does \emph{not} automate choices such as duplicating
datatypes to achieve multiple encodings of the same data---that is left to the
programmer or upstream tools.

If the LoCal compiler is passed a flag to {\em not} automatically change
datatypes, then it must use the same approach we previously used in~\cite{ecoop17-gibbon}:
insert {dummy traversals} that scan across earlier
fields to reach later ones.
%
Regardless of whether the offset or dummy-traversal strategy is used,
%
at the end of this compiler phase, we blank non-first fields in
each pattern match to ensure they are not referenced directly.
So a pattern match in our tree examples becomes
 ``\il{Node a _ -> ...}'' or   ``\il{Node offset a _ -> ...}''.
% with the location of the right child, $l_b$, computed by other means if needed.


%% \begin{code}
%%   Node' (ran : (Ptr (@\tyatlocreg{Tree}{l_b}{r_1}@)) @\tyatlocreg{l_{ran}}{r_1}@) (a : @\tyatlocreg{Tree}{l_a}{r_1}@) _ -> rightmost [@\locreg{l_b}{r_1}@] b
%% \end{code}

%% %% Depending on the result of the previous analysis, we switch the serialization format
%% %% include layout information.
%% %% This is a rather radical transformation, which affects all parts of the program.
%% %% An important thing to note is that \q{AddLayout} operates on \lamadt{} programs,
%% %% rather than \ourcalc{}.
%% %% Because once we include indirection nodes in the serialization,
%% %% the locations inferred in the previous phase are no longer valid.
%% %% For example, in the tree serialization shown in \figref{fig:tree-packed},
%% %% the left child is serialized immediately after the Node tag.
%% %% Whereas in \figref{fig:indirections-layout}, we have to skip over
%% %% the indirection node to get to the left child.
%% %% So we essentially `go back in time', update the serialization format, and then run
%% %% location inference again to ensure that the location arithmetic is valid.
%% %% In practice, \q{AddLayout} takes in a \lamadt{} program, transforms it to include indirection nodes,
%% %% and runs a few compiler passes again to get back a \ourcalc{} program.
%% %% %
%% %% \rn{Wait, is it just proper backtracking at this point? We can just draw a
%% %%   back-edge in the architecture diagram.}

%% %% Adding layout information involves the following steps:

%% %% \begin{enumerate}
%% %% \item In the current implementation, indirection nodes are encoded as
%% %%   constructors of regular packed datatypes.  So we first update the data
%% %%   types to have an additional constructor which takes in a pointer argument.
%% %% \begin{code}
%% %% data Tree = Leaf Int
%% %%           | Inner Tree Tree
%% %%           | ${\color{blue} {\q{Indirection}\ \q{Pointer}}}$
%% %% \end{code}

%% %% \item Update all data constructors having $n$ packed arguments, to accept
%% %%   additional $(n-1)$ arguments to store the indirections.

%% %% \begin{code}
%% %% data Tree = Leaf Int
%% %%           | Inner ${\color{blue} \q{Tree}}$ Tree Tree
%% %%           | Indirection Pointer
%% %% \end{code}
%% %% \end{enumerate}


%% Note that adding random-access to data-types involves changing the
%% data-constructors that construct those values and every case
%% expression that consumes them.  However, at this phase of the compiler
%% (\lamadt{} and \ourcalc), the skip-ahead pointers are {\em not} used by  the
%% operational semantics (and thus neither by an interpreter for the IR).
%% %
%% Rather, they are dormant until the conversion to the cursor-based \lamcur{} IR.
%% %
%% For instance, a data constructor application \il{Node x y}, becomes \il{Node
%%   (ptrEndOf x) x y} (using a compiler primitive \il{ptrEndOf}), and a case
%% expression matches the added pointer field, but ignores it  until later.

%% %% \begin{code}
%% %% case tree of Node $\color{blue} {(yp\ @\ {l_i} ^ {r})}$ (x $@\ {l_1} ^ {r}$) (_ $@\ {l_2} ^ {r}$) -> $\ldots$
%% %% \end{code}

%% % Update the constructor functions, (MkNode), to write the indirection nodes.

%% %% \begin{code}
%% %% addLayout : $\lamadt{}$ -> $\lamadt{}$
%% %% addLayout ex =
%% %%   case ex of
%% %%     DataConE con args ->
%% %% \end{code}

\subsubsection{Routing End Witnesses}
\label{sec:route-ends}
% \subsection{Route-Ends, To-Cursors, and Code Generation}
% ~~~~~~~~~~~~~~~~~~~~~~~~~~~~~~~~~~~~~~~~

%% Gibbon, as described in~\citep{ecoop17-gibbon}, transforms a
%% functional program operating on algebraic data structures using
%% pattern matching into a pointer-passing (or ``cursor''-passing)
%% imperative program.
%% %
%% Our compiler also must do a similar transformation, though unlike in
%% \citet{ecoop17-gibbon}, \ourcalc{} enables these steps in our compiler to be

Each of the traversal effects previously inferred proves the compiler logically reaches a
point in memory, but to realize it in the program the compiler adds an
additional return value to the function, witnessing the end-location for
traversed values (as described in \cite{ecoop17-gibbon}).
%
Here I extend the syntax to allow additional location-return values,
equivalent to returning tuples.  The \il{buildtree} example becomes:
%

\begin{code}
buildtree : forall $l^r$. Int $\xrightarrow{\{l\}}$ [$\mathit{after}(\tyatlocreg{Tree}{l}{r})$] Tree$\locreg{@l}{r}$
buildtree [$\locreg{l}{r}$]  n =
  if n == 0 then return [$\locreg{l}{r}+9$] (Leaf $\locreg{l}{r}$ 1)
  else letloc $\locreg{l_a}{r}$ = $\locreg{l}{r}$ + 1 in
       let [$\locreg{l_b}{r}$] left = buildtree [$\locreg{l_a}{r}$] (n - 1) in
       let [$\locreg{l_c}{r}$] right = buildtree [$\locreg{l_b}{r}$] (n - 1) in
       return [$\locreg{l_c}{r}$] (Node $\locreg{l}{r}$ left right)
\end{code}% \vspace{-1mm}
%% letloc $l_{ran}$ = $l_1$ + 1 in
%% letloc $l_a$ = $l_{ran}$ + 8 in
%                @{let ran = (ptr) $l_b$ in}@

%% Above, the use of \il{after} has disappeared.  With
%% the additional end-witnesses returned by the recursive calls, the previous
%% \il{sizeof} was not needed to compute $l_b$, and thus it was dropped as dead
%% code.

The \il{letloc} form for the location of the right subtree is gone, because
the first recursive call to \il{buildtree} returned $\locreg{l_b}{r}$ as an end-witness,
bound here with an extended \il{let} form.
Similarly, the final return statement returns the end-witness of the right subtree,
$\locreg{l_c}{r}$, using a new \il{return} form in the IR.

\subsubsection{Converting to \lamcur}
% ~~~~~~~~~~~~~~~~~~~~~~~~~~~~~~~~~~~~~~~~
\label{subsubsec:cursorize}

In this stage, the compiler converts programs from \ourcalc{} into \lamcur{}, switching to
imperative cursor manipulation.
%
At this stage, location arguments and return values turn into first-class cursor
values (pointers into memory buffers representing regions).  The primitive
operations on cursors read or write one atomic value, and advance the cursor to
the next spot.  The compiler drops much of the type information at this phase (see \secref{subsec:linear} for how to preserve types), and
\il{rightmost} becomes:
\begin{code}
rightmost : Cursor -> Int
rightmost cin =   -- take a pointer as input
  switch cin of   -- read one byte
    Leaf(cin1)  ->
      let (cin2,n) = readInt(cin1) in n
    Node(cin1)  -> -- only get a pointer to the 1st field
      let (cin2,ran) = readCursor(cin1) in
      rightmost ran
\end{code}\vspace{-1mm}
%
Here the \il{switch} construct is simpler than \il{case},
% is no longer pattern matching on data constructors to access fields, rather it is
reading a one byte tag, switching on
it, and binding a cursor to the very next byte in the stream
(\il{cin1 == cin + sizeof(tag) == cin+1}).

%
The key takeaway here is that, because the relationship between
location variables and normal variables representing packed data are
made explicit in the types and syntax of \ourcalc{}, this pass
does not require any complicated analysis.
%
Also, in \lamcur{} we can finally reorder writes to more often be {\em in order}
in memory, which aids prefetching and caching,
%\footnote{Linear access patterns are more friendly to prefetching \&
%  caching policies in modern processors.},
because writes are ordered only by
data-dependencies for computing \emph{locations}, with no ordering needed
on the side-effects themselves.

\subsection{Linear Cursors}\label{subsec:linear}

In the process described previously in \secref{subsubsec:cursorize}, Gibbon
transforms programs into \lamcur{}, which uses a cursor-passing style. As Gibbon
is now, these cursor-passing programs carry no type information on the cursors
themselves---the compiler has erased information about packed types. This is
convenient if the goal is to eventually generate C code, which is what Gibbon
does.

However, there are other situations where you may want cursor types to ensure
that serialized data is written and read safely, just like \ourcalc{}; for
example, if a programmer wants to embed a safe interface for programming with
serialized data into a mainstream functional language like Haskell. In this
section I will briefly present a system for typing cursors, and demonstrate its
relationship to \lamcur{} and how it may be expressed in Linear
Haskell~\cite{linear-haskell}. This initially appeared in~\cite{ecoop17-gibbon}
as a typed intermediate language for an earlier version of the Gibbon compiler,
and subsequently was adapted in~\cite{linear-haskell} as an example use case for
linear types in Haskell. In this section I borrow the presentation style
and examples from the latter.

The basic idea is that cursors are \emph{indexed} by a list of types.
Write cursors are indexed by a list of types that corresponds to the values
that must be written to that cursor, and read cursors are indexed by a list of
types that corresponds to the values that can be read from the cursor.
Cursors must be \emph{linear}, and operations that consume cursors return
new cursors with updated types. An example interface for the simple
binary tree data type we have been using is given in \Figref{fig:linpacktree},
and an interface for manipulating general packed data is given in
\figref{fig:lininterface}.

This interface is \emph{type-safe} in the sense that it provides a layer
of abstraction for consuming and producing serialized data such that
a program only reads byte-ranges at the size and type they were originally
written. For this to work, the \il{Packed} type must be abstract, so
a client working with a \il{Packed Tree} is not privy to the memory
layout of its serialization.

The code in this section uses the experimental linear types extension to
Haskell. With this extension, function types with $\multimap$ are linear
functions, while ordinary arrows remain ordinary arrows. A linear function
is constrained to consume its argument exactly once. The \il{Ur} data type
is short for \emph{unrestricted}, and is similar to $!$ in linear logic.

Because a client does not have direct access to the serialized bytes,
consuming a serialized binary tree is done with the help of a pattern
matching combinator like \il{caseTree} in \Figref{fig:linpacktree}. This
function takes the serialized tree and two continuations (one for if the
tree is a leaf, and one for if it is a node), and the types ensure that
the continuations are invoked on packed trees with appropriate structure
(the leaf case expects to find an \il{Int}, the node case expects to
find a \il{Tree} and then another \il{Tree}).
%
In \Figref{fig:linsumleaves}, \il{caseTree} is used to fold over a binary
tree and sum the values of all the leaves. Note that the packed values are linear,
and must be explicitly threaded through different recursive function applications
in the \il{go} helper function.

Reading from \il{Packed} values and writing to \il{Needs} values, as shown in
\Figref{fig:lininterface}, relies on type-level lists: reading an \il{a} from
a value of type \il{Packed (a : r)} yields a pair \il{(a, Packed r)},
and writing a value of \il{a} to a value of \il{Needs (a : r)} yields
\il{Needs r}. Once a packed value has had all of its values read, it can
be consumed with \il{done}, and once a needs cursor has had all its values
written it can be cast to a packed value with \il{finish}. Linearity
ensures that everything is read and written in the proper order.

\begin{figure}
\begin{code}
data Tree = Leaf Int | Branch Tree Tree
pack :: Tree ->. Packed [Tree]
unpack :: Packed [Tree] ->. Tree
caseTree :: Packed (Tree : r) ->.
            (Packed (Int : r) ->. a) ->
            (Packed (Tree : Tree : r) ->. a) -> a
\end{code}
\caption{A type-safe, read-only interface for computing with serialized binary trees
  in Linear Haskell}
\label{fig:linpacktree}
\end{figure}
\begin{figure}
\begin{code}
read :: Storable a => Packed (a : r) ->. (a, Packed r)
write :: Storable a => a ->. Needs (a : r) t ->. Needs r t
finish :: Needs [] t ->. Ur (Packed [t])
newBuffer :: (Needs [a] a ->. Ur b) ->. b
done :: Packed [] ->. ()
\end{code}
\caption{An interface for manipulating packed values in Linear Haskell}
\label{fig:lininterface}
\end{figure}
\begin{figure}
\begin{code}
sumLeaves :: Packed [Tree] -> Int
sumLeaves p = fst (go p)
  where go p = caseTree p
           read -- Leaf case
           (\p2 -> let (n,p3) = go p2
                       (m,p4) = go p3
                   in (n+m,p4))
\end{code}
\caption{A Linear Haskell function for summing the leaves of a packed
  binary tree}
\label{fig:linsumleaves}
\end{figure}

\begin{code}
startLeaf :: Needs (Tree : r) t ->. Needs (Int : r) t
startBranch :: Needs (Tree : r) t ->. Needs (Tree : Tree : r) t
\end{code}
%
To safely write serialized binary trees, we need a few more building blocks.
Two functions, \il{startLeaf} and \il{startBranch} write
tags to bytestrings, and leave writing the rest of the fields as future
obligations.


We can use these to write a function, \il{mapLeaves} (in \Figref{fig:linmapleaves}),
which maps a non-linear function of type \il{(Int -> Int)} over a packed
binary tree. Bernardy et al.~\cite{linear-haskell} benchmarked this function,
and found that GHC produced code that performed no Haskell heap allocations,
showing that this style of programming can be used effectively to program
efficiently with serialized data.

\begin{figure}
\begin{code}
mapLeaves :: (Int -> Int) -> Packed [Tree] ->. Packed [Tree]
mapLeaves fn pt = newBuffer (extract . go pt)
  where
    extract (inp,outp) = case done inp of () -> finish outp
    go :: Packed (Tree : r) ->. Needs (Tree : r) t ->.
          (Packed r, Needs r t)
    go p = caseTree (\p o -> let (x,p') = read p
                             in (p', writeLeaf (fn x) o))
                    (\p o -> let (p',o') = go p (writeBranch o)
                             in go p' o')
\end{code}
\caption{A Linear Haskell function for mapping over the leaves of a packed
  binary tree}
\label{fig:linmapleaves}
\end{figure}

This approach is currently \emph{not} used in the cursor-passing intermediate language
in Gibbon. It is future work to adopt something like this inside the Gibbon compiler.

\subsection{Runtime System}\label{subsec:rts}
% ================================================================================

In \ourcalc, locations track natural number positions within a region.
Abstractly, a region is an unbounded, byte-indexed storage area that can be
extended incrementally by requesting $N$ additional bytes (equivalent to
\il{malloc(N)}).
%
Each region grows monotonically, never shrinks, and can be
freed only as a whole.
%
Practically speaking, there are at many reasonable implementation strategies.
%
We always start by allocating a contiguous {\em chunk} of memory of bounded
size.  When that chunk is exhausted, we must choose whether to grow the
region by {\bf copying} (or changing memory-mapping), retaining a contiguous
address range, or by linking a new, non-contiguous chunk.


%% \begin{enumerate}
%% \item constant regions
%% \item unbounded regions: must grow to accommodate allocation.
%% \item huge regions: semantically the same as unbounded, but suspected to be very
%%       large and long-lived.
%% \end{enumerate}

We choose the latter and implement regions as a linked list of chunks: a
constant sized initial chunk, with subsequent chunks doubling in size.
%
% A reference to a region is just a pointer to its first chunk.
The runtime representation of locations (and \il{Ptr T} values)
is a direct pointer into the interior of a chunk.
(We call the writable portion of the chunk that can carry data the {\em payload}.)
Chunks linked together form regions as pictured in
\figref{fig:regions}.  Chunk metadata is stored at the \emph{end} of the
allocated area, in a footer data structure listed below:
%
\begin{lstlisting}[language=C++]
  struct footer {
    // Available bytes for serialized-data storage.
    int  size;
    // Shared reference count for this region (not chunk)
    int* refcount;
    // Set of regions we have outbound pointers into.
    ptrset  outset;
    // The chunk that follows this one (or NULL)
    footer* next;
  }
\end{lstlisting}
We avoid additional indirection by combining this metadata struct with the
payload, which is essentially an array of bytes, forming one heap object.
% \paragraph{Bounds-checking}
The reason we store the metadata as a \emph{footer}, at the end rather than the
start, is so that the payload grows \emph{towards} the struct.  Thus the pointer
to the region-chunk does double duty for bounds checking.  When the payload
space is full, we allocate a new chunk of double the size and point to it with
\il{next}.

But what do we put in the serialized bytestream to mark that the stream
continues in another chunk?  Here we implicitly add a reserved tag to each
packed data type, signaling {\em end of chunk} (EOC).\footnote{Of course, there
  are 256 possible one-byte tags, so adding indirections, random-access nodes,
  and EOC tags reduces the largest sum type supported (at least, without using
  an escape sequence to access additional tags).}
%
When the reader hits an EOC, they must use their pointer to the end of the
current payload to access the footer, follow the \il{next} pointer, and resume
reading at the head of the next chunk.

%% Here we introduce a slight variation on the concept
%% of an indirection node from \secref{sec:indirections}, a {\bf redirection
%%   node}.  When a consumer reads an \il{Indirection (Ptr T)} value from the
%% stream, it refers to the pointer to read a complete value of type \il{T} from
%% a distant address, but then it comes back to the original buffer, containing the
%% indirection, and continues.  In contrast,


\paragraph{Garbage collection}
% ~~~~~~~~~~~~~~~~~~~~~~~~~~~~~~~~~~~~~~~~

In most classic treatments, regions introduced with a \il{letregion},
are deallocated immediately upon the end of that \il{letregion}'s lexical
scope.
%
However, in this paper we choose to allow tagged indirection nodes to include
{\bf inter-region pointers}.  Thus one can keep a region alive beyond the scope of the
\il{letregion} that introduced it, by simply capturing a pointer to it within
another region.
%
This choice is critical to our ability to lift functions onto (mostly)
serialized representations without changing their asymptotic complexity.

In our setting, pointers between regions are immutable, which simplifies the job of
garbage collection.  Rather than keeping a ``remembered set'' of inter-region pointers as
in a generational collector, we can instead \emph{coarsen} the dependencies to record only
that ``chunk A points to region B''.  The \il{outset} in the \il{footer} struct above
tracks regions to which our chunk points\footnote{This set is optimized for zero or one
  elements.  A null pointer denotes the empty set, and singleton is a direct (tagged)
  pointer to the element.  Two or more elements introduce a heap data structure to store
  the out-set.}.

Both tracing or reference counting collectors would benefit from this
coarsening.  However, given that we already amortize the overheads of memory
management through coarsening, we choose reference counting for our
implementation to achieve prompt deallocation (and reuse) of chunks.
%
Thus when a region is created with \il{letregion} its reference count is set to 1,
and it is decremented on exit from the \il{letregion}.
%
Reference counts are region-level rather than chunk-level, which is why the
\il{footer} contains a pointer to the region-level reference count, rather than
a reference count directly.
%
When a region hits zero reference count, it is freed immediately via freeing its
chunks one by one.  When a chunk is deallocated, it decrements the reference
count of any regions it points to.


\floatstyle{plain}\restylefloat{figure}
\begin{figure}
  \vspace{-15mm}
  \begin{center}
    \includegraphics[width=0.75\textwidth]{region-memlayout2}
  \end{center}
  \vspace{-4mm}
  \caption{Example of multiple chunks making up a region, and of an inter-region
    indirection.  \captionscrunch{}}
  \label{fig:regions}
\end{figure}


\paragraph{Comparing against prior art's memory management}
Finally, we also implement a technique
that we call {\bf huge regions}; these are allocated as a large slab containing many
pages, and could be extended by mapping new virtual memory after hitting a guard
page capping the region end.
% This was the approach used by
% Gibbon~\cite{ecoop17-gibbon}.
%
These huge regions avoid bounds checks when writing payloads and they are
suitable for programs with a small number of large regions (especially a single
input and single output region).  But they are inappropriate for the more
general case where programs may have small and short-lived regions.
%
%% \Red{We quantify the performance impacts of this trade-off in
%%   \secref{sec:eval-litmus}.}

%% Previous compilers based on regions, such as the MLKit compiler~\citep{mlkit},
%% can represent an extreme

The choice of allocation strategy can be informed by static information the
compiler gathers about the lifetime and potential size range of the region. For
example, the region-based MLKit compiler achieved significant speedups from
statically classifying a majority of regions as
constant-sized~\cite{mlkit-retrospective}, in which case they are allocated
inside the procedure stack frame.
%
% In our case, because we are packing many logical values into a smaller number of
In our approach, we unbox constant-sized data types (e.g. numbers), and
pack recursive data-types into growable regions, so we do not observe the same
opportunity for constant-sized regions.


%% \begin{itemize}
%% \item {\bf Bounds-check} or {\bf guard page}:
%% \item {\bf Copying} or {\bf Linking}:
%% \end{itemize}

%% The latter choice is relevant to the design of our ``mostly serialized''
%% representation, because {\em contiguity} changes the design space for
%% offset-based pointers.  For instance, if using the copying strategy


\section{Applications and Evaluation}
\label{section:applications}

To evaluate the the Gibbon compiler, it was tested on a number of different
benchmarks, from simple microbenchmarks to more realistic tasks like
transforming abstract syntax trees and processing large amounts of data.
For purposes of evaluation, a couple of other points of comparison were
chosen, and the results will be given in this section.

\subsection{Microbenchmarks}

Gibbon was evaluated on a series of microbenchmarks to demonstrate how it
handles simple functions and operations on binary trees. From these results,
it is clear that Gibbon is flexible enough that it can maintain the
correct asymptotic complexity of some common data structure operations by
taking advantage of the indirections described in~\secref{subsec:indirections},
while \emph{also} producing extremely efficient code for processing
pure serialized data.

For the benchmarks in this section, Gibbon was evaluated with respect to
pointer-based C code, Haskell Compact Normal Form (compiled with GHC),
and Cap'n'Proto.
%
The full microbenchmark results are given in\tabref{tab:litmus-table}. In short,
Gibbon is 2.6 / 3.2 / 9.4 $\times$ faster than pointer-based C / Haskell CNF /
Cap'n'Proto respectively.

Gibbon is also benchmarked against itself: for the rest of this section,
Gibbon1~\footnote{The 1 in Gibbon1 signifies that the first version of the
Gibbon compiler worked like Gibbon1, while Gibbon2 represents the current
version of the Gibbon compiler.} will refer to the Gibbon compiler configured to
perform deep copies rather than insert indirections, synthesize dummy traversals
rather than insert random-access nodes, and to use fixed-size rather than
growable regions (because indirections are used to build growable regions). The
hypothesis was that Gibbon1 would be slightly faster on benchmarks that are
straightforward traversals of serialized data, while Gibbon2 would be
significantly faster in cases where indirections or random access was necessary
to preserve asymptotic complexity. This was indeed the case, as Gibbon2 was
202$\times$ faster than Gibbon1 across all benchmarks, versus 0.96$\times$ for
benchmarks with only apples-to-apples asymptotics.
% gibbon pointer cnf capn
% 202 / 2.6 / 3.2 / 9.4  $\times$ geomean faster than
% Gibbon/NonPacked/CNF/CapNP respectively, and
% 0.96 / 2.6 / 3.2 / 7.3 $\times$ faster for only apples-to-apples asymptotics.
%% $3.2\times$,
%% $9.4\times$, and
%% $202\times$


%% \tabref{tab:litmus-table} shows the results.

% \mav{Re-write to either remove or better explain difference between Gibbon1 and Gibbon2.}

% The column labeled ``Gibbon2'' shows performance of \lamadt programs
% (low-level \ourcalc{} control was not needed for any of these)
% using indirections and offsets, automatically.
% %
% ``Gibbon1'' shows the approach described in \cite{ecoop17-gibbon}.
% %
% There are two major sources of overhead for our new approach versus Gibbon1:

After investigating the C code generated for both Gibbon1 and Gibbon2,
the overhead of Gibbon2 on some benchmarks comes from two sources:
%
\begin{enumerate}
\item Growable regions: In each case, our compiler starts with smaller, growable
regions\footnote{starting at 64K bytes}, which we require to create small output
regions as in {\bf id} or {\bf treeInsert}, but we suffer the overhead of
bounds-checking. On the other hand, Gibbon1 always stores fully serialized data
in huge regions.

\item Likewise, we have found that the backend C compiler is sensitive to the
number of cases in switch statements on data constructor tags (for instance,
triggering the jump table heuristic). By including the possibility that each tag
we read may be a tagged indirection, % or end-of-chunk, we increase code size
and increase the number of cases in our switch statements.
\end{enumerate}


However, the benchmarks where indirections and random-access offsets are important (id,
rightmost, treeInsert, findMax) show a huge difference between Gibbon1 and Gibbon2,
as we would expect based on Gibbon1
requiring additional traversals to compile those functions.

\paragraph{Versus pointer-based representations}
For the pointer-based C results, labeled ``NonPacked'' in the table
Gibbon was configured to \emph{always} insert indirections
and thus emulate a traditional object representation.
% example of a traditional compilation approach
% that use one heap object per data constructor.
%
In this case, we are being overly friendly to this pointer-based representation by
allowing it to read its input (for example, the input tree to {\bf treeInsert})
in an already-deserialized, pointer-based form.  A full apples-to-apples
comparison would force the pointer-based version to deserialize the input
message and reserialize the output. I omit that here to focus only on the
cost of the tree traversals themselves.
%
%% Similarly, the earlier Gibbon work compared a subset of these programs against a
%% larger set of traditional compilers (Java, gcc, GHC, OCaml, MLton, Racket, Chez),
%% finding those compilers performance on
%% pointer-based inputs worse than the performance of lifted functions on
%% serialized inputs.

\paragraph{Versus competing libraries}

%% The CNF representation is just the traditional Haskell heap representation of
%% data constructors.  It wastes a word of header space for garbage collection
%% purposes, and it uses full 64-bit absolute pointers between heap objects.
%% The result is fast to read but not space efficient.
%% This is visible on benchmarks such as {\bf sumTree}, where CNF out-performs
%% Cap'N Proto by $3.5\times$.
%% % (/ 0.96 0.27)
%% Conversely, the Cap'N Proto representation is space efficient---using 40\% fewer
%% bytes for the binary tree---and faster to build.
%% % (- 1 (/ 805 1340.0))
%% %
%% CNF results are slow to build because they involve an extra copy: first to
%% create the data on the normal heap, second to copy it into the compact region.
%% This is why CNF's {\bf copyTree} is twice as fast as {\bf add1Leaves}, even
%% though the both computations walk the tree and build a new output tree, copy is
%% able to use a runtime system function to walk the data and copy directly from
%% input message to output message, without allocating on the regular (non-compact)
%% Haskell heap.

The biggest differences in \tabref{tab:litmus-table} are due
asymptotic complexity.
However, for constant factor performance, we see the expected
relationship---that our approach is faster than CNF and Cap'N Proto,
sometimes by an order of magnitude, \eg\,{\bf add1Leaves}.

CNF and Cap'N Proto encode some metadata in their serialization, to
support the GHC runtime, and protocol evolution, respectively.
On the other hand,
our compiler only uses offsets and tagged indirections
when needed, and the size ratio of the encodings depends on how much these features are used.
%
For example, {\bf rightmost} uses a data-encoding that includes random-access
offsets, and {\bf treeInsert} creates an output with a logarithmic number of
tagged indirections.  Thus while our size advantage over CNF is
%
$4\times$ smaller
% (/ 1340.0 335)
%
on {\bf buildTree}, it is only
%
$2.22\times$
% (/ 1340.0 603)
% (/ 1340000000.0 (+ 334000000 (* (- (expt 2 25) 1) 8)))
%
for {\bf rightmost}.
% , and $XYZ\times$ for {\bf treeInsert}.

CNF results are slow to build because they involve an extra copy: first to
create the data on the normal heap, second to copy it into the compact region.
This is why CNF's {\bf copyTree} is twice as fast as {\bf add1Leaves}, even
though the both computations walk the tree and build a new output tree, copy is
able to use a runtime system function to walk the data and copy directly from
input message to output message, without allocating on the regular (non-compact)
Haskell heap.

\paragraph{Composing traversals}

For offset-insertion, we allow the whole-program compiler to select the data
representation based on what consuming functions are applied to it.
%% In a more general setting (say, modular compilation, or messages received over
%% the network from program in another language), we would need to adopt
%% random-access nodes uniformly, which would, e.g., give {\bf leftmost} the same
%% small amount of overhead in its input tree as {\bf rightmost}.
In the presence of multiple functions traversing a single data structure,
any function demanding random access changes the representation for all of them.
% the ``weakest link'' determines how many random-access nodes are needed.
% compiler uses a data representation required to optimize the slowest one.
{\bf repMax} is one such example:
%\begin{code}[mathescape=true]
\il{ repMax t = propagateConst (findMax t) t}.
%\end{code} \vspace{-5mm}
%  repMax = propagateConst . findMax -- almost correct!
% Consider the {\bf repMax} program in which
Here {\bf findMax} only requires a partial scan (random access), but propagating
that value requires a full traversal.  In this case, the compiler would add
offsets to the datatype to ensure that `findMax' remains
logarithmic.
%
However, this causes the subsequent traversal (propagateConst) to slow
down, as it now has to unnecessarily skip over some extra bytes.  Likewise, if
we do not include {\bf findMax} in the whole program, the data remains fully
serialized, which is why {\bf propagateConst} and {\bf findMax} run separately
take less than 440ms, but run together take 480ms.  Yet the latter time is still
6$\times$ and 9$\times$ faster, respectively, than CNF and Cap'N Proto!



 \begin{table*}
  \begin{center}
    \small
      \begin{tabular}{ |c|c|c|c|c|c| }
        \hline
        Benchmark & Gibbon2 & Gibbon1 & NonPacked & CNF & CapnProto\hspace{-1mm} \\
        \hline
        % CNF: 100M iterations, calling NOINLINE id' function from NOINLINE rep.
        %      That gives 8.0ns (Allowing rep to inline, but not ID, makes it 2.15ns)
        % capnp: iter=20M median of 9 trials
        % Gibbon2: 100M iterations

        % Speedup:
        % CNF: (/ 2.1 2.1) = 1
        % CapnProto: (/ 129 2.1) = 61.42
        % Gibbon1: 152380952
        % Pointer: (/ 0.93 2.1) = 0.44
        {\bf id}:
        time, & 2.1ns &  0.32s  &  0.93ns &  2.1ns   & 129ns  \\
        complexity   & $O(1)$   &  $O(N)$ & $O(1)$    &  $O(1)$  & $O(1)$  \\
        %  output size (bytes)  &        &         &           &  8       &   8     \\
        \hline

        % CNF: 100M
        % capnp: iters = 20M median of 9

        % Speedup:
        % CNF: (/ 44 17.0) = 2.58
        % CapnProto: (/ 376 17.0) = 22.11
        % Gibbon1: (/ 18.0 17.0) = 1.05
        % Pointer: (/ 26 17.0) = 1.52
        {\bf leftmost}:
        time,        & 17ns         & 18ns        & 26ns        &    44ns      & 376ns            \\
        complexity   & $O(log(N))$  & $O(log(N))$ & $O(log(N))$ & $O(log(N))$  & $O(log(N))$ \\
        input size (bytes) & 335MB  & 335MB       & 335MB       &    1.34GB    & 805MB             \\
        \hline

        % CNF: 100M iterations
        % capnpL iter=20M, median of 9 trials
        % capnpL iter=20M, median of 9 trials
        % Speedup:
        % CNF: (/ 47 175.0) = 0.26
        % CapnProto: (/ 482 175.0) = 2.75
        % Gibbon1: 297142
        % Pointer: (/ 19 175.0) = 0.108
        {\bf rightmost}:
        time,        & 175ns        & 56ms    & 19ns        &    47ns      & 482ns \\
        complexity   & $O(log(N))$  & $O(N)$  & $O(log(N))$ & $O(log(N))$  & $O(log(N))$  \\
        input size (bytes) & 603MB & 335MB    & 335MB       &    1.34GB    & 805MB            \\
        \hline

        %capn: median of 9 trails iter=1
        % Speedup
        % CNF: (/ 4.5 0.27) = 16.66
        % CapnProto: (/ 1.8 0.27) = 6.66
        % Gibbon1: (/ 0.24 0.27) = 0.88
        % Pointer: (/ 2.7 0.27) = 10
        {\bf buildTree}:
        time,               & 0.27s     &  0.24s    & 2.7s &  4.5s  & 1.8s   \\
        complexity,         & $O(N)$    &  $O(N)$   &  $O(N)$ & $O(N)$  &  $O(N)$ \\
        output size (bytes) & 335MB     &  335MB    &  1.34GB &  1.34GB &  805MB       \\
        \hline

        % CNF: median of 9 trials:
        % capnp: median of 9 trails iter=1
        % Speedup:
        % CapnProto: (/ 3.8 0.25) = 15.2
        % CNF: (/ 2.7 0.25) = 10.8
        % Gibbon1: (/ 0.24 0.25) = 0.96
        % Pointer: (/ 3.1 0.25) = 12.4
        {\bf add1Leaves}:
        time,               & 0.25s     &  0.24s  & 3.1s     & 2.7s    &  3.8s  \\
        complexity,         & $O(N)$    &  $O(N)$ & $O(N)$   & $O(N)$  &  $O(N)$ \\
        \hline
        % CNF: median of 9 trials:
        % capn: median of 9 trails iter=1
        % Speedup:
        % CNF: (/ 0.27 0.095) = 2.84
        % CapnProto: (/ 0.96 0.095) = 10.10
        % Gibbon1: (/ 67.0 95) = 0.70
        % Pointer: 8.5
            {\bf sumTree}:
            time,               & 95ms     & 67ms     & 0.81s   &  0.27s  &  0.96s  \\
            complexity,         & $O(N)$   & $O(N)$   & $O(N)$  & $O(N)$  &  $O(N)$ \\
            \hline
            %capn: median of 9 trials iter =1
            % Speedup:
            % CNF: (/ 1.1 0.2) = 5.5
            % CapnProto: (/ 1.9 0.2) = 9.4
            % Gibbon1: (/ 0.24 0.2) = 1.2
            % pointer: (/ 3.5 0.2) = 17.5
                {\bf copyTree}:

                time,               & 0.2s      & 0.24s   & 3.5s   &  1.1s   &  1.9s       \\
                complexity,         & $O(N)$    & $O(N)$  & $O(N)$ & $O(N)$  &  $O(N)$ \\

                \hline
                \hline
                % CNF: median of 9 iters
                %capn: median of 9 trils iter=1
                % Gibbon numbers: median of 9 trials
                % Speedup:
                % CNF: (/ 4.27 0.5) = 8.54
                % CapnProto: (/ 2.1 0.5) = 4.2
                % Gibbon1: (/ 0.49 0.5) = 0.98
                % Pointer: (/ 2.96 0.5) = 5.92
                    {\bf buildSearchTree}:

                                    & 0.5s      & 0.49s     & 2.96s   &  4.27s   &  2.1s         \\
                    complexity,          & $O(N)$    & $O(N)$    & $O(N)$  & $O(N)$   &  $O(N)$ \\
                    output size (bytes)  & 603MB     & 603MB     & 1.61GB  &  1.61GB  &  805MB      \\

                    \hline

                    % CNF iters=1M
                    % This one is 2.5us with iters=1000
                    %            1.24us with iters=200K
                    %            1.04us with iters=1M
                    % NonPacked iters=1M
                    %
                    % capnp iters=20M median of 9
                    % Speedup:
                    % CNF: (/ 1 0.69) = 1.44
                    % CapnProto: (/ 1.3 0.69) = 1.88
                    % Gibbon: 144927
                    % Pointer: (/ 0.92 0.69) = 1.33
                    {\bf treeContains}:
                    time,                & 0.69$\mu$s   & 0.1s  & 0.92$\mu$s  &  1$\mu$s    & 1.3$\mu$s \\
                    complexity,          & $O(log(N))$ & $O(N)$ & $O(log(N))$ & $O(log(N))$ & $O(log(N))$ \\
                    \hline

                    % CNF iters=200K
                    % Capn median of 9 , iter=1
                    % Gibbon2 iters = 3500000
                    % Speedup:
                    % CNF: (/ 3.5 0.87) = 4.02
                    % CapnProto: (/ 150 0.87) = 172.4
                    % Gibbon: 436781
                    % Pointer: (/ 2.5 0.87) = 2.87
                    {\bf treeInsert}:

                    time,                & 0.87$\mu$s  & 0.38s  & 2.5$\mu$s   &  3.5$\mu$s   &  150$\mu$s  \\
                    complexity,          & $O(log(N))$ & $O(N)$ & $O(log(N))$ & $O(log(N))$  & $O(N)$  \\
                    avg bytes added      & 677 bytes        &  603MB  & 856 bytes   &  848 bytes  & 805MB  \\
                    \hline

                    %capnp: median of 9 trails iters=1M
                    {\bf InsertDestructive}:
                                         &  NA   & NA   & NA   & NA  &  1.37$\mu$s       \\
                    complexity,          &       &      &      &     & $O(log(N))$ \\

                    \hline

                    % Gibbon2:   100M iters
                    % Gibbon1:   20 iters
                    % Pointer:   100M iters
                    % CNF:       100M iters
                    % CapnProto: 1M iters
                    % Speedup:
                    % CNF: (/ 75 206.0) = 0.36
                    % CapnProto: (/ 597 206.0) = 2.89
                    % Gibbon: 427184
                    % Pointer: (/ 41 206.0) = 0.199
                    {\bf findMax}:
                    time,                & 206ns        & 88ms    & 41ns        & 75ns         & 597ns        \\
                    complexity           & $O(log(N))$  & $O(N)$  & $O(log(N))$ & $O(log(N))$  & $O(log(N))$  \\
                    \hline

                    % Speedup:
                    % CNF: (/ 4.2 0.43) = 9.76
                    % CapnProto: (/ 2.8 0.43) = 6.51
                    % Gibbon: (/ 0.42 0.43) = 0.97
                    % Pointer: (/ 3.3 0.43) = 7.67
                    {\bf propagateConst}:
                                         & 0.43s  &  0.42s  &  3.3s   & 4.2s   & 2.8s   \\
                    complexity,          & $O(N)$ &  $O(N)$ &  $O(N)$ & $O(N)$ & $O(N)$  \\
                    \hline

                    % Speedup:
                    % CNF: (/ 4.3 0.48) = 8.9
                    % CapnProto: (/ 2.9 0.48) = 6.04
                    % Gibbon: (/ 0.51 0.48) = 1.06
                    % Pointer: (/ 3.2 0.48) = 6.67
                    {\bf repMax}:
                    time,                & 0.48s  & 0.51s   & 3.2s    & 4.3s    & 2.9s   \\
                    complexity,          & $O(N)$ & $O(N)$  & $O(N)$  & $O(N)$  & $O(N)$  \\

                    \hline
      \end{tabular}
    \end{center}
  \vspace{-3mm}
    \caption{Tree-processing functions operating on serialized data.
      %% We are
      %% % gibbon pointer cnf capn
      %% 202 / 2.6 / 3.2 / 9.4  $\times$ geomean faster than
      %% Gibbon/NonPacked/CNF/CapNP, and
      %% 0.96 / 2.6 / 3.2 / 7.3 $\times$ faster for only apples-to-apples asymptotics.
      %% %% $3.2\times$,
      %% %% $9.4\times$, and
      %% %% $202\times$
      %% \captionscrunch{}
    }
    \label{tab:litmus-table}

\end{table*}


\subsection{Data Processing Benchmarks}

% \mav{Overview of how this technique can be applied to data processing programs.}
% \mav{If data to be processed is already serialized then this avoids marshaling cost.}
% \mav{If data to be processed is in some other format then we can still be faster by
%   transforming the data into a format that is more efficient to process (like the JSON
%   to byte array transformation necessary for the Twitter benchmark). And in these cases
%   it is often still faster than libraries that are meant to process the data in its
%   original form.}

Beyond microbenchmarks, Gibbon has been evaluated on more realistic benchmarks.
In the case of data processing, if data is already serialized then programs
may avoid the marshaling cost by operating directly on serialized data,
and if programs need to be converted from some other format it may still be
benefician to process that data into a dense, serialized form before
processing rather than processing the data in its original form

\paragraph{Twitter JSON Benchmark}

Here, we take a look at Twitter metadata consisting of user ID's and hashtags
for all tweets posted in 1 month, and count the occurrences of the hashtag
``cats'' in this dataset. The goal is to replicate and extend the CNF experiment
reported by \cite{cnf-icfp15}.

The dataset is stored on disk in JSON format, and we use RapidJSON v1.1.0
({\footnotesize\url{http://rapidjson.org/}}) as a performance baseline: a widely
recognized fast C++ JSON library.
%
In \figref{fig:twitter_slowdown_plot}, we vary the amount of data processed,
up to 1GB. (For each data-point, taking the median of 9 trials
ensures the data is already in the Linux disk cache.)
%
For fairness, all versions read the data via a single \il{mmap} call, plus
demand paging.

There are two RapidJSON versions. The ``lexer'' version never constructs an
object representing a parsed tweet, rather, it is a state-machine
that is able to count ``cats'' while tokenizing, {\em without parsing}.  It is
optimized to be as fast as possible for this particular JSON schema, with no
error detection (a non-compliant input would give silent failures and wrong
answers).
%
The ``parser'' version represents a more traditional and idiomatic situation use
of the library: calling the \il{.Parse()} method to produce a DOM object, and
then accessing its fields.
%
We have structured this benchmark to maximally advantage this parsing approach:
the 9,111,741 tweets processed in the rightmost data points of \figref{fig:twitter_slowdown_plot} are stored as one JSON object each, on each line of the input file.
%
Thus the data only needs to be read into memory once, and in a single pass the RapidJSON benchmark reads, parses, discards, and repeats.
%
Conversely, if the tweets were instead stored as a single JSON array, filling
the entire input file, then RapidJSON would have to parse the entire file
(writing the DOM tree out to memory, overflowing last level cache), then read
that same tree back into memory in a second pass to count hashtags.
%
Nevertheless, in spite of this single-pass advantage, Gibbon achieves
6$\times$ and 12$\times$ speedup over RapidJSON lexer/parser.
%
It processes the 9.1M tweets in 0.39s.


\begin{figure}
  \centering
  \input{twitter_slowdown_plot}
  \caption{Twitter data processing benchmark results}
  \label{fig:twitter_slowdown_plot}
\end{figure}

\paragraph{Point Correlation}

Point correlation  is a well-known algorithm used in data mining~\cite{gray2000n}:
given a set of points in a k-dimensional space, point correlation computes the number of points in that space  that lie within a
distance $r$ from a given point $p$.

% Rather than comparing each point in the space to the query point, one efficient way of searching such spaces is to store them in kd-trees~\cite{bentley75}. KD-trees are space-partitioning trees where the root of the kd tree represents the entire space, and each node's children represents a partition of that node's space into two subspaces.

% A kd-tree is a binary space-partitioning tree that in its simplest form  splits the space at each internal node into two sub-spaces
% around a split axes in one of the space dimensions, the left subtree stores the points to the left of the split access
% and the right subtree sotres the points to the right of the split access, the split axes alternates between the dimensions of
% the space at each level of the tree, when the number of the points in the subtree is one, a leaf node that stores
% that point is constructed.


In a naive implementation of point correlation, each point in the space needs
to be checked against the query point.
A more efficient approach is to use kd-trees~\cite{bentley75} to store the
points. KD-trees are space-partitioning trees where the root of the kd tree represents the entire space, and each node's children represents a partition of that node's space into two subspaces.
KD-trees allow the search
process to skip some regions in the space. By storing at each internal node
the boundaries within which all descendent points lies, the search process can
skip a subtree is a given point is far enough from the boundaries. As a
result, querying a kd-tree to perform point-correlation is $O(log\; n)$ instead
of $O(n)$. Note that it is exactly the process of ``skipping'' subtrees that
gives kd-tree-based point correlation its efficiency, but also that prevents a
normal packed representation from sufficing to implement the algorithm: there
is no way to skip past a subtree without performing a dummy traversal,
obviating the asymptotic performance gains.

We implemented both a standard pointer-based version of 2-point correlation in
C, as well as a version that operates over a packed representation augmented
with indirection pointers. Each interior node stores a rope-style
indirection pointer that maintains the size of its child subtrees. If a
traversal is truncated at that node, the cursor is incremented by the value in
that indirection pointer, skipping the subtrees and resuming traversal on the
rest of the tree.

\figref{fig:point_corr_plot} shows the speedup of the packed version
with respect to the standard pointer-based implementation for
different tree sizes. For each tree size, we ran 10 query points
through the tree. For small trees, the queries were performed 10000
times to produce sufficient runtime for accurate measurements. Each
experiment was performed 10 times, and the mean is reported.

For every tree size, the packed representation uses 56\% less memory
than the pointer-based trees. This reduction in memory usage has two
sources: nodes do not need to store left-child pointers; and more
efficient packing of data in the packed representation. For small
trees, the runtime performance of the packed and pointer versions are
comparable. For large trees, the packed version is up to 35\% faster
than the pointer-based version.

The relatively smaller performance improvement for this
benchmark versus the Twitter benchmarks is unsurprising. First, taking an
indirection means that any spatial locality gains from the packed
representation are lost, resulting in similar behavior to the
pointer-based version. Second, there is relatively more work to be
done per node in this benchmark, so the time spent in traversal of the
tree is relatively less, reducing the opportunity for improvement.


\begin{figure}
  \centering
  \input{point_corr_plot}
  \caption{Speedup of serialized implementation of point correlation versus
    pointer-based implementation.}
  \label{fig:point_corr_plot}
\end{figure}

\subsection{Abstract Syntax Trees}\label{subsec:ast}

One potentially fruitful application of programming with serialization is
the manipulation and processing of ASTs, or \emph{abstract syntax trees}.
This is a common task found in compilers and other programming tools that
process computer code, so it is desirable to do it quickly. There are many
situations where compile time must be minimized (for example, when a compiler
is running inside a user-facing program like a web browser), and applying
the technique of programming with serialized data has the potential to reduce
compile times by speeding up traversals of ASTs.

In addition,
it is not uncommon to represent compiled programs as \emph{bytecode}, which
resembles machine code but is intended for processing by an interpreter or
virtual machine. Languages like WebAssembly~\cite{webassembly} have
both a concrete and a bytecode specification for how to form programs
in the language, so tools that processed a language like WebAssembly would benefit
from the ability to operate directly on programs as bytecode and therefore
skipping the steps assembling and disassembling programs before manipulating them.

This section will discuss two benchmarks that test Gibbon's handling of programs
that operate on serialized abstract syntax trees: a subset of a compiler for
a simple language, and a traversal of macro-expanded Racket s-expressions.

\paragraph{Racket code benchmark}

\begin{figure}[t]
      \footnotesize
  \begin{code}
data Toplvl = DefineValues ListSym Expr | DefineSyntaxes ListSym Expr
            | BeginTop ListToplvl       | Expression Expr
data Expr = VARREF Sym | Top Sym  | Lambda Formals ListExpr  | App Expr ListExpr
      | CaseLambda LAMBDACASE     | If Expr Expr Expr        | SetBang Sym Expr
      | Begin ListExp             | Begin0 Expr ListExpr     | Quote Datum
      | QuoteSyntax Datum         | QuoteSyntaxLocal Datum
      | LetValues LVBIND ListExpr | LetrecValues LVBIND ListExpr
      | WithContinuationMark Expr Expr Expr
      | VariableReference Sym | VariableReferenceTop Sym | VariableReferenceNull
$\ldots$
\end{code}
\normalsize
  \caption{Excerpt of Racket Core AST definition, which
   follows \url{https://docs.racket-lang.org/reference/syntax-model.html}.
   There are nine data types total.}\label{fig:racket-core}
\end{figure}

In this portion of the evaluation, we look at
% two compiler passes, each of which takes a
the performance of two classes of tree walk
% (a fold and a map)
on full Racket Core syntax, an AST definition which is excerpted in \cref{fig:racket-core}.
%
These benchmarks consume a Racket abstract syntax tree as input and produce
either (1) a count of nodes, or (2) a new abstract syntax tree.

We generated a dataset of inputs by collecting all of the (macro-expanded) source
code from the main Racket distribution, which contains 4,456 files consuming
1GB of code which drops to 485MB when stripped of whitespace and comments, and
102MB once packed in our dense representation.
% (disregarding symbol tables).
%
We benchmark on this entire dataset, but report only on a subset, sampling from
a spectrum of sizes.
%
The largest single file was 1.4MB.
To simulate larger programs (as would be found in whole-program compilation), we
combined the largest $K$ files into one, varying $K$ from 1 to 4,456.
%
This is representative of a whole program compiler, which would indeed need to
load these modules as one tree.

\figref{fig:racket-core-slowdown} shows the performance of Gibbon's packed mode vs gibbon's
pointer-only mode, expressed as slowdowns of the pointer-based approaches over
packed. We measured the last level cache reference and cache misses and found
dramatic improvements in these for the packed approach (and modest differences
in the number of instructions executed).
%
Nevertheless,the performance of pointer-based
approach is good at small sizes: (1) trees are small and fit in cache, (2) the
single-threaded workload can acquire all of the last level cache, not contending
with other threads on the 16-core machine. The end result is that the system is
able to mask the bad behavior of these implementations at these sizes.
%
When the input/output tree sizes exceed the cache size, however, we see a phase
shift. Once we need to stream trees from memory, the smaller memory footprints
and linear access patterns of Gibbon's packed approach yield more significant
speedups.

\begin{figure}
  \begin{subfigure}[t]{\linewidth}
    \centering
    \input{slowdown_countnodes_plot}
    \caption{Racket core fold benchmark}
  \end{subfigure}
  \begin{subfigure}[t]{\linewidth}
    \centering
    \input{slowdown_treewalk_plot}
    \caption{Racket core map benchmark}
  \end{subfigure}
  \caption{Speedup of serialized implementation of Racket core benchmarks versus
    pointer-based implementation (y-axis is number of times faster Gibbon is than
    pointer-based C)}\label{fig:racket-core-slowdown}
\end{figure}


\paragraph{Compiler benchmark}
The compiler for this benchmark was implemented in \lamadt{}. It represents
programs as a control flow graph, and performs of a handful of simple compiler
passes before finally generating assembly code for a simple abstract machine.
The input grammar of the compiler is given
in~\figref{fig:coursecompilergrammar}, and the structure of the compiler is as
follows, roughly matching a few of the passes described
in the open source textbook Essentials of Compilation~\footnote{\url{https://github.com/IUCompilerCourse/Essentials-of-Compilation}}:
%
\begin{enumerate}
  \item \textbf{uniquify}: rename all variables to be unique across the program
  \item \textbf{optimizeJumps}: remove redundant jumps by simplifying all jump
        statements that target a block that immediately jumps somewhere else
  \item \textbf{eliminateDeadBlocks}: remove blocks that are not the target of
        any jump or conditional jump statement
  \item \textbf{assignHomes}: assign (stack) locations to local variables
  \item \textbf{codeGen}: print assembly code
\end{enumerate}

To evaluate the impact of using a packed AST on the compiler (similar to the
previous sections), I compared the benchmark times when the simple compiler was
compiled with Gibbon in both pointer-only and packed modes. On a randomly
generated input program consisting of a control-flow graph with 1000 blocks, the
packed version was 2.12$\times$ faster than the pointer-only version, andin general
the performance shown in \figref{fig:coursecompilerbench} is more consistent
(around 1.5$\times$ to 2$\times$) than the Racket core benchmarks.

\begin{figure}
\footnotesize
\begin{code}
data Program = ProgramC BlockList
data Exp = ArgC Arg | ReadC | NegC Arg | PlusC Arg Arg | NotC Arg
         | CmpC Cmp Arg Arg | $\ldots$
data Statement = AssignC Sym Exp | $\ldots$
data Tail = RetC Exp | SeqC Statement Tail | GotoC Sym
          | IfC Cmp Arg Arg Sym Sym | $\ldots$
data BlockList = BlockCons Sym Tail BlockList | BlockNil
$\ldots$
\end{code}
\caption{Excerpt of AST definition for compiler benchmark}\label{fig:coursecompilergrammar}
\end{figure}

\begin{figure}
  \input{coursecompiler_plot}
  \caption{Speedup of serialized implementation of compiler passes versus
  pointer-based implementation (y-axis is number of times faster Gibbon is than
  pointer-based C)}\label{fig:coursecompilerbench}
\end{figure}

\subsection{Parallel Programming Benchmarks}\label{subsec:parallelbench}

\begin{figure*}
  \footnotesize
  \centering
  %% \renewcommand{\arraystretch}{1.3}
  \setlength{\tabcolsep}{0.4em}
  \begin{tabular}{@{}r cc cccc r ccccc r ccccc@{}}
    \toprule
    & LoCal & \phantom{} & \multicolumn{4}{c}{Ours} & \phantom{} & \multicolumn{5}{c}{\MPL{}} & \phantom{} & \multicolumn{5}{c}{GHC} \\
    \cmidrule{2-2}
    \cmidrule{4-7}
    \cmidrule{9-13}
    \cmidrule{15-19}
    Benchmark & $T_s$ && $T_1$ & O & $T_{18}$ & S && $T_s$ & $T_1$ & O & $T_{18}$ & S && $T_s$ & $T_1$ & O & $T_{18}$ & S \\
    & (1) && (2) & (3) & (4) & (5) && (6) & (7) & (8) & (9) & (10) && (11) & (12) & (13) & (14) & (15) \\
    \midrule

    %%
    fib & 4.3 && 3.7 & -12.9 & 0.34 & 12.5 && 16 & 16.2 & 1 & 1.14 & 14 && 7 & 7.2 & 3 & 0.6 & 11.7 \\

    %%
    buildFib  & 6.8 && 5.9 & -13.6 & 0.52 & 13.1 && 25 & 25.1 & 0.2 & 1.8 & 13.9 && 12.7 & 12.7 & 0 & 1 & 12.7 \\

    %%
    buildTree & 0.77 && 0.78 & 0.54 & 0.11 & 7.1 && 1.4 & 1.9 & 31.3 & 0.4 & 3.6 && 4 & 4.4 & 9.2 & 0.57 & 7 \\

    %%
    add1Tree & 0.91 && 1.1 & 25.8 & 0.11 & 8.1 && 2.2 & 2.9 & 30.5 & 0.58 & 3.8 && 4 & 4.5 & 9.7 & 0.67 & 6 \\

    %%
    sumTree & 0.24 && 0.29 & 19.1 & 0.03 & 8.5 && 1.04 & 1.03 & -0.3 & 0.07 & 14.1 && 0.54 & 0.6 & 11.1 & 0.07 & 7.9 \\

    %%
    buildKdTree & 5.3 && 5.3 & 0 & 2.6 & 2 && 12.6 & 13.5 & 7.1 & 2.2 & 5.7 && 326.9 & 334 & 2.2 & 118.3 & 2.8 \\

    %%
    pointCorr & 0.14 && 0.14 & 0 & 0.014 & 10.1 && 0.62 & 0.62 & 0 & 0.05 & 12.9 && 0.16 & 0.18 & 18.1 & 0.014 & 11.1 \\

    %%
    barnesHut & 16.3 && 16.1 & -1.4 & 1.4 & 11.7 && 41.8 & 30.6 & -26.9 & 2.2 & 18.9 && 106.5 & 109.5 & 2.8 & 16.2 & 16.6 \\

    %%
    coins & 10.3 && 9.3 & -9.7 & 4.7 & 2.18 && 1.9 & 1.3 & -30.7 & 0.96 & 2.03 && 0.89 & 0.9 & 12.5 & 0.74 & 4.8 \\

    %%
    countnodes & 0.035 && 0.039 & 11.4 & 0.007 & 4.9 && 0.06 & 0.05 & -16.7 & 0.006 & 10 && 0.16 & 0.18 & 12.5 & 0.033 & 4.8 \\

    \bottomrule
  \end{tabular}
  \vspace{2mm}
  \normalsize
  \caption{
    Benchmark results.
    Column $T_s$ shows the run time of a sequential program.
    %% which serves the purpose of a sequential baseline.
    %
    $T_1$ is the run time of a parallel program on a single core, and
    $O$ the percentage overhead relative to $T_s$, calculated as
    $((T_1 - T_s) / T_s) * 100$.
    %
    $T_{18}$ is the run time of a parallel program on 18 cores and
    $S$ is the speedup relative to $T_s$, calculated as $T_s / T_{18}$.
    %
    The overhead (Column 3) and speedup (Column 5) for Ours are
    computed relative to sequential LoCal (Column 1).
    %
    For \MPL{} and GHC, the overheads (Columns 8 and 13) and speedups (Columns 10 and 15)
    are self-relative ---
    parallel \MPL{} and GHC programs are compared to their sequential variants.
    %
    All timing results are reported in seconds.
  }
  \label{fig:benchmark-results}
\end{figure*}

To measure the overheads of compiling parallel allocations using
fresh regions and indirection pointers, we compare our single-core performance
against the original, sequential \ourcalc{} implementation in the Gibbon compiler.
%
\ourcalc{} is also a good sequential baseline for performing speedup calculations
since its programs operate on serialized heaps, and
as shown in prior work, are significantly faster than their pointer-based counterparts.
%
Note that this chapter previously compared sequential constant factor performance
against a range of language implementations and compilers.
Since Gibbon generally
outperformed those compilers in sequential tree-traversal workloads, we focus
here on comparing against \ourcalc{} for sequential performance.

We also measure the scaling properties of our implementation by comparing its
performance to other programming languages and systems
that support efficient parallelism for recursive, functional programs
--- \MPL{}~\footnote{https://github.com/MPLLang/mpl}~\cite{MPL},
OCaml~\cite{OCaml}, and GHC.
%
\MPL{} (an extension of MLton~\footnote{http://www.mlton.org})
is a whole program optimizing compiler for the Standard ML~\cite{StandardML}
programming language,
%
and it supports nested fork/join parallelism, and generates extremely efficient code.
%% and hence serves as a baseline for comparing against a system that is pointer-based.
%
%
We compare against OCaml and GHC as the most optimized existing implementations of the
general purpose functional languages Objective Caml and Haskell respectively.

%% lscpu          -- processor info
%% free -m        -- memory
%% lsb-release -a -- OS

The experiments in this section are performed on a 48 core server made up of
2 $\times$ 2.9 GHz 24 core Intel Xeon Platinum 8268 processors, with 1.5TB of memory,
and running Ubuntu 18.04.
%
%\note{mention numactl.}
%
Each benchmark is run 9 times, and the median is reported.
%
To compile the C programs generated by our implementation we use
GCC 7.4.0 with all optimizations enabled (option \il{-O3}), and
the Intel Cilk Plus extension (option \il{-fcilkplus}) to realize parallelism.
%
To compile sequential LoCal programs, we use the Gibbon compiler but disable the
changes that add parallelism with appropriate flags.
%
For \MPL{}, we use version \il{20200220.150446-g16af66d05} compiled from its source code.
%
For OCaml, we use the Multicore OCaml compiler \cite{multicore-ocaml}
(version 4.10 with options \il{-O3}),
along with the \il{domainslib} \footnote{https://github.com/ocaml-multicore/domainslib}
library for parallelism.
%
For GHC, we use its version 8.6.5 (with options \il{-threaded -O2}) along with
the monad-par\cite{monad-par} library for parallelism.

\newcommand{\numbenchmarks}{\Red{10} }

We use the following set of of 10 benchmarks to evaluate performance.
%
For GHC, we use strict datatypes in benchmarks which generally offers the
same or better performance and avoids problematic interactions between
laziness and parallelism.

\begin{itemize}

\item \textbf{fib}:
  Compute the 48th fibonacci number with a sequential cutoff at n=30.

\item \textbf{buildFib}:
  This is an artificially designed benchmark that performs lot of parallel
  allocations, and has enough work to amortize their costs.
  %
  It constructs a balanced binary tree of depth 18, and computes
  the 20th fibonacci number at each leaf.
  %
  This benchmark is embarrassingly parallel, and it is included here to
  measure the overheads of parallel allocations under ideal conditions.
  %
  The sequential cutoff is at depth=6.

%% \item \textbf{buildTree} and \textbf{add1Tree} and \textbf{sumTree}:
%%   \il{buildTree} constructs a a balanced binary tree of depth 26 with
%%   an integer at the leaf, and \il{add1Tree} and \il{sumTree}
%%   operate on this tree.
%%   %% \il{add1Tree} is a mapper function which adds 1 to all the leaves and
%%   %% \il{sumTree} is a reducer which sums up all leaves in the tree.
%%   %% The sequential cutoff for each of these benchmarks is at depth=18.

\item \textbf{buildKdTree} and \textbf{countCorrelation} and \textbf{allNearest}:
  \il{buildKDTree} constructs a kd-tree \cite{kdtree}
  containing 1M 3-d points
  %%   %% in the Gauss–Kuzmin distribution
  in the Plummer distribution.
  %
  The sequential cutoff is at a node which contains less than 100K points.
  %
  \il{countCorrelation} takes as input a kd-tree and a list of 100 3-d points,
  and counts the number of points which are correlated to each one.
  %
  \il{allNearest} computes the nearest neighbor of all 1M 3-d points using the kd-tree.


%% \item \textbf{buildQuadTree} and \textbf{barnesHut}:
%%   \il{buildQuadTree} constructs a quad-tree
%%   containing 4M 2-d point-masses distributed uniformly within a square, and
%%   \il{barnesHut} uses this tree to run an nbody simulation over the given point-masses.
%%   %
%%   In this case, we implemented optimizations that go beyond the
%%   race-free, purely functional style of the other benchmarks.
%%   %
%%   For all three compilers, we apply point forces by updating an array in parallel, using
%%   potentially-racy mutation operations.  With library support these
%%   unsafe operations can be hidden behind a pure interface.

\item \textbf{barnesHut}:
  This benchmark uses a quad tree
  containing 1M 2-d point-masses distributed uniformly within a square
  to run an nbody simulation over the point-masses.

\item \textbf{coins}
  This benchmark is taken from GHC's NoFib \footnote{https://gitlab.haskell.org/ghc/nofib}
  benchmark suite. It is a combinatorial search problem that computes
  the number of ways in which a certain amount of money can be paid by
  using the given set of coins. It uses an append-list to store each combination
  of coins that adds up to the amount, and counts the number of non-nil elements
  in this list later. Only the time required to construct this list is measured.
  %
  The input set of coins and their quantities are
  \il{[(250,55),(100,88),(25,88),(10,99),(5,122),(1,177)]},
  and the amount to be paid is 999.
  %
  The sequential cutoff is at height=3.

\item \textbf{countNodes}:
  This benchmark taken from Gibbon's ordinary benchmark suite, as shown in \secref{subsec:ast}.
  %
  It operates on ASTs used internally in the Racket compiler, and counts the number
  of nodes in them.
  %
  The ASTs are a complicated datatype (9 mutually recursive types with 36
  data constructors) and are stored on disk as text files.
  %
  The implementations for GHC, \MPL, and OCaml have to parse these text files before
  operating on them.
  %
  For our implementation, we store the serialized data on disk in its
  binary format, and the program reads this data using a single \il{mmap} call.
  %
  To ensure an apples-to-apples comparison, we do not measure the time required
  to parse the text files for GHC, \MPL{}, and OCaml, and for our implementation,
  we run the \il{mmap}'d file through an identity function to ensure that it
  is loaded into memory.
  %
  The size of the text file is 1.2G, and that same file
  when serialized for our implementation is 356M.
  %
  The AST has around 100M nodes in it.

\item \textbf{mergeSort}:
  %
  This benchmark starts with a parallel algorithm, and then bottoms out to sequential
  quick sort at the leaves.
  %
  For our implementation, we use the \il{qsort} from the C standard library as
  the sequential sorting algorithm.
  %
  All other compilers use a sequential quick sort implemented in their source language.
  %
  All implementations use the number of elements in the array to decide when to
  bottom out to the sequential algorithm, but the exact threshold is different in each case.
  %
  For ours, it's when the array contains less than 100K elements.
  %
  For \MPL{}, it's 4096 elements.
  %
  The input in all cases is an array containing 4M random floating point numbers.
  %
  In this benchmark, we implemented optimizations that go beyond the
  race-free, purely functional style of the other benchmarks.
  %
  For all four compilers, the input array is first copied into a fresh one,
  and then this array is sorted in place, by using potentially-racy mutation operations.
  %
  With library support these unsafe operations can be hidden behind a pure interface.


%% \item \textbf{ray}:
%% \item \textbf{x86Compiler}:


\end{itemize}

We do not include some other classic benchmarks such as Mandelbrot and dense matrix
multiplication since they do not require allocating or traversing data in
serialized heaps.
%
With our compiler, it is likely these benchmarks would perform similarly to
their implementations written using C/C++, as shown by the in-place merge sort
benchmark.

\Figref{fig:benchmark-results} shows the full results of comparing performance
of programs written in parallel \ourcalc{} to the other implementations. In
general, we found that parallel \ourcalc{} programs performed as well or better
than parallel GHC and parallel MLton on a variety of benchmarks. When utilizing
18 cores, our geomean speedup is $1.87\times$ and $3.16\times$ over parallel
MLton and GHC, respectively.
%
This demonstrates that \ourcalc{} can be extended with
parallelism in a way that preserves the excellent performance of
ordinary \ourcalc{} while also scaling well to multiple cores.
%\note{Explain more.}



%% \newpage

% \appendix command is necessary to change chapter numbering.
% Appendices are optional

\appendix



\chapter{Type Safety Proof}\label{appendix:proof}

% \section{Variables and Substitution}
% %
% We use the convention that all variables for binding values,
% locations, and regions are distinct, and maintain this invariant
% implicitly.
% %
% The bindings sites of variables are summarized by the following:
% %
% \begin{itemize}
% \item Variables for binding values $\var$ are bound by function definitions
% $\FD$ and pattern matches $\pat$.

% \item Location variables $\locreg{\loc}{\reg}$ are bound by type schemes $\TS$, pattern matches
% $\pat$, and $\keywd{letloc}$ binders.

% \item Region variables $\reg$ are bound by type schemes $\TS$, pattern
% matches $\pat$, and $\keywd{letregion}$ binders.
% \end{itemize}
% %
% The use sites of variables are summarized by the following:
% \begin{itemize}
% \item Variables for binding varlues $\var$ are used by values $\VAL$.

% \item Location variables $\locreg{\loc}{\reg}$ are used by concrete locations $\concreteloc{\reg}{\ind}{\locreg{\loc}{\reg}}$,
% the argument list of function applications
% $\fapp{\overharpoon{\locreg{l}{r}}}{\overharpoon{\VAL}}$, the location
% argument of constructor applications
% $\datacon{\DC}{\keywd{\locreg{\loc}{\reg}}}{\overharpoon{\VAL}}$,
% located types $\hTYP$, and located expressions $\LE$.

% \item Region varaibles $\reg$ are used in the same places as location variables.
% \end{itemize}
% %
% We use the following conventions for variable substitution:
% %
% \begin{itemize}
% \item $\subst{\EXPR}{x}{v}$: Substitute $v$ for $x$ in $e$. We let the notation extend to vectors such that
% $\subst{\EXPR}{\overharpoon{x}}{\overharpoon{v}}$ denotes the iterated substitution $\subst{\EXPR}{\overharpoon{x_1}}{\overharpoon{v_1}} \ldots \subst{}{\overharpoon{x_n}}{\overharpoon{v_n}}$, where $n = |\overharpoon{x}| = |\overharpoon{v}|$.

% \item $\subst{\EXPR}{\locreg{\loc_1}{\reg_1}}{\locreg{\loc_2}{\reg_2}}$: Substitute location variable $\locreg{\loc_2}{\reg_2}$ for $\locreg{\loc_1}{\reg_1}$ in $\EXPR$. We extend this notation to vectors of locations in the same fashion, as described above.

% \item $\subst{\EXPR}{\reg_1}{\reg_2}$ : Substitute region variable $\reg_2$ for $\reg_1$ in $\EXPR$. We extend this notation to vectors of locations in the same fashion, as described above.

% \item Finally, we extend the aforementioned notation so that
%   substitution can act on environments $\CENV$, $\AENV$, and $\NENV$,
%   e.g.,
%   $\subst{\CENV}{\locreg{\loc_1}{\reg_1}}{\locreg{\loc_2}{\reg_2}}$.
% \end{itemize}

\paragraph{Metafunctions}

\begin{itemize}
\item $Function(f)$: An environment that maps a function $f$ to its definition $\FD$.

\item $Freshen(\FD)$: A metafunction that freshens all bound variables in function definition
$\FD$ and returns the resulting function definition.

\item $TypeOfCon(\DC):$ An environment that maps a data constructor to its type.

\item $TypeOfField(\DC,i)$: A metafunction that returns the type of the \il{i}'th field
of data constructor $\DC$.

\item $ArgTysOfConstructor(\DC)$: An environment that maps a data constructor to its field types.

\item $\allocptr{\reg}{\STOR}$: $\max \set{-1} \cup \set{ \indj \; | \; {(\reg \mapsto (\indj \mapsto \DC)) \in \STOR}}$.
\end{itemize}

% \section{Well-formedness of the Store}
% \label{sec:well-formedness}

% The well formedness of the store is defined by the top-level judgement
% \begin{displaymath}
% \storewf{\SENV}{\CENV}{\AENV}{\NENV}{\MENV}{\STOR}
% \end{displaymath}
% whose definition itself uses three other judgements.
% %
% All of these judgements are summarized in Table~\ref{tbl:swf-judgements}.

% %\floatstyle{boxed}\restylefloat{table}
% \begin{table}
% \bgroup
% \def\arraystretch{1.2}
% \setlength\tabcolsep{0.5cm}
% \begin{tabular}{lclp{6cm}}
%  & \textbf{Judgement form} & \textbf{Section} &\textbf{Summary}
%  \\\\
% \parbox[t]{3.5cm}{Store \\ well formedness} & $\storewf{\SENV}{\CENV}{\AENV}{\NENV}{\MENV}{\STOR}$ &
% \ref{sec:well-formedness} &
% The store $\STOR$ along with location map $\MENV$ are well formed with respect to
% typing environments $\SENV$, $\CENV$, and $\AENV$.
% \\\\
% End witness & $\ewitness{\TYP}{\concreteloc{\reg}{\ind_{s}}{}}{\STOR}{\concreteloc{\reg}{\ind_{e}}{}}$ &
% \ref{sec:end-witness} &
% The store address $\concreteloc{\reg}{\ind_{e}}{}$ is the position one
% after the last cell of the tree of type $\TYP$ starting at
% $\concreteloc{\reg}{\ind_{s}}{}$ in store $\STOR$.
% \\\\
% \parbox[t]{3.5cm}{Constructor-application \\ well formedness}
%  & $\storewfcfa{\CENV}{\MENV}{\STOR}$ &
% \ref{sec:well-formedness-constructors} &
% All in-flight data-constructor applications in store $\STOR$ along with location map $\MENV$
% are well formed with respect to constructor-progress typing environment $\CENV$.
% \\\\
% \parbox[t]{3.5cm}{Allocation \\ well formedness} & $\storewfca{\AENV}{\NENV}{\MENV}{\STOR}$ &
% \ref{sec:well-formedness-allocation} &
% Allocation in store $\STOR$ along with location map $\MENV$ is well formed
% with respect to allocation-typing environments $\AENV$ and $\NENV$.
% \end{tabular}
% \egroup
% \caption{Summary of judgements used to establish well formedness of the store.}
% \label{tbl:swf-judgements}
% \end{table}

% \paragraph{Notation for references to well-formedness judgements}
% Because there are many requirements specified inside the various
% well-formedness judgements, we introduce notation for referring
% to requirements individually.
% %
% For example, the notation
% \refwellformed{sec:well-formedness-allocation}{wf:impl-linear-alloc2}
% refers to the judgement
% \begin{align*}
% \storewfca{\AENV}{\NENV}{\MENV}{\STOR},
% \end{align*}
% specified in Section~\ref{sec:well-formedness-allocation},
% and in that judgement, rule number~\ref{wf:impl-linear-alloc2}.

% The definition of store well formedness follows.

% \paragraph{Judgement form}

% $\storewf{\SENV}{\CENV}{\AENV}{\NENV}{\MENV}{\STOR}$

% The well-formedness judgement specifies the valid layouts of the store by using the location
% map and the various environments from the typing judgement.
% %
% Rule~\ref{wf:map-store-consistency} specifies that, for each location in the store-typing environment,
% there is a corresponding concrete location in the location map and that concrete location satisfies
% a corresponding end-witness judgement.
% %
% Rules~\ref{wf:cfc} and~\ref{wf:ca} reference the judgements for well formedness concerning
% in-flight constructor applications (\secref{sec:end-witness}) and correct allocation in
% regions (\secref{sec:well-formedness-allocation}), respectively.
% %
% Finally, Rule~\ref{wf:impl1} specifies that the nursery and store-typing environments reference
% no common locations, which is a way of reflecting that each location is either in the process
% of being constructed and in the nursery, or allocated and in the store-typing environment, but
% never both.

% \paragraph{Definition}

% \begin{enumerate}

%     \item \label{wf:map-store-consistency} $ (\locreg{\loc}{\reg} \mapsto \TYP) \in \SENV \Rightarrow \\
%             ((\locreg{\loc}{\reg} \mapsto \concreteloc{\reg}{\ind_1}{}) \in \MENV \wedge \\
%             \ewitness{\TYP}{\concreteloc{\reg}{\ind_1}{}}{\STOR}{\concreteloc{\reg}{\ind_2}{}})
%           $

%     \item \label{wf:cfc} $\storewfcfa{\CENV}{\MENV}{\STOR}$

%     \item \label{wf:ca} $\storewfca{\AENV}{\NENV}{\MENV}{\STOR}$

%     \item \label{wf:impl1} $dom(\SENV) \cap \NENV = \emptyset $
% \end{enumerate}

% \subsection{End-Witness judgement}
% \label{sec:end-witness}

% \paragraph{Judgement form}

% $\ewitness{\TYP}{\concreteloc{\reg}{\ind_{s}}{}}{\STOR}{\concreteloc{\reg}{\ind_{e}}{}}$

% The end-witness judgement specifies the expected layout in the store of a fully
% allocated data constructor.
% %
% Rule~\ref{ewitness:impl1} requires that the first cell store a constructor
% tag of the appropriate type.
% %
% Rule~\ref{ewitness:impl2} specifies the address of the cell one past the tag.
% %
% Rule~\ref{ewitness:impl3} recursively specifies the positions of the constructor
% fields.
% %
% Finally, Rule~\ref{ewitness:impl4} specifies that the end witness of
% the overall constructor is the address one past the end of either the
% tag, if the constructor has zero fields, or the final field,
% otherwise.

% \paragraph{Definition}

% \begin{enumerate}
% \item \label{ewitness:impl1} $\STOR(\reg)(\ind_s) = \DC'$ \; \text{such that} \\
%       $\; \DATA\;\TYP = \overharpoon{\DC_1 \; \overharpoon{\sTYP}_1\;} \; | \; \ldots \; | \; \DC' \; \overharpoon{\sTYP}' \; | \; \ldots \; | \; \overharpoon{\DC_m \; \overharpoon{\sTYP}_m\;}$

% \item \label{ewitness:impl4} $\overharpoon{w_0} = \ind_s + 1$

% \item \label{ewitness:impl2}
%   $\ewitness{\overharpoon{\TYP'_1}}{\concreteloc{\reg}{\overharpoon{w_0}}{}}{\STOR}{\concreteloc{\reg}{\overharpoon{w_1}}{}} \wedge$ \\
%   $\ewitness{\overharpoon{\TYP'_{j+1}}}{\concreteloc{\reg}{\overharpoon{w_j}}{}}{\STOR}{\concreteloc{\reg}{\overharpoon{w_{j+1}}}{}}$
%   \\ where $\indj \in \set{1,\ldots,n-1} ; n = | \overharpoon{\TYP'} |$

% \item \label{ewitness:impl3}
%   $\ind_e = \overharpoon{w_n}$
% \end{enumerate}

% \subsection{Well-formedness of constructor application}
% \label{sec:well-formedness-constructors}

% \paragraph{Judgement form}

% $\storewfcfa{\CENV}{\MENV}{\STOR}$

% The well-formedness judgement for constructor application specifies the various constraints
% that are necessary for ensuring correct formation of constructors, dealing with constructor
% application being an incremental process that spans multiple \ourcalc{} instructions.
% %
% Rule~\ref{wfconstr:constraint-start} specifies that, if a location corresponding to the
% first address in a region is in the constraint environment, then there is a corresponding
% entry for that location in the location map.
% %
% Rule~\ref{wfconstr:constraint-tag} specifies that, if a location corresponding to the address one past a constructor
% tag is in the constraint environment, then there are corresponding locations for the address
% of the tag and the address after in the location map.
% %
% Rule~\ref{wfconstr:constraint-after} specifies that, if a location corresponding to the address
% one past after a fully allocated constructor application is in the constraint environment,
% then there are corresponding locations for the address one past the constructor application
% and for the address of the start of that constructor application in the location map, as well as the existence
% of an end witness for that fully allocated location.

% \paragraph{Definition}

% \begin{enumerate}
%     \item \label{wfconstr:constraint-start} $ (\locreg{\loc}{\reg} \mapsto \startr{\reg}) \in \CENV \Rightarrow \\
%             (\locreg{\loc}{\reg} \mapsto \concreteloc{\reg}{0}{}) \in \MENV $

%     \item \label{wfconstr:constraint-tag} $ (\locreg{\loc}{\reg} \mapsto (\locreg{\loc'}{\reg} + 1)) \in \CENV \Rightarrow
%             \\
%             (\locreg{\loc'}{\reg} \mapsto \concreteloc{\reg}{\ind_l}{})  \in \MENV \wedge \\
%             (\locreg{\loc}{\reg} \mapsto \concreteloc{\reg}{\ind_l + 1}{})  \in \MENV
%             $

%     \item \label{wfconstr:constraint-after} $ (\locreg{\loc}{\reg} \mapsto \afterl{\tyatlocreg{\TYP}{\locreg{\loc'}{\reg}}{\reg}}) \in \CENV \Rightarrow \\
%             ((\locreg{\loc'}{\reg} \mapsto \concreteloc{\reg}{\ind_1}{}) \in \MENV \wedge \\
%             \ewitness{\TYP}{\concreteloc{\reg}{\ind_1}{}}{\STOR}{\concreteloc{\reg}{\ind_2}{}} \wedge \\
%             (\locreg{\loc}{\reg} \mapsto \concreteloc{\reg}{\ind_2}{}) \in \MENV)
%             $

% \end{enumerate}

% \subsection{Well-formedness concerning allocation}
% \label{sec:well-formedness-allocation}

% \paragraph{Judgement form}

% $\storewfca{\AENV}{\NENV}{\MENV}{\STOR}$

% The well-formedness judgement for safe allocation specifies the various properties
% of the location map and store that enable continued safe allocation, avoiding in particular
% overwriting cells, which could, if possible, invalidate overall type safety.
% %
% Rule~\ref{wf:impl-linear-alloc} requires that, if a location is in both the allocation
% and nursery environments, i.e., that address represents an in-flight
% constructor application, then there is a corresponding location in the location
% map and the address of that location is the highest address in the store.
% %
% Rule~\ref{wf:impl-linear-alloc2} requires that, if there is an address in the allocation
% environment and that address is fully allocated, then the address of that location is the
% highest address in the store.
% %
% Rule~\ref{wf:impl-write-once} requires that, if there is an address in the nursery, then
% there is a corresponding location in the location map, but nothing at the corresponding
% address in the store.
% %
% Finally, Rule~\ref{wf:impl-empty-region} requires that, if there is a region that has been
% created but for which nothing has yet been allocated, then there can be no addresses
% for that region in the store.

% \paragraph{Definition}

% \begin{enumerate}
%     \item \label{wf:impl-linear-alloc} $ ((\reg \mapsto \locreg{\loc}{\reg}) \in \AENV \wedge \locreg{\loc}{\reg} \in \NENV) \Rightarrow \\
%           ((\locreg{\loc}{\reg} \mapsto \concreteloc{\reg}{\ind}{}) \in \MENV \wedge
% %          \ind = \max \set{0} \cup \set{\indj \; | \; \concreteloc{\reg}{\indj}{} \in \MENV} \wedge \\
%           \ind > \allocptr{\reg}{\STOR})
%           $

%     \item \label{wf:impl-linear-alloc2} $ ((\reg \mapsto \locreg{\loc}{\reg}) \in \AENV \wedge
%     \, (\locreg{\loc}{\reg} \mapsto \concreteloc{\reg}{\ind_s}{}) \in \MENV \wedge \locreg{\loc}{\reg} \not \in \NENV \wedge
%     \, \ewitness{\TYP}{\concreteloc{\reg}{\ind_s}{}}{\STOR}{\concreteloc{\reg}{\ind_e}{}}) \Rightarrow \\
%           \ind_e > \allocptr{\reg}{\STOR}
%           $

%     \item \label{wf:impl-write-once} $ \locreg{\loc}{\reg} \in \NENV \Rightarrow \\
%           ((\locreg{\loc}{\reg} \mapsto \concreteloc{\reg}{\ind}{}) \in \MENV \wedge \\
%           (\reg \mapsto (\ind \mapsto \DC)) \not \in \STOR)
%           $

%     \item \label{wf:impl-empty-region} $(\reg \mapsto \emptyset) \in \AENV \Rightarrow \\
%     \reg \not \in dom(\STOR)$
% \end{enumerate}

% \newcommand{\substlemmasubsts}[1]{\subst{#1}{\overharpoon{\var}}{\overharpoon{\VAL}} \subst{}{\overharpoon{\locreg{\loc}{\reg}}}{\overharpoon{\locreg{\loc'}{\reg'}}} \subst{}{\locreg{\loc}{\reg}}{\locreg{\loc'}{\reg'}}}

\paragraph{Progress and Preservation}
\label{proof:typesafety}

\begin{lemma}[Substitution lemma]
  \label{lemma:substitution}
  \begin{align*}
  \text{If} \quad & \TENV \cup \set{\overharpoon{\var_1} \mapsto \overharpoon{\TYP_1} \ensuremath{@} \overharpoon{\locreg{\loc_1}{\reg_1}}, \ldots, \overharpoon{\var_n} \mapsto  \overharpoon{\TYP_n} \ensuremath{@} \overharpoon{\locreg{\loc_n}{\reg_n}}}; \SENV; \CENV; \AENV; \NENV \vdash \AENV'; \NENV'; \EXPR : \tyatlocreg{\TYP}{\loc}{\reg}\\
  \text{and} \quad & \TENV; \SENV'; \CENV'; \AENV'; \NENV' \vdash \AENV'; \NENV'; \overharpoon{\VAL_i} : \overharpoon{\TYP_i} \ensuremath{@} \overharpoon{\locreg{\loc'_i}{\reg'_i}} \qquad \; i \in \set{1, \ldots, n}\\
  \text{then} \quad & \TENV; \SENV'; \CENV'; \AENV'; \NENV' \vdash \AENV'''; \NENV'''; \substlemmasubsts{\EXPR} : \tyatlocreg{\TYP}{\loc'}{\reg'}\\
   \text{where} \quad & \SENV = \SENV_0 \cup \set{\overharpoon{\locreg{\loc_1}{\reg_1}} \mapsto \overharpoon{\TYP_1}, \ldots, \overharpoon{\locreg{\loc_n}{\reg_n}} \mapsto \overharpoon{\TYP_n}}\\
   \text{and} \quad & \forall_{(\var \mapsto \TYP'' \ensuremath{@} \locreg{\loc''}{\reg''}) \in \TENV} . (\locreg{\loc''}{\reg''} \mapsto \TYP'') \in \SENV_0 \\
  \text{and} \quad & dom(\SENV) \cap \NENV = \emptyset\\
  \text{and} \quad & \NENV = \NENV_0 \cup \locreg{\loc}{\reg}\\
  \text{and} \quad & \SENV' = \SENV \cup \set{\overharpoon{\locreg{\loc'_1}{\reg'_1}} \mapsto \overharpoon{\TYP_1}, \ldots, \overharpoon{\locreg{\loc'_n}{\reg'_n}} \mapsto \overharpoon{\TYP_n}}\\
  \text{and} \quad & \CENV' = \subst{\CENV}{\overharpoon{\locreg{\loc}{\reg}}}{\overharpoon{\locreg{\loc'}{\reg'}}} \subst{}{\locreg{\loc}{\reg}}{\locreg{\loc'}{\reg'}} \\
  \text{and} \quad & \AENV' = \subst{\AENV}{\overharpoon{\locreg{\loc}{\reg}}}{\overharpoon{\locreg{\loc'}{\reg'}}} \subst{}{\locreg{\loc}{\reg}}{\locreg{\loc'}{\reg'}} \subst{}{\reg}{\reg'}\\
  \text{and} \quad & \NENV' = \subst{\NENV}{\locreg{\loc}{\reg}}{\locreg{\loc'}{\reg'}}
  \end{align*}
\end{lemma}
\begin{nproof}
  The proof is by rule induction on the given typing derivation.
  %
    \begin{bcase} \tvar{}, \tconcreteloc{}\\
    These cases discharge vacuously because
    the corresponding typing judgements cannot establish that the expression $\EXPR$ has type
    $\tyatlocreg{\TYP}{\loc}{\reg}$, as required by the premise of the lemma.
    %
    The reason is that the premise of the lemma also assumes that
    $\locreg{\loc}{\reg} \in \NENV$ and $dom(\SENV) \cap \NENV = \emptyset$, but by inversion
    on the respective typing judgements, it must be that $(\locreg{\loc}{\reg} \mapsto \TYP) \in \SENV$,
    thereby resulting in a contradiction.
  \end{bcase}

  \begin{bcase}
    \begin{mathpar}
    \rtdatacon{}
    \end{mathpar} \\
    By inversion on the typing judgement, there are three proof obligations for this case.
    %
    The first one concerns the subtitution of
    location $\locreg{\loc}{\reg}$, which changes the type of the
    term $\EXPR$ from $\tyatlocreg{\TYP}{\loc}{\reg}$ to
    $\tyatlocreg{\TYP}{\loc'}{\reg'}$.
    %
    The specific obligation is to establish that
    all uses of $\locreg{\loc}{\reg}$ in the
    typing judgement are properly substituted by $\locreg{\loc'}{\reg'}$, thereby
    satisfying the corresponding parts of the typing judgement that need to
    reflect the change in the result location.
    %
    The uses of $\locreg{\loc}{\reg}$ in the typing judgement
    are the first argument of the constructor application,
    the result type, the constraint environment $\CENV$, and environments $\AENV$, $\NENV$,
    $\AENV'$, and $\NENV'$.
    %
    The corresponding updates are established by inspection of the various substitutions
    in the consequent of the lemma, which affect $\EXPR$ and the typing environments.
    %
    The second obligation concerns the locations used by the typing judgement in $\CENV$, each
    of which is substituted as needed in the environment $\CENV'$.

    The third and final obligation is to establish typing judgements required by the premise of \tdatacon{}
    that concern the arguments of the constructor application.
    %
    To distinguish the constructor arguments from the values $\overharpoon{\VAL}$ that are being substituted,
    let the constructor arguments be $\overharpoon{\VAL'}$, and $m = |\overharpoon{\VAL'}|$.
    %
    Then the specific obligation is to establish the typing judgements
    %
    \begin{displaymath}
       \TENV;\SENV';\CENV';\AENV';\NENV' \vdash \AENV';\NENV';\substlemmasubsts{\overharpoon{\VAL'_{k}}} : \overharpoon{\tyatlocreg{\TYP_{k}'}{\loc''_{k}}{\reg}},
     \end{displaymath}
     %
     for all $k \in \set{1, \ldots, m}$, and for some suitable corresponding locations
     $\locreg{\loc''_{k}}{\reg}$.
     %
     Each value $\overharpoon{\VAL'_k}$ is either a variable or a concrete location.
    %
    \begin{itemize}
    \item Case $\overharpoon{\VAL'_k} = \yvar$, for some variable $\yvar$:\\
      \begin{itemize}
      \item Case $\yvar = \overharpoon{\var_{\indj}}$, for some $\indj$:\\
        Now, the obligation is to establish that the value resulting from the substitution of $\yvar$, namely
        $\overharpoon{\VAL_{\indj}}$, has type $\overharpoon{\TYP'_{k}} \ensuremath{@} \overharpoon{\locreg{\loc'_{\indj}}{\reg}}$.
        %
        From the premise of the lemma, we have that
        \begin{displaymath}
        \TENV; \SENV'; \CENV'; \AENV'; \NENV' \vdash \AENV'; \NENV'; \overharpoon{\VAL_j} : \overharpoon{\TYP_j} \ensuremath{@} \overharpoon{\locreg{\loc'_j}{\reg}},
        \end{displaymath}
        and, moreover, by inversion on \tdatacon{}, we can conclude that $\overharpoon{\TYP_j} = \overharpoon{\TYP'_k}$,
        thereby establishing that
       \begin{displaymath}
        \TENV; \SENV'; \CENV'; \AENV'; \NENV' \vdash \AENV'; \NENV'; \overharpoon{\VAL_j} : \overharpoon{\TYP'_k} \ensuremath{@} \overharpoon{\locreg{\loc'_j}{\reg}},
        \end{displaymath}
        and thus discharging this case.
      \item Case $\yvar \neq \overharpoon{\var_{\indj}}$, for all $\indj$:\\
        This case discharges immediately by implication of the typing judgement of the
        source term given in the premise of this lemma, and by inversion on \tvar{}.
      \end{itemize}
    \item Case $\overharpoon{\VAL'_k} = \concreteloc{\reg}{\ind'''}{\locreg{\loc''}{\reg}}$,
      for some location $\locreg{\loc''}{\reg}$, $\ind'''$\\
      \begin{itemize}
      \item Case $\locreg{\loc''}{\reg} = \overharpoon{\locreg{\loc_{\indj}}{\reg}}$, for some
        $\indj$:\\
        The specific obligation is to establish the type of
        the concrete location affected by the substitution of the location $\locreg{\loc''}{\reg}$
        for $\overharpoon{\locreg{\loc'_{\indj}}{\reg}}$, that is,
        \begin{displaymath}
        \TENV; \SENV'; \CENV'; \AENV'; \NENV' \vdash \AENV'; \NENV'; \concreteloc{\reg}{\ind'''}{\overharpoon{\locreg{\loc'_{\indj}}{\reg}}} : \overharpoon{\TYP'_k} \ensuremath{@} \overharpoon{\locreg{\loc'_j}{\reg}}.
        \end{displaymath}
        The above follows from the facts $\SENV'(\overharpoon{\locreg{\loc'_{\indj}}{\reg}}) = \overharpoon{\TYP_{\indj}}$
        and $\overharpoon{\TYP_j} = \overharpoon{\TYP'_k}$, using
        similar reasoning to the previous case, thus discharging this case.
      \item Case $\locreg{\loc''}{\reg} = \locreg{\loc}{\reg}$:\\
      Impossible, because $\locreg{\loc''}{\reg} \in dom(\SENV)$, but from the premise of this lemma,
      $\locreg{\loc}{\reg} \in \NENV$ and $dom(\SENV) \cap \NENV = \emptyset$.
      \item Case $\locreg{\loc''}{\reg} \neq \overharpoon{\locreg{\loc_{\indj}}{\reg}}$, for all
        $\indj$, and $\locreg{\loc''}{\reg} \neq \locreg{\loc}{\reg}$:\\
        This case discharges straightforwardly because, by inversion on \tconcreteloc{},
        $(\locreg{\loc''}{\reg} \mapsto \TYP'') \in \SENV$, thus implying that
        $(\locreg{\loc''}{\reg} \mapsto \TYP'') \in \SENV'$,
        as needed.
      \end{itemize}
    \end{itemize}
  \end{bcase}

  \begin{bcase}
    \begin{mathpar}
    \rtlet{}
    \end{mathpar} \\
    This case discharges via straightforward uses of the induction hypothesis
    for the let-bound expression and the body.
  \end{bcase}


  \begin{bcase} \tlregion{}, \tllstart{}, \tlltag{}, \tllafter{}, \tapp{}, \tcase{} \\
  These remaining cases discharge by similar uses of the induction hypothesis.
  \end{bcase}

$\blacksquare$
\end{nproof}

\begin{lemma}[Progress]
  \label{lemma:progress}
  \begin{displaymath}
  \begin{aligned}
  \text{if} \;\; & \emptyset;\SENV;\CENV;\AENV;\NENV \vdash \AENV';\NENV';\EXPR : \hTYP \\
  \text{and} \;\; & \storewf{\SENV}{\CENV}{\AENV}{\NENV}{\MENV}{\STOR} \\
  \text{then} \;\; & \EXPR \; \mathit{value} \\
  \text{else} \;\; & \STOR;\MENV;\EXPR \stepsto \STOR';\MENV';\EXPR'
  \end{aligned}
  \end{displaymath}
\end{lemma}

\begin{nproof}
  The proof is by rule induction on the given typing derivation.
  %
  \begin{bcase}
    \begin{mathpar}
    \rtdatacon{}
    \end{mathpar} \\

    Because $\EXPR = \datacon{\DC}{\keywd{\locreg{\loc}{\reg}}}{\overharpoon{\VAL}}$ is not
    a value, the proof obligation is to show that there is a rule in the dynamic semantics whose
    left-hand side matches the machine configuration $\STOR;\MENV;\EXPR$.
    %
    The only rule that can match is \ddatacon{}, but to establish the
    match, there remains one obligation, which is obtained
    by inversion on \ddatacon{}.
    %
    The particular obligation is to establish that
    $\concreteloc{\reg}{\ind}{} = \MENV(\locreg{\loc}{\reg})$,
    for some $\ind$.
    %
    To obtain this result, we need to use the well formedness
    of the store, given by the premise of this lemma, and in particular rule
    \refwellformed{sec:well-formedness-allocation}{wf:impl-write-once}.
    %
    But a precondition for using
    \refwellformed{sec:well-formedness-allocation}{wf:impl-write-once} that
    the location is in the nursery, i.e., $\locreg{\loc}{\reg} \in \NENV$.
    %
    This precondition is satisfied by inversion on \tdatacon{}.
    %
    Our application of rule \refwellformed{sec:well-formedness-allocation}{wf:impl-write-once}
    therefore yields the desired result, thereby discharging this case.
  \end{bcase}

  \begin{bcase}
    \begin{mathpar}
    \rtllafter{}
    \end{mathpar}

    Because $\EXPR
    = \letloc{\locreg{\loc}{\reg}}{\afterl{\tyatlocreg{\TYP'}{\loc_{1}}{\reg}}}{\EXPR'}$
    is not a value, the proof obligation is to show that
    there is a rule in the dynamic semantics whose left-hand side matches the machine
    configuration $\STOR;\MENV;\EXPR$.
    %
    The only rule that can match is \dletlocafter{}, but the match is
    dependent on two further obligations, which are
    implied by inversion on \dletlocafter{}.
    %
    The first one is to establish that
    $\concreteloc{\reg}{\ind}{} = \MENV(\locreg{\loc_1}{\reg})$.
    %
    To do so, we need to use
    rule \refwellformed{sec:well-formedness}{wf:map-store-consistency}
    of the well-formedness of the store.
    %
    This rule requires that $\SENV(\locreg{\loc_1}{\reg}) = \TYP'$,
    which is established by inversion on \tllafter{}.
    %
    As such,
    we have $(\loc_1 \mapsto \concreteloc{\reg}{\ind}{}) \in \MENV$, as needed.
    %
    The second and final obligation is to establish that, for some $\indj$,
    $\ewitness{\TYP'}{\concreteloc{\reg}{\ind}{}}{\STOR}{\concreteloc{\reg}{\indj}{}}$.
    %
    Again, we use well-formedness
    rule \refwellformed{sec:well-formedness}{wf:map-store-consistency} to
    discharge the obligation, and thus this case.
    %
  \end{bcase}

  \begin{bcase}
    \tlltag\\ Similar to the previous case.
  \end{bcase}

  \begin{bcase}
    \label{lemma:progress:trivials}
    \tllstart, \tlregion, \tapp \\ These cases discharge immediately because \dletlocstart{},
    \dletregion{}, and \dapp{} match their corresponding machine configurations unconditionally.
  \end{bcase}

  \begin{bcase}
    \tvar, \tconcreteloc \\ These cases discharge immediately because $\EXPR$ is a value.
  \end{bcase}

  \begin{bcase}
    \begin{mathpar}
    \rtlet{}
  \end{mathpar}

     Because $\EXPR = \letpack{\var : \tyatlocreg{\TYP_1}{\loc_1}{\reg_1}}{\EXPR_1}{\EXPR_2}$ is not a value, the proof obligation is to
    show that there is a rule in the dynamics whose left-hand side matches the machine
    configuration $\STOR;\MENV;\EXPR$.
    %
    If $\EXPR_1$ is a value, then the rule discharges immediately because \dletval{} matches
    $\EXPR$ unconditionally.
    %
    Otherwise, if $\EXPR_1$ is not a value, then the only other rule that can match
    is \dletexp.
    %
    To match \dletexp{}, the only requirement is to match the left-hand side of the rule
    $\STOR;\MENV;\EXPR_1 \stepsto \STOR';\MENV';\EXPR'_1$ in the premise, for some $\STOR'$, $\MENV'$, and $\EXPR'_1$.
    %
    To obtain this result, we need to use the induction hypothesis, which is in this instance
    %
    \begin{displaymath}
    \begin{aligned}
      \text{if} \;\; & \emptyset;\SENV;\CENV;\AENV;\NENV \vdash \AENV';\NENV';\EXPR_1 : \tyatlocreg{\TYP}{\loc_1}{\reg_1} \\
      \text{and} \;\; & \storewf{\SENV}{\CENV}{\AENV}{\NENV}{\MENV}{\STOR} \\
      \text{then} \;\; & \EXPR_1 \; \mathit{value} \\
      \text{else} \;\; & \STOR;\MENV;\EXPR_1 \stepsto \STOR';\MENV';\EXPR'_1.
    \end{aligned}
    \end{displaymath}
    %
    By inversion on \tlet{}, we have $\emptyset;\SENV;\CENV;\AENV;\NENV \vdash \AENV';\NENV';\EXPR_1: \tyatlocreg{\TYP_1}{\loc_1}{\reg_1}$, and, from the premise of this lemma, we have
    $\storewf{\SENV}{\CENV}{\AENV}{\NENV}{\MENV}{\STOR}$.
    %
    Thus, by the consequent of the  induction hypothesis, we have that either $\EXPR_1$ is a value (which we have
    already ruled out) or that
    $\STOR;\MENV;\EXPR_1 \stepsto \STOR';\MENV';\EXPR'_1$, thereby discharging this case.

  \end{bcase}

  \begin{bcase}
    \begin{mathpar}
    \rtcase{} \\
    \text{and} \\
    \rtpat{}
  \end{mathpar}

    Because the given expression $\EXPR = \case{\VAL}{\overharpoon{\pat}}$ is not a value, the proof obligation
    is to show that there is a rule in the dynamic semantics whose left-hand side matches the machine configuration $\STOR;\MENV;\EXPR$.
    %
    The only rule that can match is \dcase{}, and
    there are three requirements to match \dcase{}.
    %
    The first of which is that the value
    $\VAL$ is a concrete location of the form $\concreteloc{\reg'}{\ind}{\locreg{\loc'}{\reg'}}$.
    %
    Any value $\VAL$ is, by inspection of the grammar of \ourcalc{}, either a variable or a concrete location.
    %
    But because $\VAL$ is well typed with respect to the
    empty typing environment $\TENV = \emptyset$, the value $\VAL$
    cannot be a variable in this instance, owing to inversion on \tvar{} and \tconcreteloc{},
    thereby ensuring $\VAL$ is a concrete
    location, and thus discharging this requirement.
    %
    The second requirement for \dcase{} is that the tag is in the
    expected location in the store, i.e., $\STOR(\reg')(\ind) = \DC$.
    %
    To satisfy this requirement, we start by using the jugement $\storewf{\SENV}{\CENV}{\AENV}{\NENV}{\MENV}{\STOR}$,
    from the premise of this lemma,
    and in particular, unpacking from this judgement the property \refendwitness{sec:end-witness}{ewitness:impl1}.
    %
    To use this property, we need that $(\locreg{\loc'}{\reg'} \mapsto \TYP') \in \SENV$, which
    is given by inversion on the given typing rule \tcase{}.
    %
    From the unpacking, we obtain that
    \begin{align}
    (\locreg{\loc'}{\reg'} \mapsto \concreteloc{\reg'}{\ind}{} \in \MENV) \wedge \\
      \ewitness{\TYP'}{\concreteloc{\reg'}{\ind}{}}{\STOR}{\concreteloc{\reg'}{\ind'}{}} \label{prf-loc1}.
      \end{align}
    %
    From the end-witness judgement, in particular,
    \refendwitness{sec:end-witness}{ewitness:impl1},
    we establish that
    $\STOR(\reg')(\ind) = \DC$, thereby discharging the second requirement.
    %
    The third and final requirement for \dcase{} is that the arguments succeeding
    the tag are in the expected locations, i.e.,
        \begin{align*}
    & \ewitness{\overharpoon{\TYP'_{1}}}{\concreteloc{\reg'}{\ind + 1}{}}{\STOR}{\concreteloc{\reg'}{\overharpoon{w_1}}{}} \wedge \\
    & \ewitness{\overharpoon{\TYP'_{j+1}}}{\concreteloc{\reg'}{\overharpoon{w_j}}{}}{\STOR}{\concreteloc{\reg'}{\overharpoon{w_{j+1}}}{}}
    \end{align*}
    %
    The above is established by expanding the judgement obtained in~\ref{prf-loc1}, namely
    $\ewitness{\TYP'}{\concreteloc{\reg'}{\ind}{}}{\STOR}{\concreteloc{\reg'}{\ind'}{}}$,
    using in particular, the end-witness rule
    \refendwitness{sec:end-witness}{ewitness:impl2} to obtain the
    needed judgements.
    %
    This final requirement discharges the case.
  \end{bcase}

$\blacksquare$

\end{nproof}

\begin{lemma}[Preservation]
  \label{lemma:preservation}
  \begin{displaymath}
    \begin{aligned}
      \text{If} \;\; & \emptytenv;\SENV;\CENV;\AENV;\NENV \vdash \AENV';\NENV';\EXPR : \hTYP \\
      \text{and} \;\; & \storewf{\SENV}{\CENV}{\AENV}{\NENV}{\MENV}{\STOR}\\
      \text{and} \;\; & \STOR;\MENV;\EXPR \stepsto \STOR';\MENV';\EXPR' \\
      \text{then for some} \;\; & \SENV' \supseteq \SENV, \CENV' \supseteq \CENV ,\\
      & \emptytenv;\SENV';\CENV';\AENV';\NENV' \vdash \AENV'';\NENV'';\EXPR' : \hTYP \\
      \text{and} \;\; & \storewf{\SENV'}{\CENV'}{\AENV'}{\NENV'}{\MENV'}{\STOR'}\\
    \end{aligned}
  \end{displaymath}
\end{lemma}

\begin{nproof}
  The proof is by rule induction on the given derivation of the dynamic semantics.

  \begin{bcase}
    \begin{mathpar}
    \rddatacon{}
    \end{mathpar}
    \begin{itemize}
    \item
    The first of two proof obligations is to show that
    the result $\EXPR' = \concreteloc{\reg}{\ind}{\locreg{\loc}{\reg}}$ of
    the given step of evaluation is well typed, that is,
    \begin{displaymath}
    \emptytenv;\SENV';\CENV';\AENV';\NENV' \vdash \AENV''; \NENV''; \concreteloc{\reg}{\ind}{\locreg{\loc}{\reg}} : \hTYP,
    \end{displaymath}
    where $\hTYP = \tyatlocreg{\TYP}{\loc}{\reg}$.
    %
    As implied by inversion on \tconcreteloc{}, the only obligation is to establish
    that $\SENV'(\locreg{\loc}{\reg})=\TYP$.
    %
    This obligation discharges by appropriately instantiating typing
    environments: $\SENV' = \SENV \cup \set{\locreg{\loc}{\reg} \mapsto \TYP}$, so that $\SENV' \supseteq \SENV$ and
    $\SENV'(\locreg{\loc}{\reg})=\TYP$, and $\CENV' = \CENV$, so that $\CENV' \supseteq \CENV$.
    %
    \item Given the instantiations of $\SENV'$ and $\CENV'$ used by the previous step, the second obligation
    for this proof case is to show that
    \begin{displaymath}
    \storewf{\SENV'}{\CENV}{\AENV'}{\NENV'}{\MENV}{\STOR'}.
    \end{displaymath}
    The individual requirements, labeled
    \refwellformed{sec:well-formedness}{wf:map-store-consistency} -
        \refwellformed{sec:well-formedness}{wf:ca},
        are handled by the following case analysis.
    \begin{itemize}
      \item
      Case (\refwellformed{sec:well-formedness}{wf:map-store-consistency}):
      for each $(\locreg{\loc'}{\reg} \mapsto \TYP) \in \SENV'$, there exists some $\ind_1, \ind_2$ such that
      \begin{align}
      (\locreg{\loc'}{\reg'} \mapsto \concreteloc{\reg'}{\ind_1}{}) \in \MENV \wedge \\
        \ewitness{\TYP}{\concreteloc{\reg'}{\ind_1}{}}{\STOR'}{\concreteloc{\reg'}{\ind_2}{}} \label{prf:dc-cj2}
      \end{align}
      %
      The first conjunct above discharges
      by inversion on \ddatacon{}, but
      to establish the second one, we need to distinguish
      between the case in which the given location $\locreg{\loc'}{\reg'}$ is the one
      affected by the constructor application, or not.
      \begin{itemize}
      \item Case $\locreg{\loc'}{\reg'} = \locreg{\loc}{\reg}$: \\
        For this case, the obligation is to show that the constructor being
        allocated by the constructor application, namely $\locreg{\loc}{\reg}$,
        has the end witness given above.
        %
        As such, for this case, it is the case that $\reg' = \reg$ and $\ind_1 = \ind$, which is
        a consequence of inversion on \ddatacon{}.
        %
        To establish the end witness, the first obligation therein, namely \refendwitness{sec:end-witness}{ewitness:impl1},
        is to establish $\STOR'(r)(i) = \DC$.
        %
        This obligation discharges by inspection of $\STOR'$, which is obtained by inversion on \ddatacon{}.
        %
        The second part is to establish the requirement
        \refendwitness{sec:end-witness}{ewitness:impl2} of the end-witness
        judgement, which pertains to the arguments passed to the constructor.
        The specific obligation is, if $n = | \overharpoon{\TYP'} | \geq 1$, then
           \begin{align}
           \ewitness{\overharpoon{\TYP'_1}}{\concreteloc{\reg}{\ind + 1}{}}{\STOR'}{\concreteloc{\reg}{\overharpoon{w_1}}{}} \wedge \label{prf:ew-req1} \\
           \ewitness{\overharpoon{\TYP'_{j+1}}}{\concreteloc{\reg}{\overharpoon{w_j}}{}}{\STOR'}{\concreteloc{\reg}{\overharpoon{w_{j+1}}}{}} \label{prf:ew-req2}
           \end{align}
           for some $\overharpoon{w}$, where
           $\indj \in J = J' - \set{n}$, $\indj' \in J' = \set{1,\ldots,n}$, and
           $\overharpoon{\TYP'} = \kargtys(\DC)$.
           %
           To establish the above, we need
           to reason backward from what
           the corresponding typing rules establish regarding the
           arguments passed to the constructor application.
           %
           First, we establish that, for each location corresponding to a constructor argument
           $\overharpoon{\locreg{\loc}{\reg}_{\indj'}}$,
           there is a corresponding mapping in the store-typing environment, i.e.,
           $(\overharpoon{\locreg{\loc}{\reg}_{\indj'}} \mapsto \overharpoon{\TYP_{\indj'}'}) \in \SENV$.
           %
           To establish these mappings, we first
           obtain by inversion on \tdatacon{} that the
           constructor arguments are well typed:
           \begin{align*}
           \emptytenv;\SENV;\CENV;\AENV;\NENV \vdash \AENV;\NENV;\overharpoon{\VAL_{\indj'}} : \overharpoon{\tyatlocreg{\TYP_{\indj'}'}{\loc_{\indj'}}{\reg}}
           \end{align*}
           %
           Each value $\overharpoon{\VAL_{\indj'}}$ is either a variable or a concrete location, and
           as such, by inversion on the typing rules \tvar{} and \tconcreteloc{}, respectively,
           we establish the required mappings in $\SENV$.
           %
           Thus, we can now combine the well-formedness
           of the store in the premise of this lemma,
           in particular requirement \refwellformed{sec:well-formedness}{wf:map-store-consistency},
           with the mappings of constructor arguments in $\SENV$ to establish
           the end witnesses in $\overharpoon{\ind}$ corresponding to the constructor arguments:
           %
           \begin{align}
           (\overharpoon{\locreg{\loc}{\reg}_{\indj'}} \mapsto \concreteloc{\reg}{\overharpoon{\ind_{\indj'}}}{}) \in \MENV \wedge \label{prf:dc-ew1} \\
           \ewitness{\overharpoon{\TYP_{\indj'}'}}{\concreteloc{\reg}{\overharpoon{\ind_{\indj'}}}{}}{\STOR}{\concreteloc{\reg}{\overharpoon{\ind_{\indj' + 1}}}{}} \label{prf:dc-ew2}
           \end{align}
           %
           We first address the obligation pertaining to the first constructor argument,
           and then the remaining ones.
           %
           From inversion on \tdatacon{}, we establish a mapping for the location of the
           first constructor argument.
           %
           \begin{align*}
           \CENV(\overharpoon{\locreg{\loc}{\reg}_1}) = \locreg{\loc}{\reg} + 1
           \end{align*}
           %
           Now, using this result, we can establish from well
           formedness rule \refwellformed{sec:well-formedness-constructors}{wfconstr:constraint-tag}
           that the following mappings exist in the location environment.
           %
           \begin{align*}
           (\locreg{\loc}{\reg} \mapsto \concreteloc{\reg}{\ind}{})  \in \MENV \wedge \\
                      (\overharpoon{\locreg{\loc}{\reg}_1} \mapsto \concreteloc{\reg}{\ind + 1}{})  \in \MENV
           \end{align*}
           %
           Next, combining the fact from line \ref{prf:dc-ew1} above regarding $\overharpoon{\locreg{\loc}{\reg}_1}$, the
           end witness corresponding to $\overharpoon{\ind_1}$ from the end witnesses of
           constructor arguments line~\ref{prf:dc-ew2} from above, we
           establish the requirement on line \ref{prf:ew-req1} above, such that $\overharpoon{w_1} = \overharpoon{i_1}$, i.e.,
           %
           \begin{align}
           \ewitness{\overharpoon{\TYP'_1}}{\concreteloc{\reg}{\ind + 1}{}}{\STOR}{\concreteloc{\reg}{\overharpoon{w_1}}{}} \label{prf:dc-ew-f1}.
           \end{align}
%           Notice that the end witness of the first field is by definition given above $\overharpoon{w_1}$.

           For the remaining constructor arguments, the structure of the proof is similar.
           %
           We establish mappings in $\CENV$ for the locations of these constructor arguments
           by inversion on \tdatacon{}.
           %
           \begin{align*}
           \CENV(\overharpoon{\locreg{\loc}{\reg}_{\indj+1}}) = \afterl{\overharpoon{\TYP'_{\indj}} \ensuremath{@} \overharpoon{\locreg{\loc}{\reg}_{\indj}}}
           \end{align*}
           %
           The following end witnesses $\overharpoon{\ind}$ are established by combining the property
           on the constraint environment
           with the property
           \refwellformed{sec:well-formedness-constructors}{wfconstr:constraint-after},
           which is obtained from the well formedness of the store in the premise of this lemma.
           \begin{align*}
           ((\overharpoon{\locreg{\loc}{\reg}_{\indj}} \mapsto \concreteloc{\reg}{\overharpoon{\ind_{\indj}}}{}) \in \MENV \wedge \\
                       \ewitness{\overharpoon{\TYP'_{\indj}}}{\concreteloc{\reg}{\overharpoon{\ind_{\indj}}}{}}{\STOR}{\concreteloc{\reg}{\overharpoon{\ind_{\indj+1}}}{}} \wedge \\
                       (\overharpoon{\locreg{\loc}{\reg}_{\indj+1}} \mapsto \concreteloc{\reg}{\overharpoon{\ind_{\indj+1}}}{}) \in \MENV)
           \end{align*}
           %
           To isolate the indices of any constructor arguments succeeding the
           first argument, we let $\indj'' \in J - \set{1}$, and thus deduce
           from the above that the end witnesses
           \begin{align*}
           \ewitness{\overharpoon{\TYP_{\indj''+1}'}}{\concreteloc{\reg}{\overharpoon{\ind_{\indj''+1}}}{}}{\STOR}{\concreteloc{\reg}{\overharpoon{\ind_{\indj'' + 2}}}{}}.
           \end{align*}
           %
           exist.
           %
           We obtain the needed result for the remaining end witnesses
           by instantiating for $\overharpoon{w}$, yielding
           \begin{align}
           \ewitness{\overharpoon{\TYP_{\indj''+1}'}}{\concreteloc{\reg}{\overharpoon{w_{\indj''}}}{}}{\STOR}{\concreteloc{\reg}{\overharpoon{w_{\indj'' + 1}}}{}} \label{prf:dc-ew-f2}.
           \end{align}
           %
           The original end witness required by~\ref{prf:dc-cj2} is now established by letting
           $\ind_1 = \ind$ and $\ind_2 = \overharpoon{w_{n+1}}$.

           Finally, to discharge this case, the end witnesses of the constructor
           arguments established in lines~\ref{prf:dc-ew-f1} and~\ref{prf:dc-ew-f2}
           need to hold for the new store
           $\STOR' = \STOR \cup \set{\reg \mapsto (\ind \mapsto \DC)}$.
           %
           To this end, in $\STOR'$, the newly
           written tag at address $\ind$ cannot overlap with the cells
           occupied by any of the constructor arguments.
           %
           Therefore, the desired end witnesses exist in $\STOR'$, thereby
           discharging this case.
      \item Case $\locreg{\loc'}{\reg'} \neq \loc$: \\
      This case requires we establish that, for such a given location $\locreg{\loc'}{\reg'}$, its
      corresponding end witness in the original store $\STOR$ also exists
      in the new store, $\STOR'$, that is, supposing $(\locreg{\loc'}{\reg'} \mapsto \concreteloc{\reg'}{\ind_1}{}) \in \MENV$, then $\ewitness{\TYP}{\concreteloc{\reg'}{\ind_1}{}}{\STOR}{\concreteloc{\reg'}{\ind_2}{}}$ implies $\ewitness{\TYP}{\concreteloc{\reg'}{\ind_1}{}}{\STOR'}{\concreteloc{\reg'}{\ind_2}{}}$.
      %
      But the only way that any such end witness can be invalidated
      is if the write of the constructor tag at index $\ind$ in
      $\STOR' = \STOR \cup \set{\reg \mapsto (\ind \mapsto \DC)}$ affects
      any address in the end witness corresponding
      to location $\locreg{\loc'}{\reg'}$, that is, any address
      in the right-open range $[\ind_1, \ind_2)$.
      %
      The proof obligation therefore amounts to ruling out
      aliasing, that is, $\ind$ falling in the range
      $[\ind_1, \ind_2)$.
      %
      To this end, we start by working backwards from the typing of
      the location $\locreg{\loc}{\reg}$, which corresponds to address $\ind$, the (only) address
      written by the constructor application.
      %
      By inversion on \tdatacon{}, we establish that $\locreg{\loc}{\reg} \in \NENV$.
      %
      As such, given the well formedness of the store $\STOR$ in the premise
      of this lemma, we obtain
      from \refwellformed{sec:well-formedness-allocation}{wf:impl-write-once}
      that
      $(\reg \mapsto (\ind \mapsto \DC)) \not \in \STOR$.
      %
      However, by the end-witness rule, for each $\indj \in [\ind_1, \ind_2)$,
      there exists a mapping from the address in the original store to its constructor tag
      $\DC_j$, which is
      $(\reg \mapsto (\indj \mapsto \DC_j)) \in \STOR$.
      %
      Therefore, the end witness judgement remains valid in store $\STOR'$,
      thus discharging this case.
      \end{itemize}
      \item Case (\refwellformed{sec:well-formedness}{wf:cfc}):
      \begin{align*}
      \storewfcfa{\CENV}{\MENV}{\STOR'}
      \end{align*}
      The first two proof obligations of this judgement, namely \refwellformed{sec:well-formedness-constructors}{wfconstr:constraint-start} and \refwellformed{sec:well-formedness-constructors}{wfconstr:constraint-tag}, discharge immediately, because the environments used by these
      rules are unaffected in a data-constructor application.
      %
      The only remaining obligation is \refwellformed{sec:well-formedness-constructors}{wfconstr:constraint-after}, because that requirement is affected by the write of the constructor tag, which is reflected in the new store $\STOR'$.
      %
      The obligation is to establish the preservation of the
      end witnesses of the locations in the domain of $\CENV$.
      %
      A similar proof obligation was already addressed by the proof of Property~\ref{prf:dc-cj2},
      in particular the subcase for $\locreg{\loc'}{\reg'} \neq \locreg{\loc}{\reg}$.
      %
      The only difference in that case is the locations range over the
      domain of the store-typing environment $\SENV$, whereas in this case
      the obligation concerns locations
      in the domain of the constraint environment $\CENV$.
      %
      However, the same proof steps apply in both cases, thus discharging
      this case.
      \item Case (\refwellformed{sec:well-formedness}{wf:ca}):
      \begin{align*}
      \storewfca{\AENV'}{\NENV'}{\MENV}{\STOR'}
      \end{align*}
      Obligations \refwellformed{sec:well-formedness-allocation}{wf:impl-linear-alloc} and
      \refwellformed{sec:well-formedness-allocation}{wf:impl-write-once} discharge immediately
      because $\locreg{\loc}{\reg} \not \in \NENV'$.
      %
      It remains to discharge the obligation corresponding to
      \refwellformed{sec:well-formedness-allocation}{wf:impl-linear-alloc2}.
      %
      Because it is the case that
      \begin{align*}
      (\reg \mapsto \locreg{\loc}{\reg}) \in \AENV' \wedge
      (\locreg{\loc}{\reg} \mapsto \concreteloc{\reg}{\ind_1}{}) \in \MENV \wedge \locreg{\loc}{\reg} \not \in \NENV' \wedge
    \, \ewitness{\TYP}{\concreteloc{\reg}{\ind_1}{}}{\STOR'}{\concreteloc{\reg}{\ind_2}{}},
    \end{align*}
    the obligation amounts to showing that the end witness of the
    constructor application is the new highest address in the store
    $\STOR'$, i.e., $\ind_2 > \allocptr{\reg}{\STOR'}$.
    %
    There are two cases, based on the number of constructor arguments $n$:
      \begin{itemize}
      \item Case $n = 0$:\\
      We need to appeal to the well formedness of the store, as given by the premise of this lemma,
      and in particular rule \refwellformed{sec:well-formedness-allocation}{wf:impl-linear-alloc}.
      %
      To use  this rule, we need to first establish
      $(\reg \mapsto \locreg{\loc}{\reg}) \in A$ and $\locreg{\loc}{\reg} \in \NENV$,
      which follows immediately by inversion on \tdatacon{}.
      %
      It therefore follows that
      \begin{align*}
%      \ind_1 = \max \set{0} \cup \set{\indj \; | \; \concreteloc{\reg}{\indj}{} \in \MENV} \wedge \\
      \ind_1 > \allocptr{\reg}{\STOR}.
      \end{align*}
      From this property, and by inspection on $\STOR'$, we discharge
      this case by establishing that the end witness of the constructor
      application is the highest address allocated in the new store $\STOR'$, i.e.,
      \begin{align*}
      \ind_1+1 = \ind_2 > \allocptr{\reg}{\STOR'}.
      \end{align*}
      \item Case $n \geq 1$:\\
      To discharge this case, we need to show that the end witness of the
      last constructor argument, i.e., the one at position $n$,
      is the highest address in the new store $\STOR'$.
      %
      This obligation follows from the well formedness of the
      store $\STOR$ given by the premise of this lemma, and
      in particular the application of rule
      \refwellformed{sec:well-formedness-allocation}{wf:impl-linear-alloc2}
      to the end witness of the last constructor argument, i.e.,
      \begin{align*}
      (\reg \mapsto \overharpoon{\locreg{\loc}{\reg}_n}) \in A \wedge
    \, (\overharpoon{\locreg{\loc}{\reg}_n} \mapsto \concreteloc{\reg}{\overharpoon{w_n}}{}) \in \MENV \wedge
    \, \ewitness{\TYP}{\concreteloc{\reg}{\overharpoon{w_n}}{}}{\STOR}{\concreteloc{\reg}{\overharpoon{w_{n+1}}}{}}
      \end{align*}
      The first two conjuncts follow from inversion on \tdatacon{}
      and \tconcreteloc{}, respectively, and the final one from Property~\ref{prf:dc-ew-f2}.
      %
      Thus, we have that $\overharpoon{w_{n+1}} > \allocptr{\reg}{\STOR}$.
      %
      It follows that $\overharpoon{w_{n+1}} > \allocptr{\reg}{\STOR'}$, because the
      newly written address in $\STOR'$, namely $i_1$, is such that $i_1 < \overharpoon{w_{n+1}}$.
      %
      By defintion of the end witness, we discharge this case by establishing that
      $\overharpoon{w_{n+1}} = i_2 > \allocptr{\reg}{\STOR'}$.
      \end{itemize}
      The final obligation of this case concerns the requirement
      \refwellformed{sec:well-formedness-allocation}{wf:impl-empty-region}.
      %
      Part of this obligation is given by the premise of this lemma, for the original
      store $\STOR$, and yields in particular that,
      for each $(\reg' \mapsto \emptyset) \in \AENV$, it
      is the case that $\reg' \not \in dom(\STOR)$.
      %
      The remaining obligation is to show the property holds for the new store
      $\STOR'$, which discharges immediately because, although $\reg \in \STOR'$,
      by inversion on \tdatacon{}, it must be that
      $(\reg \mapsto \emptyset) \not \in \AENV$.
      \item
      Case (\refwellformed{sec:well-formedness}{wf:impl1}):
      \begin{displaymath}
      dom(\SENV') \cap \NENV' = \emptyset
      \end{displaymath}
      From the premise of the lemma, we have that the store is well formed with respect to typing
      environments $\SENV$ and $\NENV$, and as such, we have that
      $dom(\SENV) \cap \NENV = \emptyset$.
      %
      Therefore, we discharge this case by inspection of typing rule \tdatacon{}, which
      shows that $N' = N - \set{\loc}$.
    \end{itemize}
    \end{itemize}
  \end{bcase}

  \begin{bcase}
    \begin{mathpar}
    \rdcase{}
    \end{mathpar}
    \begin{itemize}
    \item
    The first of two proof obligations is to show that
    the result $\EXPR' = \subst{\EXPR}{\overharpoon{\var}}{\concreteloc{\reg}{\overharpoon{w}}{\overharpoon{\locreg{\loc}{\reg}}}}$ of
    the given step of evaluation is well typed, that is,
    \begin{displaymath}
    \emptytenv;\SENV';\CENV;\AENV;\NENV \vdash \AENV; \NENV; \EXPR' : \hTYP,
    \end{displaymath}
    %
    where $\hTYP = \tyatlocreg{\TYP}{\loc}{\reg}$.
    %
    To establish the above, we start by obtaining the type
    for the body of the pattern, then the types of the
    concrete locations being substituted into the body,
    and finally use these two results
    with the substitution lemma to discharge the case.
    %
    First, by inversion on the typing rules \tcase{} and \tpat{}, we
    establish that the body of the pattern, namely $\EXPR$, is well typed, i.e.,
    \begin{align*}
    \TENV';\SENV';\CENV;\AENV;\NENV \vdash \AENV;\NENV;\EXPR : \tyatlocreg{\TYP}{\loc}{\reg},
    \end{align*}
    where
    \begin{align*}
    \TENV' &= \set{\overharpoon{\var_1} \mapsto \overharpoon{\TYP_1} \ensuremath{@} \overharpoon{\locreg{\loc_1}{\reg}}, \ldots, \overharpoon{\var_1} \mapsto \overharpoon{\TYP_n} \ensuremath{@} \overharpoon{\locreg{\loc_n}{\reg}}} \\
    \SENV' &= \SENV \cup \set{\overharpoon{\locreg{\loc_1}{\reg}}\mapsto\overharpoon{\TYP}_1,\ldots,\overharpoon{\locreg{\loc_n}{\reg}}\mapsto\overharpoon{\TYP}_n}.
    \end{align*}
    %
    Second, we establish that the concrete locations being substituted for the
    pattern variables $\overharpoon{x}$ are well typed.
    %
    The specific obligation is, for each $i \in \set{1, \ldots, n}$, to establish that
    \begin{align*}
    \emptyset;\SENV';\CENV;\AENV;\NENV \vdash \AENV;\NENV; \concreteloc{\reg}{\overharpoon{w_i}}{\overharpoon{\locreg{\loc_i}{\reg}}} : \overharpoon{\TYP}_i \ensuremath{@} \overharpoon{\locreg{\loc_i}{\reg}}.
    \end{align*}
    %
    The above holds because, by inversion on \tconcreteloc{}, the obligation is
    to show that, for each such $i$, $(\locreg{\overharpoon{\loc_i}}{\reg} \mapsto \overharpoon{\TYP_i}) \in \SENV'$,
    which is immediate by inspection on $\SENV'$ above.
    %
    Third, and finally, to establish the typing judgement for $\EXPR'$, we use the Substitution
    Lemma \ref{lemma:substitution}, which yields
    %
    \begin{align*}
    \emptyset;\SENV';\CENV;\AENV;\NENV \vdash \AENV;\NENV; \subst{\EXPR}{\overharpoon{\var_1}}{\concreteloc{\reg}{\overharpoon{w_1}}{\overharpoon{\locreg{\loc_1}{\reg}}}}
    \ldots \subst{}{\overharpoon{\var_n}}{\concreteloc{\reg}{\overharpoon{w_1}}{\overharpoon{\locreg{\loc_n}{\reg}}}}: \hTYP,
    \end{align*}
    as needed, thereby discharging this obligation.
    \item The second obligation
    for this proof case is, given the affected environments, namely
    $\SENV'$ and $\MENV'$, to establish the well formedness
    of the resulting store, i.e.,
    %
    \begin{displaymath}
    \storewf{\SENV'}{\CENV}{\AENV}{\NENV}{\MENV'}{\STOR}.
    \end{displaymath}
    We omit most of the details of this proof obligation because they
    discharge straightforwardly.
    %
    The only part that requires attention is rule
    \refwellformed{sec:well-formedness}{wf:map-store-consistency},
    which is affected by the fresh locations in the location
    environment $\MENV'$.
    %
    This requirement discharges by inspection of \dcase{}, thereby
    discharging this obligation.
    \end{itemize}
  \end{bcase}

  \begin{bcase}
    \begin{mathpar}
    \rdletloctag{}
    \end{mathpar}
    \begin{itemize}
    \item
    The first of two proof obligations is to show that
    the result $\EXPR$ of
    the given step of evaluation is well typed, that is,
    \begin{displaymath}
    \emptytenv;\SENV;\CENV';\AENV';\NENV' \vdash \AENV''; \NENV''; \EXPR : \hTYP,
    \end{displaymath}
    where $\hTYP = \tyatlocreg{\TYP}{\loc}{\reg}$, $\AENV' = \AENV \cup \set{\reg \mapsto \locreg{\loc}{\reg}}$,
    and $\NENV' = \NENV \cup \set{\locreg{\loc}{\reg}}$.
    %
    This proof obligation follows straightforwardly by inversion
    on \tlltag{}.
    \item The second obligation for this proof case is to show that
    \begin{displaymath}
    \storewf{\SENV}{\CENV'}{\AENV'}{\NENV'}{\MENV'}{\STOR}.
    \end{displaymath}
    The individual requirements, labeled
    \refwellformed{sec:well-formedness}{wf:map-store-consistency} -
        \refwellformed{sec:well-formedness}{wf:ca},
        are handled by the following case analysis.
    \begin{itemize}
      \item
      Case (\refwellformed{sec:well-formedness}{wf:map-store-consistency}):
      for each $(\locreg{\loc'}{\reg} \mapsto \TYP) \in \SENV$, there exists some $\ind_1, \ind_2$ such that
      \begin{align*}
      (\locreg{\loc'}{\reg} \mapsto \concreteloc{\reg}{\ind_1}{}) \in \MENV' \wedge \\
        \ewitness{\TYP}{\concreteloc{\reg}{\ind_1}{}}{\STOR}{\concreteloc{\reg}{\ind_2}{}}
      \end{align*}
      %
      By the well formedness of the store given in the premise of this lemma,
      the above already holds for the location environment $\MENV$.
      %
      The obligation discharges by inspecting the only new location
      in $\MENV'$, namely $\locreg{\loc}{\reg}$, which
      is fresh and therefore cannot be in the domain of $\SENV$.
      \item Case (\refwellformed{sec:well-formedness}{wf:cfc}):
      \begin{align*}
      \storewfcfa{\CENV'}{\MENV'}{\STOR}
      \end{align*}
      Of the requirements for this judgement, the only one that is
      not satisfied immediately by the well formedness of the store
      given in the premise of the lemma is requirement
      \refwellformed{sec:well-formedness-constructors}{wfconstr:constraint-tag}
      %
      The specific requirement is to establish that
      \begin{align*}
      (\locreg{\loc'}{\reg} \mapsto \concreteloc{\reg}{\ind}{})  \in \MENV' \wedge \\
      (\locreg{\loc}{\reg} \mapsto \concreteloc{\reg}{\ind + 1}{})  \in \MENV',
      \end{align*}
      which follows immediately by inversion on \dletloctag{}.
      \item Case (\refwellformed{sec:well-formedness}{wf:ca}):
      \begin{align*}
      \storewfca{\AENV'}{\NENV'}{\MENV'}{\STOR}
      \end{align*}
        \begin{itemize}
        \item Case (\refwellformed{sec:well-formedness-allocation}{wf:impl-linear-alloc}):
        \begin{align*}
                  (\locreg{\loc}{\reg} \mapsto \concreteloc{\reg}{\ind+1}{}) \in \MENV' \wedge
%          \ind+1 = \max \set{0} \cup \set{\indj \; | \; \concreteloc{\reg}{\indj}{} \in \MENV'} \wedge \\
          \ind+1 > \allocptr{\reg}{\STOR}
        \end{align*}
        The first conjunct follows immediately from inversion on \dletloctag{}.
        %
        To establish the second, however, we first need to establish
        that the address corresponding to location $\locreg{\loc'}{\reg}$ is the highest index in
        the store $\STOR$.
        %
        To do so, we need to appeal to the well formedness of the store given by the
        premise of this lemma.
        %
        In particular, we need to use the same requirement we are trying to prove,
        namely \refwellformed{sec:well-formedness-allocation}{wf:impl-linear-alloc}, but in this case,
        instantiating for $\locreg{\loc'}{\reg}$ in the original location environment $\MENV$.
        %
        By inversion on \tlltag{}, we have that $\AENV(\reg) = \locreg{\loc'}{\reg}$ and $\locreg{\loc'}{\reg} \in \NENV$,
        and as a consequence of \refwellformed{sec:well-formedness-allocation}{wf:impl-linear-alloc},
        \begin{align*}
                  (\locreg{\loc'}{\reg} \mapsto \concreteloc{\reg}{\ind}{}) \in \MENV \wedge
%          \ind = \max \set{0} \cup \set{\indj \; | \; \concreteloc{\reg}{\indj}{} \in \MENV} \wedge \\
          \ind > \allocptr{\reg}{\STOR}.
        \end{align*}
        Using the second conjunct above, this case discharges immediately.
        \item Case (\refwellformed{sec:well-formedness-allocation}{wf:impl-linear-alloc2}):
        This obligation discharges immediately because, by inversion on \tlltag{}, $\locreg{\loc}{\reg} \in \NENV'$.
        \item Case (\refwellformed{sec:well-formedness-allocation}{wf:impl-write-once}):
        The proof obligation is to establish that, for any constructor tag $\DC$,
        \begin{align*}
         ((\locreg{\loc}{\reg} \mapsto \concreteloc{\reg}{\ind+1}{}) \in \MENV' \wedge \\
          (\reg \mapsto (\ind+1 \mapsto \DC)) \not \in \STOR)
        \end{align*}
        The first conjunct discharges by inversion on \dletloctag{},
        and the second as a consequence of having already
        established just above that $\ind+1 > \allocptr{\reg}{\STOR}$.
        \item Case (\refwellformed{sec:well-formedness-allocation}{wf:impl-empty-region}):
        The proof obligation is to establish that, for each
        $(\reg \mapsto \emptyset) \in \AENV'$, it is the case that
        $\reg \not \in dom(\STOR)$.
        %
        This case discharges because, from the premise of the
        lemma, this property holds for the original environment
        $\AENV$ and store $\STOR$, and, by inversion on \tlltag{},
        continues to hold for $\AENV'$ and $\STOR'$.
      \end{itemize}
      \item
      Case (\refwellformed{sec:well-formedness}{wf:impl1}):
      \begin{displaymath}
      dom(\SENV) \cap \NENV' = \emptyset
      \end{displaymath}
      Because it is a bound location, $\loc \not \in dom(\SENV)$, and by inversion on \tlltag{},
      $\loc \in \NENV'$, which discharges the obligation.
      \end{itemize}
    \end{itemize}
  \end{bcase}

  \begin{bcase}
    \begin{mathpar}
    \rdletlocafter{}
    \end{mathpar}
    \begin{itemize}
    \item
    The first of two proof obligations is to show that
    the result $\EXPR'$ of
    the given step of evaluation is well typed, that is,
    \begin{displaymath}
    \emptytenv;\SENV;\CENV';\AENV';\NENV' \vdash \AENV''; \NENV''; \EXPR' : \hTYP,
    \end{displaymath}
    where $\hTYP = \tyatlocreg{\TYP}{\loc'}{\reg'}$.
    %
    This proof obligation follows straightforwardly by inversion
    on \tllafter{}.
    \item The second obligation for this proof case is to show that
    \begin{displaymath}
    \storewf{\SENV}{\CENV'}{\AENV'}{\NENV'}{\MENV'}{\STOR}.
    \end{displaymath}
    The individual requirements, labeled
    \refwellformed{sec:well-formedness}{wf:map-store-consistency} -
        \refwellformed{sec:well-formedness}{wf:ca},
        are handled by the following case analysis.
    \begin{itemize}
      \item
      Case (\refwellformed{sec:well-formedness}{wf:map-store-consistency}):
      for each $(\locreg{\loc'}{\reg} \mapsto \TYP) \in \SENV$, there exists some $\ind_1, \ind_2$ such that
      \begin{align*}
      (\locreg{\loc'}{\reg} \mapsto \concreteloc{\reg}{\ind_1}{}) \in \MENV' \wedge \\
        \ewitness{\TYP}{\concreteloc{\reg}{\ind_1}{}}{\STOR}{\concreteloc{\reg}{\ind_2}{}}
      \end{align*}
      %
      By the well formedness of the store given in the premise of this lemma,
      the above already holds for the location environment $\MENV$.
      %
      The obligation discharges by inspecting the only new location
      in $\MENV'$, namely $\locreg{\loc}{\reg}$, which
      is fresh and therefore cannot be in the domain of $\SENV$.
      \item Case (\refwellformed{sec:well-formedness}{wf:cfc}):
      \begin{align*}
      \storewfcfa{\CENV'}{\MENV'}{\STOR}
      \end{align*}
      Of the requirements for this judgement, the only one that is
      not satisfied immediately by the well formedness of the store
      given in the premise of the lemma is requirement
      \refwellformed{sec:well-formedness-constructors}{wfconstr:constraint-after}
      %
      The specific requirement is to establish that
      \begin{align*}
       (\locreg{\loc_1}{\reg} \mapsto \concreteloc{\reg}{\ind}{}) \in \MENV' \wedge \\
       \ewitness{\TYP}{\concreteloc{\reg}{\ind}{}}{\STOR}{\concreteloc{\reg}{\indj}{}} \wedge \\
       (\loc \mapsto \concreteloc{\reg}{\indj}{}) \in \MENV'
      \end{align*}
      which follows immediately by inversion on \dletlocafter{}.
      \item Case (\refwellformed{sec:well-formedness}{wf:ca}):
      \begin{align*}
      \storewfca{\AENV'}{\NENV'}{\MENV'}{\STOR}
      \end{align*}
        \begin{itemize}
        \item Case (\refwellformed{sec:well-formedness-allocation}{wf:impl-linear-alloc}):
        \begin{align*}
                  (\loc \mapsto \concreteloc{\reg}{\indj}{}) \in \MENV' \wedge
%          \indj = \max \set{0} \cup \set{\indj' \; | \; \concreteloc{\reg}{\indj'}{} \in \MENV'} \wedge \\
          \indj > \allocptr{\reg}{\STOR}
        \end{align*}
        The first conjunct follows immediately from inversion on \dletlocafter{}.
        %
        To establish the second, however, we first need to establish
        that the end witness $\indj$ of location $\locreg{\loc_1}{\reg}$ is the maximum index in the
        store $\STOR$.
        %
        To do so, we need to appeal to the well formedness of the store given by the
        premise of this lemma.
        %
        In particular, we need to use the requirement \refwellformed{sec:well-formedness-allocation}{wf:impl-linear-alloc2},
        instantiating for $\locreg{\loc_1}{\reg}$ in the original location environment $\MENV$.
        %
        By inversion on \tllafter{}, we have that $\AENV(\reg) = \locreg{\loc_1}{\reg}$, $\locreg{\loc_1}{\reg} \not \in \NENV$,
        and $\ewitness{\TYP}{\concreteloc{\reg}{\ind}{}}{\STOR}{\concreteloc{\reg}{\indj}{}}$.
        %
        Thus, as a consequence of \refwellformed{sec:well-formedness-allocation}{wf:impl-linear-alloc2},
        \begin{align*}
%                  (\loc' \mapsto \concreteloc{\reg}{\ind}{}) \in \MENV \wedge
%          \ind = \max \set{0} \cup \set{\indj \; | \; \concreteloc{\reg}{\indj}{} \in \MENV} \wedge \\
          \indj > \allocptr{\reg}{\STOR}.
        \end{align*}
        Using the second and third conjuncts above, this case discharges immediately.
        \item Case (\refwellformed{sec:well-formedness-allocation}{wf:impl-linear-alloc2}):
        This obligation discharges immediately because, by inversion on \tllafter{}, $\loc \in \NENV'$.
        \item Case (\refwellformed{sec:well-formedness-allocation}{wf:impl-write-once}):
        The proof obligation is to establish that, for any constructor tag $\DC$,
        \begin{align*}
         ((\loc \mapsto \concreteloc{\reg}{\indj}{}) \in \MENV' \wedge \\
          (\reg \mapsto (\indj \mapsto \DC)) \not \in \STOR)
        \end{align*}
        The first conjunct discharges by inversion on \dletlocafter{},
        and the second as a consequence of having already
        established just above that $\indj > \allocptr{\reg}{\STOR}$.
        \item Case (\refwellformed{sec:well-formedness-allocation}{wf:impl-empty-region}):
        This case discharges straightforwardly, in a similar fashion
        to the previous case, for \dletloctag{}.
        \end{itemize}
      \item
      Case (\refwellformed{sec:well-formedness}{wf:impl1}):
      \begin{displaymath}
      dom(\SENV) \cap \NENV' = \emptyset
      \end{displaymath}
      Because it is a bound location, $\loc \not \in dom(\SENV)$, and by inversion on \tllafter{}
      $\loc \in \NENV'$, which discharges this obligation.
      \end{itemize}
    \end{itemize}
  \end{bcase}

  \begin{bcase}
    \begin{mathpar}
    \rdletlocstart{}
    \end{mathpar}
    \begin{itemize}
    \item
    The first of two proof obligations is to show that
    the result $\EXPR'$ of
    the given step of evaluation is well typed, that is,
    \begin{displaymath}
    \emptytenv;\SENV;\CENV';\AENV';\NENV' \vdash \AENV''; \NENV''; \EXPR' : \hTYP,
    \end{displaymath}
    where $\hTYP = \tyatlocreg{\TYP}{\loc'}{\reg'}$.
    %
    This obligation follows straightforwardly by inversion
    on \tllstart{}.
    \item The second obligation for this proof case is to show that
    \begin{displaymath}
    \storewf{\SENV}{\CENV'}{\AENV'}{\NENV'}{\MENV'}{\STOR}.
    \end{displaymath}
    The individual requirements, labeled
    \refwellformed{sec:well-formedness}{wf:map-store-consistency} -
        \refwellformed{sec:well-formedness}{wf:ca},
        are handled by the following case analysis.
    \begin{itemize}
      \item
      Case (\refwellformed{sec:well-formedness}{wf:map-store-consistency}):
      for each $(\loc' \mapsto \TYP) \in \SENV$, there exists some $\ind_1, \ind_2$ such that
      \begin{align*}
      (\loc' \mapsto \concreteloc{\reg}{\ind_1}{}) \in \MENV' \wedge \\
        \ewitness{\TYP}{\concreteloc{\reg}{\ind_1}{}}{\STOR}{\concreteloc{\reg}{\ind_2}{}}
      \end{align*}
      %
      By the well formedness of the store given in the premise of this lemma,
      the above already holds for the location environment $\MENV$.
      %
      The obligation discharges by inspecting the only new location
      in $\MENV'$, namely $\locreg{\loc}{\reg}$, which
      is fresh and therefore cannot be in the domain of $\SENV$.
      \item Case (\refwellformed{sec:well-formedness}{wf:cfc}):
      \begin{align*}
      \storewfcfa{\CENV'}{\MENV'}{\STOR}
      \end{align*}
      Of the requirements for this judgement, the only one that is
      not satisfied immediately by the well formedness of the store
      given in the premise of the lemma is requirement
      \refwellformed{sec:well-formedness-constructors}{wfconstr:constraint-start}.
      %
      The specific requirement is to establish that
      \begin{align*}
      (\locreg{\loc}{\reg} \mapsto \concreteloc{\reg}{0}{}) \in \MENV',
      \end{align*}
      which follows immediately by inversion on \dletlocstart{}.
      \item Case (\refwellformed{sec:well-formedness}{wf:ca}):
      \begin{align*}
      \storewfca{\AENV'}{\NENV'}{\MENV'}{\STOR}
      \end{align*}
        \begin{itemize}
        \item Case (\refwellformed{sec:well-formedness-allocation}{wf:impl-linear-alloc}):
        \begin{align*}
                  (\loc \mapsto \concreteloc{\reg}{0}{}) \in \MENV' \wedge
          0 > \allocptr{\reg}{\STOR}
        \end{align*}
        The first conjunct follows immediately from inversion on \dletlocstart{}.
        %
        To establish the second conjunct above, it suffices establish
        that $\reg \not \in dom(\STOR)$,
        because, as such, $\allocptr{\reg}{\STOR} = -1$, by the definition of $MaxIdx$.
        %
        This property follows from the well formedness of the
        store, in particular, from rule
        \refwellformed{sec:well-formedness-allocation}{wf:impl-empty-region}.
        %
        The rule guarantees that, if $(\reg \mapsto \emptyset) \in \AENV$, then
        $\reg \not \in dom(\STOR)$, as needed.
        %
        By inversion on \tllstart{}, we establish this precondition, thereby
        discharging the case.
        \item Case (\refwellformed{sec:well-formedness-allocation}{wf:impl-linear-alloc2}):
        This obligation discharges immediately because, by inversion on \tllstart{}, $\loc \in \NENV'$.
        \item Case (\refwellformed{sec:well-formedness-allocation}{wf:impl-write-once}):
        The proof obligation is to establish that, for any constructor tag $\DC$,
        \begin{align*}
         ((\locreg{\loc}{\reg} \mapsto \concreteloc{\reg}{0}{}) \in \MENV' \wedge \\
          (\reg \mapsto (0 \mapsto \DC)) \not \in \STOR)
        \end{align*}
        The first conjunct discharges by inversion on \dletlocstart{},
        and the second as a consequence of having already
        established just above that $0 > \allocptr{\reg}{\STOR}$.
        \item Case (\refwellformed{sec:well-formedness-allocation}{wf:impl-empty-region}):
        The obligation for this case is to establish that
        for each $(\reg \mapsto \emptyset) \in \AENV' = \AENV \cup \set{\reg \mapsto \locreg{\loc}{\reg}}$, it
        is the case that $\reg \not \in dom(\STOR)$.
        %
        The part of this obligation pertaining to environment $\AENV$
        is given by the premise of this lemma, and thus it only remains
        to establish that the property holds for the rest, namely
        $\set{\reg \mapsto \locreg{\loc}{\reg}}$.
        %
        This part discharges trivially, because $(\reg \mapsto \emptyset) \not \in \AENV'$,
        thereby discharging this case.
        \end{itemize}
      \item
      Case (\refwellformed{sec:well-formedness}{wf:impl1}):
      \begin{displaymath}
      dom(\SENV) \cap \NENV' = \emptyset
      \end{displaymath}
      This case discharges straightforwardly.
      \end{itemize}
    \end{itemize}
  \end{bcase}

  \begin{bcase}
    \begin{mathpar}
    \rdletregion{}
    \end{mathpar}
    \begin{itemize}
    \item
    The first of two proof obligations is to show that
    the result $\EXPR'$ of
    the given step of evaluation is well typed, that is,
    \begin{displaymath}
    \emptytenv;\SENV;\CENV';\AENV';\NENV' \vdash \AENV''; \NENV''; \EXPR' : \hTYP,
    \end{displaymath}
    where $\hTYP = \tyatlocreg{\TYP}{\loc'}{\reg'}$.
    %
    This proof obligation follows straightforwardly by inversion
    on \tlregion{}.
    \item The second obligation for this proof case is to show that
    \begin{displaymath}
    \storewf{\SENV}{\CENV}{\AENV'}{\NENV}{\MENV}{\STOR}.
    \end{displaymath}
    The individual requirements, labeled
    \refwellformed{sec:well-formedness}{wf:map-store-consistency} -
        \refwellformed{sec:well-formedness}{wf:ca},
        are handled by the following case analysis.
    \begin{itemize}
      \item
      Case (\refwellformed{sec:well-formedness}{wf:map-store-consistency}):
      for each $(\locreg{\loc'}{\reg} \mapsto \TYP) \in \SENV$, there exists some $\ind_1, \ind_2$ such that
      \begin{align*}
      (\locreg{\loc'}{\reg} \mapsto \concreteloc{\reg}{\ind_1}{}) \in \MENV \wedge \\
        \ewitness{\TYP}{\concreteloc{\reg}{\ind_1}{}}{\STOR}{\concreteloc{\reg}{\ind_2}{}}
      \end{align*}
      %
      This case discharges immediately by inversion of \tlregion{} and \dletregion{},
      because none of the relevant environments are affected by the transition.
      \item Case (\refwellformed{sec:well-formedness}{wf:cfc}):
      \begin{align*}
      \storewfcfa{\CENV}{\MENV}{\STOR}
      \end{align*}
      The case discharges in a fashion similar to the previous one.
      \item Case (\refwellformed{sec:well-formedness}{wf:ca}):
      \begin{align*}
      \storewfca{\AENV'}{\NENV}{\MENV}{\STOR}
      \end{align*}
      Of the requirements in this judgement, the only one that is affected by the
      new environment $\AENV'$ is requirement
      \refwellformed{sec:well-formedness-allocation}{wf:impl-empty-region}.
      %
      The specific obligation is to establish that,
      for each $(\reg \mapsto \emptyset) \in \AENV'$, it
      is the case that $\reg \not \in dom(\STOR)$.
      %
      By inversion on \tlregion{},
      $\AENV' = \AENV \cup \set{\reg \mapsto \emptyset}$, and therefore,
      the first part of the obligation, that is, for $\AENV$, is already
      given by the premise of this lemma.
      %
      As such, it only remains to establish that $\reg \not \in dom(\STOR)$,
      which follows from $\reg$ being a fresh region, thereby ruling out
      it being in the store, and thus discharging this case.
      \item
      Case (\refwellformed{sec:well-formedness}{wf:impl1}):
      \begin{displaymath}
      dom(\SENV) \cap \NENV' = \emptyset
      \end{displaymath}
      This case discharges straightforwardly.
      \end{itemize}
    \end{itemize}
  \end{bcase}

  \begin{bcase}
    \begin{mathpar}
    \rdletval{}
    \end{mathpar}
    \begin{itemize}
    \item
    The first of two proof obligations is to show that
    the result $\subst{\EXPR_2}{\var}{\VAL_1}$ of
    the given step of evaluation is well typed, that is,
    \begin{displaymath}
    \emptytenv;\SENV';\CENV;\AENV;\NENV \vdash \AENV; \NENV; \subst{\EXPR_2}{\var}{\VAL_1} : \tyatlocreg{\TYP_2}{\loc_2}{\reg_2}.
    \end{displaymath}
    By inversion on \tlet{}, we obtain the type of the value being bound
    \begin{align*}
    \emptyset;\SENV;\CENV;\AENV;\NENV \vdash \AENV;\NENV;\VAL_1 : \tyatlocreg{\TYP_1}{\loc_1}{\reg_1},
    \end{align*}
    and we obtain the type of the body
    \begin{align*}
    \TENV';\SENV';\CENV;\AENV;\NENV \vdash \AENV;\NENV;\EXPR_2 : \tyatlocreg{\TYP_2}{\loc_2}{\reg_2}
    \end{align*}
    where
    \begin{align*}
    \TENV' &= \set{\var \mapsto \tyatlocreg{\TYP_1}{\loc_1}{\reg_1}} \\
    \SENV' &= \SENV \cup \set{\locreg{\loc_1}{\reg_1} \mapsto \TYP_1}.
    \end{align*}
    As such we can apply the Substitution
    Lemma \ref{lemma:substitution}, as follows
    \begin{displaymath}
    \emptytenv;\SENV';\CENV;\AENV;\NENV \vdash \AENV; \NENV; \subst{\EXPR_2}{\var}{\VAL_1} \subst{}{\locreg{\loc_1}{\reg_1}}{\locreg{\loc_1}{\reg_1}} : \tyatlocreg{\TYP_2}{\loc_2}{\reg_2},
    \end{displaymath}
    which discharges our obligation, given that the substitution of the
    bound location $\locreg{\loc_1}{\reg_1}$ is the identity substitution.
    \item Given the instantiations of $\SENV'$ and $\MENV'$
    used by the previous step, the second obligation
    for this proof case is to show that
    \begin{displaymath}
    \storewf{\SENV'}{\CENV}{\AENV}{\NENV}{\MENV'}{\STOR}.
    \end{displaymath}
    The individual requirements, labeled
    \refwellformed{sec:well-formedness}{wf:map-store-consistency} -
        \refwellformed{sec:well-formedness}{wf:ca},
        are handled by the following case analysis.
    \begin{itemize}
      \item
      Case (\refwellformed{sec:well-formedness}{wf:map-store-consistency}):
      for each $(\locreg{\loc'}{\reg} \mapsto \TYP) \in \SENV' = \SENV \cup \set{\locreg{\loc_1}{\reg_1} \mapsto \TYP_1}$, there exists some $\ind_1, \ind_2$ such that
      \begin{align*}
      (\locreg{\loc'}{\reg} \mapsto \concreteloc{\reg}{\ind_1}{}) \in \MENV \wedge \\
        \ewitness{\TYP}{\concreteloc{\reg}{\ind_1}{}}{\STOR}{\concreteloc{\reg}{\ind_2}{}}
      \end{align*}
      %
      This obligation amounts to showing the above holds for the bound location
      $\locreg{\loc_1}{\reg_1}$, because the well formedness of the
      store given by the premise of this lemma guarantees the property
      holds for locations bound in $\SENV$.
      %
      The value $\VAL_1$ bound at location $\locreg{\loc_1}{\reg_1}$ is
      a value and is well typed, and as such, there are only two typing
      rules that could apply, namely \tvar{} and \tconcreteloc{}.
      %
      By inversion on these rules, we establish that
      \begin{align*}
      (\locreg{\loc_1}{\reg_1} \mapsto \TYP_1) \in \SENV.
      \end{align*}
      %
      Therefore, we can discharge this obligation by application of well formedness of the
      store, in particular, the rule
      \refwellformed{sec:well-formedness}{wf:map-store-consistency} we
      are currently considering.
      %
      Concretely, we discharge this obligation by instantiating that rule to
      %
      \begin{align*}
      (\locreg{\loc_1}{\reg_1} \mapsto \concreteloc{\reg_1}{\ind_1}{}) \in \MENV \wedge \\
        \ewitness{\TYP_1}{\concreteloc{\reg_1}{\ind_1}{}}{\STOR}{\concreteloc{\reg_1}{\ind_2}{}}.
      \end{align*}
      \item Case (\refwellformed{sec:well-formedness}{wf:cfc}):
      \begin{align*}
      \storewfcfa{\CENV}{\MENV}{\STOR}
      \end{align*}
      This case discharges immediately because the relevant environments
      are affected by neither the of the relevant typing nor the dynamic-semantic judgement.
      \item Case (\refwellformed{sec:well-formedness}{wf:ca}):
      \begin{align*}
      \storewfca{\AENV}{\NENV}{\MENV}{\STOR}
      \end{align*}
      This case discharges immediately because the relevant environments
      are affected by neither the of the relevant typing nor the dynamic-semantic judgement.
      \item
      Case (\refwellformed{sec:well-formedness}{wf:impl1}):
      \begin{displaymath}
      dom(\SENV') \cap \NENV = \emptyset
      \end{displaymath}
      This case discharges straightforwardly.
      \end{itemize}
    \end{itemize}
  \end{bcase}

  \begin{bcase}
    \begin{mathpar}
    \rdletexp{}
    \end{mathpar}
    \begin{itemize}
    \item
    The first of two proof obligations is to show that
    the result $\letpack{\var : \hTYP}{\EXPR'_1}{\EXPR_2}$ of
    the given step of evaluation is well typed, that is,
    \begin{displaymath}
    \emptytenv;\SENV;\CENV;\AENV';\NENV' \vdash \AENV''; \NENV''; \letpack{\var : \hTYP}{\EXPR'_1}{\EXPR_2} : \tyatlocreg{\TYP_2}{\loc_2}{\reg_2},
    \end{displaymath}
    %
    The induction hypothesis is
    \begin{displaymath}
      \begin{aligned}
        \text{If} \;\; & \emptytenv;\SENV;\CENV;\AENV;\NENV \vdash \AENV';\NENV';\EXPR_1 : \tyatlocreg{\TYP_1}{\loc_1}{\reg_1} \\
        \text{and} \;\; & \storewf{\SENV}{\CENV}{\AENV}{\NENV}{\MENV}{\STOR}\\
        \text{and} \;\; & \STOR;\MENV;\EXPR_1 \stepsto \STOR';\MENV';\EXPR_1' \\
        \text{then for some} \;\; & \SENV' \supseteq \SENV, \CENV' \supseteq \CENV ,\\
        & \emptytenv;\SENV';\CENV';\AENV';\NENV' \vdash \AENV'';\NENV'';\EXPR_1' : \tyatlocreg{\TYP_1}{\loc_1}{\reg_1} \\
        \text{and} \;\; & \storewf{\SENV'}{\CENV'}{\AENV'}{\NENV'}{\MENV'}{\STOR'}.
      \end{aligned}
    \end{displaymath}
    %
    By inversion on \tlet{}, we establish that
    \begin{align*}
    \emptytenv;\SENV;\CENV;\AENV;\NENV \vdash \AENV';\NENV';\EXPR_1 : \tyatlocreg{\TYP_1}{\loc_1}{\reg_1},
    \end{align*}
    and, by the premise of this lemma, we establish that
    \begin{align*}
    \storewf{\SENV}{\CENV}{\AENV}{\NENV}{\MENV}{\STOR}
    \end{align*}
    and by inversion on \dletexp{} we establish that
    \begin{align*}
    \STOR;\MENV;\EXPR_1 \stepsto \STOR';\MENV';\EXPR_1'.
    \end{align*}
    %
    Now, we can apply the above to the induction hypothesis to establish
        \begin{displaymath}
      \begin{aligned}
        \text{For some} \;\; & \SENV' \supseteq \SENV, \CENV' \supseteq \CENV ,\\
        & \emptytenv;\SENV';\CENV';\AENV';\NENV' \vdash \AENV'';\NENV'';\EXPR_1' : \tyatlocreg{\TYP_1}{\loc_1}{\reg_1} \\
        \text{and} \;\; & \storewf{\SENV'}{\CENV'}{\AENV'}{\NENV'}{\MENV'}{\STOR'}.
      \end{aligned}
    \end{displaymath}
    %
    By inversion on \tlet{}, we also have that
    \begin{align*}
    \TENV';\SENV';\CENV;\AENV';\NENV' \vdash \AENV'';\NENV'';\EXPR_2 : \tyatlocreg{\TYP_2}{\loc_2}{\reg_2},
    \end{align*}
    where
    \begin{align*}
    \TENV' &= \set{\var \mapsto \tyatlocreg{\TYP_1}{\loc_1}{\reg_1}} \\
    \SENV' &= \set{\locreg{\loc_1}{\reg_1} \mapsto \TYP_1}.
    \end{align*}
    %
    By inspection on \tlet{} and the previous two typing judgements, that is, for $\EXPR_1'$
    and $\EXPR_2$, we discharge this case.
    \item The second obligation
    \begin{displaymath}
    \storewf{\SENV'}{\CENV'}{\AENV'}{\NENV'}{\MENV'}{\STOR'}
    \end{displaymath}
    discharges immediately from the result of the induction hypothesis, which is
    established by the above.
    \end{itemize}
  \end{bcase}

  \begin{bcase}
    \begin{mathpar}
    \rdapp{}
    \end{mathpar}
    \begin{itemize}
    \item
    The first of two proof obligations is to show that
    the result $\EXPR' = \subst{\EXPR}{\overharpoon{\var}}{\overharpoon{\VAL}} \subst{}{\overharpoon{\locreg{\loc'}{\reg'}}}{\overharpoon{\locreg{\loc}{\reg}}}$ of
    the given step of evaluation is well typed, that is,
    \begin{displaymath}
    \emptytenv;\SENV';\CENV;\AENV;\NENV' \vdash \AENV'; \NENV''; \subst{\EXPR}{\overharpoon{\var}}{\overharpoon{\VAL}} \subst{}{\overharpoon{\locreg{\loc'}{\reg'}}}{\overharpoon{\locreg{\loc}{\reg}}} : \hTYP,
    \end{displaymath}
    where $\hTYP = \tyatlocreg{\TYP}{\loc}{\reg}$.
    %
    To this end, we first establish typing judgements for the body of the
    callee and then the arguments of the function, and finally
    discharge the first obligation by combining the two results using
    the substitution lemma.
    By inversion on \tfunctiondef{}, the type judgement
    \begin{align*}
    \TENV;\SENV'';\CENV;\AENV;\NENV \vdash \AENV;\NENV';
    \EXPR : \tyatlocreg{\TYP}{\loc}{\reg},
    \end{align*}
    holds for body of the callee $\EXPR$,
    with constrants
    for any caller, such that $\locreg{\loc}{\reg} \in \NENV$, $\locreg{\loc}{\reg} \not \in \NENV'$ and $\AENV(\reg) = \locreg{\loc}{\reg}$, where
    \begin{align*}
    \TENV & = \set{\overharpoon{\var_1} \mapsto \overharpoon{{\tyatlocreg{\TYP_1}{\loc'_1}{\reg'}}}, \; \ldots \; , \overharpoon{\var_n} \mapsto \overharpoon{{\tyatlocreg{\TYP_n}{\loc'_n}{\reg'}}}}\\
    \SENV'' & = \set{\overharpoon{\locreg{\loc'_1}{\reg'}} \mapsto \overharpoon{\TYP_1}, \; \ldots \; , \overharpoon{\locreg{\loc'_n}{\reg'}} \mapsto \overharpoon{\TYP_n}}.
    \end{align*}
    Regarding the arguments to the call, we obtain by inversion on \tapp{} that
    \begin{align*}
    \emptyset;\SENV;\CENV;\AENV;\NENV \vdash \AENV;\NENV;\overharpoon{\VAL_i} : \overharpoon{\tyatlocreg{\TYP_i}{\loc_i}{\reg}}
    \end{align*}
    for $\ind \in \set{1 \ldots n}$.
    Furthermore, by inversion on \tapp{}, we obtain that $\locreg{\loc}{\reg} \in \NENV$,
    $\locreg{\loc}{\reg} \not \in \NENV'$, and $\AENV(\reg) = \locreg{\loc}{\reg}$,
    which altogether satisfy the requirements of \tfunctiondef{}.
    Now, by application of the Substitution Lemma,
    we have that
    \begin{align*}
    \emptytenv;\SENV;\CENV;\AENV;\NENV' \vdash \AENV; \NENV'; \subst{\EXPR}{\overharpoon{\var_1}}{\overharpoon{\VAL_1}} \subst{}{\overharpoon{\locreg{\loc'_1}{\reg'}}}{\overharpoon{\locreg{\loc_1}{\reg}}} \ldots \subst{}{\overharpoon{\var_n}}{\overharpoon{\VAL_n}} \subst{}{\overharpoon{\locreg{\loc'_n}{\reg'}}}{\overharpoon{\locreg{\loc_n}{\reg}}} : \tyatlocreg{\TYP}{\loc}{\reg}.
    \end{align*}
    \item Given the new environment $\NENV'$
    used by the previous step, the second obligation
    for this proof case is to show that
    \begin{displaymath}
    \storewf{\SENV}{\CENV}{\AENV}{\NENV'}{\MENV}{\STOR}.
    \end{displaymath}
    The individual requirements, labeled
    \refwellformed{sec:well-formedness}{wf:map-store-consistency} -
        \refwellformed{sec:well-formedness}{wf:ca},
        are handled by the following case analysis.
    \begin{itemize}
      \item
      Case (\refwellformed{sec:well-formedness}{wf:map-store-consistency}):
      for each $(\locreg{\loc'}{\reg} \mapsto \TYP) \in \SENV$, there exists some $\ind_1, \ind_2$ such that
      \begin{align}
      (\locreg{\loc'}{\reg} \mapsto \concreteloc{\reg}{\ind_1}{}) \in \MENV \wedge \\
        \ewitness{\TYP}{\concreteloc{\reg}{\ind_1}{}}{\STOR'}{\concreteloc{\reg}{\ind_2}{}}
      \end{align}
      %
      This case discharges immediately from the well formedness of the store
      given by the premise of this lemma.
      \item Case (\refwellformed{sec:well-formedness}{wf:cfc}):
      \begin{align*}
      \storewfcfa{\CENV}{\MENV}{\STOR}
      \end{align*}
      This case discharges immediately from the well formedness of the store
      given by the premise of this lemma.
      \item Case (\refwellformed{sec:well-formedness}{wf:ca}):
      \begin{align*}
      \storewfca{\AENV}{\NENV'}{\MENV}{\STOR}
      \end{align*}
      Of the requirements pertaining to this judgement, the only one potentially
      affected by the new environment $\NENV'$ is requirement
      \refwellformed{sec:well-formedness-allocation}{wf:impl-linear-alloc2}.
      %
      The specific obligation therein is to establish that
      \begin{align*}
      ((\reg \mapsto \locreg{\loc}{\reg}) \in \AENV \wedge
    \, (\locreg{\loc}{\reg} \mapsto \concreteloc{\reg}{\ind_s}{}) \in \MENV \wedge \locreg{\loc}{\reg} \not \in \NENV' \wedge
    \, \ewitness{\TYP}{\concreteloc{\reg}{\ind_s}{}}{\STOR}{\concreteloc{\reg}{\ind_e}{}}) \Rightarrow \\
          \ind_e > \allocptr{\reg}{\STOR}.
          \end{align*}
      %
      The reason the change to environment $\NENV'$ might affect
      the above is because, if all the conjuncts above hold, then
      it remains to establish that $\ind_e > \allocptr{\reg}{\STOR}$
      holds.
      %
      However, it turns out that the fourth conjunct above does not
      hold, i.e., there is no such end witness in the store $\STOR$,
      thus relieving the obligation to establish $\ind_e > \allocptr{\reg}{\STOR}$.
      %
      The reason the end witness does not exist is yielded by
      the well formedness of the store given by the premise of this
      lemma, in particular requirement
      \refwellformed{sec:well-formedness-allocation}{wf:impl-linear-alloc}.
      %
      That is, by inversion on \tapp{}, it is the case that
      \begin{align*}
      (\reg \mapsto \locreg{\loc}{\reg}) \in \AENV \wedge \locreg{\loc}{\reg} \in \NENV.
      \end{align*}
      %
      Therefore, requirement \refwellformed{sec:well-formedness-allocation}{wf:impl-linear-alloc}
      implies that
      \begin{align*}
      \ind_s > \allocptr{\reg}{\STOR}.
      \end{align*}
      %
      As such, given that the store $\STOR$ remains unchanged
      and the above, it is straightforward to show that the end witness
      starting at $\ind_s$ cannot exist, thereby discharging this case.
      \end{itemize}
      \item
      Case (\refwellformed{sec:well-formedness}{wf:impl1}):
      \begin{displaymath}
      dom(\SENV) \cap \NENV' = \emptyset
      \end{displaymath}
      This case discharges because,
      from the well formedness of the store given by the
      premise of this lemma, $dom(\SENV) \cap \NENV = \emptyset$,
      and because $\NENV' = \NENV - \set{\locreg{\loc}{\reg}}$.
    \end{itemize}
  \end{bcase}

$\blacksquare$

\end{nproof}

%% \subsection{Type Safety}

The type safety theorem for \ourcalc{} was stated in~\ref{theorem:type-safety} and is restated here.

\begin{theorem}[Type safety]
  %% \label{theorem:type-safety}
\begin{displaymath}
  \begin{aligned}
  \text{If} \;\; & (\emptyset;\SENV;\CENV;\AENV;\NENV \vdash \AENV';\NENV';\EXPR : \hTYP) \wedge
                   (\storewf{\SENV}{\CENV}{\AENV}{\NENV}{\MENV}{\STOR}) \\
  \text{and} \;\; & \STOR;\MENV;\EXPR \stepsto^n \STOR';\MENV';\EXPR' \\
  \text{then} \;\; & (\EXPR' \; \mathit{value}) \vee
                     (\exists \STOR'', \MENV'', \EXPR'' . \; \STOR';\MENV';\EXPR' \stepsto \STOR'';\MENV'';\EXPR'')
  \end{aligned}
  \end{displaymath}
\end{theorem}

\begin{nproof}
  The type safety follows from an induction with
  \ref{lemma:progress} (progress lemma) and \ref{lemma:preservation} (preservation lemma).
\end{nproof}



\addcontentsline{toc}{chapter}{Bibliography}

\bibliographystyle{acm}
%\include{your_bibliography_name.tex}
\bibliography{refs}

% Adds a line for your CV without a page number

%% \addtocontents{toc}{%
%%   \protect\contentsline{chapter}{Curriculum Vitae}{}}
\end{document}


%\usepackage[usenames,dvipsnames]{xcolor}
%\usepackage[dvipsnames]{xcolor}
%\usepackage[]{xcolor}

% [2016.10.23] Switching to margin notes for peanut gallery:
\usepackage{todonotes}
% \presetkeys{todonotes}{inline}{}  % Activate to turn-off margin notes.

% [2016.05.26] This can just be a growing list from paper to paper:
\newcommand{\rn}[1]{\pgwrapper{RRN}{#1}}  % Ryan Newton

\newcommand{\tlm}[1]{\pgwrapper{TLM}{#1}} % Trevor
\newcommand{\bjs}[1]{\pgwrapper{BJS}{#1}} % Bo Joel Svenson
\newcommand{\chc}[1]{\pgwrapper{CHC}{#1}} % Chao-Hong Chen
\newcommand{\vc}[1]{\pgwrapper{VC}{#1}} % Vikraman Choudhury
\newcommand{\bc}[1]{\pgwrapper{BC}{#1}} % Buddhika Chamith
\newcommand{\mv}[1]{\pgwrapper{MV}{#1}} % Michael Vollmer
\newcommand{\osa}[1]{\pgwrapper{OSA}{#1}} % Omer Agacan
\newcommand{\mm}[1]{\pgwrapper{MM}{#1}} % Madan Musuvathi
\newcommand{\rgs}[1]{\pgwrapper{RGS}{#1}} % Ryan Scott
\newcommand{\csk}[1]{\pgwrapper{CSK}{#1}} % Chaitanya Koparkar

\newcommand{\mk}[1]{\pgwrapper{MK}{#1}} % Milind Kulkarni
\newcommand{\ls}[1]{\pgwrapper{LS}{#1}} % Laith Sakka

\newcommand{\spall}[1]{\pgwrapper{SS}{#1}} % Sarah Spall

\newcommand{\jd}[1]{\pgwrapper{JD}{#1}} % Joe Devietti
\newcommand{\onl}[1]{\pgwrapper{ONL}{#1}} % Omar Navarro-Leija

\newcommand{\mr}[1]{\pgwrapper{MR}{#1}} % Mike Rainey

% Add co-authors here or in a downstream file.

\InputIfFileExists{activateeditingmarks}{
}{
    \def\noeditingmarks{}
}

\definecolor{comment-red}{rgb}{0.8,0,0}
\definecolor{dark-green}{rgb}{0.0,0.4,0}
\definecolor{dark-blue}{rgb}{0.0,0.0,0.55}
\definecolor{very-dark-green}{rgb}{0.0,0.3,0}

\ifx\noeditingmarks\undefined
   %% This means someone else should take a look:
   \newcommand{\auditme}[1]{{\color{dark-green}{#1}}}

   %% This meants there's a serious problem that must be fixed before
   %% the paper goes out:
   \newcommand{\fixme}[1]{{\color{comment-red}{#1}}}

   %% This is for outlining, i.e., psuedotext that must be converted
   %% to real text:
   \newcommand{\note}[1]{{\begin{itemize}\item \textcolor{blue}{NOTE: #1} \end{itemize}}}

   %% A very poor-man's version of ``track changes'':
   \newcommand{\new}[1]{{\color{blue}{#1}}}

   \newcommand{\const}[1]{\textred{#1}}
   %\definecolor{mygrey}{rgb}{0.6,0.6,0.6}
   \definecolor{mygrey}{rgb}{0.7,0.7,0.7}

   \newcommand{\textred}[1]{{\color{comment-red}{#1}}}

   % Old version was inline:   
   % \newcommand{\pgwrapper}[2]{\textred{(#1: #2)}}
   % New version is in the margin:
   \setlength{\marginparwidth}{1.6cm}
   \newcommand{\pgwrapper}[2]{\todo[size=\scriptsize]{#1: #2}}

   %% RRN: Dim the background for Ryan's eyes:
   %% Hack: do this ONLY on Ryan's machine.
   %%  I don't know a way based on user or host name, so...
   %%  instead I just signal on a file, activategreybg.tex which I
   %%  don't then checkin.
   \InputIfFileExists{activategreybg}{
       \definecolor{comment-red}{rgb}{0.5,0,0}
       \pagecolor{mygrey}
   }{}

  % When they are their own para, todonotes annoyingly take up vertical space:
  \newcommand{\rnfloat}[1]{\pgwrapper{RRN}{#1}\vspace{-4mm}}  % Ryan Newton

\else
%   \newcommand{\textred}[1]{#1}
% Keep it on so we catch stuff to fix:
   \newcommand{\textred}[1]{{\color{comment-red}{#1}}}
   \newcommand{\auditme}[1]{#1}
   % This is more severe and should not go away:
   \newcommand{\fixme}[1]{\textred{#1}}
   \newcommand{\pgwrapper}[2]{}
%   \newcommand{\note}[1]{}
   \newcommand{\note}[1]{{\begin{itemize}\item {\textcolor{blue}{#1}} \end{itemize}}}
   % \newcommand{\new}[1]{#1}
   \newcommand{\new}[1]{{\color{blue}{#1}}}
   \newcommand{\const}[1]{#1}

   \newcommand{\rnfloat}[1]{}  % Ryan Newton
\fi

% Just an alias because sometimes we use this:
\newcommand{\Red}[1]{\textred{#1}}

% These probable should be elsewhere:
\newcommand{\ie}{{i.e.,}}
\newcommand{\eg}{{e.g.,}}
\newcommand{\etal}{\textit{et al.}}

\def\scrunchit{}
\ifx\scrunchit\undefined
  \newcommand{\captionscrunch}{}
\else
%  \newcommand{\captionscrunch}{\vspace{-4mm}}
  \newcommand{\captionscrunch}{\vspace{-3mm}}
\fi

%% \newcommand{\secref}[1]{section~\ref{#1}}
%% \newcommand{\Secref}[1]{Section~\ref{#1}}
\newcommand{\secref}[1]{\S\ref{#1}}
\newcommand{\Secref}[1]{\secref{#1}}

%% \newcommand{\Figref}[1]{Figure~\ref{#1}}
%% \newcommand{\figref}[1]{figure~\ref{#1}}
\newcommand{\Figref}[1]{Fig.~\ref{#1}}
\newcommand{\figref}[1]{Fig.~\ref{#1}}
\newcommand{\tabref}[1]{Table~\ref{#1}}


\newcommand{\appendixref}[1]{\ref{#1}}
\newcommand{\Appendixref}[1]{\ref{#1}}

\section{Overview}

This section describes \ourcalc{}, which is a programming language
for programming with serialized data. A primary use case of \ourcalc{}
is as an \emph{intermediate language} for a compiler.
%
Traditional compilers are built on a number of well-defined intermediate
abstractions and translations that close the semantic gap between source and
target. If we are building a compiler to generate code that operates on
serialized data, what we need are analogous way-points to structure compilers
that target serialized-data traversals (stream-processors, essentially). Indeed,
there is quite a semantic gap between the low-level, buffer-mutating,
pointer-bumping programs, and a source language of high-level, pure, recursive
functions on algebraic datatypes. \ourcalc{} is designed to structure the space
between, where types are augmented to track \emph{locations} within regions
(\eg{} byte offsets).

\ourcalc{} follows in the tradition of typed assembly language~\cite{TAL},
region calculi~\cite{mlkit-retrospective}, and Cyclone~\cite{cyclone-pldi}
in that it uses types to both
expose and make safe low-level implementation mechanisms.
%
%% Ultimately, our compiler output will increment pointers into memory buffers, as
%% seen in code of the previous section.
%
The basic idea of \ourcalc{} is to first establish what data \emph{share} which
logical memory regions (essentially, buffers), and in what \emph{order} those data reside,
abstracting the details of computing exact addresses.
%
For example, data constructor applications, such as \lstinline{Leaf 3}, take an extra
location argument in \ourcalc{}, specifying where the data constructor
should place the resulting value in memory:
% starts in memory, that is, where the data
% constructor itself is written:
\lstinline[mathescape]{Leaf $\;\locreg{l}{}\;$ 3}.
%
This location becomes part of type of the value: \lstinline[mathescape]{$\,\tyatlocreg{Tree}{\,l}{}$}.
Every location resides in a region, and when we want to
name that region, we write $\locreg{l}{r}$.


Locations represent information about where values are in a store, but
are less flexible than pointers. They are introduced {relative} to other
locations. A location variable is either \emph{after} another
variable, or it is at the beginning of a region,
thus specifying a serial order.
%so they essentially describe what \emph{order} values are in within a region.
If location $l_2$ is declared as
\lstinline[mathescape]{$l_2$ = after($\tyatlocreg{Tree}{l_1}{\reg}$)},
%% where $x$ is a tree at $l_1$,
then $l_2$ is \emph{after} every element of the tree
rooted at $l_1$.

{\emph{Regions} in \ourcalc{} represent the memory buffers containing serialized data
  structures.  Unlike some other region calculi, in \ourcalc{}, values in a
  region \emph{may} escape the static scope which binds and allocates that region. In
  fact, an extension introduced later in~\secref{subsec:indirections}
  specifically relies on inter-region pointers and coarse-grained garbage
  collection of regions.}

%% In this paper, we present a new method for building compilers to lift functions
%% over serialized data.
%% We also present a new calculus, \ourcalc{}, which corresponds to a simplified
%% presentation of the core intermediate language of our compiler.
%% %
%% It was designed both to form a foundation for the optimizations we describe in
%% this paper, generalizing previous work on region calculi to represent recursive
%% programs operating on serialized data. Crucially, in addition to associating
%% values with \emph{regions}, in \ourcalc{} values are associated with a

% As shown in the grammar in \figref{fig:grammar},
\ourcalc{} is a first-order,
call-by-value functional language with algebraic datatypes and pattern
matching. Programs consist of a series of datatype definitions, function
definitions, and a main expression.
%
%% The algebraic data types work essentially like one would expect, but with
%% additional machinery to handle \emph{where} a particular value is in the store
%% and the \emph{relative position} of its fields in the store.  Every invocation
%% of a data constructor is annotated with a location variable ---
%
\ourcalc{} programs can be written directly by hand, and \ourcalc{} also serves
as a practical {\em intermediate language} for other tools or front-ends that
want to convert computations to run on serialized data (essentially fusing a
consuming recursion with the deserialization loop).
%
%% We return to this use-case in \secref{sec:impl-hical}.

\paragraph{Allocating to output regions}

Now that we have seen how data constructor applications are parameterized by
locations, let us look at a more complex example than those of the prior section.
Consider \lstinline[mathescape]{buildtree}, which constructs the same trees consumed by \lstinline[mathescape]{sum}
and \lstinline[mathescape]{rightmost} above.  First, in the source language without locations:

%% \floatstyle{plain}\restylefloat{figure}
%% \begin{figure}
%% \centering
\begin{code}
buildtree : Int -> Tree
buildtree n = if n == 0
              then Leaf 1
              else Node (buildtree (n - 1))
                        (buildtree (n - 1))
\end{code}
%% \vspace{-1mm}
%% \caption{Example: Building an output data structure}
%% \label{fig:buildtree}
%% \vspace{-5mm}
% \end{figure}
% \mav{TODO: vertically align the stmts}
%
Then in {\ourcalc}, where the type scheme binds an output rather than input location:
% \floatstyle{plain}\restylefloat{figure}
%\vspace{-1mm}
%\floatstyle{boxed}\restylefloat{figure}
%\begin{figure}[h]
%\centering
\begin{code}
buildtree : forall @\locreg{l}{r}@ . Int -> @\tyatlocreg{Tree}{l}{r}@
buildtree [@\locreg{l}{r}@] n =
  if n == 0 then (Leaf @\locreg{l}{r}@ 1) -- write tag + int to output
  else -- skip past tag:
       letloc $\locreg{l_a}{r}$ = $\locreg{l}{r}$ + 1 in
       -- build left in place:
       let left : @\tyatlocreg{Tree}{l_a}{\reg}@ =
           buildtree [@\locreg{l_a}{r}@] (n - 1) in
       -- find start of right:
       letloc $\locreg{l_b}{r}$ = after(@\tyatlocreg{Tree}{l_a}{\reg}@) in
       -- build right in place:
       let right : @\tyatlocreg{Tree}{l_b}{\reg}@ =
           buildtree [@\locreg{l_b}{r}@] (n - 1) in
       -- write datacon tag, connecting things together:
       (Node @\locreg{l}{r}@ left right)
  \end{code}
%\end{figure}
%\vspace{-3mm}

\noindent
Here, we see that \ourcalc{} must represent locations that have {\em not yet
  been written}, \ie{} they are output destinations.  Nevertheless, in the
recursive calls of \lstinline[mathescape]{buildtree} this location is passed as an argument: a form
of destination-passing style~\cite{destination-passing}.
The type system guarantees that memory will be initialized and written exactly once.
%
The output location is threaded through the recursion to build the left subtree,
and then offset to compute the starting location of the right subtree.
%
%% Computing \lstinline[mathescape]{after($\tyatlocreg{Tree}{l_a}{\reg}$)} is a potentially expensive operation, and
%% implementing it---or rearranging the program to avoid needing it---is the
%% responsibility of the \ourcalc implementation (\secref{sec:impl-local}).
{It might appear that computing
\lstinline[mathescape]{after($\tyatlocreg{Tree}{l_a}{\reg}$)} could be quite expensive,
if there is a large tree at that location. This does not need to be the case.
In~\secref{sec:impl-local} I will present different techniques for efficiently
compiling \ourcalc programs without requiring linear walks through serialized data.}

%% Later
%% (in~\secref{sec:route-ends}), we describe a compilation technique that
%% makes obtaining the \lstinline[mathescape]{after} of a location in cases like this
%% straightforward by having computations return the ending locations of
%% whatever trees they traverse.
%
%% Furthermore, while the above \lstinline[mathescape]{buildTree} code writes the left and right subtrees {\em
%%   before} writing the data constructor tag into the output, we will want the
%% compiler to eventually reorder these writes to achieve a linear memory access
%% pattern on the output region.

%% \rn{May want to modify the discussion of ``size information'' below.}
{One of the goals of \ourcalc{} is to support several compilation
  strategies. One extreme is compiling programs to work with a representation of
  data structures that do not include \emph{any} pointers or indirections at
  run-time---within such a representation, the size of a value can be observed
  by threading through ``end witnesses'' while consuming packed values: for
  example, \lstinline[mathescape]{buildtree} above would \emph{return} $\locreg{l_b}{r}$, rather than computing
  it with an \lstinline[mathescape]{after} operation.
  %
  (The end-witness strategy was first used in
  {\em Gibbon}~\cite{ecoop17-gibbon} prior to the design of \ourcalc{},
  which previously compiled functions on fully serialized data,
  while not preserving asymptotic complexity.)
  %% %% Our language allows for this by ensuring that
  %% %% when location variables are introduced after values, a variable binding the
  %% %% preceeding value must be in scope.
  %% %
  %% In \secref{sec:compiler}, we will discuss in detail how the invariants
  %% enforced by \ourcalc{} make subsequent transformations and optimizations
  %% more tractable.
}
%
%% In practice, as a whole-program compiler, our compiler will decide
%% whether or not particular data constructor needs {\em random-access}
%% based on how it is used in the program. The analyses required for this
%% are described in more detail in \secref{sec:compiler}.
%
%% \mav{Forward reference whatever section where we add back limited random-access
%%   capabilities.}
%
%% Furthermore, while our compiler does translate every program into
%% \ourcalc{}, later stages in the compiler will relax or violate the
%% invariants enforced by \ourcalc{}, such as the absolute ordering of
%% values in a region.
%% %
%% In fact, by code generation time the compiler will
%% have likely transformed the program to include explicit size
%% information and inter-region pointers.
%
Next, I will present a formalized core subset of \ourcalc,
%% in more detail
%% (\secref{subsec:grammar}),
its type system (\secref{subsec:static}),
and its operational semantics (\secref{subsec:dynamic}).
% which uses a store to  represent regions and locations.
%% before moving on to implementation (\secref{sec:impl-local}, \secref{sec:impl-hical})
%% and evaluation (\secref{sec:eval}).

\section{Formal Language and Grammar}
\label{subsec:grammar}

\begin{figure}
  \begin{displaymath}
  \begin{aligned}
  &\DC \in \; \textup{Data Constructors},\:\: \TYP \in \; \textup{Type Constructors},\\
  &\var, \yvar,\fvar \in \; \textup{Variables},\:\:
  \loc, \locreg{l}{r} \in \; \textup{Symbolic Locations},\\
  &\reg \in \; \textup{Regions},\:\: \ind, \indj \in \; \textup{Region Indices},\\
  &\concreteloc{r}{i}{l} \in \; \textup{Concrete Locations}
  \end{aligned}
\end{displaymath}
\begin{displaymath}
  \begin{aligned}
    \textup{Top-Level Programs} && \PROG && \gramdef & \overharpoon{\DD} \;; \overharpoon{\FD} \;; \EXPR \\
    \textup{Datatype Declarations} && \DD && \gramdef & \DATA\;\TYP = \overharpoon{\DC \; \overharpoon{\sTYP}\;} \\
    \textup{Function Declarations} && \FD && \gramdef & \fvar : \TS ; \fvar \overharpoon{\var} = \EXPR \\
    %\textup{Data Location} && \RP && \gramdef & \loc^{\reg} \\
%    \textup{Types} && \TYP && \gramdef & \ldots \gramor \TC \gramor \ldots \\
    %\gramor \ldots \\ % maybe don't even talk about int, is it confusing to have non-packed data?
    \textup{Located Types} && \hTYP && \gramdef & \tyatlocreg{\TYP}{\loc}{\reg} \\
    \textup{Type Scheme} && \TS && \gramdef &
      \forall _{\overharpoon{\locreg{l}{r}}}.
      \overharpoon{\hTYP} \ARROW \hTYP \\
    \textup{Values} && \VAL && \gramdef & \var \gramor \concreteloc{\reg}{\ind}{\locreg{\loc}{\reg}}\\
    \textup{Expressions} && \EXPR && \gramdef & \VAL\\[-5pt]
    && && \gramor & \fapp{\overharpoon{\locreg{l}{r}}}{\overharpoon{\VAL}} \\
    && && \gramor & \datacon{\DC}{\keywd{\locreg{\loc}{\reg}}}{\overharpoon{\VAL}}\\
    %% && && \gramor & \datacon{\DC}{\keywd{\locreg{\loc}{\reg}}}{\overline{\var}} \gramor \litcon{\locreg{\loc}{\reg}}{\EXPR} \\
    && && \gramor & \letpack{\var:\hTYP}{\EXPR}{\EXPR} \\
    && && \gramor & \letloc{\locreg{\loc}{\reg}}{\LE}{\EXPR} \\
    && && \gramor & \letreg{\reg}{\EXPR} \\
    && && \gramor & \case{\VAL}{\overharpoon{\pat}} \\
    \textup{Pattern} && \pat && \gramdef &
    \caseclause{\datacon{\DC}{}{(\overharpoon{\var : \hTYP})}}{\EXPR} \\
    \textup{Location Expressions} && \LE && \gramdef
    & \startr{\reg} \\
    && && \gramor & (\locreg{\loc}{r} + 1) \\
    && && \gramor & \afterl{\hTYP}
  \end{aligned}
\end{displaymath}

  \caption{Grammar of \ourcalc{}}
  \label{fig:grammar}
\end{figure}

\Figref{fig:grammar} gives the grammar for a formalized core of \ourcalc{}.
%
Iuse the notation $\overharpoon{x}$ to denote a vector $[
x_1, \ldots, x_n]$, and $\overharpoon{x_{\ind}}$ the item at position
$\ind$.
%
To simplify presentation, the language supports
algebraic datatypes without any base primitive types, but could be extended in a straightforward
manner to represent primitives such as an $\sgramwd{Int}$ type or tuples.
%
The expression language is based on the first-order lambda calculus,
using A-normal form.
%
The use of A-normal form simplifies our formalism and proofs
without loss of generality.
%in ways that do not limit generality.
%It would be straightforward to generalize to direct style.

Like previous work on region-based memory~\cite{regioncalcs},
\ourcalc{} has a special binding
form for introducing region variables, written as
$\sgramwd{letregion}$.
%
Location variables are similarly introduced by $\sgramwd{letloc}$.
%
The pattern-matching form $\sgramwd{case}$ binds variables to
serialized values, as well as binding the location for each variable.
%
% To simplify the formalism,
It is required that each bound location in
a source program is unique.
%
% This convention rules out programs with space leaks caused by
%shadowing of \gramwd{letloc}-bound names.
% \rn{I can't easily construct hypothetical example...}

The $\sgramwd{letloc}$ expression binds locations in only three ways:
a location is either the \emph{start} of a region (meaning, the
location corresponds to the very beginning of that region), is
immediately after another location, or it occurs \emph{after} the last
position occupied by some previously allocated data constructor.
%
For the last case, the location is written to exist at
$\afterl{\tyatlocreg{\TYP}{\loc}{\reg}}$, where $\loc$ is already
bound in a region, and has a value written to it.

%% \begin{comment}
%% Second, the convention enables the dynamic semantics to treat symbolic
%% locations at runtime as a source of dynamically fresh names (i.e., a
%% gensym-like mechanism), which enables a simpler environment structure
%% for tracking locations.
%% %
%% It is straightforward to relax these requirements.
%% \end{comment}

Values in \ourcalc{} are either (non-location) variables or
\emph{concrete locations}.
%
In contrast to bound location variables, concrete locations
do not occur in source programs; rather, they appear at runtime, created by the
application of a data constructor, which has the effect of
extending the store.
%
Every application of a data constructor writes a \emph{tag} to the store, and
concrete locations allow the program to navigate through it.
%
To distinguish between concrete locations and location variables in
the formalism, I refer to the latter as \emph{symbolic locations}.
%
A concrete location is a tuple
$\concreteloc{\reg}{\ind}{\loc}$ consisting of a region, an index, and
symbolic location corresponding to its binding site.
%
The first two components are sufficient to fully describe
an \emph{address} in the store.

%% \begin{comment}
%% Rather than annotating expressions with their region (such as
%% $\sEXPR\; @ \;\sreg$) like standard region calculi, \emph{location
%% annotations} appear at various points in the program source, such as
%% in types, in function declarations, and at the site of constructor
%% application.
%% %
%% %% As a first-order language, functions are only defined at the top
%% %% level. Each function has a name, a \emph{type scheme}, an argument,
%% %% and an expression (the body of the function).  The type scheme
%% %% consists of the types of the input and output of the function,
%% %% parameterized by \emph{data locations} (location and region
%% %% pair).
%% %
%% %% {Function application similarly requires a list of data
%% %% locations.}
%% %
%% Function types are only introduced at the top level, and these
%% functions may only be polymorphic with respect to data locations.
%% \end{comment}
%
%% \mv{Explain that locations of inputs to functions must have been written,
%% and locations of outputs to functions cannot have been written.}
%
%% Data locations are annotated with an arrow indicating whether they are
%% input ($\inloc$) or output ($\outloc$) locations, which essentially is intended to represent
%% whether the location will be associated with a value that is coming
%% into the function as an input or a value that will be computed and
%% returned from the function as an output. These annotations occur both
%% in the declarations of functions and in the application of functions;
%% in the former case, as parameters for the function, and in the latter
%% case, as arguments to the function. In the case where data locations
%% occur as arguments, the locations must be in scope, and their
%% annotation of input or output must match their state at that point
%% (so, input locations have been already written to, output locations have not).

%% We elide the superscript $r$ on location variables ($\sRP$ in the
%% grammar) when that information is not relevant.
%

%% \begin{comment}
%% Naturally, a location's region matches the one it's derived from: in
%% \il{letloc $\;l^r\;$ = start $\;r$}, the locations $r$
%% match.
%% %
%% These regions model logically continuguous memory buffers that are
%% growable at one end.  This is similar to traditional region
%% systems, except that allocating to, \eg{} an MLKit region happens in
%% units of complete objects (with pointers), whereas we permit leaving
%% objects ``partially written'' for arbitrarily long periods of time ---
%% separating writing the tag of a data constructor from serializing out
%% its fields.
%% %
%% Indeed, the entire purpose of using a region system in \ourcalc{} is
%% to group objects together so that their representations can be merged
%% and compressed by ``inlining'' pointers.
%% \end{comment}

%% Pattern matching, in the form of $\sgramwd{case}$ statements, are
%% familiar from other functional languages. Each alternative in the
%% pattern match statement matches on a constructor, its location, and a
%% series of variable and annotated type pairs corresponding to each
%% field of that constructor.

%% For this presentation of \ourcalc{}, we will not be too concerned with
%% including primitive operations on, e.g., numbers, strings, etc though
%% our full implementation obviously includes these things.  At certain
%% points in this paper we will give examples of \ourcalc{} programs that
%% make use of these extensions. Extending the presentation of \ourcalc{}
%% to include such types and operations is straightforward.
%% %
%% \rn{Doesn't need that much discussion, but could combine this point
%% with a bigger point about what syntactic sugar we will allow ourselves
%% in examples.}

\subsection{Static Semantics}
\label{subsec:static}


\begin{figure}
  \begin{displaymath}
  \begin{aligned}
    \textup{Typing Env.} && \TENV && \gramdef & \; \set{\var_1 \mapsto \hTYP_1, \; \ldots \; , \var_n \mapsto \hTYP_n} \\
    \textup{Store Typing} && \SENV && \gramdef & \; \set{\locreg{\loc}{r_1}_1 \mapsto \TYP_1, \; \ldots \; , \locreg{\loc}{r_n}_n \mapsto \TYP_n} \\      
    \textup{Constraint Env.} && \CENV && \gramdef & \;
      \set{\locreg{\loc}{r_1}_1 \mapsto \LE_1, \; \ldots \; , \locreg{\loc}{r_n}_n \mapsto \LE_n} \\
    \textup{Allocation Pointers} && \AENV && \gramdef & \; \set{r_1 \mapsto ap_1, \; \ldots \; , r_n \mapsto ap_n} \\
    && && & {\text{where} \; ap = \locreg{\loc}{\reg} \gramor \emptyset} \\
    \textup{Nursery} && \NENV && \gramdef & \; \set{\locreg{\loc}{r_1}_1, \; \ldots \; , \locreg{\loc}{r_n}_n}
  \end{aligned}
\end{displaymath}

  \caption{Extended grammar of \ourcalc{} for static semantics}
  \label{fig:typegrammar}
\end{figure}
\begin{figure}
  \footnotesize
  \begin{mathpar}
    \rtvar{}\hspace{1em}
    \rtconcreteloc{}\\
    \rtlet{}\hspace{1em}
    \rtlregion{}\\
    \rtlltag{}\hspace{1em}
    \rtllstart{}\\
    \rtllafter{}\hspace{1em}
    \rtdatacon{}
  \end{mathpar}
  \normalsize
  \caption{Typing judgments for \ourcalc{} (1)}
  \label{fig:types1}
\end{figure}

%\floatstyle{boxed}\restylefloat{figure}
\begin{figure}
  \footnotesize
  \begin{mathpar}
    \rtapp{}\hspace{1em}
    \rtfunctiondef{}\\
    \rtpat{}\\
    \rtcase{}\hspace{1em}
    \rtprogram{}
  \end{mathpar}
  \normalsize
   \caption{Typing judgments for \ourcalc{} (2)}
   \label{fig:types2}

\end{figure}



In \figref{fig:typegrammar}, I extend the grammar with some extra
details necessary for describing the type system.
The typing rules for expressions in \ourcalc{} are given in
\figref{fig:types1} and \figref{fig:types2}, where the rule form is as follows:
%% $\RENV;\EENV;\CENV;\SENV;\TENV \vdash \EXPR : \TYP \tyatlocreg{\loc}{\reg} ; \RENV';\EENV'$.

\[ \TENV;\SENV;\CENV;\AENV;\NENV \vdash \AENV'; \NENV'; \EXPR : \hTYP \]

%

The five letters to the left of the turnstile are different environments.
$\TENV$ is a standard typing environment.
$\SENV$ is a store-typing environment, which maps all
\emph{materialized} symbolic
locations to their types. That is, every location in $\SENV$ {\em has been written}
and contains a value of type $\SENV(l^r)$.
$\CENV$ is a constraint environment, which keeps
track of how symbolic locations relate to each other.
$\AENV$ maps each region in scope to a location, and is used to symbolically
track the allocation and incremental construction of data structures;
$\AENV$ can be thought of as representing the
\emph{focus} within a region of the computation.
% (the exact usage of $\AENV$ will be expanded on below).
$\NENV$ is a nursery of all symbolic locations that have been allocated,
but not yet written to.
Locations are removed from $\NENV$
upon being written to, as the purpose is to prevent multiple writes to
a location.
Both $\AENV$ and $\NENV$ are threaded through the typing
rules, also occuring in the output (to the right of the
turnstile).

%% To give the typing rules, we first extend the grammar with some extra
%% details that are necessary for keeping track of location and region
%% variables. A \emph{location state} is a three-tuple of boolean values,
%% representing whether a location has been written to, whether the
%% location was introduced after some other location, and whether a
%% different location has been introduced after the location. An
%% environment maps locations to their location states.
%% %
%% The location state environment is \emph{threaded through} the
%% typing rules: it occurs as a premise (before the turnstile)
%% and alongside the type after the turnstile. This allows the typing
%% rules to encode \emph{changes} in the state of locations (for
%% example, recording that they have been written to).

The \textsc{\tvar} rule ensures that the variable is in scope, and
the symbolic location of the variable has been written to.
%
\textsc{\tconcreteloc} is very similar, and also just ensures that
the symbolic location has been written to.
%
\textsc{\tlet} is straightforward, but note that along with $\TENV$,
it also extends $\SENV$ to signify that the location $\loc$ has
materialized.

In \textsc{\tlregion}, extending $\AENV$ with an empty allocation pointer
brings the region $\reg$ in scope, and also indicates that
a symbolic location has not yet been allocated in this region.

There are three rules for introducing locations
(\textsc{\tllstart}, \textsc{\tlltag} and \textsc{\tllafter}, all shown in \Figref{fig:types1}),
corresponding to three ways of allocating a new location in a
region. A new location is either: at the start of a region, one cell
after an existing location, or after the data structure rooted at an
existing location. Introducing locations in this fashion sets up an
ordering on locations, and the typing rules must ensure that the
locations are used in a way that is consistent with this intended
ordering.
%
To this end, each such rule extends the constraint environment $\CENV$
with a constraint that is based on how the location was introduced,
and $\NENV$ is extended to indicate that the new location is in scope
and unwritten.

Additionally, the location-introduction rules use $\AENV$ to ensure that a program
must introduce locations in a certain pattern (corresponding to the
left-to-right allocation and computation of fields, as explained
in~\secref{subsec:dynamic}).
%% Recall from earlier, constructing a data structure proceeds through a
%% series of steps: allocating a location for a tag, allocating a
%% location after the tag for the first field (if there are fields), then
%% for each field alternating materializing fields and allocating after
%% them, until finally writing the initial tag.
%
In $\AENV$, each region is mapped to either the right-most allocated
symbolic location in that region (if it is unwritten), or to the
symbolic location of the most recently materialized data structure.
This mapping in $\AENV$ is used by the typing rules to ensure that:
%
(1) \textsc{\tllstart} may only introduce a location at the start
of a region once;
%
(2) \textsc{\tlltag} may only introduce a location if an unwritten location has
just been allocated in that region (to correspond to the tag of some
soon-to-be-built data structure); and
%
(3) \textsc{\tllafter} may only introduce a location if a data structure has
just been materialized at the end of the region, and the programmer wants to
allocate \emph{after} it. To attempt, for example, to allocate the location of
the right sub-tree of a binary tree \emph{before} materializing the left sub-tree would
be a type error.
%
{Each location-introduction rule also ensures that the
introduced location must be written to at some point, by checking that
it's absent from the nursery after evaluating the expression.}

%% When a fresh
%% region $\reg$ is bound, $\AENV$ is extended to map that region
%% to $\emptyset$, indicating no locations have been allocated in it,
%% and so \textsc{\tllstart} may introduce a location at the start of $\reg$
%% and update $\AENV$ to note that a location has been allocated there.
%% For allocating a location intended as the first field after a tag,
%% \textsc{\tlltag} expects $\AENV$ to map to some location $\loc'$
%% which is also in $N$, because the tag of a dat

%% and each enforces that the location will be allocated in a region in a
%% specific way, extending the location constraint environment with
%% details of where the location was allocated relative to the rest of
%% the region. $\AENV$ and $\NENV$ are also extended to reflect this allocation.

In order to type an application of a data constructor, \textsc{\tdatacon} starts by
ensuring that the tag being written and all the fields have the correct type.
Along with that, the locations of all the fields of the constructor must
also match the expected constraints. That is, the location of the first field
should be immediately after the constructor tag, and there should be
appropriate $\mathit{after}$ constraints for other fields in the location constraint environment.
After the tag has been written, the location $\loc$ is removed from the
nursery to prevent multiple writes to a location.
%
{
%
As mentioned earlier, LoCal uses destination-passing style. To guarantee destination-passing
style, it suffices to ensure that a function returns its value in a location passed from its caller.
The LoCal type system enforces this property by using constraints of the form $l' \neq l$ in the premises
of the typing rules of the operations that introduce new locations
}

{As demonstrated by \textsc{\tdatacon}, the type system enforces a
  particular ordering of writes to ensure the resulting tree is
  serialized in a certain order. Some interesting patterns are
  expressible with this restriction (for example, writing or reading
  multiple serialized trees in one function), and, as I will address
  shortly in \secref{subsec:indirections}, \ourcalc is flexible enough
  to admit extensions that soften this restriction and allow for
  programmers to make use of more complicated memory layouts.}

{A simple demonstration of the type system is shown in
Table~\ref{table:types-example}, which tracks how $\AENV$, $\CENV$,
and $\NENV$
% (allocation point, constraints, nursery)
change after each line in a simple expression that builds
a binary tree with leaf children. Introducing $\loc$ at the top
establishes that it is at the beginning of $\reg$,
$\AENV$ maps $\reg$ to $\loc$, and $\NENV$ contains $\loc$.
The location for the left sub-tree,
$\loc_a$, is defined to be $+ 1$ after it, which updates $\reg$ to
point to $\loc_a$ in $\AENV$ and adds a constraint to $\CENV$ for
$\loc_a$. Actually constructing the \texttt{Leaf} in the next line
removes $\loc_a$ to $\NENV$, because it has been written to. Once $\loc_a$
has been written, the next line can introduce a new location $\loc_b$
\emph{after} it, which updates the mapping in $\AENV$ and adds a
new constraint to $\CENV$. Once $\loc_b$ has been written and removed
from $\NENV$ in the next line, the final \texttt{Node} can be constructed,
which expects the constraints to establish that $\loc$ is before $\loc_a$,
which is before $\loc_b$.}

%% Additional rules (such as for function application, pattern matching)
%% are conventional%
%% \iftoggle{EXTND}{
%% and are in the Appendix,~\appendixref{subsec:types2}.
%% }{
%% and are available in the Appendix of the extended version~\cite{LoCal-tr}
%% }

%% Two rules (\textsc{\tcase}, \textsc{\tpat}) cover pattern matching. \ldots

%% \mav{We could also potentially explain the $\loc \neq \loc_i$ bit in \tpat.}

%All the other typing judgements are straightforward.

To finish out the typing rules, \Figref{fig:types2} contains rules for
function application and definition, as well as pattern matching.
%
Function application in \textsc{\tapp} ensures the location of the result of
the application is initially unwritten, and is considered written afterward.
%
Types and locations for the function are pulled from the function signature.
%
Pattern matching is handled by \textsc{\tcase} and \textsc{\tpat}, which
are straightforward.
%
The final rule type checks a whole program, consisting of datatype
and function definitions.

To simplify the formalism and proofs, I restricted typing rules
somewhat so that, in effect, the rules restrict well-typed expressions
so that they can return only the the result of a freshly allocated
constructor application.
%
Consequently, it is not possible, for instance, to type the following
expression, because the right-hand side is a value and, as such, does
not allocate.
%
\begin{code}
let x : @\tyatlocreg{T}{l}{r}@ = y in ...
\end{code}
%
This restriction is enforced by there being an assertion of the form
$\locreg{\loc}{\reg} \in \NENV$ in the premise of the typing rules of
the non-value expressions, such that $\tyatlocreg{\TYP}{\loc}{\reg}$
is the result type of the given expression.
%
Lifting this restriction is conceptually straightforward, but would
require either added complexity to the substitution lemma or the use
of a different factoring of the grammar and typing rules.
%
Similarly, our formalism and proofs could be extended to treat
primitive types, such as ints, bools, tuples, etc., as well as with
offsets and indirections in data constructors, with some conceptually
straightforward extensions to the formalism.


%% \setlength{\tabcolsep}{2pt}
%% \floatstyle{plaintop}\restylefloat{table}

\begin{table}[]
  \centering
%% \begin{adjustbox}{width=0.75\linewidth,center}

\begin{tabular}{llll}
  Code & $\AENV$ & $\CENV$ & $\NENV$ \\ \hline
  \begin{code}
letloc $\locreg{l}{r}$ =
    start($r$)
  \end{code} & $\{r \mapsto \locreg{l}{r} \}$ & $\emptyset$ & $\{\locreg{l}{r}\}$ \\
  \begin{code}
letloc $\locreg{l_a}{r}$ = $\locreg{l}{r}$ + 1
  \end{code} &
  $\{r \mapsto \locreg{l_a}{r} \}$ &
  $\{ \locreg{l_a}{r} \mapsto \locreg{l}{r} + 1 \}$  &
  $\{\locreg{l}{r},\locreg{l_a}{r}\}$ \\
  \begin{code}
let x : @\tyatlocreg{T}{l_a}{r}@ =
    Leaf $\locreg{l_a}{r}$ 1
  \end{code} &
  $\{ r \mapsto \locreg{l_a}{r} \}$ &
  $\{ l_a \mapsto \locreg{l}{r} + 1 \}$ &
  $\{\locreg{l}{r}\}$ \\
  \begin{code}
letloc $\locreg{l_b}{r}$ =
    after(@\tyatlocreg{T}{l_a}{r}@)
  \end{code} &
  $\{ r \mapsto \locreg{l_b}{r} \}$ &
  \makecell[cl]{$\{ \locreg{l_a}{r} \mapsto \locreg{l}{r} + 1,$\\\;$\locreg{l_b}{r} \mapsto \mathit{after}(\tyatlocreg{T}{l_a}{r})\}$} &
  $\{ \locreg{l}{r}, \locreg{l_b}{r} \}$ \\
  \begin{code}
let y : @\tyatlocreg{T}{l_b}{r}@ =
    Leaf $\locreg{l_b}{r}$ 2
  \end{code} &
  $\{ r \mapsto \locreg{l_b}{r} \}$ &
  \makecell[cl]{$\{ \locreg{l_a}{r} \mapsto l + 1,$\\\;$\locreg{l_b}{r} \mapsto \mathit{after}(\tyatlocreg{T}{l_a}{r})\}$} &
  $\{ \locreg{l}{r} \}$ \\
  \begin{code}
Node $\locreg{l}{r}$ x y
  \end{code} &
  $\{ r \mapsto \locreg{l}{r} \}$ &
  \makecell[cl]{$\{ \locreg{l_a}{r} \mapsto \locreg{l}{r} + 1,$\\\;$\locreg{l_b}{r} \mapsto \mathit{after}(\tyatlocreg{T}{l_a}{r})\}$} &
  $\emptyset$ \\
&&& \\
\end{tabular}
%% \end{adjustbox}
\caption{Step-by-step example of type checking a simple expression.}
\label{table:types-example}
\end{table}


\subsection{Dynamic Semantics}
\label{subsec:dynamic}
%\floatstyle{boxed}\restylefloat{figure}
\begin{figure}
  \begin{displaymath}
  \begin{aligned}
    \textup{Store} && \STOR && \gramdef & \set{\reg_1 \mapsto \heap_1 , \; \ldots \; , \reg_{n} \mapsto \heap_{n}} \\
    \textup{Heap} && \heap && \gramdef & \set{\ind_1 \mapsto \DC_1 , \; \ldots \; , \ind_{n} \mapsto \DC_{n}}\\
    \textup{Location Map} && \MENV && \gramdef & \set{\locreg{\loc}{\reg_1}_1 \mapsto \concreteloc{\reg_1}{\ind_1}{} , \; \ldots \; , \locreg{\loc}{\reg_n}_n \mapsto \concreteloc{\reg_n}{\ind_n}{}}
  \end{aligned}
\end{displaymath}

  \caption{Extended grammar of \ourcalc{} for dynamic semantics}
  \label{fig:opergram}
\end{figure}

\begin{figure}
  \small
  \begin{mathpar}
    \mprset{flushleft}
    \rddatacon{}

    \rdletlocstart{}\\
    \rdletloctag{}

    \rdletlocafter{}\\

    \rdletexp{}\\
    \rdletval{}

    \rdletregion{}\\
    \rdapp{}\\
    \rdcase{}\\
  \end{mathpar}
  \normalsize
  \caption{Dynamic semantics rules for \ourcalc{}}
  \label{fig:dynamic}
\end{figure}


The dynamic semantics for expressions in \ourcalc{} are given in
 \figref{fig:dynamic}, where the transition rule is as follows.
%
\begin{displaymath}
\STOR;\MENV;\EXPR \stepsto \STOR';\MENV';\EXPR'
\end{displaymath}
%
To model the behavior of reading and writing from an indexed
memory, the semantics make use of a \emph{store}, $\STOR$.
%
The store is a map from regions to \emph{heaps}, where each heap
consists of an array of \emph{cells}, which contain store values
(data constructor tags).
%
To bridge from symbolic to concrete locations, I use the
\emph{location map}, $\MENV$, to map symbolic locations
to concrete locations.
%

Case expressions are treated by the \textsc{\dcase{}} rule.
%
The objective of the rule is to load the tag of the constructor $\DC$
located at $\concreteloc{\reg}{\ind}{}$ in the store and dispatch
the corresponding case.
%
%% For simplicity, \ourcalc{} requires that all patterns are exhaustive,
%% and our type system ensures that the there always exists a constructor
%% tag at the specified location at the expected type.
%
The expression produced by the right-hand side of the rule is the body
of the pattern, in which all pattern-bound variables are replaced by
the concrete locations of the fields of the constructor $\DC$.

The concrete locations of the fields are obtained by the following
process.
%
If there is at least one field, then its starting address is the
position one cell after the constructor tag.
%
The starting addresses of subsequent fields depend on the sizes of the
trees stored in previous fields.

A feature of \ourcalc{} is the flexibility it provides to pick the
serialization layout.
%
Our formalism uses our \emph{end-witness rule} to abstract from different layout
decisions (for a more thorough explanation, see \secref{subsec:typesafety}).
%
Given a type $\TYP$, a starting address
$\concreteloc{\reg}{\ind_{s}}{}$, and store $\STOR$, the rule below
asserts that address of the end witness is
$\concreteloc{\reg}{\ind_{e}}{}$.
%
\begin{displaymath}
  \ewitness{\TYP}{\concreteloc{\reg}{\ind_{s}}{}}{\STOR}{\concreteloc{\reg}{\ind_{e}}{}}
\end{displaymath}
%
Using this rule, the starting address of the second field is obtained
from the end witness of the first, the starting address of the
third from the end witness of the second, and so on.

The allocation and finalization of a new constructor is achieved by
some sequence of transitions, starting with the \textsc{\dletloctag{}} rule, then
involving some number of transitions of the \textsc{\dletlocafter{}} rule,
depending on the number of fields of the constructor, and finally
ending with the \textsc{\ddatacon{}} transition.
%
The \textsc{\dletloctag{}} rule allocates one cell for the tag of some new
constructor of a yet-to-be determined type, leaving it to later to
write to the new location.
%
The resulting configuration binds its $\loc$ to the address
$\concreteloc{\reg}{\ind+1}{}$, that is, the address one cell past
given location $\loc'$ at $\concreteloc{\reg}{\ind}{}$.
%
Fields that occur after the first are allocated by the
\textsc{\dletlocafter{}} rule.
%
Here, its $\loc$ is bound to the address \concreteloc{\reg}{j}{} one
past the last cell of the constructor represented by its given
symbolic location $\loc_1$.
%
Like the \textsc{\dcase{}} rule, the required address is obtained by
application of end-witness rule to the starting address of the given
$\loc_1$ at the type of the corresponding field $\TYP$.
%
The final step in creating a new data constructor instance
is the \textsc{\ddatacon{}} rule.
%
It writes the specified constructor tag $\DC$ at the address in the
store represented by the symbolic location $\loc$.

The \textsc{\dletlocstart{}} rule for the \sgramwd{letloc} with \sgramwd{(start r)}.
expression binds the location to the starting address in the region
and starts running the body.

The \textsc{\dletexp{}} rule for let-expressions evaluates the let-bound
expression to a value and the \textsc{\dletval{}} rule substitutes the value for
the let-bound variable in the body.
%
The \textsc{\dapp{}} rule for function applications looks up the function by name
in the top-level environment and substitutes arguments for parameters
in the function body, substitutes argument symbolic locations for
parameter symbolic locations, then starts the resulting function body
running.
%
The \textsc{\dletregion{}} rule for the \sgramwd{letregion} expression binds the
new region and starts running the body.
%

The driver which runs an \ourcalc{} program initially loads all data
types, functions, type checks them, and if successful, then seeds the
$Function$, $\typeofcon$, and $\typeoffield$ environments.
%
Let $\EXPR_0$ be the main expression.
%
If $\EXPR_0$ type checks with respect to the \textsc{\tprogram{}} rule, then
the main program is safe to run.
%
The initial configuration for the machine with an empty store is
\begin{displaymath}
\emptyset; \set{\loc \mapsto \concreteloc{\reg}{0}{}}; \EXPR_0,
\end{displaymath}
which is, by itself, not particularly interesting or useful.
%
It is, however, straightforward to construct a type-safe initial configuration
whose store is nonempty, as long as the initial configuration
has a store that is well formed, as described in \Secref{sec:well-formedness}.
%
The program can start taking evaluation steps from this configuration.

\paragraph{Example} %: Allocating a Binary Tree.}
%
Consider this code snippet of \ourcalc{}.
%
\begin{code}
letloc @$\locreg{l_1}{\reg}$@ = @$\locreg{l_0}{\reg}$@ + 1 in
let a : @\tyatlocreg{Tree}{l_1}{r}@ = (Leaf @$\locreg{l_1}{\reg}$@) in
letloc @$\locreg{l_2}{\reg}$@ = (after (@\tyatlocreg{Tree}{l_1}{r}@)) in
let b : @\tyatlocreg{Tree}{l_2}{r}@ = (Leaf @$\locreg{l_2}{\reg}$@) in
Node @$\locreg{l_0}{\reg}$@ a b
\end{code}
%
Assume that the store starts out with a fresh heap, $\STOR = \set{\reg
  \mapsto \emptyset}$ and the location $\locreg{l_0}{\reg}$ maps to
$\concreteloc{\reg}{0}{}$ in the location map.
%
After stepping past the first line, the \textsc{\dletloctag{}} step has
allocated a cell for the tag of the interior node and bound the
location $\locreg{\loc_1}{\reg}$ to $\concreteloc{\reg}{1}{}$.
%
After the next line, the \textsc{\ddatacon{}} transition writes a leaf node to
the store at the address represented by $\locreg{\loc_1}{\reg}$:
$\STOR = \set{\reg \mapsto \set{1 \mapsto \mathtt{Leaf}}}$.
%
The second \gramwd{letloc} obtains the starting address for the second
leaf node by using end witness of the previous leaf node.
%
The write of the second leaf node appears in the store after
the next line, leaving the following store:
$
\STOR = \set{\reg \mapsto \set{1 \mapsto \mathtt{Leaf}, 2 \mapsto \mathtt{Leaf}}}$.
%
Finally, after the \textsc{\ddatacon{}} step taken for the last line, the store
contains the finalized allocation:
$\STOR = \set{\reg \mapsto \set{0 \mapsto \mathtt{Node}, 1 \mapsto \mathtt{Leaf}, 2 \mapsto \mathtt{Leaf}}}$.
%

The end-witness judgement of the new data constructor is the
following:
$\ewitness{\mathtt{Tree}}{\concreteloc{\reg}{0}{}}{\STOR}{\concreteloc{\reg}{3}{}}$
%
The judgement applies, in part, because, as expected, the tag at the
address $\concreteloc{\reg}{0}{}$ is a tag of type $\mathtt{Tree}$.
%
In addition, because the tag indicates an interior node with two
subtrees for fields, the judgement obligation extends to recursively
showing (1) that the end witness of the first leaf node (also at type
$\mathtt{Tree}$) at $\concreteloc{\reg}{1}{}$ has an end witness
(which is $\concreteloc{\reg}{2}{}$), (2) that the second field has an
end witness starting at the end witness of the first field, namely
$\concreteloc{\reg}{2}{}$, and ending at some higher address (which in
this case is $\concreteloc{\reg}{3}{}$), and (3) finally that the end
witness of the second field is the end witness of the entire
constructor, as is the case here.

\subsection{Type Safety}
\label{subsec:typesafety}

\mv{Overview of type safety of LoCal and why we should care.}
% Full type safety proof is give in the appendix of \cite{local-extended}.

\subsubsection{Variables and substitution}
I use the convention that all variables for binding values,
locations, and regions are distinct, and maintain this invariant
implicitly.
%
The bindings sites of variables are summarized by the following:
%
\begin{itemize}
\item Variables for binding values $\var$ are bound by function definitions
$\FD$ and pattern matches $\pat$.

\item Location variables $\locreg{\loc}{\reg}$ are bound by type schemes $\TS$, pattern matches
$\pat$, and $\keywd{letloc}$ binders.

\item Region variables $\reg$ are bound by type schemes $\TS$, pattern
matches $\pat$, and $\keywd{letregion}$ binders.
\end{itemize}
%
The use sites of variables are summarized by the following:
\begin{itemize}
\item Variables for binding varlues $\var$ are used by values $\VAL$.

\item Location variables $\locreg{\loc}{\reg}$ are used by concrete locations $\concreteloc{\reg}{\ind}{\locreg{\loc}{\reg}}$,
the argument list of function applications
$\fapp{\overharpoon{\locreg{l}{r}}}{\overharpoon{\VAL}}$, the location
argument of constructor applications
$\datacon{\DC}{\keywd{\locreg{\loc}{\reg}}}{\overharpoon{\VAL}}$,
located types $\hTYP$, and located expressions $\LE$.

\item Region varaibles $\reg$ are used in the same places as location variables.
\end{itemize}
%
We use the following conventions for variable substitution:
%
\begin{itemize}
\item $\subst{\EXPR}{x}{v}$: Substitute $v$ for $x$ in $e$. We let the notation extend to vectors such that
$\subst{\EXPR}{\overharpoon{x}}{\overharpoon{v}}$ denotes the iterated substitution $\subst{\EXPR}{\overharpoon{x_1}}{\overharpoon{v_1}} \ldots \subst{}{\overharpoon{x_n}}{\overharpoon{v_n}}$, where $n = |\overharpoon{x}| = |\overharpoon{v}|$.

\item $\subst{\EXPR}{\locreg{\loc_1}{\reg_1}}{\locreg{\loc_2}{\reg_2}}$: Substitute location variable $\locreg{\loc_2}{\reg_2}$ for $\locreg{\loc_1}{\reg_1}$ in $\EXPR$. We extend this notation to vectors of locations in the same fashion, as described above.

\item $\subst{\EXPR}{\reg_1}{\reg_2}$ : Substitute region variable $\reg_2$ for $\reg_1$ in $\EXPR$. We extend this notation to vectors of locations in the same fashion, as described above.

\item Finally, we extend the aforementioned notation so that
  substitution can act on environments $\CENV$, $\AENV$, and $\NENV$,
  e.g.,
  $\subst{\CENV}{\locreg{\loc_1}{\reg_1}}{\locreg{\loc_2}{\reg_2}}$.
\end{itemize}

% \subsubsection{Global environment and metafunctions}
% \begin{itemize}
% \item $Function(f)$: An environment that maps a function $f$ to its definition $\FD$.

% \item $Freshen(\FD)$: A metafunction that freshens all bound variables in function definition
% $\FD$ and returns the resulting function definition.

% \item $TypeOfCon(\DC):$ An environment that maps a data constructor to its type.

% \item $TypeOfField(\DC,i)$: A metafunction that returns the type of the \il{i}'th field
% of data constructor $\DC$.

% \item $ArgTysOfConstructor(\DC)$: An environment that maps a data constructor to its field types.

% \item $\allocptr{\reg}{\STOR}$: $\max \set{-1} \cup \set{ \indj \; | \; {(\reg \mapsto (\indj \mapsto \DC)) \in \STOR}}$.
% \end{itemize}


\subsubsection{Store typing}
\label{sec:well-formedness}
A key part of the safety of \ourcalc{} programs is the following property: if a
term $e$ is type $\tyatlocreg{\TYP}{\loc}{\reg}$, then if we look in the store
under region $\reg$ at the location represented by $\loc$, we will find the
start of a serialized data structure which is a valid serialization of a value
of type $\TYP$. In other words, the types tell us the truth about the values in
the store.
%
To achieve this, we rely on the store being \emph{well-formed}, whose definition
itself uses three other judgements (shown in Table~\ref{tbl:swf-judgements}).
%
% \begin{displaymath}
%   \storewf{\SENV}{\CENV}{\AENV}{\NENV}{\MENV}{\STOR}
% \end{displaymath}

\begin{table}
\bgroup
% \def\arraystretch{1.2}
% \setlength\tabcolsep{0.5cm}
\begin{tabular}{llp{8cm}}
 & \textbf{Judgement form} & \textbf{Summary}
 \\\\
\parbox[t]{3.5cm}{Store \\ well formedness} & $\storewf{\SENV}{\CENV}{\AENV}{\NENV}{\MENV}{\STOR}$ &

The store $\STOR$ along with location map $\MENV$ are well formed with respect to
typing environments $\SENV$, $\CENV$, and $\AENV$.
\\\\
End witness & $\ewitness{\TYP}{\concreteloc{\reg}{\ind_{s}}{}}{\STOR}{\concreteloc{\reg}{\ind_{e}}{}}$ &

The store address $\concreteloc{\reg}{\ind_{e}}{}$ is the position one
after the last cell of the tree of type $\TYP$ starting at
$\concreteloc{\reg}{\ind_{s}}{}$ in store $\STOR$.
\\\\
\parbox[t]{3.5cm}{Constructor-application \\ well formedness}
 & $\storewfcfa{\CENV}{\MENV}{\STOR}$ &

All in-flight data-constructor applications in store $\STOR$ along with location map $\MENV$
are well formed with respect to constructor-progress typing environment $\CENV$.
\\\\
\parbox[t]{3.5cm}{Allocation \\ well formedness} & $\storewfca{\AENV}{\NENV}{\MENV}{\STOR}$ &

Allocation in store $\STOR$ along with location map $\MENV$ is well formed
with respect to allocation-typing environments $\AENV$ and $\NENV$.
\end{tabular}
\egroup
\caption{Summary of judgements used to establish well formedness of the store.}
\label{tbl:swf-judgements}
\end{table}


% The store typing specifies three categories of invariants.
% %
% \begin{enumerate}
% \item The first enforces that allocations occur in the sequence
% specified by the constraint environment $\CENV$.
% %
% In particular, if there is some location $\loc$ in the domain of
% $\CENV$, then the location map and store have the expected
% allocations at the expected types.
% %
% For instance, if $(\loc \mapsto
% \afterl{\tyatlocreg{\TYP}{\loc'}{\reg}}) \in \CENV$, then $\loc'$ maps
% to $\concreteloc{\reg}{\ind_1}{}$ and $\loc$ to
% $\concreteloc{\reg}{\ind_2}{}$ in the location map, and $\ind_2$ is
% the end witness of $\ind_1$ at type $\TYP$ in the store, at region
% $\reg$.
% %
% \item The second category enforces that, for each symbolic location such
% that $(\loc \mapsto \TYP) \in \SENV$, there is some
% $\concreteloc{\reg}{\ind_1}{}$ for $\loc$ in the location map and
% $\ind_1$ has some end witness $\ind_2$ at type $\TYP$.
% %
% \item The final category enforces that each address in the store is written
% once.
% %
% This property is asserted by insisting that, if $\loc \in \NENV$, then
% there is some $\concreteloc{\reg}{\ind}{}$ for $\loc$ in the location
% map, but there is no write to for $\ind$ at $\reg$ in the store.
% %
% To support this property, there are two additional conditions which
% require that the most recently allocated location (tracked by
% to $\AENV, \NENV$) is at the end of its respective region.
% \end{enumerate}

% To support this behavior, there are two additional conditions in this
% category that enforce linearity of allocation.
% %
% The first one captures the situation when there is an allocation in
% flight, such as is the case in the postcondition of allocating
% a tag (and the corresponding binding the cell to $\loc$).
% %
% In this case, the typing rule specifies that $(\reg \mapsto \loc) \in
% A$ and $\loc \in \NENV$.
% %
% The requirement is that the address bound to $\loc$ is the maximum
% position in the location map for the region $\reg$ and that this
% position is past the end of any address already written in the store
% at $\reg$.
% The second condition captures the other situation, just after
% a field has been allocated and finalized, which is occurs
% when $(\reg \mapsto \loc) \in A$ and $(\loc \mapsto \TYP) \in \SENV$
% for the location $\loc$ representing the field.
% %
% In this case, the condition requires that the address of the end
% witness of the field is the maximum of all other addresses in the
% store at region $\reg$.



\paragraph{Notation for references to well-formedness judgements}
Because there are many requirements specified inside the various
well-formedness judgements, we introduce notation for referring
to requirements individually.
%
For example, the notation
\refwellformed{sec:well-formedness-allocation}{wf:impl-linear-alloc2}
refers to the judgement
\begin{align*}
\storewfca{\AENV}{\NENV}{\MENV}{\STOR},
\end{align*}
specified in Section~\ref{sec:well-formedness-allocation},
and in that judgement, rule number~\ref{wf:impl-linear-alloc2}.

The definition of store well formedness follows.

\paragraph{Judgement form}

$\storewf{\SENV}{\CENV}{\AENV}{\NENV}{\MENV}{\STOR}$

The well-formedness judgement specifies the valid layouts of the store by using the location
map and the various environments from the typing judgement.
%
Rule~\ref{wf:map-store-consistency} specifies that, for each location in the store-typing environment,
there is a corresponding concrete location in the location map and that concrete location satisfies
a corresponding end-witness judgement.
%
Rules~\ref{wf:cfc} and~\ref{wf:ca} reference the judgements for well formedness concerning
in-flight constructor applications (\secref{sec:end-witness}) and correct allocation in
regions (\secref{sec:well-formedness-allocation}), respectively.
%
Finally, Rule~\ref{wf:impl1} specifies that the nursery and store-typing environments reference
no common locations, which is a way of reflecting that each location is either in the process
of being constructed and in the nursery, or allocated and in the store-typing environment, but
never both.

\paragraph{Definition}

\begin{enumerate}

    \item \label{wf:map-store-consistency} $ (\locreg{\loc}{\reg} \mapsto \TYP) \in \SENV \Rightarrow \\
            ((\locreg{\loc}{\reg} \mapsto \concreteloc{\reg}{\ind_1}{}) \in \MENV \wedge \\
            \ewitness{\TYP}{\concreteloc{\reg}{\ind_1}{}}{\STOR}{\concreteloc{\reg}{\ind_2}{}})
          $

    \item \label{wf:cfc} $\storewfcfa{\CENV}{\MENV}{\STOR}$

    \item \label{wf:ca} $\storewfca{\AENV}{\NENV}{\MENV}{\STOR}$

    \item \label{wf:impl1} $dom(\SENV) \cap \NENV = \emptyset $
\end{enumerate}


\subsubsection{End-Witness judgement}
\label{sec:end-witness}

\paragraph{Judgement form}

$\ewitness{\TYP}{\concreteloc{\reg}{\ind_{s}}{}}{\STOR}{\concreteloc{\reg}{\ind_{e}}{}}$

The end-witness judgement specifies the expected layout in the store of a fully
allocated data constructor.
%
Rule~\ref{ewitness:impl1} requires that the first cell store a constructor
tag of the appropriate type.
%
Rule~\ref{ewitness:impl2} specifies the address of the cell one past the tag.
%
Rule~\ref{ewitness:impl3} recursively specifies the positions of the constructor
fields.
%
Finally, Rule~\ref{ewitness:impl4} specifies that the end witness of
the overall constructor is the address one past the end of either the
tag, if the constructor has zero fields, or the final field,
otherwise.

\paragraph{Definition}

\begin{enumerate}
\item \label{ewitness:impl1} $\STOR(\reg)(\ind_s) = \DC'$ \; \text{such that} \\
      $\; \DATA\;\TYP = \overharpoon{\DC_1 \; \overharpoon{\sTYP}_1\;} \; | \; \ldots \; | \; \DC' \; \overharpoon{\sTYP}' \; | \; \ldots \; | \; \overharpoon{\DC_m \; \overharpoon{\sTYP}_m\;}$

\item \label{ewitness:impl4} $\overharpoon{w_0} = \ind_s + 1$

\item \label{ewitness:impl2}
  $\ewitness{\overharpoon{\TYP'_1}}{\concreteloc{\reg}{\overharpoon{w_0}}{}}{\STOR}{\concreteloc{\reg}{\overharpoon{w_1}}{}} \wedge$ \\
  $\ewitness{\overharpoon{\TYP'_{j+1}}}{\concreteloc{\reg}{\overharpoon{w_j}}{}}{\STOR}{\concreteloc{\reg}{\overharpoon{w_{j+1}}}{}}$
  \\ where $\indj \in \set{1,\ldots,n-1} ; n = | \overharpoon{\TYP'} |$

\item \label{ewitness:impl3}
  $\ind_e = \overharpoon{w_n}$
\end{enumerate}

\subsubsection{Well-formedness of constructor application}
\label{sec:well-formedness-constructors}

\paragraph{Judgement form}

$\storewfcfa{\CENV}{\MENV}{\STOR}$

The well-formedness judgement for constructor application specifies the various constraints
that are necessary for ensuring correct formation of constructors, dealing with constructor
application being an incremental process that spans multiple \ourcalc{} instructions.
%
Rule~\ref{wfconstr:constraint-start} specifies that, if a location corresponding to the
first address in a region is in the constraint environment, then there is a corresponding
entry for that location in the location map.
%
Rule~\ref{wfconstr:constraint-tag} specifies that, if a location corresponding to the address one past a constructor
tag is in the constraint environment, then there are corresponding locations for the address
of the tag and the address after in the location map.
%
Rule~\ref{wfconstr:constraint-after} specifies that, if a location corresponding to the address
one past after a fully allocated constructor application is in the constraint environment,
then there are corresponding locations for the address one past the constructor application
and for the address of the start of that constructor application in the location map, as well as the existence
of an end witness for that fully allocated location.

\paragraph{Definition}

\begin{enumerate}
    \item \label{wfconstr:constraint-start} $ (\locreg{\loc}{\reg} \mapsto \startr{\reg}) \in \CENV \Rightarrow \\
            (\locreg{\loc}{\reg} \mapsto \concreteloc{\reg}{0}{}) \in \MENV $

    \item \label{wfconstr:constraint-tag} $ (\locreg{\loc}{\reg} \mapsto (\locreg{\loc'}{\reg} + 1)) \in \CENV \Rightarrow
            \\
            (\locreg{\loc'}{\reg} \mapsto \concreteloc{\reg}{\ind_l}{})  \in \MENV \wedge \\
            (\locreg{\loc}{\reg} \mapsto \concreteloc{\reg}{\ind_l + 1}{})  \in \MENV
            $

    \item \label{wfconstr:constraint-after} $ (\locreg{\loc}{\reg} \mapsto \afterl{\tyatlocreg{\TYP}{\locreg{\loc'}{\reg}}{\reg}}) \in \CENV \Rightarrow \\
            ((\locreg{\loc'}{\reg} \mapsto \concreteloc{\reg}{\ind_1}{}) \in \MENV \wedge \\
            \ewitness{\TYP}{\concreteloc{\reg}{\ind_1}{}}{\STOR}{\concreteloc{\reg}{\ind_2}{}} \wedge \\
            (\locreg{\loc}{\reg} \mapsto \concreteloc{\reg}{\ind_2}{}) \in \MENV)
            $

\end{enumerate}


\subsubsection{Well-formedness concerning allocation}
\label{sec:well-formedness-allocation}

\paragraph{Judgement form}

$\storewfca{\AENV}{\NENV}{\MENV}{\STOR}$

The well-formedness judgement for safe allocation specifies the various properties
of the location map and store that enable continued safe allocation, avoiding in particular
overwriting cells, which could, if possible, invalidate overall type safety.
%
Rule~\ref{wf:impl-linear-alloc} requires that, if a location is in both the allocation
and nursery environments, i.e., that address represents an in-flight
constructor application, then there is a corresponding location in the location
map and the address of that location is the highest address in the store.
%
Rule~\ref{wf:impl-linear-alloc2} requires that, if there is an address in the allocation
environment and that address is fully allocated, then the address of that location is the
highest address in the store.
%
Rule~\ref{wf:impl-write-once} requires that, if there is an address in the nursery, then
there is a corresponding location in the location map, but nothing at the corresponding
address in the store.
%
Finally, Rule~\ref{wf:impl-empty-region} requires that, if there is a region that has been
created but for which nothing has yet been allocated, then there can be no addresses
for that region in the store.

\paragraph{Definition}

\begin{enumerate}
    \item \label{wf:impl-linear-alloc} $ ((\reg \mapsto \locreg{\loc}{\reg}) \in \AENV \wedge \locreg{\loc}{\reg} \in \NENV) \Rightarrow \\
          ((\locreg{\loc}{\reg} \mapsto \concreteloc{\reg}{\ind}{}) \in \MENV \wedge
%          \ind = \max \set{0} \cup \set{\indj \; | \; \concreteloc{\reg}{\indj}{} \in \MENV} \wedge \\
          \ind > \allocptr{\reg}{\STOR})
          $

    \item \label{wf:impl-linear-alloc2} $ ((\reg \mapsto \locreg{\loc}{\reg}) \in \AENV \wedge
    \, (\locreg{\loc}{\reg} \mapsto \concreteloc{\reg}{\ind_s}{}) \in \MENV \wedge \locreg{\loc}{\reg} \not \in \NENV \wedge
    \, \ewitness{\TYP}{\concreteloc{\reg}{\ind_s}{}}{\STOR}{\concreteloc{\reg}{\ind_e}{}}) \Rightarrow \\
          \ind_e > \allocptr{\reg}{\STOR}
          $

    \item \label{wf:impl-write-once} $ \locreg{\loc}{\reg} \in \NENV \Rightarrow \\
          ((\locreg{\loc}{\reg} \mapsto \concreteloc{\reg}{\ind}{}) \in \MENV \wedge \\
          (\reg \mapsto (\ind \mapsto \DC)) \not \in \STOR)
          $

    \item \label{wf:impl-empty-region} $(\reg \mapsto \emptyset) \in \AENV \Rightarrow \\
    \reg \not \in dom(\STOR)$
\end{enumerate}



\subsubsection{Type safety theorem}

\newcommand{\substlemmasubsts}[1]{\subst{#1}{\overharpoon{\var}}{\overharpoon{\VAL}} \subst{}{\overharpoon{\locreg{\loc}{\reg}}}{\overharpoon{\locreg{\loc'}{\reg'}}} \subst{}{\locreg{\loc}{\reg}}{\locreg{\loc'}{\reg'}}}


% \begin{lemma}[Substitution lemma]
%   \label{lemma:substitution}
%   \begin{align*}
%   \text{If} \quad & \TENV \cup \set{\overharpoon{\var_1} \mapsto \overharpoon{\TYP_1} \ensuremath{@} \overharpoon{\locreg{\loc_1}{\reg_1}}, \ldots, \overharpoon{\var_n} \mapsto  \overharpoon{\TYP_n} \ensuremath{@} \overharpoon{\locreg{\loc_n}{\reg_n}}}; \SENV; \CENV; \AENV; \NENV \vdash \AENV'; \NENV'; \EXPR : \tyatlocreg{\TYP}{\loc}{\reg}\\
%   \text{and} \quad & \TENV; \SENV'; \CENV'; \AENV'; \NENV' \vdash \AENV'; \NENV'; \overharpoon{\VAL_i} : \overharpoon{\TYP_i} \ensuremath{@} \overharpoon{\locreg{\loc'_i}{\reg'_i}} \qquad \; i \in \set{1, \ldots, n}\\
%   \text{then} \quad & \TENV; \SENV'; \CENV'; \AENV'; \NENV' \vdash \AENV'''; \NENV'''; \substlemmasubsts{\EXPR} : \tyatlocreg{\TYP}{\loc'}{\reg'}\\
%    \text{where} \quad & \SENV = \SENV_0 \cup \set{\overharpoon{\locreg{\loc_1}{\reg_1}} \mapsto \overharpoon{\TYP_1}, \ldots, \overharpoon{\locreg{\loc_n}{\reg_n}} \mapsto \overharpoon{\TYP_n}}\\
%    \text{and} \quad & \forall_{(\var \mapsto \TYP'' \ensuremath{@} \locreg{\loc''}{\reg''}) \in \TENV} . (\locreg{\loc''}{\reg''} \mapsto \TYP'') \in \SENV_0 \\
%   \text{and} \quad & dom(\SENV) \cap \NENV = \emptyset\\
%   \text{and} \quad & \NENV = \NENV_0 \cup \locreg{\loc}{\reg}\\
%   \text{and} \quad & \SENV' = \SENV \cup \set{\overharpoon{\locreg{\loc'_1}{\reg'_1}} \mapsto \overharpoon{\TYP_1}, \ldots, \overharpoon{\locreg{\loc'_n}{\reg'_n}} \mapsto \overharpoon{\TYP_n}}\\
%   \text{and} \quad & \CENV' = \subst{\CENV}{\overharpoon{\locreg{\loc}{\reg}}}{\overharpoon{\locreg{\loc'}{\reg'}}} \subst{}{\locreg{\loc}{\reg}}{\locreg{\loc'}{\reg'}} \\
%   \text{and} \quad & \AENV' = \subst{\AENV}{\overharpoon{\locreg{\loc}{\reg}}}{\overharpoon{\locreg{\loc'}{\reg'}}} \subst{}{\locreg{\loc}{\reg}}{\locreg{\loc'}{\reg'}} \subst{}{\reg}{\reg'}\\
%   \text{and} \quad & \NENV' = \subst{\NENV}{\locreg{\loc}{\reg}}{\locreg{\loc'}{\reg'}}
%   \end{align*}
% \end{lemma}
% \begin{nproof}
% The proof is by rule induction on the given typing derivation.
% \end{nproof}

\begin{lemma}[Progress]
  \label{lemma:progress}
  \begin{displaymath}
  \begin{aligned}
  \text{if} \;\; & \emptyset;\SENV;\CENV;\AENV;\NENV \vdash \AENV';\NENV';\EXPR : \hTYP \\
  \text{and} \;\; & \storewf{\SENV}{\CENV}{\AENV}{\NENV}{\MENV}{\STOR} \\
  \text{then} \;\; & \EXPR \; \mathit{value} \\
  \text{else} \;\; & \STOR;\MENV;\EXPR \stepsto \STOR';\MENV';\EXPR'
  \end{aligned}
  \end{displaymath}
\end{lemma}
\begin{nproof}
The proof is by rule induction on the given typing derivation.
\end{nproof}

\begin{lemma}[Preservation]
  \label{lemma:preservation}
  \begin{displaymath}
    \begin{aligned}
      \text{If} \;\; & \emptytenv;\SENV;\CENV;\AENV;\NENV \vdash \AENV';\NENV';\EXPR : \hTYP \\
      \text{and} \;\; & \storewf{\SENV}{\CENV}{\AENV}{\NENV}{\MENV}{\STOR}\\
      \text{and} \;\; & \STOR;\MENV;\EXPR \stepsto \STOR';\MENV';\EXPR' \\
      \text{then for some} \;\; & \SENV' \supseteq \SENV, \CENV' \supseteq \CENV ,\\
      & \emptytenv;\SENV';\CENV';\AENV';\NENV' \vdash \AENV'';\NENV'';\EXPR' : \hTYP \\
      \text{and} \;\; & \storewf{\SENV'}{\CENV'}{\AENV'}{\NENV'}{\MENV'}{\STOR'}\\
    \end{aligned}
  \end{displaymath}
\end{lemma}
\begin{nproof}
The proof is by rule induction on the given derivation of the dynamic semantics.
\end{nproof}


The type safety of \ourcalc{} follows from the following result.

\begin{theorem}[Type safety]
  \label{theorem:type-safety}
\begin{displaymath}
  \begin{aligned}
  \text{If} \;\; & (\emptyset;\SENV;\CENV;\AENV;\NENV \vdash \AENV';\NENV';\EXPR : \hTYP) \wedge
                   (\storewf{\SENV}{\CENV}{\AENV}{\NENV}{\MENV}{\STOR}) \\
  \text{and} \;\; & \STOR;\MENV;\EXPR \stepsto^n \STOR';\MENV';\EXPR' \\
  \text{then} \;\; & (\EXPR' \; \mathit{value}) \vee
                     (\exists \STOR'', \MENV'', \EXPR'' . \; \STOR';\MENV';\EXPR' \stepsto \STOR'';\MENV'';\EXPR'')
  \end{aligned}
  \end{displaymath}
\end{theorem}

\begin{nproof}
  The type safety follows from an induction with
  the progress and preservation lemmas.
\end{nproof}

\mv{Show some cases of the proof as examples!}


\section{Extensions}\label{sec:extensions}


\subsection{Offsets and Indirections}\label{subsec:indirections}

As motivated in \secref{sec:bg}, it is sometimes desirable to be
able to ``jump over'' part of a serialized tree.
%
As presented so far, \ourcalc{} makes use of an end witness judgment
to determine the end of a particular data structure in memory.
%
The simplest computational interpretation of this technique is,
however, a linear scan through the store.
%
Luckily, extending the language to account for storing and making use
of \emph{offset} information for certain datatypes is
straightforward, and does not add conceptual difficulty to
neither the formalism nor type-safety proof.

Such an extension may use annotations on datatype declarations
that identify which fields of a given constructor are provided \emph{offsets}
and to permit cells in the store to hold offset values.
%
Because the offsets of a given constructor are known from its type,
the \textsc{\dletloctag{}} rule can allocate space for offsets when it
allocates space for the tag.
%
It is straightforward to fill the offset fields because \textsc{\ddatacon{}}
rule already has in hand the required offsets, which are provided in
the arguments of the constructor.
%
Finally, the \textsc{\dcase{}} rule can use offsets instead of
the end-witness rule.

\emph{Indirections} permit fields of data constructors to point across
regions, and thus require adding an annotation form (e.g., an
annotation on the type of a constructor field to indicate an
indirection) and extending the store to hold pointers.
%
Fortunately, as discussed later, regions in \ourcalc{} are never
collected; they are garbage collected in our implementation.
% and thus the extension adds conceptual difficulty to
% neither the formalism nor proof.
%
Every time an indirection field is constructed, space for the pointer
is allocated using a transition rule similar to the \textsc{\dletloctag{}}
rule.
%
The \textsc{\ddatacon{}} rule receives the address of the indirection in the
argument list, just like any other location and writes the indirection
pointer to the address of the destination field.
%

To type check, the type system extends with two new typing rules and a
new constraint form to indicate indirections.
%
To maintain type safety in the presence of offsets and indirections,
the store typing rule needs to be extended to include them.
%
Because the programmer is not manually managing the creation or use of offsets
or indirections (they are below the level of abstraction, indicated by
annotating the datatype, but not changing the code), the store-typing rule
generalizes straightforwardly and the changes preserve type safety.
%\rn{Still might be good to show some kind of concrete example of these annotations.}

In datatype annotations each field can be marked to store its offset in the constructor {\em
  or} be represented by an indirection pointer (currently not both):
\begin{code}
data T = K1 T (Ind T) | K2 T (Offset T) | K3 T
\end{code}
Type annotations would also be the place to express {\em permutations} on fields
that should be serialized in a different order, (e.g., postorder).  But it is
equivalent to generating \ourcalc{} with reordered fields in the source program.

\subsection{Parallelism}\label{subsec:parallelism}

So far, this section has presented a language for performing tree traversals
over serialized data (with some potential indirection). While this data representation
strategy works well for sequential programs, there is an intrinsic tension if we
want to parallelize these tree traversals. As the name implies, efficiently
serialized data must often be read serially. To change that, first, enough
\emph{indexing} data must be left in the representation in order for parallel tasks to
``skip ahead'' and process multiple subtrees in parallel. Second, the allocation
areas must be bifurcated to allow allocation of outputs in parallel.

% Here I will outline how \ourcalc{} can be extended to support parallel
% computation. In this strategy, form follows function: data representation is
% random-access only insofar as parallelism is needed, and both data
% representation and control flow ``bottom out'' to sequential pieces of work.
% That is, granularity-control in the data mirrors traditional granularity-control
% in parallel task scheduling.

% We implemented this in the Gibbon compiler.
% % and found that it performed as well or better than parallel GHC and parallel
% % MLton on a variety of benchmarks. When utilizing 18 cores, our geomean speedup
% % is $1.87\times$ and $3.16\times$ over parallel MLton and GHC, respectively.
% A thorough evaluation of the performance of parallel \ourcalc{} is
% in~\secref{subsec:parallelbench}.

There are several opportunities for parallelism in \ourcalc{} programs.
%
The first kind of parallelism is available when \ourcalc{} programs access
the store in a read-only fashion, such as the program that calculates
the size of a binary tree (as shown in \Figref{fig:sizefunc}).
\begin{figure}
\begin{code}
size : forall @\locreg{l}{r}@ . @\tyatlocreg{Tree}{l}{r}@ -> Int
size [@\locreg{l}{r}@] t = case t of
              Leaf -> 1
              Node (a : @\tyatlocreg{Tree}{l_a}{r}@) (b : @\tyatlocreg{Tree}{l_b}{r}@)
               -> (size [@\locreg{l_a}{r}@] a) + (size [@\locreg{l_b}{r}@] b)
\end{code}
\caption{\ourcalc{} function computing the size of a binary tree}
\label{fig:sizefunc}
\end{figure}
%
However, even though the recursive calls in the \il{Node} case can
safely evaluate in parallel, there is a subtelty: parallel evaluation
is efficient only if the \il{Node} constructor stores offset
information for its child nodes.
%
If it does, then the address of \il{b} can be calculated in constant
time, thereby allowing the calls to proceed immediately in parallel.
%
If there is no offset information, then the overall tree traversal is
necessarily sequential, because the starting address of \il{b} can be
obtained only after a full traversal of \il{a}.
%
As such, there is a tradeoff between space and time, that is, the cost
of the space to store the offset in the \il{Node} objects versus the
time of the sequential traversal (e.g., of \il{a}) forced by the
absence of offsets.

Programs that write to the store also provide opportunities for
parallelism.
%
The most immediate such opportunity exists when the program performs
writes that affect different regions.
%
For example, the writes to construct the leaf nodes for \il{a} and
\il{b} can happen in parallel because different regions cannot overlap
in memory.
%
\begin{code}
letregion @ra@ in
letregion @rb@ in
letloc @\locreg{la}{ra}@ = start @ra@ in
letloc @\locreg{lb}{rb}@ = start @rb@ in
let a : @\tyatlocreg{Tree}{la}{ra}@ = Leaf @\locreg{la}{ra}@ in
let b : @\tyatlocreg{Tree}{lb}{ra}@ = Leaf @\locreg{lb}{rb}@ in
@\ldots@
\end{code}
%
There is another kind of parallelism that is more challenging
to exploit, but is at least as important as the others:
%
the parallelism that can be realized by allowing different fields
of the same constructor to be filled in parallel.
%
This is crucial in \ourcalc{} programs, where large, serialized data
frequently occupy only a small number of regions, and yet there are
opportunities to exploit parallelism in their construction.
%
Consider the \il{buildtree} function, mentioned earlier in this chapter, which
creates a binary tree of a given size \il{n} in a given region \il{r}.
%
If we want to access the parallelism between the recursive calls,
we need to break the data dependency that the right branch has on
the left.
%
The starting address of the right branch, namely $\locreg{l_b}{r}$, is
assigned to be end witness of the left branch by the \il{letloc}
instruction.
%
But the end witness of the left branch is, in general, known only
after the left branch is completely filled, which would effectively
sequentialize the computation.
%
One non-starter would be to ask the programmer to specify the
size of the left branch up front, which would make it possible to
calculate the starting address of the right branch.
%
Unfortunately, this approach would introduce safety issues, such as
incorrect size information, of exactly the kind that \ourcalc{}
is designed to prevent.
%
Instead, here I will explore an approach that is safe-by-construction and
efficient.
%
% Parallel \ourcalc{} overview continues here.
%
A full formalization and proof of type safety for parallel \ourcalc{}
is future work, so this section will outline a sketch of
how the operational semantics of \ourcalc{} can be extended
to safely allow parallel execution.



The expression syntax is the same, and no changes are necessary to the type system.
The operational semantics do require some changes, most notably the addition
of a richer form of indexing in regions.


% \begin{displaymath}
%   \begin{aligned}
%     \textup{Store} && \STOR && \gramdef & \set{\reg_1 \mapsto \heap_1 , \; \ldots \; , \reg_{n} \mapsto \heap_{n}} \\
%     \textup{\new{Heap Values}} && \heapval && \gramdef & \DC \gramor \indirection{\reg}{\concreteind{\ind}} \\
%     \textup{Heap} && \heap && \gramdef & \set{\concreteind{\ind}_1 \mapsto \heapval_1 , \; \ldots \; , \concreteind{\ind}_{n} \mapsto \heapval_{n}}\\
%     \textup{Location Map} && \MENV && \gramdef & \set{\locreg{\loc}{\reg_1}_1 \mapsto \concretelocvar{}_1 , \; \ldots \; , \locreg{\loc}{\reg_n}_n \mapsto \concretelocvar{}_n} \\
%     \textup{Sequential States} && \SEQSTATE && \gramdef & \STOR ; \MENV ; \EXPR \\
%     \textup{Parallel Tasks} && \TASKVAR && \gramdef & (\hTYP, \concretelocvar{}, \SEQSTATE) \\
%     \textup{Task Set} && \TASKSET && \gramdef & \set { \TASKVAR_1, \ldots, \TASKVAR_n } \\
%     \textup{Concrete Locations} && \concretelocvar{} && \gramdef & \concreteloc{\reg}{\indbef{\ind}}{\loc}\\
%     \textup{\new{Extended Region Indices}} && \indbef{\ind}, \indbef{\indj} && \gramdef & \concreteind{\ind} \gramor \indivar{\ind} \gramor \indirection{\reg}{\concreteind{\ind}}
%   \end{aligned}
% \end{displaymath}

Take, as an example, a \ourcalc{} computation involving a binary tree, where the
left child \il{a} at location $\locreg{\loc_a}{\reg}$ is to be computed in
parallel with the right child \il{b}.
Each task has its own private view of memory, which is realized by
giving the child and parent task copies of the store $\STOR$ and
location map $\MENV$.
%
These copies differ in one way, however: each sees a different mapping
for the starting location of \il{a}. %, namely $\locreg{\loc_a}{\reg}$.
%
The child task sees the mapping
$\locreg{\loc_a}{\reg} \mapsto \concreteloc{\loc_a}{1}{}$, which
is the ultimate starting address of \il{a} in the heap.

The parent task sees a different mapping for $\locreg{\loc_a}{\reg}$,
namely $\concreteloc{\reg}{\indivar{1}}{}$.
%
This location is an \emph{ivar index}: it behaves exactly like an I-Var~\cite{IStructures},
and, in our example, stands in for the completion of the memory being
filled for \il{a}, by the child task.
%
Any expression in the body of the let expression that tries to read
from this location blocks on the completion of the child task.
%
When \il{a} is used, it will force the parent task to join with
its child task. The ivar index $\concreteloc{\reg}{\indivar{1}}{}$
will be substituted with $\concreteloc{\loc_a}{1}{}$, and
all the new entries in the location map and store map of the
child task are merged into the corresponding environments in
the parent task. Finally, the results are wired together using
indirections, as covered in~\secref{subsec:indirections}.

To make parallel \ourcalc{} practical when adding it to Gibbon, it was
necessary to add \il{spawn} and \il{sync} forms to the language, so
that it is explicit where parallelism should be exploited, but these
annotations are not strictly necessary in the semantics of
parallel \ourcalc{}.
%
A thorough evaluation of the performance of parallel \ourcalc{} is
in~\secref{subsec:parallelbench}.
